%%%%%%%%%%%%%%%%%%%%%%%%%%%%%%%%%%%%%%%%%%%%%%%%%%%%%%%%%%%%%%%%%%%%%%%%
%%            This file was generated from DICTIONARY.dict.txt
%%
%%                         DO NOT EDIT
%%%%%%%%%%%%%%%%%%%%%%%%%%%%%%%%%%%%%%%%%%%%%%%%%%%%%%%%%%%%%%%%%%%%%%%%

\entry{a}{pron.}{\pf{quoi}}{\textit{Interrogative and relative pronoun}.\\\s{indef} What?\\\s{def} Who? Whom?\\\s{indef} \textit{or} \s{def} Which, who, that \textit{(see grammar)}.}{}
\entry{á}{n.}{\pf{âme}}{Spirit.}{}
\entry{aa}{interj.}{\textnf{onomatopoeic}}{Ah, oh.}{\s{nd} ā}
\entry{aḅ}{v.}{\pf{appeler}}{\\+\s{acc} \textit{and} \s{abs} To call, name, nominate, give a name to. {\itshape{}In PF, this verb used to take a double accusative, but this usage disappeared early on in UF, with the second accusative, denoting the name, naturally being replaced by the absolutive, likely to avoid ambiguity that was starting to manifest as a result of UF’s increasingly free word order.}\ex \s{Snet’h v.2} \w{jdap rác’hsaý’adâ} ‘I call you a liar’. {\itshape{}Even in the writings of \s{Snet’h}, the double accusative is no longer attested.}\\\textit{(archaic)} To call oneself, be called. {\itshape{}Replaced by \w{nvẹ́} in Early Modern UF in the reflexive sense.}}{}
\entry{áb}{n.}{\pf{ambre}}{Amber.}{}
\entry{ab’há}{conj.}{\pf{avant que}}{+\s{opt} Before.}{}
\entry{áb’há}{n.}{\pf{enfant}}{Child.}{}
\entry{ab’haḍ}{v.}{\pf{abattre}}{\\To cut down, fell, knock down, shoot down.\\To butcher, cut apart violently.}{\s{fut} ab’haḍrẹ́, \s{subj} ab’has}
\entry{abhár}{n.}{\pf{apartement}}{House.}{}
\entry{áb’has}{v.}{\pf{en face}}{To be the front of sth.}{}
\entry{ab’hásy’ô}{n.}{\pf{aviation}}{Aviation.}{}
\entry{abhaúrḍ}{v.}{\pf{apporter}}{+\s{acc} To bring (+\s{all} to sbd.).}{}
\entry{áb’haý’}{v.}{\pf{emballer}}{To coil, wind.}{\s{fut} áb’haý’ẹ́, \s{subj} áb’haý’s}
\entry{áb’hé}{adv.}{\pf{enfin}}{Finally, at last, at the end.}{}
\entry{ábhec}{v.}{\pf{empêcher}}{+\s{acc} To prevent, stop (sth. from happening).}{\s{fut} ábhece, \s{subj} ábhecs}
\entry{ab’hèc’h}{v.}{\pf{affecter}}{+\s{acc} To affect, influence.}{\s{fut} ab’hèc’hre, \s{subj} ab’hè\-c’hes}
\entry{áb’hẹḍ}{v.}{\pf{embêtter}}{\\+\s{acc} To disturb, inconvenience sbd.\\+\s{part} To harass, bother sbd.}{}
\entry{abhérś}{v.}{\pf{apercevoir}}{To behold, descry (+\s{part}).}{}
\entry{ab’héy’}{n.}{\pf{abeille}}{Bee.}{}
\entry{ab’hiḍy’él}{v.}{\pf{habituel}}{To be usual, customary, common.}{}
\entry{ab’hínéb’heḅaý’évrâ}{v.}{\pf{habit ne fait pas le moi\-ne}}{To judge based on appearances.}{\s{fut} ab’hínéb’heḅaý’év́ẹ́, \s{subj} ab’hínéb’heḅaý’\-év́\-ás}
\refentry{ábhírâ}{ábhíré}
\entry{ábhíré}{adj.}{\pf{empyrée}}{To be divine, heavenly, empyrean, celestial.}{\s{ptcp} ábhírâ}
\entry{áb’hóhẹ}{v.}{\pf{enfoncer}}{To push, press, shove, drive (+\s{ill} into).}{}
\entry{aḅrâ}{v.}{\pf{apprendre}}{\\+\s{part} \textit{or} \s{intr.} To learn.\\+\s{acc} To teach (+\s{dat} sbd.). {\itshape{}See also.}}{\s{fut} aḅrâdé, \s{subj} aḅrâs}
\entry{aḅraúc̣}{v.}{\pf{approcher}}{To approach, come near, walk up to (+\s{all} sbd./sth.).}{\s{fut} aḅraúc̣é, \s{subj} aḅraúc̣s}
\entry{aḅrdvê}{adv.}{\pf{après-demain}}{The day after tomorrow. {\itshape{}The prefix \w{aḅr} can be prepended as often as necessary, e.g. \w{aḅraḅraḅrdvê} would be ‘in four days’.}}{}
\entry{ac}{n.}{\pf{hache}}{Axe, hatchet.}{}
\entry{ac’h}{n.}{\pf{acte}}{Act, action.}{}
\refentry{ach’es}{\w{a} + \w{c’hes}}
\entry{act’he}{v. tr.}{from \w{ac}}{\\To cut or cleave with an axe.\\+\s{acc} To bring an end to.\\+\s{acc def} \textit{of \w{árb} intr. (other than literal)} To get to the point, cut to the chase.\\+\s{acc def} \textit{of \w{árb} and \s{acc}} To bring to light, reveal. {\itshape{}Originally, this idiom did not take a double \s{acc}, but was instead formed with the \s{acc} of ‘tree’ and the \s{ill} of the object, meaning something along the lines of ‘to bring down the tree(s) on sth’—the image here being that of cutting down trees in a wood until only a clearing remains or is ‘brought to light’.}}{\s{fut} acḍe, \s{subj} act’hes}
\entry{ac̣t’he}{v. tr.}{\pf{acheter}}{To buy.}{\s{fut} ac̣ḍrẹ́, \s{subj} ac̣t’hes}
\entry{ad’he}{v.}{\pf{vader}}{To go.}{\s{fut} í, \s{subj} al}
\entry{ad’hór}{v. tr. or n.}{\pf{adore}}{\\+\s{acc} To love, adore.\\+\s{part} To be in love with, have a crush on.\\+\s{gen} To desire, yearn for sbd./sth.\ex \s{Snet’h iv.17} \w{jad’hóré ávvaúríhe} ‘I yearned to remember’. {\itshape{}Compare \w{jad’hóré devvaúríhe} ‘I loved to remember’.}\\\textit{n.} Love, adoration.}{\s{fut} ad’hórérẹ́, \s{subj} ad’hórs}
\entry{ád’húy’}{n.}{\pf{andouille}}{Fool, buffoon, idiot.}{}
\entry{ad’hyl}{v.}{\pf{adulte}}{To be adult, grown-up.}{\s{fut} ad’hyle, \s{subj} ad’hyls}
\entry{ád’hýr}{v.}{\pf{endure}}{To resist, endure, withstand (+\s{acc} sth.).}{}
\entry{aḍrá}{v.}{\pf{attraper}}{\\+\s{acc} To take.\\+\s{part} To take or remove a part of.\\\w{aḍrá faúr} \textit{intr.} To take shape, take form.}{}
\entry{áḍrébh}{n.}{\pf{entrepôt}}{Warehouse, depot.}{}
\entry{áḍrén}{v.}{\pf{entraîner}}{\\+\s{acc} To train, coach.\\\textit{refl.} To exercise.}{}
\entry{aḍríb’h}{n.}{\pf{atribut}}{Property, attribute.}{}
\entry{ádróid}{n.}{\pf{androïde}}{Android.}{}
\entry{ádrrá}{n.}{\pf{endroit}}{Location, place, locale.}{}
\entry{advnés}{v.}{\pf{admonester}}{To admonish, reprimand, scold.}{}
\entry{áḍy’ẹ́}{num. frac. or n.}{\pf{entier}}{\\Entire, whole. {\itshape{}Seldom used metaphorically, for that see \w{ḍẹ} instead, e.g. \w{vaûd ḍẹ} ‘all the world’, i.e. everyone, as opposed to \w{vaúd áḍy’ẹ́} ‘the world as a whole’, i.e. the entire world considered as a physical object.}\\\textit{n.} Entirety, wholeness.}{}
\entry{ady’ŷ}{v. or interj.}{\pf{adieu}}{\\Goodbye, farewell.\\+\s{gen} To say goodbye to sbd., bid sbd. farewell.}{}
\entry{áẹ}{n.}{\pf{en-haut}}{Sky. {\itshape{}Often plural, especially in a religious sense.}}{}
\entry{ah}{n.}{\pf{assez}}{\textit{sufficient comparative prefix; see §~\ref{subsec:comparison}}.}{}
\entry{áháb}{adv.}{\pf{ensemble}}{Together.}{}
\entry{áhaúr}{conj.}{\pf{encore}}{+\s{subj} Even though.}{}
\entry{áhaúr}{adv.}{\pf{encore}}{\\\textit{pos.} Still, again.\\\textit{neg.} Not yet.}{}
\entry{áhâłát’hẹ}{v.}{\pf{ensanglanté}}{To be (very) bloody, bloodstained.}{}
\entry{áhén}{v.}{\pf{enseigner}}{\\\textit{intr.} To teach, be a teacher.\\To teach (+\s{acc} sth.) (+\s{dat} to sbd.). {\itshape{}Historically, this word used to govern the \s{part} instead, but it later changed to the \s{acc}, most likely under influence from \w{aḅrâ}.}}{}
\entry{ahíví}{v. tr.}{\pf{assimiler}}{\\+\s{acc} To assimilate.\\+\s{part} To liken, equate (+\s{gen} to sth.).\\+\s{acc} (\textit{of food}) To digest.}{\s{fut} ahívíý’ẹ́, \s{subj} ahívíý’s}
\entry{ahúr}{v.}{\pf{assurer}}{To ensure.}{\s{fut} ahúré, \s{subj} ahúrs}
\entry{aír}{v.}{\pf{hair}}{To hate, abhor, detest, loathe, despise (+\s{acc} sbd./sth.).}{}
\entry{ajut’h}{v.}{\pf{ajouter}}{To add, append.}{}
\entry{ânb’hé}{adv.}{\pf{en effet}, via metathesis from *\w{âné\-b’he}}{Verily, indeed, in fact.}{}
\entry{ánẹ́}{n.}{\pf{année}}{Year.}{}
\entry{ânérb}{v.}{\pf{enherber}}{To poison.}{}
\entry{ánvá}{n.}{\pf{animal}}{Animal.}{}
\entry{ánvé}{v. tr.}{\pf{animer}}{+\s{acc} To bring to life, animate.}{}
\entry{ár}{n.}{\pf{arme}}{Weapon.}{}
\entry{aráráhúr}{adj.}{\pf{à bras raccourcis}}{Ferocious, violent.}{}
\entry{árás}{v.}{\pf{agacer}}{+\s{acc} To annoy, anger.}{}
\entry{árásaû}{n.}{from \w{árás}}{Anger, wrath, ire.}{}
\entry{áraú}{v.}{\pf{haro}}{\\To cry for help.\\\textit{interj.} A cry for help.}{}
\entry{árb}{n.}{\pf{arbre}}{Tree.}{}
\entry{arc’hais}{n.}{\pf{archaïsme}}{Archaism.}{}
\entry{árḍihyl}{n.}{\pf{particule}}{Particle.}{}
\entry{áré}{n.}{\pf{arrêt}}{Stop, stopping.}{}
\entry{áríb’h}{v.}{\pf{arriver}}{To arrive, reach a destination.}{}
\entry{árjá}{n.}{\pf{argent}}{Silver.}{}
\entry{ârjisḍ}{v.}{\pf{enregistrer}}{To record, make a record of.}{\s{fut} ârjisḍrẹ́}
\entry{árrih}{n.}{\pf{arbrisseau}}{Forest, woods, woodland.}{}
\entry{ársl}{v.}{\pf{harceler}}{To attack, assail, beset, bully (+\s{acc} sbd.).}{}
\entry{ârýý’}{v.}{\pf{enrouler}}{To wrap (+\s{acc} around sth.).}{}
\entry{aśár}{v.}{\pf{asseoir}}{\\+\s{acc} To seat, sit sbd. down.\\\textit{refl.} To sit, be seated.}{}
\entry{asḍrál}{n.}{\pf{aster} + \pf{étoile}}{Star.}{}
\entry{asy’ẹ́}{n.}{\pf{acier}}{Steel.}{}
\entry{ásy’ê}{v.}{\pf{ancien}}{To be ancient.}{\s{fut} ásy’êr, \s{subj} ásy’ês}
\entry{asý’ýâ}{particle}{\pf{pas absolument}}{Not, no. {\itshape{}Commonly \w{’sý’ýâ} after vowels and verbs. This particle is used only in the indicative; see also \w{sá}, \w{t’hé}.}}{}
\entry{at’hád}{v.}{\pf{attendre}}{\\+\s{acc} To wait for, await.\\+\s{part} To expect.}{}
\entry{át’hád}{v.}{\pf{entendre}}{To hear, perceive (+\s{part} sbd./sth.).}{\s{fut} át’hádé, \s{subj} át’hás}
\entry{át’has}{v.}{\pf{entasser}}{\\\s{+acc} To heap, accumulate.\\\textit{refl.} To pile up, heap.}{}
\entry{át’hér}{v.}{\pf{enterrer}}{+\s{acc} To bury, inter.}{}
\entry{au}{conj.}{\pf{aussi}}{\\And, also, as well, too.\\\w{au} \ldots{} \w{au} \ldots{} (\w{’ḍ}) ‘both \ldots{} and \ldots’ {\itshape{}When an adjective is to be applied to a conjunction consisting of noun phrases, the particle \w{’ḍ}, historically a reduced form of \w{ḍẹ} ‘all’, may be used to avoid repetition of the adjective.}\ex \w{Au árb au raû rá} ‘The tree and the big log’.\ex \w{Au árb rá au raû} ‘The big tree and the log’.\ex \w{Au árb rá au raû rá} ‘The big tree and the big log’.\ex \w{Au árb au raû’ḍ rá} ‘The big tree and the big log’.}{}
\entry{aú}{n.}{\pf{homme}}{Man, human.}{}
\entry{aû}{particle}{\pf{non}}{Not-. {\itshape{}Used to negate nouns, adjectives, and adverbs; see §~\ref{subsubsec:noun-negation}.}}{}
\entry{aub}{n.}{\pf{aube}}{Dawn, sunrise, daybreak.}{}
\refentry{aubhaus}{ní}
\entry{aub’hèḍ}{adv.}{\pf{au fait}}{By the way, btw, incidentally.}{}
\entry{aub’heír}{v. (in)tr.}{\pf{obéir}}{To obey (+\s{part} sbd.).}{}
\entry{aublit’hér}{v.}{\pf{oblitérer}}{\\To defeat, vanquish, obliterate (+\s{acc} sbd./sth.).\\To be better than, ‘beat’ (+\s{inf} sbd/sth.).}{}
\entry{auc’hsid’h}{v. or n.}{\pf{occident}}{\\\textit{v.} To be western.\\\textit{n.} \w{auc’hsid’h} West.}{}
\entry{auḍ}{adj.}{\pf{autre}}{Other, another.}{}
\entry{auḍ}{v.}{from \pf{haute}, the \textit{fem.} form of \pf{haut}}{To be high.}{}
\entry{auḍé}{v.}{\pf{obtenir}}{\\To obtain, get, acquire.\\+\s{abl} To gain purchase on or hei\-ght or distance from.}{\s{fut} auḍy’édrẹ́}
\entry{auha}{conj.}{\pf{au cas où}}{+\s{opt} In case.}{}
\entry{auhéy’bhaún}{n.}{From \pf{oseille} + \pf{pognon}}{Money.}{}
\entry{auhit’h}{v.}{\pf{aussitôt}}{To be early.}{}
\entry{auhybh}{v.}{\pf{occuper}}{\\\textit{tr.} +\s{iness} To inhabit, occupy.\\\textit{intr.} To be busy (engaged in an activity).\\\textit{intr.} To be crowded, busy (of an area or street).\\\textit{intr.} To be occupied, taken.}{}
\entry{aujúrdy’í}{adv.}{\pf{aujourd’hui}}{\textit{(archaic)} Today, nowadays. {\itshape{}See also \w{júrdy’í}.}}{}
\entry{aúr}{n.}{\pf{or}}{Gold.}{}
\entry{aúrâ}{v.}{\pf{orange}}{To be orange.}{}
\entry{aúráj}{n.}{\pf{orage}}{\\\textit{(usually pl.)} Storm, tempest, thunderstorm.\ex \s{Snet’h ii.7} \w{phárýaúráj téríbâ} ‘like a terrible storm’.\\\textit{fig.} Upheaval, turmoil, crisis.}{}
\entry{aúraúr}{n.}{\pf{aurore}}{\\Dusk, sunset.\\Aurora.}{}
\entry{aúrd}{n.}{\pf{ordre}}{\\Order (arrangement).\\Order, tidiness.\\Order, command, instruction.\\Order (military group).\\\s{orn} \w{vé(t’hý)naúrd} Orderly, tidy.\\\s{def gen} In the region of, around, approximately. {\itshape{}Always followed by an ordinal number in the \s{abs} and then by a noun phrase also in the \s{abs}. Since the number is usually not 1, the noun phrase is usually in the \s{pl}.}\ex \w{áaúrd dizy’ê t’halẹ} ‘around 10 tables’, ‘10 or so tables’. {\itshape{}Literally ‘tables of the 10th order’.}}{}
\entry{aúry’ât’h}{n.}{from \w{aúry’ât’hal}}{East.}{}
\entry{aúry’ât’hal}{v.}{\pf{oriental}}{To be eastern. {\itshape{}See also \w{aúry’ât’h}.}}{}
\entry{aus}{n.}{\pf{os}}{Bone.}{}
\entry{ausc’hýr}{v.}{\pf{obscur}}{To be dark.}{}
\entry{aut’heldývér}{n.}{\pf{hôtel (de ville) + du maire}}{Town hall.}{}
\entry{auu}{interj.}{\textnf{onomatopoeic}}{Wow, whoa, ooh. {\itshape{}Written with as many ‘u’s as necessary, e.g. \w{auuuuu}, but always at least two to differentiate it from \w{au}.}}{\s{nd} aū}
\entry{aúú}{interj.}{\textnf{onomatopoeic}}{Hmm. {\itshape{}Written with as many ‘ú’s as necessary, e.g. \w{aúúúú}, but usually at least two, paralleling \w{auu}.}}{\s{nd} aǔ}
\entry{aúvaúḅlaḍ}{n.}{\pf{omoplate}}{\\Shoulder.\\Scapula.}{}
\entry{aúłau}{n.}{\pf{horloge}}{Time. {\itshape{}In the SD, this word has replaced earlier \w{ḍá} entirely except in set phrases.}}{}
\entry{áv́}{v.}{\pf{armer}}{To arm (+\s{acc} sbd.).}{}
\entry{av́ár}{v. irreg.}{\pf{avoir}}{+\s{acc} To have. {\itshape{}This usually denotes inalienable possession.}}{\s{pres ant} and \s{pret} y, \textit{obsolete} \s{pret} ab’hẹ, \s{fut} aúrẹ́, \s{subj} ès}
\entry{av́árḷý}{v.}{\pf{avoir lieu}}{To take place, happen.}{\s{fut} lav́árḷýé, \s{subj} lav́árḷýs}
\entry{áv́áy’é}{v.}{\pf{envoyer}}{\\To send.\\\w{Ádác’hebèc’h áv́áy’é} To kill someone. {\itshape{}Lit. ‘to send to the Promised Land’.}}{\s{fut} áv́áy’érẹ́, \s{subj} áv́áy’és}
\entry{áví}{n.}{\pf{ami}}{Friend.}{}
\entry{ávrê}{conj.}{\pf{à moins que}}{+\s{opt} Unless.}{}
\entry{aý’a}{particle}{from earlier \w{aý’afrá}}{At the same time, at once, both. {\itshape{}Typically postpositive, i.e. \w{X Y aý’a} ‘both X and Y’.}\ex \w{nárb ýrŷâ rísá aý’a} ‘a tree both happy and sad’.}{}
\entry{aý’afrá}{particle}{\pf{à la fois}}{\textit{Obsolete form of \w{aý’a}}.}{}
\entry{aý’aúr}{conj.}{\pf{alors}}{While, as (temporal), because.}{}
\entry{Aý’èc’hsád}{n.}{\pf{Alexandre}}{\textit{Male given name}.}{}
\entry{aý’ívât’h}{n.}{\pf{alimenter}}{Food.}{}
\entry{áł}{v.}{from earlier *\w{ḅał} < \pf{parler}}{To speak.}{}
\entry{âłut’h}{v.}{\pf{engloutir}}{+\s{acc} To swallow up, engulf.}{}
\entry{áȷ́éd}{v.}{\pf{enjoindre}}{To order, enjoin, command.}{}
\entry{ḅ}{n.}{\pf{époux}}{Spouse, wife, husband, partner.}{}
\entry{ba}{v.}{\pf{baser}}{To base on, found on.}{\s{fut} bare, \s{subj} bas}
\entry{ḅá nórávíc’h}{n. archaic}{\pf{Panoramix}}{Druid. {\itshape{}Only the \w{nórávíc’h} is inflected; infixing of adj. is attested.}\ex \s{Snet’h}, \s{iii.2}: \w{derúb’h phá ráinórávíc’h} ‘to find the great druid’, with infixed \w{rá}.}{}
\entry{ḅabh}{n.}{\pf{papa}}{Father.}{}
\entry{ḅaḅrás}{n.}{\pf{paperasse}}{Paper.}{}
\entry{baḍ}{v.}{\pf{battre}}{To beat, strike, hit (+\s{acc} sbd./sth.).}{}
\entry{ḅáhẹ}{n.}{\pf{pensée}}{Thought, reflection, meditation, faculty of thinking.}{}
\entry{ḅaj}{n.}{\pf{page}}{Page.}{}
\entry{ḅánár}{n.}{\pf{panard}}{Foot.}{}
\entry{bár}{v.}{\pf{parier}}{\\To bet, wager.\\+\s{aci} To be sure of something.}{}
\entry{ḅará}{n.}{\pf{parent}}{Parent.}{}
\entry{bárc’húr}{v.}{\pf{parcourir}}{\\+\s{perl} To pass, pass through, move through.\\+\s{perl} To cross a border.\\\textit{intr.} To spend time.\\+\s{part} To take a test, sit an exam.\\To pass an exam.}{}
\entry{ḅarḍ}{v.}{\pf{partir}}{To leave, go away, depart.}{\s{fut} ḅarẹ́, \s{subj} ḅars}
\entry{ḅárḍáḍ}{v.}{\pf{partante}}{(+\s{aci}) To be interested in, willing to, ready to, prepared for.}{}
\entry{ḅárḍáḍaû}{n.}{From \w{ḅárḍáḍ}}{Interest.}{}
\entry{ḅárḍẹ}{n.}{\pf{partie}}{\\Part, portion, piece, fraction of a whole.\\Stretch (e.g. of a road).}{}
\entry{ḅárḍihibhá}{n.}{\pf{participant}}{Participant.}{}
\entry{ḅárḍihyw}{v.}{\pf{particulier}}{\\To be particular.\\To be specific.}{}
\entry{ḅáréḍ}{v.}{\pf{parraitre}; future stem from \pf{sembler}}{To seem, appear.}{\s{fut} sáb}
\entry{ḅas}{conj.}{\pf{parce que}}{+\s{subj} Because. {\itshape{}Often used to explain motivation rather than cause, as in e.g. ‘We did that because\ldots’}}{}
\entry{baú}{v. irreg.}{\pf{bon}}{\\To be good, well, healthy.\\To be right, correct, appropriate.\\\textit{usually intr.} To satisfy, fullfill, gratify.}{\s{fut} baúré, \s{subj} véy’ýrs; \s{comp} lẹvéy’ýr, y’ŷvéy’ýr, rêvéy’ýr; \s{sup} révéy’ýr, râdvâv\-éy’\-ýr}
\entry{ḅaú}{n.}{\pf{pont}}{Bridge.}{}
\entry{ḅaú}{n.}{\pf{pomme}}{Apple.}{}
\entry{ḅauheŷnlabhé}{v.}{\pf{poser un lapin}}{To forsake, abandon.}{\s{fut} ḅauheŷnlabhére, \s{subj} ḅauheŷnlabhés}
\entry{ḅauhib}{v.}{\pf{impossible}}{To be impossible, unfeasible.}{\s{fut} ḅauhibre, \s{subj} ḅauh\-ibes}
\entry{ḅaúr}{n.}{\pf{port}}{Port, harbour.}{}
\entry{ḅaúr}{n.}{\pf{porte}}{Door, gate.}{}
\entry{ḅaúrḍ}{v.}{\pf{porter}}{To carry, bear.}{}
\entry{Baúré}{n.}{\pf{Borée}}{Boreas, the North Wind.}{}
\entry{ḅaúvḍér}{n.}{\pf{pomme de terre}}{Potato.}{}
\entry{ḅauz}{n.}{\pf{pose}}{Pose, posture.}{}
\entry{baz}{n.}{\pf{base}}{Base, bottom part of sth.}{}
\entry{ḅáł}{v.}{\pf{parler}}{To speak, talk.}{\s{fut} báłérẹ́}
\entry{ḅáłýr}{n.}{\pf{parleur}}{Speaker, interlocutor.}{}
\entry{bẹ}{v.}{\pf{bas}}{To be low.}{}
\entry{bèḍ}{n.}{\pf{bête}}{Beast.}{}
\entry{bèl}{v.}{\pf{belle}}{To be beautiful, attractive, pretty.}{}
\entry{ḅelbec}{n.}{\pf{pelle} + \pf{bêche}}{Shovel.}{}
\entry{ḅér}{n.}{\pf{père}}{\textit{(informal)} Father, dad.}{}
\entry{Bèrḍrá}{n.}{\pf{Bertrand}}{\textit{Male given name}.}{}
\entry{ḅérs}{v.}{\pf{percer}}{To pierce, penetrate.}{}
\entry{besy’ál}{v.}{\pf{spécial}}{To be special.}{}
\entry{ḅẹt’hẹ}{v. irreg.}{\pf{petit}}{To be small, little.}{\s{fut} rêdẹ́, \s{subj} ḅẹt’hes; \s{comp} lẹrêd, y’ŷrêd, rêrêd; \s{sup} rérêd, râdvârêd}
\entry{ḅét’hýr}{v.}{\pf{peinture}}{To paint.}{}
\entry{ḅeý’as}{v.}{\pf{épellation}}{To spell, spell out.}{}
\entry{ḅéy’í}{n.}{\pf{pays}}{Country, land, region, nation. {\itshape{}Usually political, see also \w{rẹ́jy’aû}.}}{}
\entry{b’há}{n.}{\pf{vent}}{Wind, breeze.}{}
\entry{b’hac̣}{v.}{\pf{évacuer}}{To evacuate.}{}
\entry{b’hac̣aû}{v.}{from \w{b’hac̣}}{Evacuation.}{}
\entry{b’hár}{n.}{\pf{vague}}{\\Wave.\\\textit{pl.} Ripples, undulations.}{}
\entry{b’hát’hiý’at’hýr}{v.}{\pf{ventilateur}}{To blow.}{}
\entry{b’hauḍ}{v.}{\pf{vôtre}}{To be yours (\s{pl}).}{\s{fut} b’hauḍre, \s{subj} b’haus}
\entry{b’haul}{v.}{\pf{voler}}{To hover, float.}{}
\entry{b’hauý’ý}{n.}{\pf{volume}}{\\Volume.\\\s{orn} \w{vé(t’hý)nb’hauý’ý} Overweight, obese, fat. {\itshape{}Lit. ‘voluminous’.}}{}
\entry{b’hây’ér}{adv.}{\pf{avant-hier}}{The day before yesterday. {\itshape{}The prefix \w{b’hâ} can be prepended as often as necessary, e.g. \w{b’hâb’hâb’hây’ér} would be ‘four days ago’.}}{}
\entry{b’hał}{v.}{\pf{valoir}}{\\+\s{gen} To be worth, equal in worth to.\\\textit{refl. pl.} To be the same or equivalent to each other.\\\textit{intr.} \w{b’hał áhuf} To be worth the trouble, worth it, worthwhile. {\itshape{}Literally ‘to be worth the pain’.}}{\s{fut} b’hałárẹ́, \s{subj} b’hałás}
\entry{b’he}{conj.}{\pf{envers}}{+\s{subj} So that, so as to, to, in order to. {\itshape{}Commonly enclitic \w{’b’h} after vowels.}}{}
\entry{b’hé}{n.}{\pf{vin}}{Grape.}{}
\entry{b’hénvâ}{n.}{\pf{évènement}}{Event, occurrence.}{}
\entry{b’hér}{v.}{from \pf{verser} and \w{fér}, see \senseref{2}}{\\+\s{part} \textit{(archaic)} To pour out.\\\s{pass} + \s{part/acc} \textit{(of a state or condition)} to obtain; \textit{(of precipitation)} to fall. {\itshape{}Exclusively used in the passive of the weather or a situation or state of affairs, with the noun for the condition or preciptation as the ‘object’ and no subject (see examples below). \par In Middle UF, \w{fér} with a noun in the \s{abs} was often used for this purpose, hence also the irregular \s{fut} and \s{subj}, e.g. *\w{sfér cẹaû} ‘It is hot’; similarly, weather conditions used to be their own verbs, e.g. *\w{slýv́á} ‘it rains’. \par The latter were eventually replaced with a construction using \w{b’hér} in the passive with the noun in the \s{part}, e.g. *\w{dŷnḷẹ syb’hér} ‘it rains’ (lit. ‘some rain is being poured’). This was later extended to any state (of the weather), e.g. \w{dŷncẹaû syb’hér} (lit. ‘some heat is being poured’). The \s{acc} can be used instead as an intensifier, e.g. \w{sc̣ẹaû syb’hér} ‘it is scorching hot’. \par With precipitation, the \s{pl} is most common in modern times, e.g. \w{dýḷẹ syb’hér} instead of older *\w{dŷnḷẹ syb’hér}. The old weather verbs, e.g. *\w{lýv́á}, are altogether obsolete.}\ex \w{dýḷẹ syb’hér} ‘it rains’.\ex \w{dŷncẹaû syb’hér} ‘it is hot’.\ex \w{dyraúvá léâ syb’hér} ‘it’s a full moon’.}{\s{fut} b’hẹ́, \s{subj} b’hés}
\entry{b’hér}{v.}{\pf{vert}}{To be green.}{\s{nd} b’haǐ}
\entry{b’hér}{n.}{\pf{verre}}{Glass (substance).}{}
\entry{b’hér}{n.}{\pf{verbe}}{Verb.}{}
\entry{b’hérḍy’ŷ}{v.}{\pf{vertueux}}{To be virtuous.}{}
\entry{b’hẹ́ríf}{v.}{\pf{vérifier}}{To check, inspect, examine.}{}
\entry{b’hért’he}{n.}{from \pf{vérité}; the /i/ disappeared during Early Middle UF}{\\Truth.\\Fact.}{}
\entry{b’héy’}{v.}{\pf{veiller}}{\\\textit{intr.} To keep watch, keep guard.\\+\s{spress} To watch over, guard, keep an eye on.}{}
\entry{b’heý’au}{n.}{from archaic \w{b’heý’auhic’h}}{Bicycle.}{}
\entry{b’heý’auhic’h}{n. archaic}{\pf{vélo-cycle}}{Bicycle.}{}
\entry{b’heý’o}{v.}{back-formation from *\w{b’heý’os}, reanalysed as a subjunctive; from \pf{véloce}}{To be quick, fast.}{\s{fut} b’heý’o\-se, \s{subj} b’heý’os}
\entry{b’hí}{n.}{\pf{vigne}}{Vine.}{}
\entry{b’hib’hárjá}{n.}{\pf{vif-argent}}{Mercury, quicksilver.}{}
\entry{b’hic’hḍrár}{n.}{\pf{victoire}}{Victory.}{}
\entry{b’hid}{v.}{\pf{vide}}{To be empty.}{}
\entry{b’hiḍ}{n.}{\pf{vitre}}{\\Pane of glass.\\\s{nd} Glass (substance).}{}
\entry{b’hil}{n.}{\pf{ville}}{Town, city.}{}
\entry{b’hiý’a}{n.}{\pf{village}}{Small Town.}{}
\entry{b’hizy’ô}{n.}{\pf{vision}}{Vision.}{}
\entry{b’hóy’ẹ}{v.}{\pf{voler}}{To fly. Flight.}{}
\entry{b’huḍ}{n.}{\pf{voûte}}{Vault, arched ceiling.}{}
\entry{b’hýd’hír}{v.}{from \pf{veut dire}, \s{1sg} of \pf{vouloir dire}}{To mean, signify.}{}
\entry{b’hýlnẹ́r}{v.}{\pf{invulnérable}}{+\s{instr} To be incapable of being af\-fec\-ted by, invulnerable to.}{\s{fut} b’hýlnẹ́rẹ́, \s{subj} b’hýlnẹ́rs}
\entry{b’hŷnnúb’hâ}{adv.}{old \s{all} of \w{núb’hâ}}{Anew.}{}
\entry{ḅih}{v.}{\pf{piquer}}{To sting.}{}
\entry{ḅínár}{n.}{\pf{pinard}}{Wine.}{}
\refentry{bír}{vaúb’hẹ}
\entry{biwaú}{n.}{\pf{billion}}{\textit{(obsolete)} Billion (long scale, i.e. 10\Sup{12}). {\itshape{}Replaced with modern \w{dýwaú}).}}{}
\entry{ḅíy’ýrt’haúb}{n.}{\pf{pilleur de tombe}}{Archaeologist.}{}
\entry{ḅlyc̣}{v. tr.}{\pf{éplucher}}{+\s{circabl} To peel. {\itshape{}This verb is construed from the peel’s point of view, i.e. the peel is removed ‘from around’ something, hence the use of the \s{circabl} for the object.}}{}
\entry{ḅré}{conj.}{\pf{après que}}{+\s{opt} After.}{}
\entry{ḅrýb’h}{n.}{\pf{épreuve}}{\\Test. {\itshape{}For ‘exam’, see \w{ḍèsrzávê} instead.}\\Trial, challenge, ordeal.}{}
\entry{ḅúrc’hrá}{pron.}{\pf{pourquoi}}{\textit{(obsolete)} Why. {\itshape{}For usage, see \w{c’hrá}.}}{}
\entry{ḅusy’ér}{n.}{\pf{poussière}}{Dust.}{}
\entry{but’héy’}{n.}{\pf{bouteille}}{Litre, liter (metric unit).}{}
\entry{ḅýt’hèḍ}{adv.}{\pf{peut-être}}{+\s{subj} Perhaps, maybe.}{}
\entry{bźé}{v.}{\pf{besoin}}{+\s{part} To need, require.}{}
\entry{cá}{n.}{\pf{chambre}}{Room, chamber.}{}
\entry{cabh}{n.}{\pf{chapeau}}{Hat.}{}
\entry{cabharḍrár}{n.}{\pf{échappatoire}}{\\Escape route.\\\textit{fig.} A way out, loophole, solution.}{}
\entry{cac’h}{adj.}{\pf{chaque}}{Each, every.}{}
\entry{caḍráy’ẹ́}{v.}{\pf{chatoyer}}{To shimmer, iridesce.}{}
\entry{cah}{v.}{\pf{chasser}}{To hunt.}{\s{fut} cahe, \s{subj} cas}
\entry{cáhaú}{n.}{\pf{chanson}}{Song.}{}
\entry{cahý}{pron. pl. indef.}{\pf{chacun}}{Each other, one another.}{}
\entry{Cár}{n.}{}{Charles, Kyle. {\itshape{}Male given name. Often declined like a regular noun, i.e. with \s{nom} \w{Lác̣ár}.}}{}
\entry{cáry’aú}{n.}{\pf{chariot}}{Cart, carriage.}{}
\entry{cás}{n.}{\pf{chance}}{\\Luck.\\Chance.}{}
\entry{cát’h}{v.}{\pf{chanter}}{(+\s{part}) To sing.}{}
\entry{cb’hal}{n.}{\pf{cheval}}{Horse.}{}
\entry{cẹ}{v.}{\pf{chaud}}{To be hot. {\itshape{}For weather, \w{cẹaû} is used instead; see also \w{b’hér}.}}{}
\entry{će}{v.}{\pf{échouer}}{\\+\s{part} To stumble, do a bad job at.\\+\s{acc} \textit{or} \s{aci} To fail, flunk, not pass.}{\s{fut} ćere, \s{subj} ćes}
\entry{cẹaû}{n.}{from \w{cẹ}}{\\Heat.\\\s{part} + \s{pass} of \w{b’hér} To be hot.\ex \w{Dŷncẹaû syb’hér.} ‘It is hot.’}{}
\entry{cèc}{n.}{phonetic respelling of \w{cèc’h}}{\textit{(chess)} Check.}{}
\entry{cèc’h}{n.}{\pf{échec}}{Failure, defeat.}{}
\entry{cér}{v.}{\pf{cher}}{\\To be dear, important (+\s{dat} to sbd.). {\itshape{}Possession of a noun qualified with this adjective verb is generally construed with the dative rather than the genitive, e.g. \w{asvẹ áví cérâ} or \w{áví cérâvé} ‘my dear friend’, rather than *\w{vaú áví cérâ}.}\\\textit{with \w{áví} ‘friend’} To be friends with.\ex \w{áví lẹcérvé} ‘he is a (dear) friend of mine’.}{\s{fut} céré, \s{subj} cés}
\entry{cér}{n.}{\pf{chair}}{Flesh.}{}
\entry{cévê}{n.}{\pf{chemin}}{Street.}{}
\entry{c’há}{n.}{\pf{camp}}{Camp.}{}
\entry{c’hábh}{v. intr.}{\pf{camper}}{To camp.}{}
\entry{c’habhahit’hẹ}{n.}{\pf{capacité}}{Skill, capacity, ability.}{}
\entry{c’haḍ}{num.}{}{Four.}{}
\entry{c’haḍríy’ê}{num.}{}{Fourth.}{}
\entry{c’hah}{v.}{\pf{casser}}{\\+\s{acc} To break, shatter, smash.\\+\s{part} To crack, make a crack in.}{}
\entry{c’hánár}{n.}{\pf{canard}}{\\Ship, boat.\\\textit{adv.} \w{b’henc’hánár} By boat.}{}
\entry{c’hánaú}{n.}{\pf{canot}}{Duck (bird).}{}
\entry{c’háraúciḍ}{v.}{\pf{les carrotes sont cuites}}{To end for good, put to a permanent end.}{\s{fut} c’hár\-aúc\-re, \s{subj} c’háraúc}
\entry{c’hasḅesy’ál}{n.}{\pf{cas spécial}}{Exception.}{}
\entry{c’hasḍaúr}{n.}{\pf{castor}}{Beaver.}{}
\entry{c’haú}{adj.}{see sense 2}{\\Holy.\\\w{c’haú-}\L{}. {\itshape{}The so-called ‘religious prefix’ is prepended in derivation to nouns that have a religous connotation; this is historically a back-formation from \w{c’haúfrér} and \w{c’haúhýr} which happen to both start with this ‘prefix’.}}{}
\entry{c’haúáł}{n.}{\w{c’haú} + \w{áł}}{Prophecy.}{}
\entry{c’haúbhárrás}{n.}{\w{c’haú} + \pf{paroisse}}{Parish.}{}
\entry{c’haúbhausy’ô}{n.}{\pf{composition}}{Composition, arrangement, structure.}{}
\entry{c’haúb’héc’h}{v.}{\pf{convaincre}}{To persuade.}{}
\entry{c’haúbhèłínáj}{n.}{\w{c’haú} + \pf{pèlerinage}}{Pilgrimage.}{}
\entry{c’haúb’hír}{v.}{\pf{confirmer}}{To confirm, verify.}{}
\entry{c’haúbhýríf}{n.}{\w{c’haú} + \pf{purifier}}{To purify (+\s{acc} sbd./sth.).}{}
\entry{c’haúḅlér}{v.}{\pf{complaire}}{To be complacent; to be accepting in the presence of +\s{gen} sbd./sth. perceived as negative.}{\s{fut} c’haúḅlére, \s{subj} c’haúḅlés}
\entry{c’haúḅrâd}{v.}{\pf{comprendre}}{+\s{part} To comprehend, understand, gr\-asp.}{\s{fut} c’haúḅrâdrẹ́, \s{subj} c’haúḅrâs}
\entry{c’haud}{n.}{\pf{côte}}{Rib.}{}
\refentry{c’haúḍé}{ní}
\entry{c’haúḍrêd’hẹ}{n.}{\pf{compte-rendu}}{Account, rec\-ord.}{}
\entry{c’haúfí}{v.}{\pf{confiner}}{To contain.}{}
\entry{c’haúfrér}{n.}{\pf{confrère}}{Brother (religious). {\itshape{}Masc. or pl. only, see also \w{c’haúhýr}.}}{}
\entry{c’haúhaúvnaút’hẹ}{n.}{\w{c’haú} + \pf{com\-mu\-nau\-té}}{Mo\-nastery.}{}
\entry{c’haúhýr}{n.}{\pf{consœur}}{Sister (religious). {\itshape{}Fem. only, see also \w{c’haúfrér}.}}{}
\entry{c’haul}{n.}{\pf{école}}{School.}{}
\entry{c’haul}{n.}{\pf{col}}{Pass, mountain pass.}{}
\entry{c’haúnéhás}{n.}{\pf{connaissance}}{Knowledge.}{}
\entry{c’haúr}{conj.}{\pf{car} + \pf{comme}}{+\s{subj} As, because, since.}{}
\entry{c’haút’h}{v.}{\pf{content}}{To be content. {\itshape{}This is perceived as weaker than \w{baú} ‘satisfy’.}}{c’haút’hé, c’haúss}
\entry{c’haút’hínâ}{v.}{\pf{continent}}{Continent.}{}
\entry{c’haut’hó}{n.}{\pf{coton}}{Cotton.}{}
\entry{c’haúv́ájẹ}{n.}{\w{c’haú} + \pf{magie}}{Magic.}{}
\entry{c’haúvnaút’hẹ}{n.}{\pf{communauté}}{Community.}{}
\entry{c’haúvs}{conj.}{\pf{comme si}}{+\s{subj} As if, as though.}{}
\entry{c’haúy’ê}{adv.}{\pf{combien}}{\\+\s{gen} How much, how many.\\\w{c’haúy’ê sýná} How long, for how long, since how long ago. {\itshape{}Literally ‘how much time’. This is generally used with either the present or future tense. The present tense is used when the focus is how long ago an ongoing event started in the past, and the future tense when discussing how far it will extend into the future.}}{}
\entry{c’hauý’ó}{n.}{\pf{colonne}}{(\textit{architecture}) Column.}{}
\entry{c’hd’hal}{adv.}{\pf{que dalle}}{Naught, absolutely no\-thing.}{}
\entry{c’hẹ}{n.}{\pf{écho}}{Echo.}{}
\entry{C’hebèc’h}{n.}{\pf{Québec}}{The Promised Land.}{}
\refentry{c’heb’hḍ’ráb’h’}{ráb’haut’h}
\entry{c’heḍ}{n.}{\pf{quête}}{Question.}{}
\refentry{c’heḍ’ráb’h’}{ráb’haut’h}
\entry{c’hèl}{det. postpos.}{\pf{quelques}}{Some, any, a few, a couple of.}{}
\entry{c’hèl}{particle}{\pf{quel}}{\\\s{indef} \textit{(interrog.)} Which. {\itshape{}Replaces \w{c’hes} in this sense.}\\\s{def} \textit{(interrog.)} Which one. {\itshape{}Replaces \w{c’hes} in this sense.}\\+\s{abs} What, what a. {\itshape{}Used in exclamations, always infixed between a noun and its adjectives; if no adjective is present, \this{} is placed after the noun and followed by \w{ḍèl}.}\ex \w{Árb c’hèl rá !} ‘What a big tree!’ vs \w{Árb rá c’hèl ?} ‘Which big tree?’\ex \w{Árb c’hèl ḍèl !} ‘What a tree!’ vs \w{Árb c’hèl ?} ‘Which tree?’}{}
\entry{c’hèlc’hý}{pron.}{\pf{quelqu’un}}{Someone, somebody, anyone, anybody.}{}
\entry{c’hes}{quest. part.}{\pf{qu’est-ce que}}{\textit{See grammar}.}{\textit{often enclitic \w{-c’h’s} in older texts.}}
\entry{c’hesse}{}{contraction of \w{c’hes} + \w{se}}{Is it? {\itshape{}Also substituted for other forms of ‘to be’ in questions, particularly for the plural neuter; stressed on the first syllable.}}{}
\entry{c’hlýr}{v.}{\pf{inclure}}{\\+\s{part} To include.\\+\s{acc} To possess, have \textit{(alienably)}.\\+\s{gen} \textit{usually} \s{indef} To sell, offer, have in stock.}{\s{fut} c’hlýré, \s{subj} c’hlýrs}
\entry{c’hóbhár}{v.}{\pf{comparer}}{+\s{acc} To compare (+\s{gen} with sth.).}{\s{fut} c’hóbhárre, \s{subj} c’hó\-bhárs}
\entry{c’hóhid’hẹ́}{v.}{\pf{considérer}}{\\+\s{part} To consider, think ab\-out, ponder. {\itshape{}For the sense of ‘thinking that something is the case’, see \w{rrá} instead.}\\+\s{acc} To think through.}{\s{fut} c’hóhid’hẹ́rẹ́, \s{subj} c’hóhid’hés}
\entry{c’hóný}{adj.}{\pf{connu}}{Known, familiar, well-kn\-own.}{}
\entry{c’hór}{n.}{\pf{corps}}{Body.}{}
\entry{c’hóvâ}{v.}{\pf{commencer}}{\\(\s{+ part}) To start, commence, begin.\\\s{+ gen} To start out as.\\\w{âc’hóvâ} \s{def} Beginning, start. {\itshape{}Lit. ‘that which is being begun’.}}{\s{fut} c’hóvârẹ́, \s{subj} c’hóv\-ás}
\entry{c’hóvníc’h}{v.}{\pf{communiquer}}{\\To communicate (+\s{instr} with sbd.).\\Communication.}{\s{fut} c’hóvníc’hre, \s{subj} c’hóvníc’hes}
\entry{c’hrá}{pron.}{from earlier \w{ḅúrc’hrá}}{Why. {\itshape{}Replaces \w{c’hès} in questions.}}{}
\entry{c’hrír}{v.}{\pf{écrire}}{To write.}{\s{fut} c’hrírẹ́, \s{subj} c’hrís}
\refentry{c’h’s}{c’hes}
\entry{c’hubh}{v.}{\pf{couper}}{\\+\s{acc} To cut up, chop, sever.\\+\s{part} To cut into, incise.\\+\s{acc} \textit{fig.} To stop, prevent.\\+\s{acc} (\textit{of a river}) To flow through.\\+\s{perl} \textit{fig.} To take a shortcut through, cut through.}{}
\entry{c’hulvâ}{n.}{\pf{écoulement}}{Flow.}{}
\entry{c’hur}{n.}{\pf{cours}}{Course (of events).}{}
\entry{c’húr}{v. tr.}{\pf{court}}{To shrink, reduce in size, narrow. {\itshape{}Always transitive or reflexive.}}{}
\entry{c’húr}{v. intr.}{\pf{courrir}}{To run.}{}
\entry{c’húraû}{n.}{\pf{couronne}}{Crown.}{}
\entry{c’huv́}{v.}{\pf{couvrir}}{+\s{acc} To cover, cover up.}{}
\entry{c’huý’ýr}{n.}{\pf{couleur}}{\\Colour, color.\\(\textit{card games}) Suit.}{}
\entry{c’hý}{n.}{\pf{queue}}{Tail.}{}
\entry{c’hýr}{n.}{\pf{cœur}}{Heart.}{}
\entry{cif}{n.}{\pf{chiffre}}{\\Digit (number).\\Cipher.}{}
\entry{coḅ}{n.}{\pf{échope}}{Market.}{}
\entry{cyc̣aut’h}{n.}{\pf{chuchoter}}{Secret.}{}
\entry{cyḍ}{n.}{\pf{chute}}{\textit{(grammar)} Case.}{}
\entry{da}{n.}{Originally \s{nd}, from \pf{état}}{State. {\itshape{}This covers both the sense of ‘condition’ and ‘polity’. Originally, this was a \s{nd} word.}}{}
\refentry{da’}{dahaúr}
\entry{dá}{n.}{\pf{dent}}{Tooth.}{}
\entry{ḍá}{conj.}{\pf{tandis}}{Whereas, while.}{}
\entry{ḍá}{n.}{\pf{temps}}{\textit{(obsolete or ND)} Time. {\itshape{}In the ND, this word never merged with \w{dá} ‘tooth’ and thus remains in common use until today.}}{}
\entry{ḍaḅ}{n.}{\pf{étape}}{\\Stop (on a journey).\\Stage, step (in a process).}{}
\entry{ḍád}{n.}{\pf{stand}}{Stand, stall, booth.}{}
\entry{ḍad’hat’haú}{n.}{\pf{tas d’atomes}}{Mole (SI unit).}{}
\entry{dahaúr}{particle}{\pf{d’accord}}{Sure, ok, agreed, fine.}{\s{abbr} da’}
\entry{ḍalẹ}{n.}{\pf{tableau}}{Table.}{}
\entry{ḍaléraû}{n.}{from \w{ḍalẹ} + \w{-(é)raû}}{Carpenter.}{}
\entry{ḍalisvâ}{n.}{\pf{établissement}}{Establishment, institution.}{}
\entry{dár}{v.}{\pf{darder}}{To throw, cast, yeet (+\s{acc} sth.).}{}
\entry{ḍár}{adj.}{\pf{tardif}}{To be late, belated.}{}
\entry{ḍárd}{v.}{\pf{tarder}}{To tarry, delay.}{}
\entry{ḍát’h}{adv.}{\pf{tantôt}}{Soon, presently.}{}
\entry{ḍaú}{n.}{\pf{tonne}}{Weight.}{}
\refentry{daú’b’h}{daú(c’h) \textnf{+} b’he}
\entry{ḍaúb’h}{v. intr.}{\pf{tomber}}{To fall, drop.}{}
\entry{daúb’hedwébhó}{v.}{\pf{tomber dans les pommes}}{To faint.}{\s{fut} daúb’hedwébhóre, \s{subj} daúb’hedwébhós}
\entry{daú(c’h)}{particle}{\pf{donc}}{Therefore, then, thus.}{}
\entry{ḍauḍ}{def. pron.}{from earlier \w{ḍẹ auḍ}}{Everything else, any other (one).}{}
\entry{daúr}{v.}{\pf{dormir}}{To sleep.}{}
\entry{daúvê}{n.}{\pf{domaine}}{Domain, field. {\itshape{}In an abstract sense; for an actual physical field, see \w{ráhó}.}}{}
\entry{Daúvníc’h}{n.}{}{\textit{Male or female given name, equivalent to English ‘Dominic’}.}{}
\entry{db’hid’h}{n.}{\pf{individu}}{Person, individual.}{}
\entry{de}{conj.}{\pf{dès que}}{+\s{subj} Once, when once, as soon as.}{}
\entry{dẹ́}{particle}{from \w{Provençal} \textit{den}}{Then (sequential), next.}{}
\entry{dẹ}{n. obsolete}{\pf{état}}{State. {\itshape{}Now replaced with \w{da}.}}{}
\entry{ḍẹ}{adj.}{\pf{tout}}{\\All, every, whole, entire.\\\w{ḍẹ auḍ} Obsolete form of \w{ḍauḍ}.}{}
\entry{deb’haúr}{v.}{\pf{dévorer}}{(\textit{lit. and fig.}) To devour.}{}
\entry{deb’hlau}{v.}{\pf{développer}}{To develop.}{\s{fut} deb’hlaubhe, \s{subj} deb’hlaubhs}
\entry{dẹb’hní}{v.}{\pf{devenir}}{\\+\s{transl} To become, turn into. {\itshape{}The subject is in the \s{abs} case.}\\+\s{exess} To cease being. {\itshape{}The subject is in the \s{abs} case.}}{}
\entry{deḅlér}{v.}{\pf{déplaire}}{To displease (+\s{acc} sbd.), be displeasing.}{}
\entry{dèc’h}{adj.}{\pf{dextre}}{Right (side), right-handed.}{}
\entry{ḍèc’hníc’hvâ}{adv.}{\pf{techniquement}}{Technically.}{}
\entry{dec̣ír}{v.}{\pf{déchirer}}{\\+\s{part} To tear, rip, rend.\\+\s{acc} To rend asunder, tear to pieces.}{\s{fut} dec̣irrẹ, \s{subj} dec̣írs}
\entry{ḍédv́ér}{interj.}{\pf{putain de merde}}{Fuck. {\itshape{}Generic expletive.}}{}
\entry{dẹh}{v.}{\pf{dessous}}{To be below, beneath.}{}
\entry{dehab’híy’}{v. tr.}{\pf{déshabiller}}{To undress +\s{acc} sbd.}{}
\entry{dehád}{v.}{\pf{descendre}}{\textit{(dated)} (+\s{acc}) To kill.}{\s{fut} dehád, \s{subj} dehás}
\entry{dẹhẹ}{n.}{\pf{dessus}}{\\Top, upper side.\\Surface of a body of water.}{}
\entry{dehid}{v.}{\pf{décider}}{To decide (+\s{inf} to do sth.).}{}
\entry{dehín}{v.}{\pf{dessiner}}{\\To draw, sketch.\\\textit{refl.} To appear slowly, take shape.}{}
\entry{dej}{particle}{from \w{dejẹ}}{\textit{Emphatic particle; only used in the preterite}.\\\s{pret} + \w{dej} \textit{roughly} To have ever done sth.}{}
\entry{dejẹ}{adv.}{\pf{déjà}}{Already.}{}
\entry{ḍèl}{particle}{\pf{tel}}{\textit{Emphatic particle, used as an intensifier, often postpositive after the verb, but not so much intensifying the verb directly as it does the entire clause.}.\\Such. {\itshape{}When placed after a noun phrase.}\\So, indeed, verily. {\itshape{}When applied to a verb or an entire clause.}\ex \s{Snet’h}, \s{ii.34}: \w{lá-árb srýlé dèl} ‘so it was that the tree was burning’ or ‘the tree was burning fiercely’, or ‘indeed, the tree was burning’.}{}
\entry{ḍénéb}{n. pl.}{\pf{ténèbres}}{Darkness. {\itshape{}This nouns is exclusively plural, e.g. \w{lḍénéb}.}}{}
\entry{ḍéní}{v.}{\pf{tenir}}{+\s{acc}To hold.}{}
\entry{dénúb’h}{v.}{\pf{de nouveau}}{\\+\s{acc} To repeat, iterate.\\+\s{aci} To do sth. repeatedly.}{}
\entry{dêr}{adj.}{\pf{dingue}}{(\textit{informal}) Crazy, mad, wild, nuts. {\itshape{}The ‘ê’ may be repeated, e.g. \w{dêêêr} ‘craaazy’.}}{}
\entry{ḍér}{n.}{\pf{terme}}{Term.}{}
\entry{ḍèr}{v.}{\pf{taire}}{To silence, shut up.}{\s{fut} ḍérẹ́}
\entry{ḍéraúj}{v.}{\pf{interroger}}{To demand.}{}
\entry{ḍérésḍ}{v.}{\pf{terrestre}}{To be terrestrial, earth-based.}{\s{fut} ḍérésḍrẹ́, \s{subj} ḍérésḍs}
\entry{ḍẹ́ríb}{v.}{\pf{terrible}}{To be terrible (all senses).}{\s{fut} ḍẹ́ríre, \s{subj} ḍẹ́rís}
\entry{ḍérnít’h}{n.}{\pf{éternité}}{Eternity.}{}
\entry{dérny’é}{adj.}{\pf{dernier}}{Last, final, ultimate.}{}
\entry{dérny’ẹ́huf}{n.}{\pf{dernier} + \pf{souffle}}{Death.}{}
\entry{ḍérsèd}{v.}{\pf{intercéder}}{To intercede.}{}
\entry{ḍèrvíc’h}{n.}{\pf{thermique}}{Heat, warmth.}{}
\entry{déry’é}{v.}{\pf{derrière}}{\\+\s{gen} To be the back of. {\itshape{}For ‘\textit{at} the back of’, the \s{pstess} is used instead.}}{}
\entry{deslẹ}{v.}{\pf{déceler}}{To detect, discover, uncover, reveal.}{\s{fut} deslẹre, \s{subj} deslẹs}
\entry{ḍèsrzávê}{n.}{From \pf{test} + \pf{examen}}{Test, exam, examination.}{}
\entry{det’hérvn}{v.}{\pf{determiner}}{To determine, figure out.}{}
\entry{déví}{num. frac. or n.}{\pf{demi}}{Half.}{}
\entry{dẹ́vná}{v.}{\pf{déménager}}{\\+\s{acc} To move.\\\textit{intr.} To move house.}{}
\entry{dévýr}{v.}{\pf{demeurer}}{\\To remain, stay.\\+\s{iness} To live in, dwell.}{}
\entry{ḍeý’ebhat’hẹ}{n.}{\pf{télépathie}}{Telepathy.}{}
\entry{ḍeý’ebhat’hic’h}{v.}{\pf{télépathique}}{To be telepathic.}{\s{fut} ḍeý’ebh\-at’hic’hre, \s{subj} ḍeý’ebhat’hic’hes}
\refentry{dib’hat’h’}{dib’hat’hiv}
\entry{dib’hat’hiv}{n.}{from earlier \w{dib’hbat’hiv}}{Building, edifice. {\itshape{}Shortened to just \w{dib’hat’h’} before ordinals.}\ex \w{dib’hat’h’ révy’ẹ́} ‘Building №1’.}{}
\entry{dib’hbat’hiv}{n.}{from \pf{édifice} + \pf{bâtiment}}{\textit{(obsolete)} Building, edifice.}{}
\entry{di’bhẹ́}{v.}{\pf{différer}}{\\(+\s{gen}) To differ, be different from sth.}{}
\entry{dib’hih}{v.}{\pf{diviser}}{To divide.}{}
\entry{dír}{v. tr.}{\pf{dire}}{+\s{acc} To say, tell (+\s{dat} someone).}{\s{fut} dírẹ́, \s{subj} díss}
\entry{díréc’h}{v.}{\pf{direct}}{To be direct.}{}
\entry{díríj}{v.}{\pf{diriger}}{+\s{acc} To direct, run, oversee, operate (a business or establishment).}{\s{fut} díríje, \s{subj} díríjs}
\entry{dis}{num.}{}{Ten.}{}
\entry{ḍis}{v.}{\pf{tisser}}{To weave.}{\s{fut} ḍiś, \s{subj} ḍiss}
\entry{disḍas}{n.}{\pf{distance}}{Distance.}{}
\entry{dizy’ê}{num.}{}{Tenth.}{}
\entry{dónẹ́}{v.}{\pf{donner}}{\s{+ dat \& acc/part} To endow, bestow, give. {\itshape{}The \s{acc} is used when talking about concrete, measurable, and finite objects or sums; the partitive to talk about abstract concepts or parts of objects; the \s{dat} is the person being endowed with.}}{\s{fut} dónrẹ́, \s{subj} dónés}
\entry{ḍrâ}{adj.}{\pf{étrange}}{Strange, foreign.}{}
\entry{ḍrálfi’yá}{n.}{\pf{étoile filante}}{Falling star, shooting star.}{}
\entry{ḍréd’hb’h}{v. intr.}{\pf{être debout}}{To stand, be upright.}{}
\entry{du}{v.}{\pf{doux}}{To be mild, gentle (e.g. of a slope or person).}{}
\entry{dub}{num.}{}{Twice.}{}
\entry{dubâ}{num.}{}{Twofold.}{}
\entry{ḍúr}{adv.}{\pf{toujours}}{\\\textit{(positive context)} Always.\\\textit{(negative context)} Still.}{}
\entry{ḍúr}{n.}{\pf{tour}}{Tower.}{}
\entry{ḍúrn}{v.}{\pf{tourner}}{To change.}{\s{fut} ḍúrnẹ́, \s{subj} ḍúrs}
\entry{duý’ýr}{v.}{\pf{douleur}}{To suffer, be in pain.}{}
\entry{dý}{num.}{}{Two.}{}
\entry{ḍydy’ẹ́}{v.}{\pf{étudier}}{To study.}{}
\entry{dy’ê}{v.}{\pf{tien}}{To be yours (\s{sg}).}{\s{fut} dy’êrẹ́, \s{subj} dy’ês}
\entry{dýr}{v.}{\pf{dur}}{To be hard, firm.}{\s{nd} sy̌}
\entry{dývrê}{particle}{\pf{du moins}}{At least. {\itshape{}As in e.g. ‘At least, I think that \ldots’}}{}
\entry{dýzy’ê}{num.}{}{\\Second.\\\w{dýzy’êâ} Secondary.}{}
\entry{e}{n.}{\pf{eau}}{Water.}{}
\entry{ebhẹ}{v.}{\pf{épais}}{To be thick.}{\s{fut} ebhrẹ, \s{subj} ebhes}
\entry{ec̣}{n.}{\pf{péché}}{Sin, transgression, wrongdoing.}{}
\entry{ec’hlér}{v. or n.}{\pf{éclairer}}{\\To shine, glimmer.\\\textit{n.} Shine, glimmer.}{}
\entry{ed}{particle}{\pf{et}}{\textit{Used in numbers, see §~\ref{subsec:numerals}}.}{}
\entry{eḍ}{v. irreg.}{\pf{être}}{To be. {\itshape{}Active only, see §~\ref{subsec:ed-paradigm}.}}{}
\entry{eḍrrá}{v.}{\pf{étroit}}{Pointy.}{}
\entry{Eḍy’ê}{n.}{}{\textit{Male given name, equivalent to English ‘Ste\-phen’}.}{}
\entry{ee}{interj.}{\textnf{onomatopoeic}}{Hey. {\itshape{}Pronounced /eː/ Used to call for attention.}}{\s{nd} ē}
\entry{ehehe}{interj.}{\textnf{onomatopoeic}}{Hehe. {\itshape{}Mischievous laughter.}}{}
\entry{ehyó}{n.}{\pf{écusson}}{Shield.}{}
\entry{el}{n.}{\pf{ailles}}{Wing, blade, fin.}{}
\entry{ér}{n.}{\pf{ère}}{Era, age, aeon.}{}
\entry{-(é)raû}{affix}{\pf{-eron}}{-maker, -wright. {\itshape{}Used in derivation, see §~\ref{subsec:diachrony-and-derivation}.}\ex \w{ḍaléraû} ‘carpenter’ from \w{ḍalẹ} ‘table’.}{}
\entry{ẹ́v}{v.}{from \pf{aime}, \s{1sg} of \s{aimer}}{\textit{(poetic, \s{nd})} +\s{acc} To love. {\itshape{}Exclusively used in reference to a person. Diachronically, this verb was already obsolete in Middle UF and was rediscovered later on by a number of Early Modern UF poets, which led it to skip some initial sound changes. \w{ad’hór} is generally much preferred in common parlance in the \s{sd}, though \this{} is has entered common use in the \s{nd}.}\ex \s{nd} \w{s’htẹ́v} ‘I love you’.}{}
\entry{ez-}{pron.}{\pf{ses}}{Its, her, his, their.}{}
\entry{F}{adj.}{from \pf{fẹ}}{\textit{(Logic)} False. {\itshape{}Always capitalised.}}{}
\entry{fac’h}{v.}{\pf{factice}}{To be fake, artificial, false.}{}
\entry{fahaú}{conj.}{\pf{de façon que}}{+\s{opt} In such a way that, such that, so much so.}{}
\entry{fahiý’it’h}{n.}{\pf{facilité}}{Ease, easiness.}{}
\entry{fasḍau}{v.}{\pf{fastoche}}{Easy.}{}
\entry{fát’haú}{n.}{\pf{fantôme}}{Mind.}{}
\entry{faú}{adv.}{\pf{fort}}{Very, right, really. {\itshape{}Postpositive intensifier placed after adjectives, particularly in the comparative or superlative degrees.}}{}
\entry{faúr}{adj.}{\pf{fort}}{\textit{obsolete, except in proverbs} Strong, mighty.}{}
\entry{faúr₁}{n.}{\pf{force}}{\\Force, strength, power.\\\w{Faúr} \s{def} \textit{(science fiction, Star Wars)} The Force.}{}
\entry{faúr₂}{n. or v. tr.}{\pf{forme}}{\\Shape, form. {\itshape{}In this sense sometimes also spelt \w{fór}.}\\+\s{acc} To give form to, shape.}{}
\entry{faúrḍ}{n.}{\pf{fortune}}{Fortune, destiny, faith.}{}
\entry{fávíy’}{n.}{\pf{famille}}{Family.}{}
\entry{fé}{n.}{\pf{fin}}{End.}{}
\entry{fẹ́}{v.}{\pf{fin}}{To be fine, thin, small. {\itshape{}This word is derived from a northern pronunciation of \pf{fin}, unlike \w{fé}, which uses the southern pronunciation.}}{}
\entry{fẹ}{v.}{\pf{faux}}{To be false, incorrect, wrong.}{\s{fut} faure, \s{subj} faus}
\entry{fẹhab}{v.}{\pf{faisable}}{To be possible, feasible.}{\s{fut} fẹhabre, \s{subj} fẹhas}
\entry{fèhẹ}{n.}{\pf{faisceau}}{\\Bundle, bunch, cluster.\\Beam, ray.}{}
\entry{fér}{v.}{\pf{faire}}{\\To do, make, build, construct, erect.\\\textit{Expletive; see §~\ref{subsubsec:personal-pronouns}}.}{\s{fut} fẹ́, \s{subj} fés}
\entry{fér}{n.}{\pf{ferme}}{Farm.}{}
\entry{féráy’}{n.}{\pf{ferraille}}{Iron.}{}
\entry{férḍufraú}{v.}{\pf{en faire tout un fromage}}{To make a big fuss a\-bout something.}{\s{fut} fér\-ḍu\-fraúrẹ́, \s{subj} férḍufraús}
\entry{férr-rásvát’h}{n.}{\pf{faire la grasse mat’}}{A long, deep sleep.}{}
\entry{férv}{v.}{\pf{fermer}}{To close, shut.}{}
\entry{féłeb’hut’hic’h}{v. intr.}{\pf{faire les boutiques}}{To shop, especially leisurely.}{}
\entry{fic’h}{v.}{back-formation from *\w{fic’hs}, reinterpreted as a subjunctive stem; from \pf{fixer}}{To fix, set, establish.}{\s{fut} fic’hre, \s{subj} fic’hs}
\entry{fihas}{v.}{\pf{efficace}}{To be efficient.}{}
\entry{fínít’h}{n.}{\pf{infinité}}{Infinity, endlessness.}{}
\entry{fírýr}{n.}{\pf{figure}}{Figure.}{}
\entry{fis}{n.}{\pf{fils}}{Son.}{}
\entry{fíy’}{n.}{\pf{fille}}{Daughter.}{}
\entry{fíy’é}{n.}{\pf{fillette}}{Girl, young woman.}{}
\entry{flij}{n.}{\pf{infliger}}{Cause.}{}
\refentry{fór}{faúr₂}
\entry{fórvẹ́}{v.}{\pf{informer}}{To inform (+\s{acc} sbd.) (+\s{aci} of sth.).}{\s{fut} fórv́ẹ́, \s{subj} fórvẹ́s}
\entry{fúr}{v.}{\pf{fournir}}{To deliver, provide (+\s{dat} sbd.) (+ \s{acc} with sth.).}{}
\entry{fý}{n.}{\pf{feu}}{Fire.}{}
\entry{fýy’}{n.}{\pf{feuille}}{Plant.}{}
\entry{hab’híy’}{v. tr.}{\pf{habiller}}{To dress +\s{acc} sbd.}{}
\entry{í}{n.}{\pf{hymne}}{Legend, myth.}{}
\entry{id’h}{n.}{\pf{idée}}{Idea.}{}
\entry{il}{n.}{\pf{île}}{Island, isle.}{\s{nd} ī}
\refentry{ís}{ub’hrá}
\entry{isḍrár}{n.}{\pf{histoire}}{Story, tale.}{}
\entry{íváj}{n. archaic}{\pf{image}}{Image, picture.}{}
\entry{iý’ývî}{v.}{\pf{illuminer}}{To light up, illuminate.}{}
\entry{Já}{n.}{}{\textit{Male or female given name, equivalent to English ‘John’ or ‘Joan’}.}{}
\entry{Jac’h}{n.}{\pf{Jacques}}{\textit{Male given name}.}{}
\entry{jaú}{v.}{\pf{jaune}}{To be yellow.}{}
\entry{jávé}{adv.}{\pf{jamais}}{\\\textit{neg.} Never, at no time.\\\textit{pos.} Ever, always.\\\w{dwájávé} For ever, forever.}{}
\entry{jaý’aú}{n.}{\pf{jalon}}{\\Nail.\\\textit{(obsolete)} Stake, pole.}{}
\entry{Jed’háy’}{n.}{\pf{Jedi}}{Jedi (Star Wars).}{}
\entry{jehír}{v. intr.}{\pf{gésir}}{To lie, rest in a horizontal position.}{}
\entry{ju}{v.}{\pf{jouer}}{\\(+\s{part}) To play (a game or role).\\(\textit{refl.}) To be decided, be a matter of time.\\+\s{spress} To affect.\\+\s{gen} To take advantage of, make fun of.}{}
\entry{júrdy’í}{adv.}{from archaic \pf{aujúrdy’í}}{Today, nowadays.}{}
\entry{jý}{v.}{\pf{jeune}}{To be young.}{}
\entry{jý}{n.}{\pf{jeu}}{\\Game.\\Play (theatrical).}{}
\entry{jys}{conj.}{\pf{jusqu’à ce que}}{+\s{opt} Until.}{}
\entry{jys}{adv.}{\pf{juste}}{Just, only, merely.}{}
\entry{jys}{v.}{\pf{injuste}}{To be unjust, unfair.}{\s{fut} jysre, \s{subj} jyss}
\entry{lá}{v.}{\pf{planer}}{To fly.}{}
\entry{lá}{v.}{\pf{lent}}{To be slow.}{}
\entry{lá}{n.}{\pf{plan}}{\\Level or flat surface.\\Tool used for determining whether sth. is level.\\Map. {\itshape{}Mainly used for street and city maps.}\\\textit{(geometry)} Plane.}{}
\entry{lá}{v.}{\pf{plan}}{To be level or flat.}{}
\entry{lab’h}{v.}{\pf{laver}}{\\To wash, clean (+\s{acc} sth.).\\\textit{refl.} To wash oneself, take a bath, have a shower.}{}
\entry{Lác}{n.}{}{\textit{Female given name, equivalent to English ‘Bi\-anca’}.}{}
\entry{lác}{v.}{\pf{blanche}}{To be white.}{}
\entry{lac’h}{n.}{\pf{lac}}{Lake.}{}
\entry{láḍ}{n.}{\pf{plante}}{\\Blade of grass.\\\textit{pl.} Grass.}{}
\entry{lahaú}{n.}{\pf{glaçon}}{Icicle.}{}
\entry{lánẹ́}{v.}{\pf{flâner}}{To meander.}{}
\entry{lár}{v.}{\pf{large}}{Wide, broad.}{}
\entry{lârdávrá}{n.}{\pf{langue de bois}}{Evasive, unclear, or overly formal speech.}{}
\entry{las}{v.}{\pf{placer}}{To place, put, set (+\s{acc} sth.).}{}
\entry{las}{n.}{\pf{classe}}{Class, type.}{}
\entry{lasy’ér}{n.}{\pf{glacière}}{Ice.}{}
\entry{laú}{v.}{\pf{long}}{Long. {\itshape{}Often in compounds \w{laú-} ‘long-’.}}{}
\entry{laúrs}{conj.}{\pf{lorsque}}{When (temporal only).}{}
\entry{laúrvé}{conj.}{from \w{laúrs} + \w{vé}}{\textit{(contraction)} But when. {\itshape{}Stressed on the first syllable.}}{}
\entry{laut’h}{v.}{\pf{flotter}}{Fl\-oat, hover, levitate.}{\s{fut} laut’hre, \s{subj} laut’hes}
\entry{laút’há}{adv.}{\pf{longtemps}}{\\Since long ago, for a long time, over a long period. {\itshape{}When describing ongoing processes that started a long time ago, this is construed with the present tense instead of the present anterior.}\ex \w{laút’há ḍaléraû vy’í} ‘I have been a carpenter for a long time’, lit. ‘I am a carpenter for a long time’.\\\textit{(with comparative prefix)} Since long before, For longer, for a longer time.\ex \w{dau lẹlaút’há ḍaléraû vy’í} ‘I have been a carpenter much longer than you’.}{}
\entry{lávýr}{n.}{\pf{clameur}}{Sound, noise.}{}
\entry{le}{v.}{\pf{laisser} > *\w{lehe}}{To let, allow, permit. {\itshape{}Used chiefly in questions or imperative.}}{\s{fut} lere, \s{subj} les}
\entry{lé}{n.}{\pf{plaine}}{Plain, plains.}{}
\entry{lé}{v.}{\pf{plein}}{\\To be full.}{}
\entry{ḷẹ}{n. pl.}{\pf{pluie}}{Rain. {\itshape{}Starting in Early Modern UF, this noun only occurs in the plural.}\ex \w{dýḷẹ syb’hér} ‘it rains’ (lit. ‘rains are being poured’).}{}
\entry{lẹ}{n.}{\pf{clef}}{Key.}{}
\entry{lẹ-}{prefix}{\pf{plus}}{\textit{Affirming comparative prefix. See grammar}.}{}
\entry{lec’hḍraúvnẹ́t’hic’h}{v.}{\pf{électromagnétique}}{To be electromagnetic.}{\s{fut} lec’hḍraúvnẹ́t’hic’hre, \s{subj} lec’hḍraúvnẹ́t’hic’hes}
\entry{leḍ}{n.}{\pf{lettre}}{\\Letter (of the alphabet).\\\w{lý’aúleḍ} By the book.}{}
\entry{lèheb’h}{v.}{\pf{laisser-faire}}{+\s{aci} To let happen.}{}
\entry{lẹhuvud}{n.}{\pf{coup de foudre}}{Love at first sight.}{}
\entry{lejy’aû}{n.}{\pf{légion}}{\textit{(Ancient Rome, military)} Legion.}{}
\entry{lér}{v.}{\pf{clair}}{To be evident, obvious, frank, clear.}{}
\entry{lér}{v.}{\pf{plaire}}{To please (+\s{acc} sbd.), be pleasing.}{}
\entry{lí}{v.}{\pf{lire}}{\\+\s{part} To read from.\\+\s{acc} To peruse, read entirely.}{\s{fut} lírẹ́, \s{subj} lís}
\entry{lí}{n.}{\pf{ligne}}{Row (of a table).}{}
\entry{lis}{n.}{\pf{liste}}{List.}{}
\entry{lit’hijy’}{v.}{\pf{litigier}}{To litigate, be at law with (+\s{dat} sbd.).}{}
\entry{lívnád}{n.}{\pf{limonade}}{Lemonade.}{}
\entry{liv́uhé}{n.}{\pf{livre} + \pf{bouquin}}{Book.}{}
\entry{lúr}{v.}{\pf{lourd}}{To be bulky, oversized, heavy.}{}
\entry{ly}{particle}{\pf{plus}}{\textit{obsolete variant of \w{lẹ}, sometimes leniting}.}{}
\entry{lý}{n.}{\pf{plume}}{Pen, quill.}{}
\entry{lý}{v.}{\pf{bleu}}{To be blue.}{}
\entry{ḷý}{n.}{\pf{lieu}}{\textit{Base of the spatial correlatives. In senses 2–5, case affixes are attached before \this, e.g. sense 2 \s{all} \w{sẹb’héḷý} ‘hither’}.\\Place, location.\\\w{sẹ}\ldots\this{} \s{def} [from \w{sẹh}] Here, hither, hence, \&c. {\itshape{}Proximal demonstrative (all cases).}\\\w{sý’\L}\ldots\this{} \s{def} [from \w{sý’ẹ}] There, thither, thence, \&c. {\itshape{}Distal demonstrative (all cases).}\\\this\w{hes} \s{indef} [from \w{c’hes}] Where, whither, when\-ce, \&c. {\itshape{}Interrogative (locative cases only).}\\\w{s’}/\w{sá\L}\ldots\this{} \s{indef} [from \w{sá}] No\-where, from no\-where, \&c. {\itshape{}Negative (locative cases only).}}{}
\entry{lybhárdyt’há}{adv.}{pluspart du temps}{Often.}{}
\entry{lýr}{pron.}{\pf{leur}}{Their.}{}
\entry{lýrḍ}{v.}{from \pf{leur}; the \w{ḍ} was added in analogy with \w{naúḍ} and \w{b’hauḍ}}{To be theirs.}{\s{fut} lýrḍre, \s{subj} lýrs}
\entry{lys}{adv.}{\pf{plus} /plys/}{\textit{neg. only} No longer, not any more. {\itshape{}The meaning of this and \w{lẹ} swapped at some point for unknown reasons.}}{}
\entry{lýv́á}{v. obsolete, \s{3rd} person only}{\pf{pleuvoir}}{To rain. {\itshape{}Replaced with \w{b’hér} in Middle UF.}}{\s{fut} lýv́áre, \s{subj} lýv́ás}
\entry{lývný}{v.}{\pf{lumineux}}{To be light, bright, luminous.}{}
\entry{lývy’ér}{n.}{\pf{lumière}}{\\Light, visible electromagnetic radiation.\\A source of light, such as a lamp or torch.}{}
\entry{lyzy’ýr}{adj.}{\pf{plusieurs}}{Several.}{}
\entry{n}{n.}{\pf{haine}}{Hate, hatred, loathing.}{}
\entry{nájẹ}{v.}{\pf{nager}}{To swim.}{\s{fut} náȷ́ẹ, \s{subj} nájes}
\entry{nárrahóḍ}{v.}{\pf{raconter} + \pf{narrer}; subj. from \pf{filer}}{To narrate, recount \s{+part sth.}, tell (\s{+dat} sbd.) a story.}{\s{fut} nárrahóḍe, \s{subj} fils}
\entry{nát’hýr}{n.}{\pf{nature}}{\\\textit{(chiefly)} \s{indef} Nature, the natural world.\\\s{def} The way something is.}{}
\entry{naú}{n.}{\pf{nom}}{\\Name.\\Noun.}{}
\entry{naûb}{n.}{\pf{nombre}}{\\Amount, number.\\\w{naûb vú} + \s{gen indef} So many, so much, such a great amount of. {\itshape{}In informal speech often contracted to \w{naûvú}.}}{}
\entry{naúḍ}{v.}{\pf{nôtre}}{To be ours.}{\s{fut} naúḍre, \s{subj} naús}
\entry{naúḍ}{n.}{\pf{note}}{Note.}{}
\entry{naúrvál}{v.}{\pf{normal}}{To be normal.}{}
\refentry{naûvú}{naûb vú}
\entry{nẹ́b’h}{n.}{\pf{névé}}{Snow.}{}
\entry{néḍ}{v. dep.}{\pf{naître}}{To be born. {\itshape{}This is a deponent verb whose subject takes the \s{acc} and which only takes passive affixes.}}{\s{sub} néhs}
\entry{néḍnírc’haýb’h}{v.}{\pf{ne tenir qu’à un fil}}{To hang by a thread.}{}
\entry{néhás}{n.}{\pf{naissance}}{Birth.}{}
\entry{nérjẹ}{n.}{\pf{énergie}}{Energy.}{}
\entry{nés}{adj.}{from earlier \w{nésḍ}}{Left (side), left-handed.}{}
\entry{nésḍ}{adj. archaic}{\pf{senestre}}{Left (side), left-handed.}{}
\entry{ní}{v.}{\pf{nier}, \s{fut} from \pf{contrer}, \s{subj} from \pf{oposer}}{To deny, ref\-use, reject, rebut (+\s{acc} sbd./sth.).}{\s{fut} c’haúḍé, \s{subj} aubhaus}
\entry{ní}{conj.}{\pf{ni}}{\\Neither, nor. {\itshape{}Requires a negative context and thus frequently appears as \w{ní’sý’ýâ}. If there is no verb, the negation may be ommitted. Often paired with another \textit{ní} or preceding negated clause, in which case \w{ní} generally replaces the negation particle of the later clause.}\\\ldots{} \w{ní(’sý’ýâ)} \ldots{} \w{ní} Neither \ldots{} nor \ldots{}. {\itshape{}Unlike \w{au} and \w{u}/\w{ra}, this conjunction is postpositive and usually appears in the same position as particles rather than before the things it joins.}\ex \w{Jsydyrár ní’sý’ýâ, jsydyát’hád ní} ‘I neither see nor hear it’.}{}
\entry{níb’hẹ}{n.}{\pf{niveau}}{\\Level, floor.\\\textit{fig.} Level, degree.\\+\s{gen} \w{dwáníb’hẹ} On the level of.}{}
\entry{níb’hý’}{v.}{\pf{niveler}}{+\s{acc} To level, flatten.}{}
\entry{nír}{v.}{\pf{venir}}{To come.}{}
\entry{nór}{v.}{back-formation from *\w{nórâ} from \pf{ignorant}}{To be ignorant, unaware, oblivious.}{}
\entry{nóráv}{n.}{from archaic \w{ḅá nórávíc’h}}{Druid.}{}
\entry{núb’h}{v.}{\pf{nouveau}}{To be new.}{\s{fut} núb’he, \s{subj} núb’hs}
\entry{nvẹ́}{v. intr.}{\pf{nommer}}{\textit{(refl. or intr.)} To be called, name oneself. {\itshape{}Usually placed after the name. This word replaced \w{aḅ} in the reflexive sense in Early Modern UF. Outside of literature, this verb is frequently simply intransitive rather than reflexive.}\ex \w{Aý’èc’hsád j(v)nvé} ‘My name is Aý’èc’hsád’.\ex \w{(A) ḍẹnvẹ́c’h’s ?} ’What’s your name?’, lit. ‘What are you called?’. {\itshape{}While \w{-c’h’s} for \w{c’hes} is archaic nowadays, this is a set phrase that preserves this form. Because this renders this phrase rather recognisable, the \w{a} is often dropped in informal speech. Note also that this is a passive, with the plural being \w{A b’hýnvẹ́ c’hes}.}}{}
\entry{nýb’hy’ê}{num.}{}{Ninth.}{}
\entry{nýt’h}{num.}{}{Nine.}{}
\refentry{p-}{ḅ-}
\refentry{ph-}{ḅ-}
\refentry{p’h-}{bh-}
\entry{R}{adj.}{from \pf{ré}}{\textit{(Logic)} True. {\itshape{}Always capitalised.}}{}
\entry{r}{n.}{\pf{air}}{Air. {\itshape{}Frequently in the plural.}}{}
\entry{r}{v.}{\pf{créer}}{+\s{acc} To invent, come up with.}{\s{fut} re, \s{subj} ś}
\entry{ra}{conj.}{\pf{swa} > *\w{rá}}{\\Or. {\itshape{}Exclusive, see also~\w{u}.}\\\w{u}/\w{ra} \ldots\ \w{ra} \ldots\ ‘either \ldots\ or \ldots’ \textit{(exclusive)}.}{}
\entry{râ}{v.}{\pf{gagner}}{To win, gain, earn (+\s{acc} sth.).}{}
\entry{râ}{n.}{\pf{gant}}{Glove.}{}
\entry{rá₁}{n.}{\pf{loi}}{Law, rule, regulation.}{}
\entry{rá₂}{adj.}{\pf{grand}}{Big, large, great, tall.}{}
\entry{rá₃}{n.}{\pf{mois}}{Month.}{}
\entry{rá₄}{n.}{\pf{voix}}{Voice.}{}
\entry{rá₅}{n.}{\pf{bras}}{Arm.}{}
\entry{râḅ}{v. tr.}{\pf{remplir}}{\\+\s{acc/part} To fill, fill up.\\+\s{acc/part} To fill in.}{\s{fut} râḅle, \s{subj} râḅles}
\entry{Ráb’h}{n.}{from earlier \w{Réáb’h}; presumably the name of some celebrity or local deity}{\\\textit{indecl.} \s{def sg} \textit{always} \s{nom/voc} Ráb’h. {\itshape{}Main god of the ULTRAFRENCH pantheon; usually male. Old-fashioned also often all-caps \w{RÁB’H}.}\ex \s{Snet’h}, \s{i.17}: \w{au lebálá daú RÁB’H} ‘and thus spake Ráb’h’.\\\w{Ráb’h sénýr} \s{def sg} Lord Ráb’h. {\itshape{}Used for \senseref{1} in all other cases; as with all names, only \w{sénýr} is inflected. Old-fashioned often \w{RÁB’H Sénýr}.}\ex \s{Snet’h}, \s{8.1}: \w{au labraúc RÁB’H naút B’héhénýr} ‘and they came to our Lord Ráb’h’.\\\textit{(archaic) interj.} \w{Ráb’h hénýr} Oh god, oh my god. {\itshape{}Old-fashioned form of \w{ráb’hénýr}. The \s{voc} Ráb’h without \w{hénýr} is never used on its own other than to address the deity directly.}\\\textit{(rarely)} \w{réáb’h} The main god of another culture. {\itshape{}Only attested figuratively. Not capitalised in this sense, and declined like a regular word.}\ex \s{Snet’h}, \s{ii.3}: \w{ledéraújá’z derévôt’he láréáb’h} ‘their god demanded they return’.}{}
\entry{rábh}{v.}{\pf{frapper}}{+\s{acc} To smite, strike down.}{}
\refentry{ráb’h}{v́ár}
\entry{râbh}{v.}{\pf{ramper}}{To slither.}{}
\entry{ráb’hau}{n.}{From \w{ráb’haut’h}}{\textit{(woodworking)} Plane. {\itshape{}This refers to the tool used in woodworking. For the mathematical concept of a plane, see \w{lá}.}}{}
\entry{ráb’haut’h}{n.}{\pf{raboter}}{\\\textit{(woodworking)} +\s{acc} To plane, flatten with a plane.\\\textit{(informal)} \w{c’heḍ’ráb’h’ !} Fuck off!, get lost! {\itshape{}Abbreviated form of \w{c’heḍẹráb’haut’h}, literally ‘plane yourself’. First attested in the works of the Early UF comedian \s{J. A. B. Snet’h}. The plural form is generally \w{c’heb’hḍ’ráb’h’}.}}{}
\entry{ráb’háy’}{v.}{\pf{travailler}, \s{fut} and \s{subj} from \pf{bos\-ser}}{To work.}{\s{fut} bohér, \s{subj} bos}
\entry{ráb’hé}{n.}{\pf{ravin}}{Ravine.}{}
\entry{ráb’hẹ}{interj.}{\pf{bravo}}{Well done, bravo.}{}
\entry{ráb’hén(ýr)}{interj.}{contraction of earlier \w{Ráb’h hénýr}}{Oh god, oh my god.}{}
\entry{ráb’hér}{v.}{\pf{traverser}}{+\s{perl} To traverse, cross, move cross.}{}
\entry{rác’hánár}{n.}{from \w{ráhe} + \w{c’hánár}}{Airship, dirigible.}{}
\entry{rác’hsaý’ad}{v.}{\pf{raconter des salades}}{To lie, tell tall tales, overexaggerate.}{\s{fut} rác’h\-sa\-ý’e, \s{subj} rác’hsaýs}
\entry{rád}{v. tr.}{\pf{rendre}}{To surrender +\s{acc} sth. (\s{dat} to sbd.).}{}
\entry{ráḍ}{n.}{\pf{boîte}}{Box.}{}
\entry{râd}{v.}{\pf{prendre}}{+\s{acc} \textit{or} \s{part} To grab, grasp. {\itshape{}The \s{part} usually implies that only a part or some of a larger whole is grabbed, e.g. a handful of sand).}}{}
\entry{râd’h}{n.}{\pf{grandir}}{\\\textit{intr.} To grow, become larger.\\+\s{acc} To magnify, increase, heighten.}{}
\entry{râd’hẹḅy’é}{n.}{\pf{gant de pied}}{Shoe. {\itshape{}Literally ‘footglove’.}}{}
\entry{rád’hérn}{n. def.}{from \w{rá} + \w{dérny’é}}{Last month.}{}
\entry{rád’hsy’ô}{n.}{\pf{traditon}}{Tradition, custom.}{}
\entry{rád’hyc’hsy’ô}{n.}{\pf{traduction}}{Translation.}{}
\entry{rád’hyr}{n.}{\pf{bois d’œuvre}}{Wood.}{}
\entry{râdrásôn}{v.}{\pf{prendre ses jambe à son cou}}{To run.}{\s{fut} râdrásônre, \s{subj} râdrásôns}
\entry{rádrénẹ́}{v. + \s{aci}}{\pf{les doigts dans le nez}}{To put no effort into.}{\s{fut} rádrénrẹ́, \s{subj} rádrénẹ́s}
\entry{râdvâ-}{prefix}{\pf{grandement}}{\textit{Superlative prefix. See grammar}.}{}
\entry{ráhál}{v.}{\pf{rassembler}}{\\+\s{acc} \textit{or refl.} To assemble.}{}
\entry{ráhaú}{v.}{\pf{boisson}}{\\(+\s{part}) To drink, drink from.\\+\s{acc} To empty (by drinking).}{}
\entry{râhaúḍ}{v.}{\pf{recontrer}}{To meet, encounter, come face to face (+\s{all} with sbd.).}{\s{fut} râhaúḍre, \s{subj} râhaús}
\entry{ráhávnýḍ}{n.}{\pf{soixante} + \pf{minute}}{Hour.}{}
\entry{ráhe}{n.}{\pf{oiseau}}{Bird.}{}
\entry{ráhé}{n.}{from \w{ráhe} + \w{ráhó}}{Flying fish.}{}
\entry{ráhé}{n.}{\pf{voisin}}{Neighbour.}{}
\entry{ráhẹ}{conj.}{\pf{quoique}}{+\s{subj} Although, though.}{}
\entry{râhẹ}{n.}{\pf{français}}{Human, person.}{}
\entry{ráhír}{v.}{\pf{choisir}}{+\s{acc} To choose, select.}{}
\entry{ráhis}{v.}{\pf{raciste}}{To be racist.}{\s{fut} ráhise, \s{subj} ráhiss}
\entry{ráhó}{n.}{\pf{poisson}}{Fish. {\itshape{}UF has a series of proverbs around fish drowning (in water!), despite the fact that fish quite literally breathe water and therefore are incapable of ‘drowning’.}\ex \w{Láráhó slẹlúrá.} Now you’ve done it. {\itshape{}Literally ‘the fish was too bulky [to swim to the surface, so it drowned]’.}\ex \w{Áhaúr’sý’ýâ láráhó sráy’éá.} There is more to this. There is a story behind this. {\itshape{}Literally ‘the fish hasn’t drowned yet’.}}{}
\entry{ráhó}{n.}{\pf{gazon}}{Grassland, grassy field, meadow.}{}
\entry{rahúrs}{v.}{\pf{raccourcir}}{To be short.}{}
\entry{ráhut’h}{n.}{\pf{grand} + \pf{couteau}}{Sword, blade.\ex \w{áráhut’h’t ilý ly b’haúr} ‘the pen is mightier than the sword’. {\itshape{}This was originally a fossilised, obsolete \s{aci}: \w{á\-hut’h\-rá éḍ ilý lẹb’haúr}.}}{}
\entry{rál}{n.}{\pf{toile}}{Canvas.}{}
\entry{rár}{v. tr.}{\pf{voir}}{+\s{part} To see.}{\s{fut} b’hérẹ́, \s{subj} rárs}
\entry{rár}{v.}{\pf{noir}}{To be black.}{}
\entry{rárd}{v.}{\pf{regarder}}{\\+\s{acc} To watch.\\+\s{part} To look at.}{\s{fut} rárdre, \s{subj} rárds}
\entry{rársaú}{n.}{\pf{garçon}}{Boy, young man.}{}
\entry{râsḅár}{v.}{\pf{transparent}}{To be transparent, clear.}{}
\entry{râsír}{v.}{\pf{transpirer}}{\\+\s{aci} To come to light, become known, transpire.\\+\s{aci} \s{pres ant} For it to be clear, apparent, evident that \ldots {\itshape{}Lit. ‘it has come to light that \ldots’}}{\s{fut} râsírẹ́, \s{subj} râsírs}
\entry{rát’hẹ}{particle}{\pf{vois-tu}}{You see, you know.}{}
\entry{rát’hýr}{n.}{\pf{voiture}}{Car, wagon.}{}
\entry{Raû}{n}{\pf{Rome}}{Rome.}{}
\entry{raû}{n. archaic}{\pf{tronc}}{Log (of a tree).}{}
\entry{raû}{interj.}{\pf{gône}}{Kid. {\itshape{}This is grammatically a vocative—not that one could tell since it looks identical to the absolutive.}}{}
\entry{raû}{v.}{\pf{rond}}{To be round.}{}
\entry{raúḅ}{det.}{\pf{propre}}{Own. {\itshape{}Always used after a possessive pronoun.}\ex \w{Ez raúḅ árb} ‘His own tree’.}{}
\entry{raúb’ha}{v.}{\pf{probable}}{To be probable, likely.}{}
\entry{raúb’haú}{v.}{\pf{profond}}{\\To be deep.\\To be profound.}{}
\entry{raú(b’hc’h)-}{prefix}{from \w{rób’hoc’h}}{\textit{Causative prefix, see §~\ref{subsec:diachrony-and-derivation}}.}{}
\entry{raúb’hẹ}{n.}{\pf{robot}}{Robot.}{}
\entry{raúb’hys}{v.}{\pf{robuste}}{To be strong, powerful, mighty.}{}
\entry{raúḅríy’ét’h}{n.}{\pf{propriété}}{\\Property, something that is owned.\\Ownership.}{}
\entry{raúc̣}{n.}{\pf{crochet}}{Fang.}{}
\entry{raûc}{n.}{\pf{tronche}}{Head.}{}
\entry{raúc̣é}{v.}{\pf{prochain}}{\\To be upcoming.\\To be nearby.}{}
\entry{raúc’hlá}{v.}{\pf{proclamer}}{+\s{acc/aci} To proclaim, announce, declare.}{}
\entry{raûd’hárb}{n.}{\pf{tronc d’arbre}}{Log (of a tree).}{}
\entry{raúd’yí}{v. or n.}{\pf{produire}}{\\+\s{acc} To produce, yield.\\Product.}{}
\entry{raúhérẹ́}{v.}{from \w{raú-} + \w{sérẹ́}}{\\+\s{acc} To tighten sth., make tighter (+s\w{circlat} around sth.).\\\textit{refl.} To tighten.}{\s{fut} raúhérrẹ́, \s{subj} raúhérẹ́s}
\entry{raúhy’b’h}{v.}{from \w{raú-} + \w{sy’b’h}}{To raise, lift up (+\s{acc} sth.) (+\s{ela} from sth.).}{}
\entry{raúj}{v.}{\pf{ronger}}{+\s{part} To gnaw.}{}
\entry{raúj}{n.}{\pf{projet}}{Plan, project.}{}
\entry{raúl}{n.}{\pf{parole}}{\\Language, speech, word.\\\w{Raúl} (\s{def} (\textit{only}) Short for \w{T’hebhaú Raúl}. \textit{\s{nom sg} irreg. \w{Raúl}; all other forms are regular}.}{}
\entry{raúlaú}{v.}{\pf{prolonger}}{\\\textit{intr.} To continue, persist.\\+\s{aci} \textit{tr.} To continue.}{\s{fut} raúlaúr, \s{subj} raúlaús}
\entry{raûlaú}{n.}{\pf{tromblon}}{Gun, firearm.}{}
\entry{raúlé}{n.}{\pf{problème}}{Problem, issue.}{}
\entry{raúsy’é}{v.}{\pf{grossier}}{\\To be coarse.\\(\textit{of a person}) To be rude, uncouth.}{}
\entry{raút’h}{n.}{\pf{rôtie}}{\\\s{sg} A loaf of bread.\\\s{pl} Bread.}{}
\entry{raút’hal}{n.}{\pf{crotale}}{Serpent, snake.}{}
\entry{raúvá}{n.}{\pf{fromage}}{Moon.}{}
\entry{raúvúr}{v.}{from \w{raú-} + \w{vúr}}{(+\s{acc}) To kill.}{}
\entry{raúz}{v.}{\pf{rose}}{To be pink.}{}
\entry{raûz}{n.}{\pf{bronze}}{Bronze.}{}
\entry{ráv́â}{adv.}{\pf{rarement}}{\textit{neg. only} Seldom, rarely (ever).}{}
\entry{rávẹ́}{v.}{\pf{ramer}}{(+\s{acc}) To row (a boat).}{}
\entry{rávér}{n.}{\pf{grammaire}}{\\Grammar, the grammatical rules of a language.\\A textbook describing the grammar of a language.}{}
\entry{ráy’á}{v. or n.}{\pf{voyage}}{\\To travel, go on a journey.\\\textit{n.} Travel, voyage, journey.}{}
\entry{ráy’ál}{v.}{\pf{royal}}{To be majestic.}{}
\entry{ráy’é}{v.}{\pf{noyer}}{To drown.}{}
\entry{ráy’ẹ́}{n.}{\pf{foyer}}{Home, homeland.}{}
\entry{ráy’ê}{n.}{\pf{moyen}}{\\Way, means, method.\\\w{ráy’ê y’aúhý} + \s{aci} There is no way, that \ldots{}.\\\s{instr pl} \w{b’hehráy’ê} How, by what means, in this way.}{}
\entry{ráý’ẹ}{v.}{\pf{râler}}{To complain, grumble.}{}
\entry{ráy’él}{v.}{\pf{voyelle}}{Vowel.}{}
\entry{ráz}{n.}{\pf{phrase}}{Sentence.}{}
\entry{ré}{v.}{\pf{vrai}}{To be true, correct, right.}{\s{fut} rẹ́, \s{subj} rés}
\entry{ré}{n.}{\pf{rai}}{Ray, beam.}{}
\entry{ré}{v.}{\pf{créer}, \s{subj} from \pf{fabriquer}}{To create, make (\s{+acc} sth.).}{\s{fut} rẹ́éré, \s{subj} faríc’hs}
\entry{ré}{adj.}{\pf{près}}{Near, close, nearby.}{}
\entry{ré}{v. intr.}{\pf{errer}}{To wander, roam (+\s{perl} across sth.).}{}
\entry{ré}{adv.}{en vain}{In vain, for nothing. {\itshape{}Usually preceded directly by the verb it applies to.}}{}
\entry{ré}{n.}{\pf{souhait}}{Wish.}{}
\entry{ré-}{prefix}{\pf{très}}{\textit{Superlative prefix. See grammar}.}{}
\entry{rê}{conj.}{\pf{bien que}}{+\s{subj} Although, though.}{}
\entry{rê}{v.}{\pf{trine}}{To be composed of three parts or people; triune.}{\s{fut} rêrẹ́, \s{subj} rês}
\entry{rê}{n.}{\pf{airain}}{Copper.}{}
\entry{rê}{n.}{\pf{point}}{\\Point (small mark).\\Point (in a score).}{}
\entry{rê-}{prefix}{\pf{moins}}{\textit{Neutral comparative prefix. See grammar}.}{}
\refentry{Réáb’h}{Ráb’h}
\entry{réaû}{n.}{from \w{ré}}{Creation, making.}{}
\entry{réḅ}{interj.}{\pf{ouaip}}{Yeah, yep, yes. {\itshape{}Never used in response to a yes-no question.}}{}
\entry{rébh}{v.}{\pf{préparer}}{\\+\s{part} To anticipate.\\To prepare (+\s{acc} sbd./sth.) (+\s{dat} for sbd.). {\itshape{}This verb can take up to two complements. Intransitively, it means ‘to be prepared’; with a direct object, the sense is ‘to prepare sbd./sth.’, with an indirect object ‘to prepare oneself for sth.’, and with both it becomes ‘to prepare sbd/sth. for sbd.’}}{}
\entry{rẹ́b’h}{v. or n.}{\pf{rêver}}{\\To dream (+\s{gen} of sth.).\\Dream, a dreaming.}{\s{fut} rẹ́v́e, \s{subj} rẹ́b’hs}
\entry{réb’hád’hih}{v.}{\pf{revandiquer}}{+\s{acc} To claim ownership of.}{}
\entry{rébhauz}{v.}{\pf{reposer}}{\\+\s{acc} To set down, put down, place.\\\textit{refl.} To rest, relax, repose.}{}
\entry{réb’hní}{v.}{\pf{prévenir}}{\\To prevent, stop (+\s{acc} sth. from happening).\\To forewarn (+\s{part} of sth.).}{\s{fut} réb’hníre, \s{subj} réb’hnís}
\entry{rébhós}{n.}{\pf{réponse}}{Answer, response, reply.}{}
\entry{rẹ́bhysc’hyl}{n.}{\pf{crépuscule}}{Twilight.}{}
\entry{réd}{v.}{\pf{raide}}{To be steep, abrupt.}{}
\entry{réḍ}{v.}{\pf{souhaiter}}{To wish (+\s{acc/aci} for sth.).}{}
\entry{réḍ}{v.}{\pf{frette}}{To be cool, cold.}{}
\entry{rêd}{v.}{\pf{craindre}}{+\s{opt} To fear, lest \ldots {\itshape{}Construed with the negated optative.}}{\s{fut} rêdrẹ́, \s{subj} rês}
\refentry{rêd}{ḅẹt’hẹ}
\entry{rêd’hes}{particle}{\pf{bien sûr}}{Of course, naturally, as a matter of course.}{}
\entry{réḍní}{v.}{\pf{retenir}}{\\+\s{acc} To keep, retain, mainain possession of.\\\textit{refl.} To hold back, restrain oneself.}{}
\entry{rêdrsýrśẹ}{v.}{\pf{prendre sur soi}}{\\+\s{aci} To take upon onself to do sth.\\+\s{pci} To take upon oneself to start doing sth.}{}
\entry{rẹ́dy’í}{v.}{\pf{réduire}}{To reduce (+\s{acc} \textit{or pass.} sbd./sth.) (+\s{all} to sth.).}{\s{fut} rẹ́dy’ré, \s{subj} rẹ́dy’ís}
\entry{rẹ́el}{v. archaic}{\pf{réel}}{\textit{Archaic \s{sd} form of \w{rẹ́il}}.}{}
\entry{rẹ́flec̣}{v.}{\pf{réfléchir}}{To think (+\s{part} sth.).}{}
\entry{réhál}{v.}{\pf{ressembler}}{\\+\s{gen} To resemble.\\\textit{refl.} To resemble each other.}{}
\entry{réhèḍ}{n.}{\pf{recette}}{Income.}{}
\entry{réhẹv́}{v.}{\pf{recevoir}}{To receive.}{\s{fut} réhẹv́é, \s{subj} rẹsy}
\entry{rêhibhal}{adj.}{\pf{principal}}{Main, key, principal.}{}
\entry{réhyý’}{v.}{\pf{reculer}}{To be far. {\itshape{}Both \this{} and \w{rêt’hé} can mean ‘far’, but they differ in perspective: \this{} is used to describe a great distance or the crossing of one, whereas \w{rêt’hé} is used to describe something far away.}\ex \w{Jráy’áé réhyý’vâ.} ‘I have travelled far.’\ex \w{Jráy’áé ádŷnḅéy’í rêt’hê.} ‘I travelled to a distant land.’\ex \w{disḍas réhyý’â} ‘a great distance’.}{}
\entry{réhýy’ír}{v.}{\pf{recueillir}}{+\s{acc} To gather, collect.}{}
\entry{rẹ́il}{v.}{originally \s{nd}, compare \w{rẹ́el}}{To be real, true, genuine, authentic.}{}
\entry{rẹ́jy’aû}{n.}{\pf{région}}{Area, region. {\itshape{}Usually geographical, see also \w{ḅéy’í}.}}{}
\entry{rél}{n.}{\pf{moelle}}{Marrow, bone marrow.}{}
\entry{rénéhás}{n.}{\pf{renaissance}}{Rebirth, renaissance.}{}
\entry{rér}{n.}{\pf{guerre}}{War.}{}
\entry{rêr}{n.}{\pf{fringues}}{\\An article of clothing, garment, piece of clothing.\\\textit{pl.} Clothes, garments.}{}
\entry{rés}{n.}{\pf{reste}}{Rest, remainder.}{}
\entry{résc’hil}{v.}{\pf{presqu’île}}{Peninsula.}{}
\entry{rét’h}{n.}{\pf{traité}}{Contract, treaty, deal.}{}
\entry{rét’hád}{v.}{\pf{prétendre}}{To claim, allege.}{\s{fut} rét’hádrẹ́, \s{subj} rét’h\-ádes}
\entry{rét’hẹ}{v.}{\pf{traiter}}{To handle, take care of, deal with.}{\s{fut} rét’hẹre, \s{subj} rét’hes}
\entry{rêt’hé}{v.}{\pf{lointain}}{To be distant, far-off. {\itshape{}For usage, see \w{réhyý’}.}}{}
\entry{rét’hír}{v.}{\pf{retirer}}{\\(+\s{acc}) To pull, draw, withdraw.\\+\s{part} To pull on sth. without actually moving it; to try to pull sth.}{\s{fut} rét’hírẹ́, \s{subj} rét’hírs}
\entry{révôt’hẹ}{v.}{\pf{remonter}}{To return, come back.}{}
\entry{révy’ẹ́}{num.}{}{\\First.\\\w{révy’ẹ́â} Primary.}{}
\entry{réý’í}{n.}{from \pf{trait} + \pf{ligne}}{\\Rope, thread.\\Line, straight path.}{}
\entry{rełi}{v.}{\pf{réglisse}}{To be bitter.}{}
\entry{ríb}{n.}{\pf{bribe}}{Scrap, bit.}{}
\entry{ríḅ}{num.}{}{Thrice, three times.}{}
\entry{ríḅâ}{num.}{}{Threefold.}{}
\entry{ríb’há}{v.}{\pf{suivant}}{To be next.}{}
\entry{ríb’hy’ér}{n.}{\pf{rivière}}{River.}{}
\entry{ríh}{v.}{\pf{crisser}}{To screech.}{\s{onom} ríí \textnf{/ɰĩː/}}
\entry{ríj}{v. tr.}{\pf{ériger}}{+\s{acc} To erect, raise, set up.}{}
\entry{rír}{v.}{\pf{rire}}{To laugh (+\s{gen} at sbd./sth.).}{}
\entry{rís}{v.}{\pf{triste}}{To be sad.}{}
\entry{rís}{v.}{\pf{grise}}{To be grey, gray.}{}
\entry{rívnél}{n.}{\pf{criminel}}{Scoundrel, someone without virtue.}{}
\entry{ríy’ẹ́}{v.}{\pf{crier}}{\\To call, cry out, yell, shout.\\+\s{gen} To call for someone.}{}
\entry{ríy’ŷrệ}{n.}{\pf{prieuré}}{Priory.}{}
\entry{rjẹ}{n.}{\pf{Hergé}}{Comic book.}{}
\entry{rób’hoc’h}{v.}{\pf{provoquer}; future from \pf{infliger}}{\s{+acc} To cause, make happen.}{\s{fut} flijé, \s{subj} rób’hoc’hs}
\entry{róc}{n.}{From \pf{roche}; spelt with \w{ó} to distinguish it from \w{raúc̣} ‘fang’.}{\\Rock, mass of stone. {\itshape{}See also \w{róc̣}.}\\\s{indef} Rock (material).}{}
\entry{róc̣}{n.}{\pf{rocher}}{Boulder, large rock. {\itshape{}See also \w{róc}.}}{}
\entry{rrá}{v.}{\pf{croire}}{+\s{part} To believe, think that. {\itshape{}For ‘thinking’ in the sense of ‘thinking about something’ rather than ‘thinking that something is the case’, see \w{c’hóhid’hẹ́}.}}{\s{fut} rrẹ́, \s{subj} rrás}
\entry{rrá}{n.}{\pf{croix}}{Cross (shape).}{}
\entry{rrá}{num.}{}{Three.}{}
\entry{rrád’hahánár}{n.}{\pf{froid de canard}}{Extreme cold, coldness.}{}
\entry{rráḍraúc}{n.}{\pf{droit} + \pf{gauche}}{Side.}{}
\entry{rrázy’ê}{num.}{}{\\Third.\\\w{rrázy’êâ} Tertiary.}{}
\entry{rú}{n.}{\pf{roue}}{Wheel.}{}
\entry{rúḅ}{n.}{\pf{groupe}}{Group.}{}
\entry{rúb’h}{v.}{\pf{trouver}}{+\s{acc/part} To find, discover. {\itshape{}The \s{acc} is generally preferred, but in cases where it is either questionable who discovered something first, or whether something is ‘discovered’ as opposed to ‘invented’, the \s{part} may be used instead.}}{}
\entry{rúj}{v.}{\pf{rouge}}{To be red.}{}
\entry{rvá}{interj.}{of unknown origin}{Alas, woe, oh. {\itshape{}Exclamation of distress, surprise, sadness, or regret.}}{\textit{after words that end with ‘r’, this is spelt \w{-vá} instead}}
\entry{Rýb’hihaú}{n.}{\pf{Rubicon}}{Rubicon. {\itshape{}River in Italy.}}{}
\entry{rýc̣ér}{v.}{\pf{requerir}}{To ask, question.}{}
\entry{rýd}{v.}{\pf{rude}}{To be uneven, rough, rugged.}{}
\entry{ry’él}{v.}{\pf{cruel}}{To be cruel.}{}
\entry{rýl}{v.}{\pf{brûler}}{\\\s{+acc} To burn.\\\s{+part} To scorch, singe.}{}
\entry{rýl}{n.}{\pf{gueule}}{Face.}{}
\entry{rýrŷ}{v.}{\pf{rugueux}}{To be rough, rugged.}{}
\entry{rýsḍ}{v.}{\pf{frustrer}}{To frustrate, vex, annoy.}{}
\entry{rývýr}{n.}{\pf{rumeur}}{History.}{}
\entry{rýý’ẹ́}{v.}{\pf{céruléen}}{To be cerulean, sky-blue.}{}
\entry{rzáḅ}{n.}{\pf{exemple}}{\\Example.\\\s{perl} \w{lý’ýnrzáḅ} For example, for instance.}{}
\entry{rzaúsḍ}{v.}{\pf{exhaustif}}{\\To be exhaustive, comprehensive, complete.\\To be finished, completed.}{\s{fut} rzaúsḍre, \s{subj} rzaúsḍs}
\entry{s}{conj.}{\pf{si}}{If, when, whenever.}{}
\refentry{s’}{sá}
\refentry{ś}{r}
\entry{sá}{particle}{\pf{sans}}{Not, no. {\itshape{}Always enclitic \w{s’} before vowels. This particle is used only in the subjunctive; see also \w{asý’ýâ}, \w{t’hé}.}}{}
\entry{sá}{conj.}{\pf{sans que}}{+\s{subj} Without (doing sth.).}{}
\entry{sáb’hé}{v.}{\pf{sans fin}}{To be endless, unending, infinite.}{}
\entry{sáḍy’ér}{n.}{\pf{sanctuaire}}{Sanctuary, shrine.}{}
\entry{sáhẹ}{v.}{\pf{insensé}}{To be preposterous, absurd, nonsensical.}{\s{fut} sáhere, \s{subj} sáhes}
\entry{sáhyẹ}{adj.}{\pf{sens dessus dessous}}{To be upside down.}{}
\entry{saj}{v.}{\pf{sage}}{To be wise, prudent.}{}
\entry{sajès}{n.}{\pf{sagesse}}{Wisdom.}{}
\entry{Sásc’hríḍ}{n.}{\pf{sanskrit}}{The Sanskrit language. {\itshape{}Never lenited.}}{}
\entry{sásy’él}{v.}{\pf{essentiel}}{To be essential.}{\s{fut} sásy’élẹ́, \s{subj} sásy’éls}
\entry{sát’hás}{n.}{\pf{sentence}}{Verdict, judgement, court sentence.}{}
\entry{sat’hýrn}{n.}{\pf{saturnin}}{Lead (metal).}{}
\entry{sauc’h}{conj.}{\pf{sauf que}}{+\s{subj} Except that.}{}
\entry{saul}{n.}{\pf{sol}}{Sun.}{}
\entry{saúr}{n.}{\pf{sorte}}{\\Kind, sort, type, form.\\\s{def + gen} (some) kind(s) of.}{}
\entry{saut’h}{v. intr. or tr.}{\pf{sauter}}{To teleport, translocate, warp (+\s{acc} sth.).}{}
\entry{sauz}{n.}{\pf{chose}}{Thing, object.}{}
\entry{sauz-aud}{adj.}{\pf{autre chose}}{Something else, another thing.}{}
\refentry{sauzaud}{sauz-aud}
\entry{sav́á}{v.}{\pf{savoir}}{To know (+\s{part/acc} sth.). {\itshape{}The case depends on the depth of the speaker’s understanding: the \s{acc} indicates ‘complete’ or expert-level understanding, the \s{part} only some understanding. In questions, the \s{acc} is rather harsh (similar to ‘do you have even the slightest idea ...’) with the \s{part} preferred if a more neutral tone is intended. In negated statements, the \s{acc} indicates a complete lack of understanding (‘I have no idea whatsoever ...’), whereas the \s{part} is again neutral.}}{\s{fut} saúr, \s{subj} sac}
\entry{Sávýy’él}{n.}{\pf{Samuel}}{\textit{Male given name}.}{}
\entry{say’ẹ}{interj.}{\pf{salut}}{\textit{(informal)} Hi, hello.}{}
\entry{sḅas}{n.}{\pf{espace}}{Space.}{}
\entry{sḅé}{v.}{\pf{espérer}}{\\To want (+\s{acc/inf} sth.).\\+\s{opt} To wish, want, desire.}{\s{fut} sḅérẹ́, \s{subj} sḅés}
\entry{sb’hé}{v.}{\pf{se baigner}}{To bathe.}{}
\entry{sḅrí}{n.}{\pf{espirit}}{Soul.}{}
\entry{sd’hehis}{v.}{\pf{se désister}}{To withdraw, back out, stand down.}{\s{fut} sd’hehise, \s{subj} sd’hehiss}
\entry{sd’hévâd’h}{v.}{\pf{se demander}}{+\s{pci} To wonder, ponder.}{}
\entry{sẹ}{particle}{\pf{ainsi}}{So, thus, now, well then, as a result.}{}
\entry{séḅ}{v.}{\pf{simple}}{To be plain, simple.}{\s{fut} séḅrẹ́, \s{subj} séḅs}
\entry{séḅ}{num.}{}{Once.}{}
\entry{seb’haúd}{v. intr.}{\pf{s’effondrer}}{To cave in, collapse.}{}
\entry{séc̣é}{num.}{}{Fifth.}{}
\entry{sèc’h}{v.}{\pf{sec}}{To be dry.}{}
\entry{séc’h}{num.}{}{Five.}{}
\entry{sèḍ}{num.}{}{Seven.}{}
\entry{seḍáḍ}{n.}{from \w{seḍáḍríy’}}{North.}{}
\entry{seḍáḍríy’}{v.}{\pf{septentrion}}{To be northern.}{}
\entry{sèd’h}{part.}{from \pf{c’est du}}{It is due to (+\s{gen} sth. / +\s{aci} the fact that...).}{}
\entry{sèḍy’ê}{num.}{}{Seventh.}{}
\entry{sẹh}{det.}{\pf{ceci}}{+\s{def} \textit{noun} This, these. {\itshape{}Precedes and is attached to nouns.}}{}
\entry{sehár}{n.}{\pf{César}}{\\King, queen, monarch. {\itshape{}Usually used regardless of the person’s gender.}\\\w{Sehár} Caesar.}{}
\entry{sehárê}{n.}{From \w{César} + \pf{reine}}{Queen.}{}
\entry{sehárês}{n.}{From \w{César} + \pf{prince}}{Prince, princess. {\itshape{}Usually used regardless of the person’s gender.}}{}
\entry{sehárêsê}{n.}{From \w{sehárês}; the final \w{ê} was added in analogy with how \w{sehárê} derives from \w{sehár}.}{Princess.}{}
\entry{sẹhérél}{v.}{\pf{se quereller}}{To quarrel, argue, fight about (+\s{part}).}{}
\entry{sehul}{v.}{\pf{s’écouler}}{To flow.}{}
\entry{sẹhúr}{v.}{\pf{secourir}}{To help, succour, give aid (+\s{dat} to sb.) (+\s{aci}/\s{acc} with sth.).}{\s{fut} sẹhúrre, \s{subj} sẹhús}
\entry{sèl}{n.}{\pf{sel}}{Salt.}{}
\entry{sénýr}{n.}{\pf{seigneur}}{\\Lord.\\\textit{Short for \w{Ráb’h sénýr}}.}{}
\entry{sẹrád}{v. intr.}{\pf{se rendre}}{To surrender.}{}
\entry{sérḍé}{det.}{\pf{certain}}{Certain, particular but not specified.}{}
\entry{sérẹ́}{v.}{\pf{serré}}{\\To be tight, close-fitting, snug.\\\s{def} \textit{usually} \s{instr} \w{c’hýr sérệ} A heavy heart. {\itshape{}The use of the \s{instr} instead of the \s{ess} case in this sense is an established idiom; in Middle UF, the \s{ess} was more common in this sense.}}{\s{fut} sérrẹ́, \s{subj} sérẹ́s}
\entry{sérí}{n.}{\pf{serie}}{\\Column (of a table).\\Column, newspaper article.}{}
\entry{sérvâdrráḍ}{n.}{\pf{segment de droute}}{Segment, line segment.}{}
\entry{sès}{v.}{\pf{cesser}}{To cease, stop.}{}
\entry{sèt’h}{v.}{\pf{sentir}}{To feel.}{\s{fut} sèt’he, \s{subj} sès}
\entry{set’hád}{v. chiefly trans.}{from \pf{s’étendre}; see also \senseref{3}}{\\+\s{acc/part} To spread, extend, expand. {\itshape{}The \s{acc} is used for concrete objects or substances, the \s{part} for abstract or intangible concepts.}\\\textit{refl.} To sprawl, stretch out, extend oneself, reach for (+\s{dat} sth.).\\\textit{intr. (archaic) Same as \senseref{2}, but implicitly reflexive even though it is morphologically only intransitive. This sense has been largely replaced by the morphological reflexive}.}{}
\entry{sévê}{n.}{\pf{semaine}}{Week.}{}
\entry{séy’ẹ́}{v.}{\pf{essayer}}{+\s{part} \textit{or} \s{inf} To try, attempt.}{\s{fut} séy’ẹ́rẹ́, \s{subj} séy’ẹ́s}
\entry{sib’haú}{v.}{\pf{siphonner}}{\textit{(lit. or fig.)} +\s{part} To siphon; to steal or remove small amounts of sth.}{\s{fut} sib’haún, \s{subj} sib’haús}
\entry{sib’hil}{n.}{\pf{civil}}{Civilian, layperson.}{}
\entry{siḍ}{n.}{\pf{site}}{Facility, site.}{}
\entry{sid’hẹ}{v.}{\pf{ci-dessus}}{To be above.}{}
\entry{siránó}{n.}{\pf{Cyrano (de Bergerac)}}{Nose.}{}
\entry{sis}{num.}{}{Six.}{}
\entry{sisḍé}{n.}{\pf{système}}{System.}{}
\entry{Sit’h}{n.}{\pf{Sith}}{Sith (Star Wars).}{}
\entry{sit’há}{conj.}{\pf{si tant est que}}{+\s{opt} Supposing that; if, assuming that.}{}
\entry{sívý’ér}{v.}{\pf{similaire}}{To be similar, alike (+\s{gen} to sth.).}{}
\entry{sizy’ê}{num.}{}{Sixth.}{}
\entry{Snet’h}{n.}{}{\textit{Family name, equivalent to English ‘Smyth’}.}{}
\entry{’só}{abbr.}{}{\textit{Informal, abbreviated form of \w{aúsó}, \s{1pl} of \w{eḍ}}.}{}
\entry{sol}{n.}{\pf{sol}}{Ground, floor, earth, soil. {\itshape{}The plural may be used to indicate a large quantity of soil.}}{}
\entry{suḍ}{v.}{\pf{soutenir}}{\\+\s{acc} To support, hold up.\\+\s{part} To help support, hold up part of.}{}
\entry{sud’hénvâ}{adv.}{\pf{soudainement}}{Suddenly.}{}
\entry{suf}{n.}{\pf{souffre}}{Pain.}{}
\entry{sufb’h}{n.}{\pf{souffle} + \pf{vie}}{Life.}{}
\entry{suh}{n.}{\pf{souci}}{Fear, dread.}{}
\entry{susḍré}{v.}{\pf{soustraire}}{To subtract.}{}
\entry{susy’é}{v.}{\pf{soucier}}{+\s{part, pci} To care about, worry about.}{\s{fut} susy’ére, \s{subj} susy’és}
\entry{swi}{det.}{\pf{celui}}{The one, that one, this one.}{}
\refentry{sy̌}{dýr}
\entry{sý’a}{pron.}{}{\textit{Archaic form of \w{sý’ẹ}}.}{}
\entry{sy’b’h}{v. intr.}{\pf{se lever}}{To rise (+\s{ela} from sth.).}{}
\entry{sybhérfih}{n.}{\pf{superficie}}{\\Surface.\\(\textit{maths}) Area.}{}
\entry{sybhẹ́rýr}{v.}{\pf{supérieur}}{\textit{intr. or} +\s{gen} To be superior to, better than, higher than.}{\s{fut} sybhẹ́rýrẹ́, \s{subj} sybhẹ́rýrs}
\entry{syb’hír}{v.}{\pf{suffire}}{To suffice, be enough. {\itshape{}Governs the subjunctive when used when an \s{aci}.}}{}
\entry{syḅlẹ}{v.}{\pf{suppléer}}{\\To supplement (\s{acc} sth.) (+\s{instr} with sth.). {\itshape{}If no \s{instr} is present, the subject is implied to be the supplement.}\\\w{syḅlâ} \textit{adj.} Additional, extra.}{}
\entry{sy’ê}{v.}{\pf{sien}}{To be his, hers, its.}{\s{fut} sy’êrẹ́, \s{subj} sy’ês}
\entry{sý’ẹ}{det.}{\pf{cela}}{+\s{def} \textit{noun} That, those. {\itshape{}Precedes and is attached to nouns; often \w{sý’} before vowels, with one apostrophe, not two.}}{}
\entry{syhýr}{v. or n.}{\pf{susurrer}}{\\To whisper.\\Whisper, whispering, rustling.}{}
\entry{syhyý’á}{v.}{\pf{succulent}}{To be succulent, delicious.}{\s{fut} syhyý’áré, \s{subj} syhyý’ás}
\entry{syl}{v.}{\pf{seul}}{\\To be the only one.\\To be lone, alone.}{\s{fut} syle, \s{subj} syls}
\entry{sýr}{adj. or v.}{\pf{sûr}}{\\Sure.\\To be sure, certain. {\itshape{}This is both an adjective verb and an adjective: the adjective form is always \w{sýr}, not *\w{sýrâ}, but it can also be conjugated, e.g. \w{jsýr} ‘I am sure’.}\\\textit{adv.} \w{sýr(é)vâ} Surely, certainly.}{}
\refentry{’sý’ýâ}{asý’ýâ}
\refentry{t-}{ḍ-}
\entry{t’hé}{conj.}{\pf{de peur que} > *\w{dbhýrc’h} > *\w{dýrc’h} > *\w{dc’hý} > \this}{Not, no. {\itshape{}Always \w{t’h’\N} before vowels, but does not nasalise if the ‘é’ is still present. This particle is used only in the optative; see also \w{asý’ýâ}, \w{sá}.}}{}
\entry{T’hebhaú}{n. or adj.}{from \w{t’hebhaúz}}{(ULTRA-) France, (ULTRA-)French.}{}
\entry{T’hebhaú Raúl}{n. def. sg.}{from \w{t’hebhaúz} + \w{raúl}}{The ULTRAFR\-ENCH language. {\itshape{}Only \w{T’hebhaú} is declined as though the entire phrase were one word. In informal speech and writing, this is typically shortened to \w{Raúl}.}}{\s{nom sg} \textit{irreg.} \w{T’hebhaú Raúl}}
\entry{t’hebhaúz}{v.}{\pf{jeter l’éponge}}{To be (ULTRA-)French.}{\s{fut} t’hebhaúźe, \s{subj} t’hebhaúś}
\entry{t’hiy’e}{v.}{from \w{yt’hiy’ihẹ}; \s{subj} via ba\-ck-formation from the \s{fut}}{+\s{part} To use, make use of.}{\s{fut} t’hiźe, \s{subj} \s{t’hizes}}
\entry{u}{conj.}{\pf{ou}}{\\Or. {\itshape{}Inclusive, see also \w{ra}.}\\\w{u} \ldots\ \w{u} \ldots\ ‘\ldots\ or \ldots’ \textit{(inclusive)}.}{}
\entry{ub’h}{v.}{\pf{ouvrir}}{To open.}{\s{fut} uv́, \s{subj} ub’hs}
\entry{ub’hrá}{v.}{\pf{pouvoir}}{\\+\s{inf/aci} To be able to, can. {\itshape{}Never construed with an \s{inf} if it in and of itself is the infinitive of an \s{aci} or \s{pci}, in which case the variant with the \s{part} (\senseref{2}) is used instead.}\\+\s{part} To be capable of \ldots\\\s{opt cond i + aci} To be possible; may. {\itshape{}Dynamic or epistemic, never deontic; this and sense 4 are essentially a more emphatic optative.}\\\s{opt cond ii + aci} Might. {\itshape{}Dynamic or epistemic, never deontic.}}{\s{fut} úrẹ́, \s{subj} ís}
\entry{ulíy’ẹ́}{v.}{\pf{oublier}}{To forget.}{\s{fut} ulíy’ẹ́rẹ́, \s{subj} ulíy’ẹ́s}
\entry{úrbh}{conj.}{\pf{pour peu que}}{+\s{opt} Provided that, so long as.}{}
\entry{urdálbhaúrḍ}{n.}{\pf{avoir un oursin dans le portefeuille}}{A very rich person; billionaire.}{}
\refentry{úrẹ́}{ub’hrá}
\entry{uy’ed’háb’hrí}{v.}{\pf{rouler dans la farine}}{To scam, cheat, swindle.}{\s{fut} uy’e\-d’háv́e, \s{subj} uy’ed’háb’hrís}
\entry{vá}{n.}{\pf{mât}}{Mast.}{}
\refentry{vá}{rvá}
\entry{váćár}{n.}{\pf{mâchoire}}{Jaw, jawbone.}{}
\entry{vádłabhaud’hávúrsab’hád’háváb’hrárḍuẹ}{v.}{\pf{vendre la peau de ours avant de avoir tué}}{\textit{(literary)} To depend on predictions of the future. {\itshape{}Of disputed origin; first attested in the works of the Early UF comedian \s{J. A. B. Snet’h}.}}{\s{fut} vád\-ła\-bhau\-d’há\-vúr\-sa\-b’há\-d’há\-vá\-b’hrár\-ḍu\-re, \s{s} vád\-ła\-bhau\-d’há\-vúr\-sa\-b’há\-d’há\-vá\-b’hrár\-ḍus}
\entry{vâhẹ}{v.}{\pf{manquer}}{\\+\s{gen} To lack, want.\\+\s{part} \textit{or} \s{pass} To miss. {\itshape{}The object and subject of this verb are swapped compared to English ‘to miss’, e.g. \w{b’hývvâhé} (\s{2pl.act} + \s{1sg.pass}) ‘I miss you (\s{pl})’, lit. roughly ‘you (\s{pl}) are wanting to me’).}\\+\s{acc} To miss out on.}{\s{fut} vâhérẹ́, \s{subj} vâhés}
\entry{váj}{n.}{from \w{íváj}}{Image, picture.}{}
\entry{vâj}{v.}{\pf{manger}}{To eat.}{}
\entry{vájaúr}{n.}{\pf{majorité}}{\\Majority (i.e. more than 50 \% of sth.).\\Adulthood.}{}
\entry{válḍrét’hás}{n.}{\pf{maltraitance}}{Torture.}{}
\entry{válfèz}{v.}{\pf{malfaisant}}{To be malfeasant, evil, malevolent.}{\s{fut} válfèź, \s{subj} válfès}
\entry{válv́áy’}{v.}{\pf{malvoyant}}{To be blind.}{}
\entry{válvê}{v.}{\pf{malmener}}{To mistreat, torture.}{\s{fut} válv́e, \s{subj} válvês}
\entry{vár}{n.}{\pf{marque}}{Mark.}{}
\entry{vár}{n.}{\pf{marche}}{\\March (way of walking).\\March (music).\\Step (of a stair or ladder).}{}
\entry{v́ár}{v. irreg.}{\pf{devoir}}{\\\s{pass} +\s{aci} Must, have to, be obliged to. {\itshape{}The subject is always in the passive in this sense only.}\\+\s{dat} To owe sbd. (+\s{acc} sth.).\\\s{cond i + aci} Even if.\ex \w{aúrdyssa dẹće} ‘even if he should fail’.}{\s{cond i, ii} dy, \s{fut} dv́e, \s{subj} ráb’h}
\entry{váráhé}{n.}{\pf{magazin}}{Shop, store.}{}
\entry{váraû}{v. or n.}{\pf{marron}}{\\To be brown.\\\textit{n.} Chestnut.}{}
\entry{várc}{v.}{\pf{marcher}}{\\To march, walk.\\\textit{fig.} To work, function.}{}
\entry{várḍeý’}{v.}{\pf{marteler}}{+\s{acc} To forge.}{}
\entry{vás}{n. \s{pl def}}{\pf{masses}}{The masses, the people.}{}
\entry{vát’hé}{n.}{\pf{matin}}{Morning.}{}
\entry{vaú}{v.}{\pf{mot}}{Word.}{}
\entry{vaúb’hẹ}{v. irreg.}{\pf{mauvais}}{\\To be bad.\\To be wrong, incorrect, inappropriate.}{\s{fut} bíré, \s{subj} bíres; \s{comp} lẹbír, y’ŷbír, rêbír; \s{sup} réb’hír, râdvâbír}
\entry{vaûd}{n.}{\pf{monde}}{\\World.\\\w{vaûd ḍẹ} Everyone. {\itshape{}Other than literal ‘the whole world’.}}{}
\entry{vaûḍ}{v.}{\pf{montrer}}{To show, display (+\s{acc} sth.).}{}
\entry{vaúd’hèl}{n.}{\pf{modèle}}{Model, pattern.}{}
\entry{vaúd’hér}{v.}{\pf{modérer}}{To be moderate.}{}
\entry{vaût’há}{n.}{\pf{montagne}}{Mountain.}{}
\entry{vaút’hif}{n.}{\pf{motif}}{\\Motive.\\Motif, pattern, design.}{}
\entry{vaúý’ehyl}{n.}{\pf{molécule}}{Molecule.}{}
\entry{váý’eb’his}{n.}{\pf{maléfice}}{Vice.}{}
\entry{váý’ýr}{n.}{\pf{malheur}}{Tragedy, misfortune.}{}
\entry{váłé}{conj.}{\pf{malgré que}}{+\s{subj} Despite that, in spite of.}{}
\entry{vé}{conj.}{\pf{mais}}{But, however, although.}{}
\entry{vê₁}{adv.}{\pf{demain}}{Tomorrow.}{}
\entry{vê₂}{n.}{\pf{main}}{Hand.}{}
\entry{vê₃}{v.}{\pf{même}}{\\To be the same, identical, alike.\\\textit{(emphatic)} Oneself. {\itshape{}Placed either directly after a noun or infixed after (the prefix part of) a pronoun.}\ex \w{Aúłau vê ssèhá’z ivúb’hvâ} ‘Time itself stood still’. {\itshape{}Lit. ‘Time itself ceased its movement’.}\ex \w{Jvêsyráré} ‘I saw it myself’.\\\w{vêvâ} Even. {\itshape{}In this sense, \w{vêvâ} usually precedes the noun (phrase) it qualifies.}\ex \s{H. P. Lovecraft:} \w{Lavúrer’sý’ýâ là dwájávé sdaúr~/ Ádȅr trâ vêvâ Dérny’éhuf laúv’raú.} ‘That is not dead which can eternal lie~/ And with strange æons even death may die’.}{}
\entry{véc}{n.}{\pf{mèche}}{\\A strand of hair.\\\s{pl} Hair.\\Wick (of a candle).\\Fuse (of a bomb).}{}
\entry{véc’h}{n.}{\pf{mec}}{\textit{(informal)} Dude, mate, guy, chap, bloke, bud, buddy.}{}
\entry{véḍ}{v.}{\pf{mettre}}{To lay, put, place (+\s{acc} sth.).}{}
\entry{véḍ}{n.}{\pf{maître}}{Master, expert.}{}
\entry{véḍrâḅla}{v.}{\pf{mettre en place}}{\\+\s{acc} To organise, set up.\\+\s{aci} To arrange for.}{}
\entry{vẹ́ḍy’ẹ́}{n.}{\pf{métier}}{Loom.}{}
\entry{véhýr}{conj.}{\pf{dans la mesure où}}{Insofar as.}{}
\entry{véhýr}{v/n.}{\pf{mesure}}{\\To measure.\\Measurement.}{\s{fut} véhýrẹ́, \s{subj} véhýrs}
\entry{vér}{n.}{\pf{mère}}{\textit{(informal)} Mum, mom.}{}
\entry{vẹ́ríd}{n.}{from \w{vẹ́ríd’yê}}{South.}{}
\entry{vẹ́ríd’yê}{v.}{\pf{méridien}}{To be southern.}{}
\entry{vérjet’hic’h}{v.}{\pf{énergétique}}{To be vigorous, energetic.}{}
\entry{vérr}{n.}{\pf{mer}}{Sea, ocean.}{}
\entry{vérs}{interj.}{\pf{merci}}{\\Thank you. (+\s{gen} for sth.).\\\w{dyvérs fér} To thank (+\s{dat} sbd.) (+\s{gen} for sth.).}{}
\refentry{vérvá}{vér + vá}
\entry{vẹ́t’he}{n.}{\pf{météo}}{Weather.}{}
\entry{vêt’hnâ}{adv.}{from \pf{maintenant}, lenited for unknown reasons}{Now.}{}
\refentry{véy’ýr}{baú}
\entry{víd’hẹ}{n.}{\pf{midi}}{Noon, midday.}{}
\entry{Víd’hic’hlaúry’ê}{n.}{\pf{Midichlorien}}{Midichlorian (Star Wars).}{}
\entry{vísy’ô}{n.}{\pf{émission}}{\\Emission.\\Programme, broadcast, show.}{}
\entry{víwý}{v. or n.}{\pf{milieu}}{\\To be central, in the middle of.\\\textit{n.} Centre, center.}{}
\entry{vnásḍér}{n.}{\pf{monastère}}{Castle.}{}
\entry{vnaúr}{n.}{\pf{minorité}}{\\Minority (i.e. less than 50 \% of sth.).\\Youth.}{}
\entry{vnýḍ}{n.}{\pf{minute}}{Minute.}{}
\entry{vú}{adj.}{\pf{moult}}{Many, much, a lot of.}{}
\entry{vú}{v.}{\pf{mou}}{To be soft, squishy.}{}
\entry{vúb’hvâ}{n.}{\pf{movement}}{Movement, motion.}{}
\entry{vúr}{v.}{\pf{mourir}}{To die.}{\s{pres ptcp} vaûr, \s{fut} vúre}
\entry{vúr}{n.}{Most likely borrowed from Friulian \w{mûr}}{Wall.}{}
\entry{vúslihé}{n.}{\pf{mousse} + \pf{lichen}}{Moss.}{}
\entry{vúy’ẹ́}{v.}{\pf{mouillé}}{To be wet, moist.}{}
\entry{vvâ}{n.}{\pf{maman}}{Mother.}{}
\entry{vvâ}{n.}{\pf{moment}}{Moment, instant.}{}
\entry{vvaúríhe}{v.}{\pf{mémoriser}}{To remember.}{\s{fut} vvaúríźe, \s{subj} vvaúríhes}
\entry{vŷ}{v.}{\pf{mener}}{To lead.}{\s{fut} menre, \s{subj} mens}
\entry{výḍy’él}{v.}{\pf{mutuel}}{To be mutual, reciprocal.}{}
\entry{výhic’h}{n.}{\pf{musique}}{Music.}{}
\entry{výlḍiḅḷ}{v.}{\pf{multiplier}}{To multiply.}{}
\entry{výlḍiḅlihit’h}{n.}{\pf{multiplicité}}{Plural.}{}
\entry{w}{v.}{\pf{enlever}}{To remove (+\s{acc} sth.).}{}
\entry{ý}{num.}{}{One.}{}
\entry{y’ác’hraúníc’h}{v.}{\pf{diachronique}}{To be diachronic.}{\s{fut} y’ác’hraú\-níc’hre, \s{subj} y’ác’hraúníc’hes}
\entry{y’âs}{n.}{\pf{science}}{Science.}{}
\entry{y’aúhý}{particle}{\pf{il n’y a aucun}}{There is no, there are no, there is none. {\itshape{}See also \w{ý’aúhý}.}}{}
\entry{ý’aúhý}{particle}{\pf{il y a aucun}}{\\There is, there are. {\itshape{}See also \w{y’aúhý}.}\\\w{ý’aúhý c’hes} Is there?, are there?}{}
\entry{y’aúý’}{v.}{back-formation from \w{y’aúý’vâ}, displacing earlier \w{y’aúý’á}}{To be violent, vehement.}{}
\entry{y’aúý’á}{v. archaic}{back-formation from \w{y’aúý’ávâ}}{To be violent, vehement.}{}
\entry{y’aúý’ávâ}{adv. archaic}{\pf{violament}}{Violently, vehemently.}{}
\entry{y’aúý’è}{v.}{\pf{violette}}{To be purple.}{}
\entry{y’aúý’vâ}{adv.}{back-formation from \w{y’aúý’á}, displacing earlier \w{y’aúý’ávâ}}{Violently, vehemently.}{}
\entry{y’é}{pron.}{\pf{rien}}{Nothing. {\itshape{}Like most negative polarity items, this induces negation of the verb.}}{}
\entry{y’ẹ́}{v.}{\pf{nier}}{To forbid, deny.}{\s{fut} y’ẹ́rẹ́, \s{subj} y’ẹ́s}
\entry{y’ê}{v.}{\pf{mien}}{To be mine.}{\s{fut} y’êrẹ́, \s{subj} y’ês}
\entry{y’éc’h}{n.}{\pf{siècle}}{Century.}{}
\entry{y’éjúré}{n.}{\pf{siège} + \pf{tabouret}}{Chair, seat.}{}
\entry{y’ér}{adv.}{\pf{hier}}{Yesterday.}{}
\entry{y’ér}{n.}{\pf{pierre}}{\\\s{indef} Stone (subtance).\\Stone, pebble.}{}
\entry{yhí}{n.}{\pf{usine}}{Machine, device.}{}
\entry{yhícáry’aú}{n.}{From \w{yhí} + \w{cáry’aú}}{Truck.}{}
\entry{y’í}{n.}{\pf{nuit}}{Night.}{}
\entry{y’í}{n.}{\pf{puits}}{Well (water source).}{}
\entry{y’î}{ruine}{\pf{n.}}{Ruin.}{}
\entry{y’íb’h}{v.}{\pf{suivre}}{\\(+\s{acc}) To follow. {\itshape{}In its literal sense.}\\(+\s{gen}) To follow, get, understand.}{}
\entry{y’íḍ}{num.}{}{Eight.}{}
\entry{y’íḍy’ê}{num.}{}{Eighth.}{}
\entry{y’íhá}{v.}{\pf{puissant}}{To be powerful, mighty, puissant.}{}
\entry{y’íhás}{n.}{\pf{puissance}}{Might, power.}{}
\entry{y’íhó}{n.}{\pf{buisson}}{Bush, shrub.}{}
\entry{y’ír}{v.}{\pf{ouïr}}{\\To understand, listen (+\s{part} to sbd.).\\\textit{(rarely)} +\s{acc} To hear.}{\s{fut} aúré, \s{subj} rás}
\entry{y’ís}{conj.}{\pf{puisque}}{Considering that, since, because. {\itshape{}Unlike \w{c’haúr}, this does not take the subjunctive; it is used to indicate the (potential) cause of something.}}{}
\entry{yl}{n.}{\pf{hublot}}{Window.}{}
\entry{ýnít’h}{n.}{\pf{unité}}{\\Unity.\\Unit.}{}
\entry{ýr}{v.}{\pf{heurter}}{To hit, strike.}{\s{fut} ýrḍ, \s{subj} ýrs}
\entry{ýrŷ}{v.}{\pf{heureux}}{To be happy.}{}
\entry{yt’hiy’ihẹ}{v.}{\pf{utiliser}}{+\s{part} \textit{(archaic)} To use, make use of.}{\s{fut} yt’hiy’iźe, \s{subj} yt’hiy’\-ihẹs}
\entry{y’úḅłẹ}{particle}{\pf{s’il vous plait}}{Please.}{}
\entry{y’úr}{n.}{\pf{jour}}{\\Day.\\\textit{adj.} \w{cahýny’úr} Every day, daily.\\\textit{adv.} \w{órdy’úr ád’y’úr} Day after day. {\itshape{}Contracted \s{ela} and \s{ill}.}}{}
\entry{y’ŷ}{n.}{from \w{y’ŷvéłáfrí}}{Eye.}{}
\entry{y’ŷ}{v.}{\pf{vieux}}{To be old.}{}
\entry{y’ŷ-}{prefix}{\pf{mieux}}{\textit{Denying comparative prefix. See grammar}.}{}
\entry{Yý’is}{n.}{\pf{Ulysse}}{\textit{Male given name}.}{}
\entry{y’ŷvéłáfrí}{n. pl. archaic}{\pf{yeux de merlan frit}}{Eyes.}{}
\entry{Zauḅ}{n.}{\pf{Ésope}}{Aesop.}{}

