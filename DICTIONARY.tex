%% This is a generated file
%% Do not edit this file manually
%%
%% To update this file, edit DICTIONARY.txt and rerun
%% GENERATE-DICTIONARY.sh

\entry{ab’haḍ}{v.}{\pfabbr abattre}{1. To cut down, fell, knock down, shoot down. 2. To butcher, cut apart violently}{\s{fut} *ab’haḍrẹ́*, \s{subj} *ab’has*}
\entry{ab’hásy’ô}{n.}{\pfabbr aviation}{Aviation}{}
\entry{ab’héy’}{n.}{\pfabbr abeille}{Bee}{}
\entry{ab’hínéb’heḅaý’évrâ}{v.}{\pfabbr habit ne fait pas le moine}{To judge based on appearances}{\s{fut} *ab’hínéb’heḅaý’év́ẹ́*, \s{subj} *ab’hínéb’heḅaý’év́ás*}
\entry{ac}{n.}{\pfabbr hache}{Axe, hatchet}{}
\entry{aċt’he}{v. tr.}{\pfabbr acheter}{To buy}{\s{fut} *aċḍrẹ́*, \s{subj} *aċt’hes*}
\entry{act’he}{v. tr.}{\nf from *ac*}{1. To cut or cleave with an axe. 2. +\s{acc} To bring an end to. 3. +\s{acc def} *of* ***árb*** *intr. (other than literal)* To get to the point, cut to the chase. 4. +\s{acc def} *of* ***árb*** *and* \s{acc} To bring to light, reveal *(originally, this idiom did not take a double \s{acc}, but was instead formed with the \s{acc} of ‘tree’ and the \s{ill} of the object, meaning something along the lines of ‘to bring down the tree(s) on sth’—the image here being that of cutting down trees in a wood until only a clearing remains or is ‘brought to light’)*}{\s{fut} *acḍe*, \s{subj} *act’hes*}
\entry{ad’he}{v.}{\pfabbr vader}{To go}{\s{fut} *ad’hrẹ́*, \s{subj} *ad’hes*}
\entry{ad’hór}{v. tr.}{\pfabbr adore}{1. To love, adore. 2. +\s{part} To be in love with, have a crush on. 3. +\s{gen} To desire (someone)}{\s{fut} *ad’hórérẹ́*, \s{subj} *ad’hórs*}
\entry{áhaúr}{conj.}{\pfabbr encore}{+\s{subj} Even though}{}
\entry{ânb’hé}{adv.}{\pfabbr en effet\nf, via metathesis from \**ânéb’he*}{Verily, indeed, in fact}{}
\entry{ánvé}{v. tr.}{\pfabbr animer}{To bring to life, animate}{}
\entry{árb}{n.}{\pfabbr arbre}{Tree}{}
\entry{asý’ýâ}{particle}{\pfabbr pas absolument}{Not, no. *Commonly* ***’sý’ýâ*** *after vowels. This particle is used only in the indicative; see also* ***sá***}{}
\entry{aub’heír}{v. (in)tr.}{\pfabbr obéir}{To obey}{}
\entry{au}{conj.}{\pfabbr aussi}{And, also, as well, too}{}
\entry{auḍé}{v.}{\pfabbr obtenir}{1. To obtain, get, acquire. 2. +\s{abl} To gain purchase on or height or distance from}{\s{fut} *auḍy’édrẹ́*}
\refentry{aúfý}{eḍ}
\refentry{aúsó}{eḍ}
\entry{aý’aúr}{conj.}{\pfabbr alors}{While, as (temporal)}{}
\entry{ḅáł}{v.}{\pfabbr parler}{To speak, talk, say}{\s{fut} *báłérẹ́*}
\entry{ḅárḍáḍ}{v.}{\pfabbr partante}{ (+ \s{aci}) To be interested in, willing to, ready to, prepared for}{}
\entry{ḅarḍ}{v.}{\pfabbr partir}{ To leave, go away, depart}{\s{fut} *ḅarẹ́*, \s{subj} *ḅars*}
\entry{ḅas}{conj.}{parce que}{+\s{subj} Because *(often used to explain motivation rather than cause as in ‘We did that because\ldots’)*}{}
\entry{ḅauheŷnlabhé}{v.}{\pfabbr poser un lapin}{To forsake, abandon}{\s{fut} *ḅauheŷnlabhére*, \s{subj} *ḅauheŷnlabhés*}
\entry{ḅẹt’hẹ}{v.}{\pfabbr petit}{To be small, little}{}
\entry{b’he}{conj.}{\pfabbr envers}{+\s{subj} To, so as to, in order to, so that. *Commonly enclitic* ***’b’h*** *after vowels*}{}
\refentry{b’heḍ}{eḍ}
\refentry{b’heḍy’é}{eḍ}
\entry{b’hóy’ẹ}{v.}{\pfabbr voler}{To fly. Flight}{}
\refentry{b’hu}{eḍ}
\entry{Cár}{n.}{}{*male given name, equivalent to English ‘Kyle’ or ‘Charles’*}{}
\entry{c’hánár}{n.}{\pfabbr canard}{Ship, boat}{}
\entry{c’háraúciḍ}{v.}{\pfabbr les carrotes sont cuites}{To end for good, put to a permanent end }{\s{fut} *c’háraúcre*, \s{subj} *c’háraúc*}
\entry{c’haúr}{conj.}{\pfabbr car + comme}{+\s{subj} As, because, since}{}
\entry{c’hd’hal}{adv.}{\pfabbr que dalle}{Naught, absolutely nothing}{}
\entry{c’hes}{part.}{\pfabbr qu'est-ce que}{*interrogative particle*}{}
\entry{c’hóný}{adj.}{\pfabbr connu}{Known, familiar, well-known}{}
\entry{c’hór}{n.}{\pfabbr corps}{Body}{}
\entry{c’húr}{v.}{\pfabbr court}{To shrink, reduce in size, narrow}{}
\entry{c’hýr}{n.}{\pfabbr corps}{Heart}{}
\entry{ḍá}{conj.}{\pfabbr tandis}{Whereas}{}
\entry{ḍalẹ}{n.}{\pfabbr tableau}{Table}{}
\entry{daúb’hedwébhó}{v.}{\pfabbr tomber dans les pommes}{To faint}{\s{fut} *daúb’hedwébhóre*, \s{subj} *daúb’hedwébhós*}
\entry{daúc’h}{conj.}{\pfabbr donc}{+\s{subj} So, therefore, thus}{}
\entry{Daúvníc’h}{n.}{}{*male or female given name, equivalent to English ‘Dominic’*}{}
\entry{ḍẹ}{adj.}{\pfabbr tout}{All, every, whole, entire}{}
\entry{de}{conj.}{\pfabbr dès que}{+\s{subj} Once, when once, as soon as}{}
\refentry{ḍe}{eḍ}
\entry{dẹhẹ}{n.}{\pfabbr dessus}{1. Top, upper side. 2. Surface of a body of water}{}
\entry{ḍèr}{v.}{\pfabbr taire}{To silence, shut up}{\s{fut} *ḍérẹ́*}
\refentry{ḍet’h}{eḍ}
\entry{dír}{v. tr.}{\pfabbr dire}{To say, tell}{\s{fut} *dírẹ́*}
\refentry{ḍyf}{eḍ}
\entry{ebhẹ}{v.}{\pfabbr épais}{To be thick}{\s{fut} ebhrẹ, \s{subj} ebhes}
\entry{eċ}{n.}{\pfabbr péché}{Sin, transgression, wrongdoing}{}
\refentry{éḍ}{eḍ}
\entry{eḍrrá}{v.}{\pfabbr étroit}{Pointy}{}
\entry{eḍ}{v. irreg. }{\pfabbr être}{To be}{*active only*. \s{\textbf{pres:} sg} *vy’í*, *ḍe*, *le*, *lle*, *s*; \s{pl} *aúsó*, *b’heḍ*, *lẹsó*, *llẹsó*, *lasó*; \s{inf} *éḍ*. \s{\textbf{pres ant:} sg} *vẹ*, *ḍyf*, *leb’h*, *lleb’h*, *seb’h*; \s{pl} *aúfý*, *b’hu*, *lẹfýr*, *llẹfýr*, *lafýr*; \s{inf} *éfyḍ*. \s{\textbf{pret:} sg} *vet’h*, *ḍet’h*, *let’h*, *llet’h*, *set’h*; \s{pl} *weḍy’ó*, *b’heḍy’é*, *let’he*, *llet’he*, *laet’h*; \s{inf} *ét’hẹd*}
\entry{Eḍy’ê}{n.}{}{*male given name, equivalent to English ‘Ste\-phen’*}{}
\refentry{éfyḍ}{eḍ}
\entry{ehyó}{n.}{\pfabbr écusson}{Shield}{}
\entry{el}{n.}{\pfabbr ailles}{Wing, blade, fin.}{}
\entry{e}{n.}{\pfabbr eau}{Water}{}
\refentry{ét’hẹd }{eḍ}
\entry{ez-}{pron.}{\pfabbr ses}{Its, her, his, their}{}
\entry{férḍufraú}{v.}{\pfabbr en faire tout un fromage}{To make a big fuss about something}{\s{fut} *férḍu\-fraúrẹ́*, \s{subj} *férḍufraús*}
\entry{férrrásvát’h}{n.}{\pfabbr fer la grasse matinée}{A long, deep sleep}{}
\refentry{ís }{ub’hrá}
\entry{Já}{n.}{}{*male or female given name, equivalent to English ‘John’ or ‘Joan’*}{}
\entry{jávé}{adv.}{\pfabbr jamais}{Never, at no time}{}
\entry{Lác}{n.}{}{*female given name, equivalent to English ‘Bi\-anca’*}{}
\refentry{laet’h}{eḍ}
\refentry{lafýr}{eḍ}
\entry{lârdávrá}{n.}{\pfabbr langue de bois}{Evasive, unclear, or overly formal speech}{}
\entry{lár}{v.}{\pfabbr large}{Wide, broad}{}
\refentry{lasó}{eḍ}
\entry{laúrs}{conj.}{\pfabbr lorsque}{When (temporal only)}{}
\entry{laú}{v.}{\pfabbr long}{Long}{}
\refentry{leb’h}{eḍ}
\refentry{le}{eḍ}
\refentry{lẹfýr}{eḍ}
\entry{lẹhuvud}{n.}{\pfabbr coup de foudre}{Love at first sight}{}
\entry{lẹ}{prefix}{\pfabbr plus}{*Denying comparative prefix. See grammar*}{}
\refentry{lẹsó}{eḍ}
\refentry{let’h}{eḍ}
\refentry{let’he}{eḍ}
\refentry{lleb’h}{eḍ}
\refentry{lle}{eḍ}
\refentry{llẹfýr}{eḍ}
\refentry{llẹsó}{eḍ}
\refentry{llet’h}{eḍ}
\refentry{llet’he}{eḍ}
\entry{lúr}{v.}{\pfabbr lourd}{To be bulky, oversized, heavy}{}
\entry{lýr}{pron.}{\pfabbr leur}{Their}{}
\entry{nájẹ}{v.}{\pfabbr nager}{To swim}{\s{fut} náȷ́ẹ, \s{subj} nájes}
\entry{rác’hsaý’ad}{v.}{\pfabbr raconter des salades}{To lie, tell tall tales, overexaggerate}{\s{fut} *rác’hsa\-ý’e*, \s{subj} *rác’hsaýs*}
\entry{râdrásôn}{v.}{\pfabbr prendre ses jambe à son cou}{To run}{\s{fut} *râdrásônre*, \s{subj} *râdrásôns*}
\entry{rádrénẹ́}{v. + \s{aci}}{\pfabbr les doigts dans le nez}{To put no effort into}{\s{fut} *rádrénrẹ́*, \s{subj} *rádrénẹ́s*}
\entry{râdvâ}{prefix}{\pfabbr grandement}{*Superlative prefix. See grammar*}{}
\entry{ráhẹ}{conj.}{\pfabbr quoique}{+\s{subj} Although, though}{}
\entry{ráhé}{n.}{{\nf from} ráhe + ráhó}{Flying fish}{}
\entry{ráhe}{n.}{\pfabbr oiseau}{Bird}{}
\entry{ráhó}{n.}{\pfabbr poisson}{Fish}{}
\entry{rá}{n.}{\pfabbr loi}{Law, rule, regulation}{}
\entry{raúl}{n.}{\pfabbr parole}{1. Language, speech, word 2. **Rául** *definite only* Short for *T’hebhaú Rául* *(\s{nom sg} irreg.* **Rául***; all other forms are regular)*}{}
\entry{ráv́â}{adv.}{\pfabbr rarement}{Seldom, rarely (ever)}{}
\entry{rá}{v.}{\pfabbr grand}{Big, large, great}{}
\entry{ráy’ê}{n.}{\pfabbr moyen}{1. Way, means, method. 2. *ráy’ê y’aúhý* + \s{aci} There is no way, that..}{}
\entry{ráy’é}{v.}{\pfabbr noyer}{To drown}{}
\entry{ráý’ẹ}{v.}{\pfabbr râler}{To complain, grumble}{}
\entry{rê}{conj.}{\pfabbr bien que}{+\s{subj} Although, though}{}
\entry{rê}{prefix}{\pfabbr moins}{*Neutral comparative prefix. See grammar*}{}
\entry{ré}{prefix}{\pfabbr très}{*Superlative prefix. See grammar*}{}
\entry{rrád’hahánár}{n.}{\pfabbr froid de canard}{Coldness}{}
\entry{rvá}{interj.}{of unknown origin}{Alas, woe, oh. *Exclamation of distress, surprise, sadness, or regret*}{ *after words that end with ‘r’, this is spelt ‘-vá’ instead*}
\entry{sá}{conj.}{\pfabbr sans que}{+\s{subj} Without (doing sth.)}{}
\entry{sá}{particle}{\pfabbr sans}{Not, no. *Commonly s’\N before words starting with a variant of ‘a’ or ‘e’. This particle is used only in the subjunctive; see also* ***asý’ýâ***}{}
\entry{sauc’h}{conj.}{\pfabbr sauf que}{+\s{subj} Except that}{}
\entry{s}{conj.}{\pfabbr si}{If, when, whenever}{}
\refentry{seb’h}{eḍ}
\refentry{s}{eḍ}
\refentry{set’h}{eḍ}
\entry{sol}{n.}{\pfabbr sol}{Ground, floor, earth, soil}{}
\refentry{’sý’ýâ }{asý’ýâ}
\entry{T’hebhaú}{n. or adj.}{{\nf from} t’hebhaúz}{France, French}{}
\entry{T’hebhaú Rául}{n. def. sg. (only {\nf T’hebhaú} is declined as though the entire phrase were one word) }{{\nf from} t’hebhaúz + rául}{The Ultrafrench language *(in informal speech and writing, this is typically shortened to* **Raúl***)*}{\s{nom sg} *T’hebhaú Rául* *(regular in all other cases)*}
\entry{t’hebhaúz}{v.}{\pfabbr jeter l’éponge}{To be French}{ \s{fut} *t’hebhaúźe*, \s{subj} *t’hebhaúś*}
\entry{ub’hrá}{v.}{\pfabbr pouvoir}{1. To be able to, can. 2. +\s{part} Capable of ..}{\s{fut} *úrẹ́*, \s{subj} *ís*}
\entry{urdálbhaúrḍ}{n.}{\pfabbr avoir un oursin dans le portefeuille}{A very rich person; billionaire}{}
\refentry{úrẹ́}{ub’hrá}
\entry{uy’ed’háb’hrí}{v.}{\pfabbr rouler dans la farine}{To scam, swindle, cheat}{\s{fut} *uy’e\-d’háv́e*, \s{subj} *uy’ed’háb’hrís*}
\entry{vádłabhaud’hávúrsab’hád’háváb’hrárḍuẹ}{v. literary}{\pfabbr vendre la peau de ours avant de avoir tué}{ To depend on predictions of the future *(of disputed origin; first attested in the works of the Early UF comedian J. A. B. Smyth)*}{ \s{fut} *vádłabhaud’hávúrsab’hád’háváb’hrárḍure*, \s{s} *vádłabhaud’hávúrsab’hád’háváb’hrárḍus*}
\entry{váłé}{conj.}{\pfabbr malgré que}{+\s{subj} Despite that, in spite of}{}
\refentry{vá }{rvá}
\entry{vaûd}{n.}{\pfabbr monde}{World}{}
\entry{vaût’há}{n.}{\pfabbr montagne}{Mountain}{}
\refentry{vẹ}{eḍ}
\refentry{vet’h}{eḍ}
\entry{vvaúríhe}{v. (in)tr.}{\pfabbr mémoriser}{To remember}{\s{fut} *vvaúríźe*}
\refentry{vy’í}{eḍ}
\refentry{weḍy’ó}{eḍ}
\entry{y’aúhý}{inconj., postpos.}{\pfabbr il n’y a aucun}{There is no, there are no, there is none}{}
\entry{ý’aúhý}{inconj., postpos.}{\pfabbr il y a aucun}{There is, there are}{}
\entry{y’é}{adv.}{\pfabbr rien}{Nothing}{}
\entry{y’ír}{v. (in)tr.}{\pfabbr ouïr}{To hear, understand, listen}{}
\entry{y’ís}{conj.}{\pfabbr puisque}{Considering that, since, because *(unlike* **c’haúr***, this does not take the subjunctive; it is used to indicate the (potential) cause of something)*}{}
\entry{y’ŷ}{n.}{{\nf from} y’ŷvéłáfrí}{Eye}{}
\entry{y’ŷ}{prefix}{\pfabbr mieux}{*Affirming comparative prefix. See grammar*}{}
\entry{y’ŷvéłáfrí}{n. pl. archaic}{yeux de merlan frit}{Eyes}{}
