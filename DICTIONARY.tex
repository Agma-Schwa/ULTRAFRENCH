\entry{a}{pron. indef.}{\pfabbr quoi}{What, which, that *(relative pronoun)*}{}{}
\entry{á}{conj.}{\pfabbr tandis}{Whereas}{}{}
\entry{ab’há}{conj.}{\pfabbr avant que}{+\s{opt} Before}{}{}
\entry{ab’haḍ}{v.}{\pfabbr abattre}{1. To cut down, fell, knock down, shoot down. 2. To butcher, cut apart violently}{\s{fut} *ab’haḍrẹ́*, \s{subj} *ab’has*}{}
\entry{ab’hásy’ô}{n.}{\pfabbr aviation}{Aviation}{}{}
\entry{ab’hèc’h}{v.}{\pfabbr affecter}{+\s{acc} To affect, influence}{\s{fut} *ab’hèc’hre*, \s{subj} *ab’hèc’hes*}{}
\entry{ab’héy’}{n.}{\pfabbr abeille}{Bee}{}{}
\entry{ab’hínéb’heḅaý’évrâ}{v.}{\pfabbr habit ne fait pas le moine}{To judge based on appearances}{\s{fut} *ab’hínéb’heḅaý’év́ẹ́*, \s{subj} *ab’hínéb’heḅaý’év́ás*}{}
\entry{ac}{n.}{\pfabbr hache}{Axe, hatchet}{}{}
\entry{act’he}{v. tr.}{\nf from *ac*}{1. To cut or cleave with an axe. 2. +\s{acc} To bring an end to. 3. +\s{acc def} *of* ***árb*** *intr. (other than literal)* To get to the point, cut to the chase. 4. +\s{acc def} *of* ***árb*** *and* \s{acc} To bring to light, reveal *(originally, this idiom did not take a double \s{acc}, but was instead formed with the \s{acc} of ‘tree’ and the \s{ill} of the object, meaning something along the lines of ‘to bring down the tree(s) on sth’—the image here being that of cutting down trees in a wood until only a clearing remains or is ‘brought to light’)*}{\s{fut} *acḍe*, \s{subj} *act’hes*}{}
\entry{aċt’he}{v. tr.}{\pfabbr acheter}{To buy}{\s{fut} *aċḍrẹ́*, \s{subj} *aċt’hes*}{}
\entry{ad’he}{v.}{\pfabbr vader}{To go}{\s{fut} *ad’hrẹ́*, \s{subj} *ad’hes*}{}
\entry{ad’hór}{v. tr.}{\pfabbr adore}{1. To love, adore. 2. +\s{part} To be in love with, have a crush on. 3. +\s{gen} To desire (someone)}{\s{fut} *ad’hórérẹ́*, \s{subj} *ad’hórs*}{}
\entry{áhaúr}{conj.}{\pfabbr encore}{+\s{subj} Even though}{}{}
\entry{ânb’hé}{adv.}{\pfabbr en effet\nf, via metathesis from \**âné\-b’he*}{Verily, indeed, in fact}{}{}
\entry{ánvé}{v. tr.}{\pfabbr animer}{To bring to life, animate}{}{}
\entry{árb}{n.}{\pfabbr arbre}{Tree}{}{}
\entry{árḍihyl}{n.}{\pfabbr particule}{Particle}{}{}
\entry{asý’ýâ}{particle}{\pfabbr pas absolument}{Not, no. *Commonly* ***’sý’ýâ*** *after vowels and verbs This particle is used only in the indicative; see also* ***sá****, ****t’hé***}{}{}
\entry{au}{conj.}{\pfabbr aussi}{1. And, also, as well, too 2. *au* ... *au* ... ‘both ... and ...’}{}{}
\entry{aû}{particle}{\pfabbr non}{Not- *(used to negate nouns)*}{}{}
\entry{aub’heír}{v. (in)tr.}{\pfabbr obéir}{To obey}{}{}
\entry{aud}{adj.}{\pfabbr autre}{Other, another}{}{}
\entry{auḍé}{v.}{\pfabbr obtenir}{1. To obtain, get, acquire. 2. +\s{abl} To gain purchase on or height or distance from}{\s{fut} *auḍy’édrẹ́*}{}
\refentry{aúfý}{eḍ}
\entry{auha}{conj.}{\pfabbr au cas où}{+\s{opt} In case}{}{}
\refentry{aúsó}{eḍ}
\entry{áv́áy’é}{v.}{\pfabbr envoyer}{To send}{\s{fut} *áv́áy’érẹ́*, \s{subj} *áv́áy’és*}{}
\entry{ávrê}{conj.}{\pfabbr à moins que}{+\s{opt} Unless}{}{}
\entry{aý’aúr}{conj.}{\pfabbr alors}{While, as (temporal)}{}{}
\entry{áȷ́éd}{v.}{\pfabbr enjoindre}{To order, enjoin, command}{}{}
\entry{ḅáhẹ}{n.}{\pfabbr pensée}{Thought, reflection, meditation, faculty of thinking}{}{}
\entry{ḅarḍ}{v.}{\pfabbr partir}{To leave, go away, depart}{\s{fut} *ḅarẹ́*, \s{subj} *ḅars*}{}
\entry{ḅárḍáḍ}{v.}{\pfabbr partante}{(+ \s{aci}) To be interested in, willing to, ready to, prepared for}{}{}
\entry{ḅárḍẹ}{n.}{\pfabbr partie}{Part, portion, piece, faction of a whole}{}{}
\entry{ḅas}{conj.}{parce que}{+\s{subj} Because *(often used to explain motivation rather than cause as in ‘We did that because\ldots’)*}{}{}
\entry{baú}{v. irreg.}{\pfabbr bon}{1. To be good, well. 2. To be right, correct, appropriate}{\s{fut} *baúrẹ́*, \s{subj} *véy’ýrs*; \s{comp} *lẹvéy’ýr*, *y’ŷvéy’ýr*, *rêvéy’ýr*; \s{sup} *révéy’ýr*, *râdvâvéy’ýr*}{}
\entry{ḅaú}{n.}{\pfabbr pont}{Bridge}{}{}
\entry{ḅauheŷnlabhé}{v.}{\pfabbr poser un lapin}{To forsake, abandon}{\s{fut} *ḅauheŷnlabhére*, \s{subj} *ḅauheŷnlabhés*}{}
\entry{ḅauhib}{v.}{\pfabbr impossible}{To be impossible, unfeasible}{\s{fut} *ḅauhibre*, \s{subj} *ḅauhibes*}{}
\entry{ḅáł}{v.}{\pfabbr parler}{To speak, talk, say}{\s{fut} *báłérẹ́*}{}
\entry{ḅáłýr}{n.}{\pfabbr parleur}{Speaker, interlocutor}{}{}
\entry{ḅẹt’hẹ}{v. irreg.}{\pfabbr petit}{To be small, little}{\s{fut} *rêdẹ́*, \s{subj} *ḅẹt’hes*; \s{comp} *lẹrêd*, *y’ŷrêd*, *rêrêd*; \s{sup} *rérêd*, *râdvârêd*}{}
\refentry{bír}{vaúb’hẹ}
\entry{ḅré}{conj.}{\pfabbr après que}{+\s{opt} After}{}{}
\entry{b’he}{conj.}{\pfabbr envers}{+\s{subj} To, so as to, in order to, so that. *Commonly enclitic* ***’b’h*** *after vowels*}{}{}
\refentry{b’heḍ}{eḍ}
\refentry{b’heḍy’é}{eḍ}
\entry{b’hóy’ẹ}{v.}{\pfabbr voler}{To fly. Flight}{}{}
\refentry{b’hu}{eḍ}
\entry{b’hýlnẹ́r}{v.}{\pfabbr invulnérable}{+\s{instr} To be incapable of being affected by, invulnerable to}{\s{fut} *b’hýlnẹ́rẹ́*, \s{subj} *b’hýlnẹ́rs*}{}
\entry{cahý}{pro. \s{pl indef} only; declined like a regular noun}{\pfabbr chacun}{Each other, one another}{}{}
\entry{Cár}{n.}{}{*male given name, equivalent to English ‘Kyle’ or ‘Charles’*}{}{}
\entry{c’habhahit’hẹ}{n.}{\pfabbr capacité}{Skill, capacity, ability}{}{}
\entry{c’hánár}{n.}{\pfabbr canard}{Ship, boat}{}{}
\entry{c’háraúciḍ}{v.}{\pfabbr les carrotes sont cuites}{To end for good, put to a permanent end}{\s{fut} *c’háraúcre*, \s{subj} *c’háraúc*}{}
\entry{c’haúbhausy’ô}{n.}{\pfabbr composition}{Composition, arrangement, structure}{}{}
\entry{c’haúḅrâd}{v.}{+\s{part} To comprehend, understand, grasp}{\s{fut} *c’haúḅrâdrẹ́*, \s{subj} *c’haúḅrâs*}{}{}
\entry{c’haúr}{conj.}{\pfabbr car + comme}{+\s{subj} As, because, since}{}{}
\entry{c’hd’hal}{adv.}{\pfabbr que dalle}{Naught, absolutely nothing}{}{}
\entry{C’hebèc’h}{n.}{\pfabbr Québec}{The Promised Land}{}{}
\entry{c’hes}{part.}{\pfabbr qu'est-ce que}{*interrogative particle*}{}{}
\entry{c’hesse}{}{}{*contraction of* **c’hes** *+* **se** Is it? *(Also substituted for other forms of to be in questions, particularly for the plural neuter)*}{}{}
\entry{c’hóný}{adj.}{\pfabbr connu}{Known, familiar, well-known}{}{}
\entry{c’hór}{n.}{\pfabbr corps}{Body}{}{}
\entry{c’húr}{v.}{\pfabbr court}{To shrink, reduce in size, narrow}{}{}
\entry{c’hýr}{n.}{\pfabbr corps}{Heart}{}{}
\entry{ḍalẹ}{n.}{\pfabbr tableau}{Table}{}{}
\entry{daúb’hedwébhó}{v.}{\pfabbr tomber dans les pommes}{To faint}{\s{fut} *daúb’hedwébhóre*, \s{subj} *daúb’hedwébhós*}{}
\entry{daúc’h}{conj.}{\pfabbr donc}{+\s{subj} So, therefore, thus}{}{}
\entry{Daúvníc’h}{n.}{}{*male or female given name, equivalent to English ‘Dominic’*}{}{}
\entry{de}{conj.}{\pfabbr dès que}{+\s{subj} Once, when once, as soon as}{}{}
\refentry{ḍe}{eḍ}
\entry{dẹhẹ}{n.}{\pfabbr dessus}{1. Top, upper side. 2. Surface of a body of water}{}{}
\entry{deslẹ}{v.}{\pfabbr déceler}{To detect, discover, uncover, reveal}{\s{fut} *deslẹre*, \s{subj} *deslẹs*}{}
\entry{dír}{v. tr.}{\pfabbr dire}{To say, tell}{\s{fut} *dírẹ́*, \s{subj} *díss*}{}
\entry{dónẹ́}{v.}{\pfabbr donner}{\s{+ dat \& acc/part} To endow, bestow *(the \s{acc} is used when talking about concrete, measurable, and finite objects or sums; the partitive to talk about abstract concepts or parts of objects)*}{\s{fut} *dónrẹ́*, \s{subj} *dónés*}{}
\entry{e}{n.}{\pfabbr eau}{Water}{}{}
\entry{ẹ}{adj.}{\pfabbr tout}{All, every, whole, entire}{}{}
\entry{ebhẹ}{v.}{\pfabbr épais}{To be thick}{\s{fut} ebhrẹ, \s{subj} ebhes}{}
\entry{eċ}{n.}{\pfabbr péché}{Sin, transgression, wrongdoing}{}{}
\entry{eḍ}{v. irreg.}{\pfabbr être}{To be}{*active only*. \s{\textbf{pres:} sg} *vy’í*, *ḍe*, *le*, *lle*, *se*; \s{pl} *aúsó*, *b’heḍ*, *lẹsó*, *llẹsó*, *lasó*; \s{inf} *éḍ*.           \s{\textbf{pres ant:} sg} *vẹ*, *ḍyf*, *leb’h*, *lleb’h*, *seb’h*; \s{pl} *aúfý*, *b’hu*, *lẹfýr*, *llẹfýr*, *lafýr*; \s{inf} *éfyḍ*.        \s{\textbf{pret:} sg} *vet’h*, *ḍet’h*, *let’h*, *llet’h*, *set’h*; \s{pl} *weḍy’ó*, *b’heḍy’é*, *let’he*, *llet’he*, *laet’h*; \s{inf} *ét’hẹd*}{}
\refentry{éḍ}{eḍ}
\entry{eḍrrá}{v.}{\pfabbr étroit}{Pointy}{}{}
\entry{Eḍy’ê}{n.}{}{*male given name, equivalent to English ‘Ste\-phen’*}{}{}
\refentry{éfyḍ}{eḍ}
\entry{ehyó}{n.}{\pfabbr écusson}{Shield}{}{}
\entry{el}{n.}{\pfabbr ailles}{Wing, blade, fin.}{}{}
\entry{èr}{v.}{\pfabbr taire}{To silence, shut up}{\s{fut} *ḍérẹ́*}{}
\refentry{et’h}{eḍ}
\refentry{ét’hẹd}{eḍ}
\entry{eý’ebhat’hẹ}{n.}{\pfabbr télépathie}{Telepathy}{}{}
\entry{eý’ebhat’hic’h}{v.}{\pfabbr télépathique}{To be telepathic}{\s{fut} *ḍeý’ebhat’hic’hre*, \s{subj} *ḍeý’ebhat’hic’hes*}{}
\entry{ez-}{pron.}{\pfabbr ses}{Its, her, his, their}{}{}
\entry{F}{adj.}{{\nf from} fẹ}{*(logic)* False, $\bot$}{}{}
\entry{fahaú}{conj.}{\pfabbr de façon que}{+\s{opt} In such a way that}{}{}
\entry{faúr}{n.}{\pfabbr force}{1. Force, strength, power. 2. \s{def} *(science fiction)* The Force}{}{}
\entry{fẹ}{v.}{\pfabbr faux}{To be false, incorrect, wrong}{\s{fut} *faure*, \s{subj} *faus*}{}
\entry{fẹhab}{v.}{\pfabbr faisable}{To be possible, feasible}{\s{fut} *fẹhabre*, \s{subj} *fẹhas*}{}
\entry{fèhẹ}{n.}{\pfabbr faisceau}{1. Bundle, bunch, cluster. 2. Beam, ray}{}{}
\entry{fér}{v.}{\pfabbr faire}{To do, make, build, construct, erect}{\s{fut} *fẹ́*, \s{subj} *fés*}{}
\entry{férḍufraú}{v.}{\pfabbr en faire tout un fromage}{To make a big fuss about something}{\s{fut} *férḍu\-fraúrẹ́*, \s{subj} *férḍufraús*}{}
\entry{férrrásvát’h}{n.}{\pfabbr fer la grasse matinée}{A long, deep sleep}{}{}
\refentry{ís}{ub’hrá}
\entry{Já}{n.}{}{*male or female given name, equivalent to English ‘John’ or ‘Joan’*}{}{}
\entry{jávé}{adv.}{\pfabbr jamais}{Never, at no time}{}{}
\entry{jys}{conj.}{\pfabbr jusqu'à ce que}{+\s{opt} Until}{}{}
\entry{jys}{adv.}{\pfabbr juste}{Just, only, merely}{}{}
\entry{Lác}{n.}{}{*female given name, equivalent to English ‘Bi\-anca’*}{}{}
\refentry{laet’h}{eḍ}
\refentry{lafýr}{eḍ}
\entry{lár}{v.}{\pfabbr large}{Wide, broad}{}{}
\entry{lârdávrá}{n.}{\pfabbr langue de bois}{Evasive, unclear, or overly formal speech}{}{}
\refentry{lasó}{eḍ}
\entry{laú}{v.}{\pfabbr long}{Long *(often in compounds* **laú-** *‘long-’)*}{}{}
\entry{laúrs}{conj.}{\pfabbr lorsque}{When (temporal only)}{}{}
\entry{laut’h}{v.}{\pfabbr flotter}{Float, hover, levitate}{\s{fut} *laut’hre*, \s{subj} *laut’hes*}{}
\entry{le}{v.}{\pfabbr laisser {\nf > \**lehe*}}{*(chiefly in questions or imperative)* To let, allow, permit}{\s{fut} *lere*, \s{subj} *les*}{}
\refentry{le}{eḍ}
\entry{lẹ}{prefix}{\pfabbr plus}{*Denying comparative prefix. See grammar*}{}{}
\refentry{leb’h}{eḍ}
\entry{lec’hḍraúvnẹ́t’hic’h}{v.}{\pfabbr électromagnétique}{To be electromagnetic}{\s{fut} *lec’hḍraúvnẹ́t’hic’hre*, \s{subj} *lec’hḍraúvnẹ́t’hic’hes*}{}
\refentry{lẹfýr}{eḍ}
\entry{lẹhuvud}{n.}{\pfabbr coup de foudre}{Love at first sight}{}{}
\refentry{lẹsó}{eḍ}
\refentry{let’h}{eḍ}
\refentry{let’he}{eḍ}
\entry{lí}{v.}{\pfabbr lire}{1. +\s{part} To read from. 2. +\s{acc} To peruse, read entirely}{\s{fut} *lírẹ́*, \s{subj} *lís*}{}
\entry{liv́uhé}{n.}{\pfabbr livre + bouquin}{Book}{}{}
\refentry{lle}{eḍ}
\refentry{lleb’h}{eḍ}
\refentry{llẹfýr}{eḍ}
\refentry{llẹsó}{eḍ}
\refentry{llet’h}{eḍ}
\refentry{llet’he}{eḍ}
\entry{lúr}{v.}{\pfabbr lourd}{To be bulky, oversized, heavy}{}{}
\entry{lýr}{pron.}{\pfabbr leur}{Their}{}{}
\entry{lývy’ér}{n.}{\pfabbr lumière}{Light}{}{}
\entry{nájẹ}{v.}{\pfabbr nager}{To swim}{\s{fut} náȷ́ẹ, \s{subj} nájes}{}
\entry{nérjẹ}{n.}{\pfabbr énergie}{Energy}{}{}
\entry{R}{adj.}{{\nf from} ré}{*(logic)* True, $\top$}{}{}
\entry{ra}{conj.}{\pfabbr swa (> \**rá*)}{1. Or *(exclusive, see also*~**u***)*. 2. **u**/**ra** \ldots\ **ra** \ldots\ ‘either \ldots\ or \ldots’ *(exclusive)*}{}{}
\entry{rá}{v.}{\pfabbr grand}{Big, large, great}{}{}
\entry{rá}{n.}{\pfabbr loi}{Law, rule, regulation}{}{}
\entry{rác’hsaý’ad}{v.}{\pfabbr raconter des salades}{To lie, tell tall tales, overexaggerate}{\s{fut} *rác’hsa\-ý’e*, \s{subj} *rác’hsaýs*}{}
\entry{râdrásôn}{v.}{\pfabbr prendre ses jambe à son cou}{To run}{\s{fut} *râdrásônre*, \s{subj} *râdrásôns*}{}
\entry{rádrénẹ́}{v. + \s{aci}}{\pfabbr les doigts dans le nez}{To put no effort into}{\s{fut} *rádrénrẹ́*, \s{subj} *rádrénẹ́s*}{}
\entry{râdvâ}{prefix}{\pfabbr grandement}{*Superlative prefix. See grammar*}{}{}
\entry{ráhe}{n.}{\pfabbr oiseau}{Bird}{}{}
\entry{ráhé}{n.}{{\nf from} ráhe + ráhó}{Flying fish}{}{}
\entry{ráhẹ}{conj.}{\pfabbr quoique}{+\s{subj} Although, though}{}{}
\entry{râhẹ}{n.}{\pfabbr Français}{Human, person}{}{}
\entry{ráhis}{v.}{\pfabbr raciste}{To be racist}{\s{fut} *ráhise*, \s{subj} *ráhiss*}{}
\entry{ráhó}{n.}{\pfabbr poisson}{Fish}{}{}
\entry{rár}{v.}{\pfabbr voir}{To see}{\s{fut} *b’hérẹ́*, \s{subj} *rárs*}{}
\entry{rát’hẹ}{particle}{\pfabbr vois-tu}{You see, you know}{}{}
\entry{raúb’hẹ}{n.}{\pfabbr robot}{Robot}{}{}
\entry{raúl}{n.}{\pfabbr parole}{1. Language, speech, word 2. **Raúl** *definite only* Short for *T’hebhaú Raúl* *(\s{nom sg} irreg.* **Raúl***; all other forms are regular)*}{}{}
\entry{ráv́â}{adv.}{\pfabbr rarement}{Seldom, rarely (ever)}{}{}
\entry{ráy’é}{v.}{\pfabbr noyer}{To drown}{}{}
\entry{ráy’ê}{n.}{\pfabbr moyen}{1. Way, means, method. 2. *ráy’ê y’aúhý* + \s{aci} There is no way, that..}{}{}
\entry{ráý’ẹ}{v.}{\pfabbr râler}{To complain, grumble}{}{}
\entry{ré}{prefix}{\pfabbr très}{*Superlative prefix. See grammar*}{}{}
\entry{ré}{n.}{\pfabbr rai}{Ray, beam}{}{}
\entry{ré}{v.}{\pfabbr vrai}{To be true, correct, right}{\s{fut} *rẹ́*, \s{subj} *rés*}{}
\entry{rê}{prefix}{\pfabbr moins}{*Neutral comparative prefix. See grammar*}{}{}
\entry{rê}{conj.}{\pfabbr bien que}{+\s{subj} Although, though}{}{}
\refentry{rêd}{ḅẹt’hẹ}
\entry{rêd}{v.}{\pfabbr craindre}{+s{opt} To fear, lest \ldots *(construed with the negated optative)*}{\s{fut} *rêdrẹ́*, \s{subj} *rês*}{}
\entry{rêd’hes}{particle}{\pfabbr bien sûr}{Of course, certainly, surely}{}{}
\entry{rét’hád}{v.}{\pfabbr prétendre}{To claim, allege}{\s{fut} *rét’hádrẹ́*, \s{subj} *rét’hádes*}{}
\entry{rét’hẹ}{v.}{\pfabbr traiter}{To handle, take care of, deal with}{\s{fut} rét’hẹre, \s{subj} rét’hes}{}
\entry{rrá}{v.}{\pfabbr croire}{Believe (something or someone)}{\s{fut} *rrẹ́*, \s{subj} *rrás*}{}
\entry{rrád’hahánár}{n.}{\pfabbr froid de canard}{Coldness}{}{}
\entry{rvá}{interj.}{of unknown origin}{Alas, woe, oh. *Exclamation of distress, surprise, sadness, or regret*}{*after words that end with ‘r’, this is spelt ‘-vá’ instead*}{}
\entry{s}{conj.}{\pfabbr si}{If, when, whenever}{}{}
\entry{sá}{conj.}{\pfabbr sans que}{+\s{subj} Without (doing sth.)}{}{}
\entry{sá}{particle}{\pfabbr sans}{Not, no. *Always s’ before vowels. This particle is used only in the subjunctive; see also* ***asý’ýâ****, ****t’hé***}{}{}
\entry{sáhẹ}{v.}{\pfabbr insensé}{To be preposterous, absurd, nonsensical}{\s{fut} *sáhere*, \s{subj} *sáhes*}{}
\entry{Sásc’hríḍ}{n.}{\pfabbr Sanskrit}{The Sanskrit language}{}{}
\entry{sásy’él}{v.}{\pfabbr essentiel}{To be essential}{\s{fut} *sásy’élẹ́*, \s{subj} *sásy’éls*}{}
\entry{sauc’h}{conj.}{\pfabbr sauf que}{+\s{subj} Except that}{}{}
\entry{saúr}{n.}{\pfabbr sorte}{1. Kind, sort, type, form 2. \s{def + gen} (some) kind(s) of}{}{}
\entry{sauz}{n.}{\pfabbr chose}{Thing, object}{}{}
\entry{sauz-aud}{adj.}{\pfabbr autre chose}{Something else, another thing}{}{}
\refentry{sauzaud}{sauz-aud}
\entry{sav́á}{v.}{\pfabbr savoir}{To know}{\s{fut} *saúr*, \s{subj} *sac*}{}
\entry{sḅé}{v.}{\pfabbr espérer}{+\s{opt} To wish, want, desire}{\s{fut} *sḅérẹ́*, \s{subj} *sḅés*}{}
\refentry{se}{eḍ}
\refentry{seb’h}{eḍ}
\entry{sẹh}{det.}{\pfabbr ceci}{+\s{def} *noun* This, these *(precedes and is attached to nouns)*}{}{}
\entry{sérḍé}{det.}{\pfabbr certain}{Certain, particular but not specified}{}{}
\refentry{set’h}{eḍ}
\entry{sisḍé}{n.}{\pfabbr système}{System}{}{}
\entry{sit’há}{conj.}{\pfabbr si tant est que}{+\s{opt} Supposing that; if, assuming that}{}{}
\entry{sol}{n.}{\pfabbr sol}{Ground, floor, earth, soil}{}{}
\entry{susy’é}{v.}{\pfabbr soucier}{+\s{part, pci} To care about, worry about}{\s{fut} susy’ére, \s{subj} susy’és}{}
\entry{swi}{det.}{\pfabbr celui}{The one, that one, this one}{}{}
\entry{sý’ẹ}{det.}{\pfabbr cela}{+\s{def} *noun* That, those *(precedes and is attached to nouns; generally* **sý’** *before vowels, with one apostrophe, not two)*}{}{}
\refentry{s’}{sá}
\entry{t’hé}{conj.}{\pfabbr de peur que {\nf > \**dbhýrc’h* > \**dýrc’h* > \**dc’hý* > \**t’hé*}}{Not, no. *Always t’h’\N before vowels. This particle is used only in the optative; see also* ***asý’ýâ****, ****sá***}{}{}
\entry{T’hebhaú}{n. or adj.}{{\nf from} t’hebhaúz}{France, French}{}{}
\entry{T’hebhaú Raúl}{n. def. sg. (only {\nf T’hebhaú} is declined as though the entire phrase were one word)}{{\nf from} t’hebhaúz + raúl}{The Ultrafrench language *(in informal speech and writing, this is typically shortened to* **Raúl***)*}{\s{nom sg} *T’hebhaú Raúl* *(regular in all other cases)*}{}
\entry{t’hebhaúz}{v.}{\pfabbr jeter l’éponge}{To be French}{\s{fut} *t’hebhaúźe*, \s{subj} *t’hebhaúś*}{}
\entry{t’hiy’e}{v.}{{\nf from} yt’hiy’ihẹ}{+\s{part} To use, make use of}{\s{fut} *t’hiźe*, \s{subj} *(via back-formation from the \s{fut})* \s{t’hizes}}{}
\entry{u}{conj.}{\pfabbr ou}{1. Or *(inclusive, see also* **ra***)*. 2. **u** \ldots\ **u** \ldots\ ‘\ldots\ or \ldots’ *(inclusive)*}{}{}
\entry{ub’hrá}{v.}{\pfabbr pouvoir}{1. +\s{inf} To be able to, can. 2. +\s{part} Capable of .. *(***ub’hrá** *is never construed with an \s{inf} if it in and of itself is the infinitive of an \s{aci} or \s{pci}, in which case the variant with the \s{part} is used instead)*}{\s{fut} *úrẹ́*, \s{subj} *ís*}{}
\entry{ulíy’ẹ́}{v.}{\pfabbr oublier}{To forget}{\s{fut} *ulíy’ẹ́rẹ́*, \s{subj} *ulíy’ẹ́s*}{}
\entry{úr}{adv.}{\pfabbr toujours}{1. *(in positive context)* Always. 2. *(in negative context)* Still}{}{}
\entry{úrbh}{conj.}{\pfabbr pour peu que}{+\s{opt} Provided that, so long as}{}{}
\entry{urdálbhaúrḍ}{n.}{\pfabbr avoir un oursin dans le portefeuille}{A very rich person; billionaire}{}{}
\refentry{úrẹ́}{ub’hrá}
\entry{uy’ed’háb’hrí}{v.}{\pfabbr rouler dans la farine}{To scam, swindle, cheat}{\s{fut} *uy’e\-d’háv́e*, \s{subj} *uy’ed’háb’hrís*}{}
\refentry{vá}{rvá}
\entry{vádłabhaud’hávúrsab’hád’háváb’hrárḍuẹ}{v. literary}{\pfabbr vendre la peau de ours avant de avoir tué}{To depend on predictions of the future *(of disputed origin; first attested in the works of the Early UF comedian J. A. B. Smyth)*}{\s{fut} *vádłabhaud’hávúrsab’hád’háváb’hrárḍure*, \s{s} *vádłabhaud’hávúrsab’hád’háváb’hrárḍus*}{}
\entry{vâhẹ}{v.}{\pfabbr manquer}{1. +\s{gen} To lack, want. 2. +\s{part} *or* \s{pass} To miss *(the object and subject of this verb are swapped compared to English ‘to miss’, e.g.* **b’hývvâhé** *(\s{2pl.act} + \s{1sg.pass}) ‘I miss you (\s{pl})’, lit. roughly ‘you (\s{pl}) are wanting to me’)*. 3. +\s{acc} To miss out on}{\s{fut} *vâhérẹ́*, \s{subj} *vâhés*}{}
\entry{vaúb’hẹ}{v. irreg.}{\pfabbr mauvais}{1. To be bad 2. To be wrong, incorrect, inappropriate}{\s{fut} *bíré*, \s{subj} *bíres*; \s{comp} *lẹbír*, *y’ŷbír*, *rêbír*; \s{sup} *réb’hír*, *râdvâbír*}{}
\entry{vaûd}{n.}{\pfabbr monde}{World}{}{}
\entry{vaût’há}{n.}{\pfabbr montagne}{Mountain}{}{}
\entry{váłé}{conj.}{\pfabbr malgré que}{+\s{subj} Despite that, in spite of}{}{}
\entry{vé}{conj.}{\pfabbr mais}{But, however, although}{}{}
\refentry{vẹ}{eḍ}
\entry{véhýr}{conj.}{\pfabbr dans la mesure où}{Insofar as}{}{}
\entry{véhýr}{v/n.}{\pfabbr mesure}{1. To measure, 2. Measurement}{\s{fut} *véhýrẹ́*, \s{subj} *véhýrs*}{}
\refentry{vet’h}{eḍ}
\refentry{véy’ýr}{baú}
\entry{vísy’ô}{n.}{\pfabbr émission}{1. Emission. 2. Programme, broadcast, show}{}{}
\entry{vú}{adj.}{\pfabbr moult}{Many, much, a lot of}{}{}
\entry{vvaúríhe}{v. (in)tr.}{\pfabbr mémoriser}{To remember}{\s{fut} *vvaúríźe*}{}
\refentry{vy’í}{eḍ}
\refentry{weḍy’ó}{eḍ}
\refentry{yf}{eḍ}
\entry{yt’hiy’ihẹ}{v.}{\pfabbr utiliser}{+\s{part} *(archaic)* To use, make use of}{\s{fut} *yt’hiy’iźe*, \s{subj} *yt’hiy’ihẹs*}{}
\entry{y’aúhý}{inconj., postpos.}{\pfabbr il n’y a aucun}{There is no, there are no, there is none}{}{}
\entry{ý’aúhý}{inconj., postpos.}{\pfabbr il y a aucun}{There is, there are}{}{}
\entry{y’é}{adv.}{\pfabbr rien}{Nothing}{}{}
\entry{y’ẹ́}{v.}{\pfabbr nier}{To forbid, deny}{\s{fut} *y’ẹ́rẹ́*, \s{subj} *y’ẹ́s*}{}
\entry{y’ír}{v. (in)tr.}{\pfabbr ouïr}{To hear, understand, listen}{}{}
\entry{y’ís}{conj.}{\pfabbr puisque}{Considering that, since, because *(unlike* **c’haúr***, this does not take the subjunctive; it is used to indicate the (potential) cause of something)*}{}{}
\entry{y’ŷ}{n.}{{\nf from} y’ŷvéłáfrí}{Eye}{}{}
\entry{y’ŷ}{prefix}{\pfabbr mieux}{*Affirming comparative prefix. See grammar*}{}{}
\entry{y’ŷvéłáfrí}{n. pl. archaic}{yeux de merlan frit}{Eyes}{}{}
\refentry{’sý’ýâ}{asý’ýâ}
