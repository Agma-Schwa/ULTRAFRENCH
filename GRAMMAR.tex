\documentclass[a4paper, 12pt, oneside, final]{article}
\usepackage[margin=2cm]{geometry}
\usepackage{fontspec}
\usepackage{unicode-math}
\usepackage[english]{babel}
\usepackage{csquotes}
\usepackage{array, tabularx, multirow}
\usepackage{longtable}
\usepackage{float}
\usepackage{tabularray}
\usepackage{graphicx}
\usepackage{wasysym}
\usepackage{xcolor}

%\definecolor{bgcolor}{HTML}{2D2A2E}
%\definecolor{fgcolor}{HTML}{FAFCFC}
%\pagecolor{bgcolor}
%\color{fgcolor}

\setmainfont[Numbers=OldStyle]{Minion 3}
\setmathfont{latinmodern-math.otf}
\setmathfont[range=\mathit]{Minion 3 Italic}

%% TODO: Remove this and fix overfull boxes.
\hfuzz=10000pt
%\geometry{showframe}

%% Normalise to NFC.
\XeTeXinputnormalization=1

%% %%%%%%%%%%%%%%%%%%%%%%%%%%%%%%%%%%%%%%%%%%%%%%%%%%%%%%%%%%%%%%%%%%%%%%%%%%%%%
%%  Environment and Layout
%% %%%%%%%%%%%%%%%%%%%%%%%%%%%%%%%%%%%%%%%%%%%%%%%%%%%%%%%%%%%%%%%%%%%%%%%%%%%%%
\ExplSyntaxOn
\makeatletter

\cs_generate_variant:Nn \seq_set_split_keep_spaces:Nnn { Nnx }

\def \dlabelstyle #1 {
    \def\descriptionlabel ##1 {\hspace\labelsep \normalfont #1 ##1}
}

\cs_new:Npn \__two_cols:nnnnn #1 #2 #3 #4 #5 {
    \ifvmode\else\unskip\par\fi
    \noindent\leavevmode
    \hbox to \hsize {
        \hbox to #3 { \vtop {#1} }
        \hskip   #4
        \hbox to #5 { \vtop {#2} }
    } \par
}

\NewDocumentCommand \TwoCols {
    D[]{.475\hsize}
    D[]{.475\hsize}
    D[]{0pt plus 1fill}
    +m
    +m
} {
    \__two_cols:nnnnn{#4}{#5}{#1}{#3}{#2}
}

\cs_new:Npn \__gloss_insert_table_header: {
    %% Generating columns doesn’t seem to work, so this hack will do. If you
    %% need a gloss with more columns than this, I suggest you pause and take
    %% a moment to reflect on your life choices.
    \begin {tabular} { @{} *{100}l }
}

\cs_new:Npn \__gloss_table_start: {
    \ifvmode\else\unskip\par\fi
    \addvspace { 8pt }
    %\vtop{\iffalse}\fi
    \noindent
}

\cs_new:Npn \__gloss_table_end: {
    \end {tabular}
    \ifvmode\else\par\fi
    \addvspace { 8pt }
    %\iffalse{\fi}
    \everypar { \setbox\z@\lastbox \everypar{} }
}

%% Rescan a token list.
\cs_new:Npn \__gloss_rescan:n #1 {
    { \tl_rescan:nn {} { #1 } }
}

%% Format a single line of a gloss.
%%
%% The line number must be stored in \g_tempb_int.
%%
%% This takes the line #1 and formats it, inserting the contents of #2
%% at the start of each line, and #3 at the beginning of each column.
%% #3 can be a macro that takes one argument, in which case it will be
%% passed the contents of the column.
\cs_new:Npn \__gloss_format_line:nnn #1 #2 #3 {
    #2

    \seq_set_split:Nnn \l_tmpa_seq { | } { #1 }
    \bool_set_true:N \l_tmpa_bool

    \seq_map_inline:Nn \l_tmpa_seq {
        \bool_if:NF \l_tmpa_bool { & }
        \bool_set_false:N \l_tmpa_bool
        #3 { \__gloss_rescan:n {##1} }
    }
}

%% Format a gloss.
%%
%% This iterates over all lines in #1 and calls #2 on it.
\cs_new:Npn \__gloss_format:nn #1 #2 {
    \int_gset:Nn \g_tmpb_int { 1 }
    \tl_map_inline:nn { #1 } {
        \tl_if_blank:nF { ##1 } {
            %% Insert \cr or table header if this is the first line. We
            %% need to emit this here, as otherwise, TeX will encounter
            %% unexpandable tokens before the formatter (#2) is executed,
            %% which will start the cell before any actual content has
            %% been inserted; this means we can no longer tell TeX to
            %% \omit the cell header, which causes \multicolumn to break
            %% horribly.
            %%
            %% By emitting the table header here, we ensure that instead,
            %% the formatter is the first thing TeX gets to see after the
            %% initial table header and after every \cr.
            \int_compare:nNnTF { \g_tmpb_int } > { 1 } { \\ } { \__gloss_insert_table_header: }
            #2 { ##1 }
            \int_gincr:N \g_tmpb_int
        }
    }
}

%% Two lines means object language + gloss.
\cs_new:Npn \__gloss_two_lines:n #1 {
    \cs_gset:Npn \__formatter:n ##1 {
        \__gloss_format_line:nnn { ##1 } {} {
            \int_compare:nNnT { \g_tmpb_int } = { 1 } { \itshape }
        }
    }

    \__gloss_format:nn { #1 } { \__formatter:n }
}

%% Full gloss (text, object language, pronunciation, gloss, and translation).
\cs_new:Npn \__gloss_five_lines:n #1 {
    \cs_gset:Npn \__formatter_i:n ##1 { \multicolumn {100} {@{}l} {\itshape\bfseries\__gloss_rescan:n {##1}} }
    \cs_gset:Npn \__formatter_ii:n ##1 { \__gloss_format_line:nnn {##1} {} {\itshape} }
    \cs_gset:Npn \__formatter_iii:n ##1 { \__gloss_format_line:nnn {##1} {} {} }
    \cs_gset:Npn \__formatter_iv:n ##1 { \__gloss_format_line:nnn {##1} {} {} }
    \cs_gset:Npn \__formatter_v:n ##1 { \multicolumn {100} {@{}l} {\__gloss_rescan:n {##1}} }

    \__gloss_format:nn { #1 } {
        \cs:w __formatter_ \int_to_roman:n \g_tmpb_int :n \cs_end:
    }
}

%% Full gloss w/o IPA (text, object language, gloss, and translation).
\cs_new:Npn \__gloss_four_lines:n #1 {
    \cs_gset:Npn \__formatter_i:n ##1 { \multicolumn {100} {@{}l} {\itshape\bfseries \__gloss_rescan:n {##1} } }
    \cs_gset:Npn \__formatter_ii:n ##1 { \__gloss_format_line:nnn {##1} {} {\itshape} }
    \cs_gset:Npn \__formatter_iii:n ##1 { \__gloss_format_line:nnn {##1} {} {} }
    \cs_gset:Npn \__formatter_iv:n ##1 { \multicolumn {100} {@{}l} {\__gloss_rescan:n {##1}} }

    \__gloss_format:nn { #1 } {
        \cs:w __formatter_ \int_to_roman:n \g_tmpb_int :n \cs_end:
    }
}

\NewDocumentCommand \gloss {
    > { \exp_args:Nx \SplitList { \iow_char:N \^^M } } +v
} {
    \__gloss_table_start:

    %% Count lines.
    \int_gset:Nn \g_tmpa_int { 0 }
    \tl_map_inline:nn { #1 } {
        \tl_if_blank:nF { ##1 } {
            \int_gincr:N \g_tmpa_int
        }
    }

    %% Dispatch the appropriate number of lines.
    \int_case:nnF { \g_tmpa_int } {
        2 { \__gloss_two_lines:n { #1 } }
        4 { \__gloss_four_lines:n { #1 } }
        5 { \__gloss_five_lines:n { #1 } }
    }

    %% Any other line count is an error.
    {
        \msg_new:nnn { gloss } { too-many-lines } {
            Too~many~lines~in~gloss:~expected~2,~got~\int_use:N \g_tmpa_int
        }

        \msg_error:nn { gloss } { too-many-lines }
    }

    %% Close the table.
    \__gloss_table_end:
}

%% Process two words. Used by \multigloss.
\cs_new:Npn \__uf_multigloss_word:nn #1 #2 {
    \allowbreak

    \hbox {
        \begin{tabular}{@{}l}
            \itshape \tl_rescan:nn {} {#1} \\
            \noalign{\vskip-6pt}
            \tl_rescan:nn {} {#2}          \\
        \end{tabular}
    }

    \space
}

%% Process two lines. Used by \multigloss.
\cs_new:Npn \__uf_multigloss:NN #1 #2 {
    \ifvmode\noindent\leavevmode\fi

    %% Split the lines into words.
    \seq_set_split_keep_spaces:Nnx \l_tmpa_seq { | } { #1 }
    \seq_set_split_keep_spaces:Nnx \l_tmpb_seq { | } { #2 }

    %% Iterate over each word in the two lines.
    \seq_mapthread_function:NNN \l_tmpa_seq \l_tmpb_seq \__uf_multigloss_word:nn
}

%% Typeset two-line glosses across multiple lines.
\NewDocumentCommand \multigloss {
    > { \exp_args:Nx \SplitList { \iow_char:N \^^M } } +v
} {
    \ifvmode\else\unskip\par\fi
    \begingroup
    \linespread { 1.5 } \selectfont
    \raggedright
    \begin{sloppypar}

    %% Iterate over each line, two lines at a time.
    \bool_gset_true:N \g_tmpa_bool
    \tl_map_inline:nn { #1 } {
        %% Ignore empty lines entirely.
        \tl_if_blank:nF { ##1 } {
            \bool_if:NTF \g_tmpa_bool {
                \tl_gset:Nn \g_tmpa_tl { ##1 }
            } {
                \tl_gset:Nn \g_tmpb_tl { ##1 }
                \__uf_multigloss:NN \g_tmpa_tl \g_tmpb_tl
            }

            %% Flip.
            \bool_gset_inverse:N \g_tmpa_bool
        }
    }

    \end{sloppypar}
    \endgroup
}

\def \footnoterule {
    \kern -3\p@
    \hrule \@width .4\columnwidth
    \kern 2.6\p@
}

\def \@makefntext #1 {
    \setlength \parindent { 1em }
    \noindent {
        \mbox {
            \llap { {}\textsuperscript{\@thefnmark} \kern.5pt }
        } { #1 }
    }
}

\newlength{\EnumItemSep} \EnumItemSep-3pt

\newenvironment { enum } [1] [0] {
    \vspace { -.5em }
    \settowidth \leftmargini { 99.\hskip\labelsep }
    \begin { enumerate }
    \setcounter { enumi } { #1 }
    \itemsep \EnumItemSep
} {
    \end { enumerate }
    \vspace { -.5em }
}

\newenvironment { dlist } [1] [{}] {
    \vspace { -.5em }
    \begingroup
    \def\descriptionlabel ##1 {\hspace\labelsep \normalfont #1 ##1}
    \settowidth \leftmargini { 99.\hskip\labelsep }
    \begin { description }
    \itemsep \EnumItemSep
} {
    \end { description }
    \endgroup
    \vspace { -.5em }
}

%% Make it so *...* works just like in Markdown.
\char_set_catcode_active:N \*
\def\* { \detokenize{*} }
\cs_new:Npn * { \__md_star:w }

\cs_set_protected:Npn \__md_star:w {
    \peek_charcode_remove:NTF * { \__md_starstar:w } { \__md_singlestar:w }
}

\cs_set_protected:Npn \__md_starstar:w {
    \peek_charcode_remove:NTF * { \__md_triplestar:w } { \__md_doublestar:w }
}

\cs_set_protected:Npn \__md_singlestar:w #1*   { \textit{#1} }
\cs_set_protected:Npn \__md_doublestar:w #1**  { \textbf{#1} }
\cs_set_protected:Npn \__md_triplestar:w #1*** { \textbf{\textit{#1}} }

\cs_new:Npn \items {
    \ifvmode\else\unskip\par\fi
    \addvspace\medskipamount
    \begingroup
    \itemize\kern-\topsep
    \itemsep0pt
}

\cs_new:Npn \enditems {
    \enditemize\kern-\topsep
    \endgroup
    \addvspace\medskipamount
}


%% %%%%%%%%%%%%%%%%%%%%%%%%%%%%%%%%%%%%%%%%%%%%%%%%%%%%%%%%%%%%%%%%%%%%%%%%%%%%%
%%  Settings and Utility
%% %%%%%%%%%%%%%%%%%%%%%%%%%%%%%%%%%%%%%%%%%%%%%%%%%%%%%%%%%%%%%%%%%%%%%%%%%%%%%
\def \UF { \bfseries \itshape }
\let \nf \normalfont

\def \d {ḍ}
\def \D {Ḍ}
\def \b {ḅ}
\def \B {Ḅ}
\def \L {\textsuperscript{L}}
\def \N {\textsuperscript{N}}
\long \def \s #1 {{\normalfont\scshape #1 }}

\let \Sl \textbackslash
\let \Sub \textsubscript
\def \parheading #1 { \noindent \textbf{#1} }

\def \Item #1 {
    \item [ \textsc{\textbf{#1}} ]
}

\def \Paragraph #1 {
    \ifvmode\else\unskip\par\fi
    \addvspace \bigskipamount
    \noindent \leavevmode \ignorespaces \textbf{#1} \par
    \everypar { \setbox 0 \lastbox \everypar {} }
    \nobreak
}

\frenchspacing

\AtBeginDocument {
    \def \today {
        \int_value:w \day \space
        \int_case:nn { \month } {
             1 { January }
             2 { February }
             3 { March }
             4 { April }
             5 { May }
             6 { June }
             7 { July }
             8 { August }
             9 { September }
            10 { October }
            11 { November }
            12 { December }
        } \space
        \int_value:w \year
    }
}

\makeatother
\ExplSyntaxOff

%% %%%%%%%%%%%%%%%%%%%%%%%%%%%%%%%%%%%%%%%%%%%%%%%%%%%%%%%%%%%%%%%%%%%%%%%%%%%%%
%%  Document
%% %%%%%%%%%%%%%%%%%%%%%%%%%%%%%%%%%%%%%%%%%%%%%%%%%%%%%%%%%%%%%%%%%%%%%%%%%%%%%
\title{A Comprehensive Diachronic Grammar of Modern ULTRAFRENCH}
\author{Ætérnal \& Agma Schwa}
\date{\today}

\begin{document}
\maketitle
\thispagestyle{empty}
\clearpage
\setcounter{page}{1}

\tableofcontents
\clearpage

\section{Phonology and Evolution from Modern Pseudo-French}\label{sec:phonology}{\def\arraystretch{1.25}\setlength{\tabcolsep}{.4em}
\noindent\begin{tabular}{@{}|l|l|l|l|l|l|l@{\quad}|l|l|l|}                                                   \cline{1-6} \cline{8-10}
                       & Labial & Coronal  & Palatal  & Velar & Glottal &&            & Front        & Back        \\ \cline{1-6} \cline{8-10}
    Stop               & b, bʱ  & d        &          &       &         && Close      & i ĩ ĩ̃ i̥      & u ũ ũ̃ u̥ \\ \cline{1-6} \cline{8-10}
    Nasal              &        & n        &          &       &         && Near-close & ʏ ʏ̃ ʏ̃̃ ʏ̊      &             \\ \cline{1-6} \cline{8-10}
    Fricative          & ɸ β, ʋ̃ & s z, θ ð & ç ɕ ʑ    & x χ   & h       && Close-mid  & e ẽ ẽ̃ e̥      & o o̥         \\ \cline{1-6} \cline{8-10}
    Fric. (ʁ-coloured) & βʶ     & sʶ zʶ ɮ̃ʶ & ɕʶ ʑʶ    &       &         && Mid        & ə ə̣          &             \\\cline{1-6} \cline{8-10}
    Trill              &        &          &          & ʀ     &         && Open-mid   & ɛ ɛ̃ ɛ̃̃ ɛ̥      & ɔ̃ ɔ̃̃         \\ \cline{1-6} \cline{8-10}
    Approximant        &        &          & ɥ ɥ̃, j̊   & ɰ ɰ̃   &         && Near-open  & ɐ ɐ̥          &             \\ \cline{1-6} \cline{8-10}
    Lateral Fricative  &        & ɮ̃        & ʎ̝̃        &       &         && Open       &              & ɑ̃ ɑ̃̃         \\ \cline{1-6} \cline{8-10}
\end{tabular}}\bigskip

\parheading{Legend}\par\noindent
Ṽ = nasalised vowel, Ṽ̃ = nasal vowel, V = any vowel (or, in conjunction with Ṽ/Ṽ̃, oral vowel)\\
N = nasal consonant, C̃ = nasalised consonant (e.g. /ɰ̃/, but not true nasals), C = any consonant.\medskip
\def\scalpha{\kern-2pt\raisebox{2pt}{\Sub α}}

%% NOTE: In case the changes below and the ones listed
%% in the Lexurgy file differ, the latter are authoritative,
%% as I may forget to update these here sometimes.

\TwoCols[.45\hsize][.45\hsize][0pt]{
\parheading{Preliminary Changes}
\begin{enum}
    \item g, w > ɰ ⟨r⟩
    \item œ, œ̃, ø > y, ỹ, ỹ
    \item ɔ > o
    \item u > v / \_o
    \item y > j / \_(\#)V
    \item V\scalpha > $\emptyset$ / \_\#V\scalpha
    \item lj, lɥ > ʎ
    \item j > ɥ ⟨y’⟩
    \item ɰ > ɥ / \_i
    \item ʁʁ > ʀ
    \item sʁ, ʃʁ, zʁ, ʒʁ > sʶ, ʃʶ, zʶ, ʒʶ
    \item vʁ > vʶ
    \item ʁ > ɰ
    \item C > $\emptyset$ / \#\_C
    \item C > $\emptyset$ / C\_\#
    \item k > x ⟨c’h⟩
    \item ʃ, ʃʶ, ʒ, ʒʶ > ɕ, ɕʶ, ʑ, ʑʶ
    \item nt > nθ
    \item t > \d{} [d] (‘hard /d/’)
    \item p > \b{} [b] (‘hard /b/’)
    \item f, v, vʶ > ɸ ⟨f⟩, β ⟨b’h⟩, βʶ ⟨v́⟩
\end{enum}
}{
\parheading{Great Nasal Shift}
\begin{enum}[15]
    \item Ṽl > ɰ̃ ⟨w⟩
    \item V > Ṽ̃ / [NC̃ɥɰ]\_N\#
    \item V, Ṽ > Ṽ, Ṽ̃ / \_[NC̃ɥɰ], [NC̃ɥɰ]\_
    \item ə̃, ə̃̃, ã, ã̃, õ, õ̃ > ɛ̃, ɛ̃̃, ɑ̃, ɑ̃̃, ɔ̃, ɔ̃̃
    \item N, C̃ > $\emptyset$ / V\_\#
    \item ɲ, ŋ > n
    \item V, Ṽ > $\emptyset$ / N \_ N
    \item m, l, ʎ > ʋ̃ ⟨v⟩, ɮ̃ ⟨l⟩, ʎ̝̃ ⟨ḷ⟩
    \item ɮ̃ɰ, ɰɮ̃ > ɮ̃ʶ ⟨ł⟩
\end{enum}

\parheading{Intervocalic Lenition (/ V\_V is implied)}
\begin{enum}[21]
    \item x, s, z > h
    \item ɕ, ɮ̃, ʎ̝̃ > j̊ ⟨ç̇⟩, ɥ̃, ɰ̃
    \item nθ > n
    \item d, \d{}, b, \b{} > ð ⟨d’h⟩, θ ⟨t’h⟩, β, bʱ ⟨bh⟩
    \item ɸ > β / V\_V
\end{enum}

\parheading{Late Changes}
\begin{enum}[25]
    \item C[+stop, -alveolar]C\scalpha > C\scalpha
    \item C[+stop]C\scalpha[+stop] > C\scalpha
    \item h > $\emptyset$ / hV\_
    \item ə > $\emptyset$ / C\_C
    \item V[-nasalised, -nasal] > ə̥ / \_\#
\end{enum}
}\medskip

\subsection{Pronunciation, Allophony, and Stress}\label{subsec:pronunciation-allophony-and-stress}
There is not a lot of allophony in UF, save that /x/ is realised as [χ] around back vowels and [ɕ] elsewhere, e.g.
*c’húr* /xũɰ/ ‘to shrink’ is pronounced [χũˑˠ]. Furthermore, /h/ is [ç] before variants of /i/ and /y/, and [h] elsewhere.

The vast majority PF words are stressed on the last syllable of the root, e.g. *ad’hór* ‘to love’ /aˈðɔ̃ɰ/, but *b’had’hóré*
‘you (\s{pl}) love’ /βaˈðɔ̃.ɰɛ̃/. The stress is not indicated in writing, neither in actual texts, nor in this
grammar or in dictionaries. The main exception to this are names, which are generally stressed on the first syllable,
and receive secondary stress on the last syllable,\footnote{That is, unless the name ends in an obvious suffix, in which case the last
syllable before any such suffixes receives secondary stress; however, this is generally quite rare.} e.g. *Daúvníc’h* /ˈdɔ̃ʋ̃ˌnĩx/.

The only exception to this rule are certain particles and irregular verbs, some of which have irregular stress; for instance,
the forms of *eḍ* ‘to be’ are all stressed on the first syllable. Any such words that deviate from the norm will be pointed
out in this grammar and in dictionaries.

Oral vowels before the stressed syllable are often somewhat muted or reduced, albeit still audible, and stressed vowels are lengthened if they
are nasalised, e.g. the pronunciation of *ad’hór*, which we just transcribed as /aˈðɔ̃ɰ/, is actually closer to [ɐ̯ˈðɔ̃ˑɰ].
Word-final voiceless *ẹ* is always /ə̥/. Finally, non-back vowels that are followed by /ɰ/ or /ɰ̃/ are retracted, e.g. *y’ẹ́rẹ́*, the future
stem of *y’ẹ́* ‘forbid’, is phonemically /ɥẽ'ɰẽ/, but pronounced [ɥɘ̃'ɰẽ].

Oral vowels have a nasalised and nasal counterpart. /i/, /y/—which is actually [ʏ]—and /u/ do not vary in quality when nasalised.
/a/ is normally [ɐ],
but becomes [ɑ] when nasalised or nasal. Similarly, /e/ becomes [ɛ], and /o/ becomes [ɔ]. Note that nasalised [ẽ] exists, but it’s
rare. The quality never changes when going from nasalised to nasal. The schwa has no nasal(lised) counterpart. Lastly, oral vowel
also have voiceless counterparts, whose quality is the same as that of the base vowel.

The difference between nasalised vowels and nasal vowels is that the former are merely coarticulated with nasalisation, whereas
the latter are completely and utterly *in the nose*—no air escapes through the mouth when a nasal vowel is articulated, and all
the air flows just through the nose. Middle UF and some modern dialects also distinguish between sinistral and dextral nasal
vowels,\footnote{Sinistral nasal vowels are articulated with the left nostril, and dextral nasal vowels with the right nostril.}
but this distinction is no longer present in the modern standard language.

Furthermore, as indicated in that same example, word-final /ɰ/ is often realised as velarisation of the preceding vowel;
the same, however, is not the case for /ɰ̃/. Initial /ɰ/ is sometimes elided after words that end with /ɰ/, particularly
in particles (e.g. *rvá* ‘alas’).

Lenition causes the changes marked above as ‘Intervocalic Lenition’ to be applied to a consonant; furthermore,
ʁ-coloured consonants are replaced with their regular counterparts, and *h* disappears completely.

\subsection{Orthography}
The spelling of most UF sounds is indicated above; the less exotic consonants are spelt as
one might expect. That is, /b, d, n, ɸ, s, z, h/ are spelt ⟨b, d, n, f, s, z, h⟩, respectively.

Several fricatives are spelt with an apostrophe followed by a ‘h’, viz. /x/ ⟨c’h⟩, /θ/ ⟨t’h⟩, /ð/ ⟨d’h⟩,
and /β/ ⟨b’h⟩. Conventional letters are used for rather unconventional sounds, mostly for diachronic reasons:
/l/ does not exist in UF, so ⟨l⟩ is either /ɮ̃/ or /ʎ̝̃/, ⟨v⟩ is /ʋ̃/, ⟨j⟩ is /ʑ/, ⟨r⟩ is /ɰ/, ⟨w⟩ is /ɰ̃/. The vowel
/y/ is spelt ⟨y⟩, and its consonantal equivalent /ɥ/ as well as nasalised /ɥ̃/ are spelt with an apostrophe, that is
⟨y’⟩ and ⟨ý’⟩. The ʁ-fricated fricatives /βʶ, ɮ̃ʶ, sʶ, ɕʶ, ʑʶ, zʶ/
are spelt ⟨v́, ł, ś, ć, ȷ́, ź⟩, respectively.

Double consonant letters indicate a lengthened consonant; these are rare, but they can occur in any position. The only
exception to this is ⟨rr⟩, which is not /ɰː/, but rather /ʀ/. UF does not have phonemic vowel length (though recall
that phonetic lengthening occurs in some situations), so a double vowel letter is always pronounced as two separate vowels.

The vowels are mostly spelt as one might expect; nasalised vowels are indicated by an acute, and nasal vowels by a circumflex.
The variants of /i, y, u, a, e/ are spelt with ⟨i, y, u, a, e⟩ as their base letters. Nasal /ẽ/ and /ẽ̃/ as well as Schwa are
indicated by adding a dot below the ⟨e⟩ in grammars and dictionaries only; the vowel /o/ is spelt ⟨au⟩ or ⟨o⟩ for diachronic reasons;\footnote{As is always the
case in cases like this, hypercorrection is frequent, and ⟨au⟩ is often preferred word-initially, even if the
PF root was spelt with ⟨o⟩. In general, UF speakers seem to prefer ⟨au⟩ over ⟨o⟩, except word-finally and after ⟨w⟩, except
that in verb affixes, *au* is quite common word-finally. The sequence ⟨wau⟩ does not exist in UF.} in the case of
⟨au⟩, the acute and circumflex are added to the ⟨u⟩. The diphthong /au/ is spelt ⟨äu⟩, ⟨aü⟩, or with accents on both vowels. Oral
/ɛ/ is rare and is spelt ⟨è⟩. Word-initially and word-finally, a grave indicates that the vowel is voiceless. Word-final
voiceless /ə/ is always voiceless.\footnote{Thus, a word-final ⟨e⟩
can be /e/, such as in *vvaúríhe* /ʋ̃ːɔ̃ɰĩˈhe/ ‘to remember’, or /ə̥/, such as in *dale* /daɮ̃ə̥/ ‘table’. As a rule of thumb, it is
usually /e/ at the end of verb stems—but not verb forms in general—and /ə̥/ elsewhere. Fortunately they are differentiated by a
dot below in dictionaries and in this grammar: *vvaúríhe* vs *ḍalẹ*.}

The ‘hard’ voiced *ḅ*, *ḍ* which are pronounced exactly like their regular counterparts, are normally also spelt ⟨b⟩ and
⟨d⟩. However, the dot is commonly used in dictionaries and grammatical material to distinguish between the two
as they differ from one another in how they are lenited. Furthemore, a dot below or above a letter is commonly to indicate
a variety of different things, depending on the letter:
\begin{items}\itemsep .5ex plus .1ex minus .1ex\relax
\item a dot below in *ḅ*, *ḍ* indicates that they are the ‘hard’ variants of the letter, which are pronounced
      the same, but lenited differently;
\item a dot below in *ḷ* indicates that it is palatal /ʎ̝̃/ instead of alveolar /ɮ̃/;
\item a dot below in *ẹ* indicates that it is a schwa;
\item a dot below nasalised *ẹ́*, *ệ* indicates that they are /ẽ/, /ẽ̃/ instead of /ɛ̃/, /ɛ̃̃/;
\item a dot above in *ċ* indicates that it is lenited /j̊/.
\end{items}

\noindent Thus, in non-grammatical writing, the following are indistinguishable:
\begin{items}\itemsep .5ex plus .1ex minus .1ex\relax
\item *l* can be palatal /ʎ̝̃/ or alveolar /ɮ̃/;
\item *e* can be a schwa, or /e/;
\item *é*, *ê* can be /ɛ̃/, /ɛ̃̃/ or /ẽ/, /ẽ̃/;
\item *c* can be /ɕ/ or /j̊/.
\end{items}

\noindent Elided initial /ɰ/ is indicated by omitting the *r* in writing and attaching the word to the previous one with a hyphen,
e.g. *-vá* ‘alas’.

UF seldom uses hyphens to separate or join words and instead prefers to spell them as one word instead; an exception
to this is that affixes that end with a vowel are typically separated from the word they are attached to with a hyphen
if that word starts with (a variant of) the same vowel. For example, the \s{def nom sg} of *el* ‘wing’
is *láel*, but the plural is *lé-el*.

\subsubsection{Lenition and Nasalisation}
Certain morphological elements subject surrounding context to lenition or nasalisation. Nasalisation affects vowels,
which become more nasal (that is, (voiceless) oral vowels become nasalised, and nasalised vowels become nasal; nasal
vowels are unaffected), as well as *ḍ*, which becomes *n*.

Lenition is more complicated; it affects only consonants and causes a softening similar to what happened diachronically
between vowels. All ʁ-fricated consonants simply lose their ʁ-frication. Furthermore, the
following consonants are also affected by lenition:

\begin{table}[H]
\centering
\itshape
\begin{tabular}{l|lll|l|l|l|ll|l|l|l|}
\bf Consonant & x & s & z       & c & l  & ḷ & b & f          & ḅ   & d   & ḍ    \\\hline
\bf Lenited & \multicolumn{3}{c|}{h} & ċ & ý’ & w & \multicolumn{2}{c|}{b’h} & bh  & d’h & t’h  \\
\end{tabular}
\nf
\caption{Consonants Affected by Lenition}\label{tab:lenition}
\end{table}

\noindent Note that double consonants are typically unaffected by morphological lenition, e.g. *dír* ‘to say’,
whose subjunctive stem is *díss*, forms *aúdíssâ* (rougly ‘we should have said’), not \**aúdíhhâ*.

\subsubsection{Glossing}
To simplify glosses, cases are assumed to be definite and singular unless otherwise stated, and verb forms are
assumed to be indicative, present tense, and active, unless otherwise stated.

\section{Accidence}\label{sec:accidence}

\subsection{Noun Morphology}\label{subsec:noun-morphology}
UF has 4 declensions. A definite and indefinite vocalic declension, and a definite and indefinite consonantal declension.
As their names might suggest, the former two are used for nouns that start with a vowel, and the latter two for nouns
that start with a consonant. UF has no morphologically separate articles; rather, the old PF articles have been incorporated
into the declensions. Furthermore, UF no longer has a gender distinction in nouns.

\subsubsection{Declension}\label{subsubsec:declension}
The table below shows the affixes of the definite and indefinite declensions. The declensions are mostly identical,
except that, as with the conjugation of verbs, the consonantal prefixes often end in a vowel (marked below with
parentheses), which are not present in the vocalic declension.

\begin{table}[H]
\centering
\noindent\begin{tabular}{@{}|l|>{\it}l|>{\it}l|l|l|>{\it}l|>{\it}l|}\cline{1-3}\cline{5-7}
Definite    &\nf Sg&\nf Pl && Indefinite       &\nf Sg&\nf Pl\\\cline{1-3}\cline{5-7}

Absolutive    & $\emptyset$     & l-      &&Absolutive    & $\emptyset$-\N & $\emptyset$-\L   \\\cline{1-3}\cline{5-7}
Nominative    & lá-\L   & lé-\L   &&Nominative    & ŷn-\N & ý-\L      \\\cline{1-3}\cline{5-7}
Vocative      & $\emptyset$-\L  & $\emptyset$-\L  &&Vocative      & / & /             \\\cline{1-3}\cline{5-7}
Partitive     & dy-\L   & dẹ-\L   &&Partitive     & dŷn-\N & dý-\L    \\\cline{1-3}\cline{5-7}
Accusative    & i-\L    & sý-\L   &&Accusative    & s-\L & s-         \\\cline{1-3}\cline{5-7}
Genitive      & á-\L    & abh-\L  &&Genitive      & sý-\N & sý-\L     \\\cline{1-3}\cline{5-7}
Dative        & as-\L   & a-\L    &&Dative        & an-\N & an-\L     \\\cline{1-3}\cline{5-7}
Inessive      & dwá-    & dwé-    &&Inessive      & dáhŷn- & dáhŷ-    \\\cline{1-3}\cline{5-7}
Interessive   & aḍá-    & aḍé-    &&Inessive      & aḍŷn- & aḍŷ-    \\\cline{1-3}\cline{5-7}
Ablative      & rê(d)-  & rês-    &&Ablative      & rêdýn- & rêdý-    \\\cline{1-3}\cline{5-7}
Allative      & b’hé-\L & b’hér-  &&Allative      & b’hŷn-\N & b’hý-\L  \\\cline{1-3}\cline{5-7}
Considerative & słá-    & słé-    &&Considerative & sý’óýn- & sý’óý-  \\\cline{1-3}\cline{5-7}
Instrumental  & b’hel-  & b’he-   &&Instrumental  & b’hehý(n)- & b’heh-  \\\cline{1-3}\cline{5-7}
...           &         &         &&              & &                 \\\cline{1-3}\cline{5-7}
\end{tabular}
\caption{UF Declension.}\label{tab:table-uf-declension}
\end{table}

\noindent Most of these forms cause lenition in the initial consonant of the noun, e.g. *ḍalẹ* ‘table’ to
\s{def acc sg} *s’thalẹ*; this lenition is blocked in the \s{indef acc pl} due to the presence of a hypercorrected ‘s’
in PF \**ces*, e.g. *s’ḍalẹ* ‘the tables (\s{acc})’ (not *s’thalẹ*, which is the singular), as well as in
less commonly used forms such as the \s{def iness} *dwáḍalẹ* ‘on the table’.

The \s{indef nom sg} *ŷn-* prefix and some other forms nasalise nouns; as a reminder, this means that in
nouns starting with *ḍ*, the *ḍ* is deleted, e.g. *ŷnalẹ* ‘a table’;
it causes nasalisation in words that start with a vowel e.g. *ehyó* ‘shield’ to *ŷnéhyó* ‘a shield’. The indefinite \s{voc}
does not exist, as that would make little sense. As lenition, nasalisation too is blocked in rarer forms, e.g. \s{indef iness}
*dáhŷnḍalẹ* ‘on a table’.

The \s{abs} case is used for the predicate noun of predicative sentences, e.g. *Aúsó ḍe ráhó* ‘We are all fish’.
The \s{cons} case can be translated as ‘according to’, or ‘in the opinion of’, and is used to express the opinion of
the speaker or point out something as an opinion, belief, or hypothesis of someone or something.

Both the \s{part} and the \s{acc} can be used to mark the direct object of a verb; some verbs, e.g. *ub’hrá* ‘to be
able to’ always take a \s{part}, and some always take an \s{acc}, but for most verbs, the difference is semantic: the
\s{acc} indicates that an action is being or has been performed in its entirety or to completion, e.g. *jlí sliv́uhé* ‘I peruse
a book’ vs *jlí dŷnliv́uhé* ‘I read (\s{pres}) from a book’ or ‘I am reading a book’. Consequently, \s{pres ant}
forms, which are mainly perfective, generally take the \s{acc}, e.g. *jlíé iliv́uhé* ‘I’ve read the book’, whereas \s{pret} forms,
which are mainly imperfective, generally take the \s{part}, e.g. *jlíá dyliv́uhé* ‘I was reading (from) the book’.


The *d* in the \s{def abl sg} is omitted if the noun starts with a consonant, e.g. *rêḍalẹ* ‘from the table’; be careful
especially with words that start with *s*, whose \s{abl sg} is often mistaken for a plural, e.g. *rêsol* ‘from the floor’,
but *rêssol* ‘from the floors’.

The diachrony of these forms is mostly from the PF definite and indefinite pronouns as well as from PF prepositions, though
some forms, such as the
accusative, are borrowed from demonstratives instead (\s{def} from PF \**celui* and \s{indef} from PF \**ce*); the definite
partitive forms are from the PF partitive article, and
the indefinite forms are formed with an additional *d-* by analogy to the definite forms. The locative cases are combinations
of the articles and PF prepositions. The ablative is from PF \**loin de* ‘away from’. The diachrony of the genitive singular
is unclear.

\begin{table}[H]
\centering
\noindent\begin{tabular}{@{}|l|>{\it}l|>{\it}l|l|l|>{\it}l|>{\it}l|}\cline{1-3}\cline{5-7}
Definite    &\nf Sg&\nf Pl && Indefinite&\nf Sg&\nf Pl\\\cline{1-3}\cline{5-7}

Nominative  & lát’halẹ  & lét’halẹ   &&Nominative & ŷnalẹ & ýt’halẹ         \\\cline{1-3}\cline{5-7}
Vocative    & t’halẹ    & t’halẹ    &&Vocative   & / & /                      \\\cline{1-3}\cline{5-7}
Partitive   & dyt’halẹ  & dẹt’halẹ &&Partitive   & dŷnalẹ & dýt’halẹ                     \\\cline{1-3}\cline{5-7}
Accusative  & it’halẹ & sýt’halẹ  &&Accusative & st’halẹ & sḍalẹ                       \\\cline{1-3}\cline{5-7}
...         &  &  &&           & &              \\\cline{1-3}\cline{5-7}
Inessive    & dwáḍalẹ & dwéḍalẹ &&Inessive   & dáhŷnḍalẹ & dáhýḍalẹ                    \\\cline{1-3}\cline{5-7}
\end{tabular}
\caption{Consonantal declension of *ḍalẹ*.}\label{tab:vocalic-declension}
\end{table}

\subsection{Adjectives}
UF does not have many actual adjectives. Most words in UF are either nouns or verbs, and most ‘adjectives’ are just
participles, which can always be used like adjectives. Indeed, there are a lot of verbs whose meaning is something
along the lines of ‘to be X’, whose present participle behaves like the adjective ‘X’, e.\,g. *ḅẹt’hẹ* ‘to be small’
to *ḅẹt’hâ* ‘small’ (literally ‘being small’).

Adjectives generally follow the noun they modify and are never inflected, e.g. *át’halẹ ḅẹt’hâ* ‘of a small table’.
There is no established order of adjectives.

%% ‘lẹ-’: from PF ‘plus’. Comparative prefix.
\subsubsection{Comparison}\label{subsubsec:comparison}
Unlike in many other languages, there are 3 comparatives in UF: The affirming comparative, so called
because it affirms the positive (‘better, and also good’); the denying comparative, which denies the positive
(‘better, but not good’), and the neutral comparative, which does not make any statement about the positive
(‘better’).

To illustrate the difference between the three: We might say that an ant is ‘bigger’ than a grain of sand, but
an ant is still not big, all things considered. By contrast, an elephant may be ‘smaller’ than a mountain,
but that doesn’t mean that an elephant is small.

In UF, the comparatives are expressed by three infixes, which are prefixed directly to the stem. The affirming
comparative prefix is *lẹ*, the denying comparative prefix is *y’ŷ*, and the neutral comparative prefix is *rê*.
Thus, we have *ḅẹt’hâ* ‘small’, *lẹḅẹt’hâ* ‘smaller, and also small’, *y’ŷḅẹt’hâ* ‘smaller, but not small’, and
*rêḅẹt’hâ* ‘smaller’.

The comparative prefixes can also be applied to verbs, though they usually only make sense for the aforementioned
‘adjective verbs’, e.g. *jy’ŷḅẹt’hẹ* ‘I am smaller, but still big’. Note that these prefixes
might cause a verb’s forms to change from vocalic to consonantal, e.g. *ebhẹ* ‘to be thick’ (future stem *ebhrẹ*)
is vocalic *náy’ebhraú* ‘we shall be thick’ in the positive, but consonantal *aúnraûy’ŷebhraû* ‘we shall be
thicker, but not thick’ in the negative comparative.

The affirming comparative can also be used absolutely, with the meaning of ‘to a large degree’. Thus,
we have *ḅẹt’hâ* ‘small’, and *lẹḅẹt’hâ* ‘tiny’; sometimes, this also leads to a slight change in meaning
or perception, e.g. *ebhâ* ‘thick’, but *lẹ-ebhâ* ‘thicc’.

The affirming and denying comparative can also mean ‘too X’ and ‘not X enough’, respectively; thus, *lẹḅẹt’hâ*
can also mean ‘too small’, and *y’ŷḅẹt’hâ* can also mean ‘not small enough’, though this meaning is somewhat
uncommon in isolation and most commonly found in constructions (see below).

The superlative is formed with one of two prefixes: *ré\L* and *râdvâ*. Be careful not to confuse the former
with the neutral comparative *rê*! The two prefixes are largely interchangeable, however, the former is more
literary and also older. The latter is a more recent development to reduce potential ambiguity with the
neutral comparative. Note that *ré* lenites, whereas *râdvâ* does not. Thus, we have *rébhẹt’hâ* or
*râdvâḅẹt’hâ* ‘smallest’.

\subsubsection{Constructions}
The comparative can be used together with an infinitive, \s{aci}, or \s{pci}. The affirming comparative here has the meaning
of ‘too X to \ldots’, and the denying comparative has the meaning of ‘not X enough to \ldots’. A good illustrative
example of this is the following UF proverb:

\gloss {
    Láráhó slẹlúrá b’héd’hẹhẹ dẹnájẹ.
    Lá-ráhó|s-lẹ-lúr-á|b’hé\Sl d’hẹhẹ|dẹ-nájẹ
    \s{nom}-fish|\s{3n}-\s{aff.comp}-bulky-\s{3sg.pres.ant}|\s{all}\Sl surface|\s{inf}-swim
    ‘The fish was too bulky to swim to the surface’\footnotemark
}

\footnotetext{This is a very common proverb (often also just \textit{láráhó slẹlúr} ‘The fish is too bulky’)
and roughly means that something has gone too far or gone on for too long (‘Now you’ve done it’ or ‘Now
it’s too late’). Variations of it exists; in the optative, for instance, this proverb means ‘Let’s not overdo this’.}

\subsection{Verbal Morphology}\label{subsec:verbal-morphology}
Verbs in UF are inflected for person, number, tense, aspect, mood, and voice. Verbal inflexion is mainly done
by means of concatenating a vast set of prefixes onto a verb, with the occasional suffix and circumfix making
its appearance. This chapter details these affixes, their meanings, uses, forms, and restrictions.


\subsubsection{Active/Passive Affixes}\label{subsubsec:active-passive-affixes}
UF has a set of active/subject as well as passive/object prefixes which can be used on their own or in combination
with one another, though at most one active and one passive prefix may be combined with a verb.\footnote{Irrespective
of whether they are personal or infinitive prefixes. For instance, it would also be illegal to combine e.g. the active
infinitive prefix with the first person active singular prefix.} Table~\ref{tab:active-passive-prefixes}
below lists those prefixes, two of which are actually circumfixes.

\begin{table}[H]
\centering
\noindent\begin{tabular}{@{}|>{}l|>{\it}l|>{\it}l|>{}l|>{}l|>{\it}l|>{\it}l|}\cline{1-3}\cline{5-7}
 Active&\nf Sg&\nf Pl& & Passive&\nf Sg&\nf Pl\\\cline{1-3}\cline{5-7}
1st&j-&aú-/r-/w- -(y’)ó&&1st&v-&aú-/r-/w-\\\cline{1-3}\cline{5-7}
2nd&\d{}(ẹ)-&b’h(y)- -(y’)é&&2nd&\d{}(ẹ)-&b’h(y)-\\\cline{1-3}\cline{5-7}
3rd m&l(ẹ)-&l(ẹ)-&&3rd m&y’-&lý-\\\cline{1-3}\cline{5-7}
3rd f&ll(a)-&ll(ẹ)-&&3rd f&y’- &lý-\\\cline{1-3}\cline{5-7}
3rd n&s- &l(a)-&&3rd n&sy-&lý-\\\cline{1-3}\cline{5-7}
Infinitive&\multicolumn{2}{c|}{\it d(ẹ)-}&&Infinitive&\multicolumn{2}{c|}{\it à-/h-}\\\cline{1-3}\cline{5-7}
Participle&\multicolumn{2}{c|}{\it -â}&&Participle&\multicolumn{2}{c|}{\it â-}\\\cline{1-3}\cline{5-7}
\end{tabular}
\caption{Active (left) and passive (right) verbal affixes.}\label{tab:active-passive-prefixes}
\end{table}

\noindent A great degree of syncretism can be observed in the third-person forms. The gender distinction in the
\s{3sg} that diachronically resulted from gendered personal pronouns is almost non-existent in the
plural; the reason for this development is that those forms are derived from the old dative form, which lacked
this distinction altogether.

The \s{act 1pl, 2pl} forms are only distinguished from their passive counterparts by
the presence of additional suffixes in the former. The \s{3sg n} in the active and passive is derived from the PF
demonstrative \**ce* and its variants; the \s{3pl n} is derived from the other \s{3pl} forms.

\Paragraph{Usage Notes}
\begin{items}
\Item{1pl}
The \s{1pl} prefix varies if there is a vowel following it: if it is
any vowel that is *not* a variant of ‘o’, the prefix is realised as *r-* instead, e.g. *ad’hór* ‘love’ to
*rad’hóró* ‘we love’. If the vowel a variant of ‘o’, the prefix is realised as *w-* instead, e.g. *aub’heír* ‘obey’
to *wob’heíró* ‘we obey’.\footnote{Diachronically, the base form of this prefix is \**o-*, whence e.g.
\**oad’hóró* > *rad’hóró* and \**oob’heíró* > *wob’heíró*.} Note that this also leads to a change in spelling: stem-initial
⟨au⟩ is changed to ⟨o⟩.

\Item{1,2 pl}
The *y’* in the suffix parts of the \s{1pl, 2pl act} are dropped if the verb ends with a consonant, e.g. *ad’hór*
to *b’hád’hóré*, or if it ends with a vowel that is a variant of ‘o’ in the case of the \s{1pl} and ‘e’ in the case
of the \s{2pl}, in which cases the vowels are contracted and a level of nasalisation is added, e.g. *vvaúríhe*
‘to remember’ to *b’hyvvaúríhé* ‘you (\s{pl}) remember’ (not \**b’hyvvaúríhy’é*). In all other cases, the *y’* is retained,
e.g. *aúvvaúríhey’ó* ‘we remember’.

\Item{inf}
The \s{inf pass} prefix *à-* coalesces with any vowel following it: it becomes *á* if it
is followed by a non-nasal variant of ‘a’, e.g. *ad’hór* to *ád’hór* ‘to be loved’; *â* if it is
followed by a nasal variant of ‘a’, e.g. *ánvé* ‘give life to’ to *ânvé* ‘to be animated’; and *h-* if it is
followed by any other vowel, e.g. *aub’heír* to *haub’heír* ‘to be obeyed’.

\Item{part}
The participle affixes are commonly used to form adjectives since the vast majority of adjectives in UF are actually ‘adjective verbs’
with a meaning of ‘to be X’. The participle can be used to convert such a verb back into a regular adjective, e.g. *lár* ‘to be wide’
to *lárâ* ‘wide’. Like the passive infinitive affix, the participle affixes coalesce with vowels and always form a maximally
nasal vowel, e.g. *vvaúríhe* ‘to remember’ forms *vvaúríhê* ‘remembering’, and *ad’hór* forms *âd’hór* ‘being loved’. As with other
coalescence rules, the *-â* instead *replaces* final or initial *ẹ*, and *ẹ* only: e.g. *ḅẹt’hẹ* ‘to be small’ becomes *ḅẹt’hâ* ‘being
small’. Note that if the word already ends with a maximally nasal vowel, no coalescence occurs, e.g. *rê* ‘to be triune’ becomes
*rêâ* ‘triune’.

\Item{{\upshape\textit{-ẹ-}}}
The parenthesised vowels are used if the prefix is followed by a consonant, e.g. *dír* ‘say’ to *llẹ{}dír*
‘they (\s{f}) say’ and *b’hydíré* ‘you (\s{pl}) say’, but *ad’hór* to *llad’hór* ‘they (\s{f}) love’ and *b’had’hóré* ‘you
(\s{pl}) love’. The prefixes *aú-* and *à-* retain their main forms if followed by a consonant,
e.g. *dír* ‘say’ to *aúdíró* ‘We say’ and *àdír* ‘to be said’.

\Item{{\upshape\textit{-y-}}}
The exception to this is that \s{2pl} *b’h(y)-*
drops the *y* if followed by a glide, e.g. *y’ír* ‘to hear’ to *b’hy’íré* ‘you (\s{pl}) hear’ (not \**b’hyy’íré*).
\end{items}

\Paragraph{Combining Prefixes}
When multiple prefixes are used together, active prefixes precede passive prefixes, except that infinitive and participle prefixes
always come first, e.g. *ad’hór* ‘love’ to *jvad’hór* ‘I love myself’ (not \**vjad’hór*) and *b’hy’ad’hóré* ‘you (\s{pl}) love him/her’,
but *dẹvad’hór* ‘to love me’ and *àb’had’hóré* ‘to be loved by you (\s{pl})’. Recall that at most one infinitive prefix
and at most one participle affix may be used.

\Paragraph{Example Paradigms}
By way of illustration, consider the paradigm of the verb *ad’hór* as shown in Table~\ref{tab:adhor-paradigm} below.
Since this word starts with a vowel, the parenthesised vowels in Table~\ref{tab:active-passive-prefixes} above
are not used. Furthermore, since it starts with a non-nasal ‘a’-like vowel, the *aú-* prefix is realised as *r-*
and the *à-* prefix coalesces with the initial ‘a’ of the stem to form *á*.

% TEMPLATE:
%\noindent\begin{tabular}{@{}|>{}l|>{\it}l|>{\it}l|>{}l|>{}l|>{\it}l|>{\it}l|}\cline{1-3}\cline{5-7}
%\nf Active&\nf Sg&\nf Pl&\nf &\nf Passive&\nf Sg&\nf Pl \\\cline{1-3}\cline{5-7}
%1st       &   &  &&1st   &   &   \\\cline{1-3}\cline{5-7}
%2nd       &   &  &&2nd   &   &   \\\cline{1-3}\cline{5-7}
%3rd m     &   &  &&3rd m &   &   \\\cline{1-3}\cline{5-7}
%3rd f     &   &  &&3rd f &   &   \\\cline{1-3}\cline{5-7}
%3rd n     &   &  &&3rd n &   &   \\\cline{1-3}\cline{5-7}
%Infinitive& \multicolumn{2}{c|}{\it }  && Infinitive & \multicolumn{2}{c|}{\it } \\\cline{1-3}\cline{5-7}
%\end{tabular}

\begin{table}[H]
\centering
\noindent\begin{tabular}{@{}|>{}l|>{\it}l|>{\it}l|>{}l|>{}l|>{\it}l|>{\it}l|}\cline{1-3}\cline{5-7}
\nf Active&\nf Sg&\nf Pl&\nf &\nf Passive&\nf Sg&\nf Pl\\\cline{1-3}\cline{5-7}
1st&jad’hór&rad’hóró&&1st&vad’hór&rad’hór\\\cline{1-3}\cline{5-7}
2nd&\d{}ad’hór&b’had’hóré&&2nd&\d{}ad’hór&b’had’hór\\\cline{1-3}\cline{5-7}
3rd m&lad’hór&lad’hór&&3rd m&y’ad’hór&lýad’hór\\\cline{1-3}\cline{5-7}
3rd f&llad’hór&llad’hór&&3rd f&y’ad’hór &lýad’hór\\\cline{1-3}\cline{5-7}
3rd n&sad’hór&lad’hór&&3rd n&ý’ad’hór&lýad’hór\\\cline{1-3}\cline{5-7}
Infinitive&\multicolumn{2}{c|}{\it dad’hór}&&Infinitive&\multicolumn{2}{c|}{\it ád’hór}\\\cline{1-3}\cline{5-7}
Participle&\multicolumn{2}{c|}{\it ad’hórâ}&&Participle&\multicolumn{2}{c|}{\it âd’hór}\\\cline{1-3}\cline{5-7}
\end{tabular}
\caption{Paradigm of the Verb \emph{ad’hór}.}\label{tab:adhor-paradigm}
\end{table}

\noindent For comparison, the paradigm of the verb *vvaúríhe* ‘remember’ is shown in Table~\ref{tab:vvorihe-paradigm} below.
Since it starts with a consonant, the parenthesised vowels in Table~\ref{tab:active-passive-prefixes} are used, and any
prefixes that end with a vowel remain unchanged.

\begin{table}[H]
\centering
\noindent\begin{tabular}{@{}|>{}l|>{\it}l|>{\it}l|>{}l|>{}l|>{\it}l|>{\it}l|}\cline{1-3}\cline{5-7}
\nf Active&\nf Sg&\nf Pl&\nf &\nf Passive&\nf Sg&\nf Pl\\\cline{1-3}\cline{5-7}
1st&jvvaúríhe&aúvvaúríhey’ó&&1st&vvvaúríhe&aúvvaúríhe\\\cline{1-3}\cline{5-7}
2nd&ḍẹvvaúríhe&b’hyvvóríhé&&2nd&ḍẹvvaúríhe&b’hyvvaúríhe\\\cline{1-3}\cline{5-7}
3rd m&lẹvvaúríhe&lẹvvaúríhe&&3rd m&y’vvaúríhe&lývvaúríhe\\\cline{1-3}\cline{5-7}
3rd f&llavvaúríhe&llẹvvaúríhe&&3rd f&y’vvaúríhe&lývvaúríhe\\\cline{1-3}\cline{5-7}
3rd n&ý’vvaúríhe&lavvaúríhe&&3rd n&ý’vvaúríhe&lývvaúríhe\\\cline{1-3}\cline{5-7}
Infinitive&\multicolumn{2}{c|}{\it dẹvvaúríhe}&&Infinitive&\multicolumn{2}{c|}{\it àvvaúríhe}\\\cline{1-3}\cline{5-7}
Participle&\multicolumn{2}{c|}{\it vvaúríhê}&&Participle&\multicolumn{2}{c|}{\it âvvaúríhe}\\\cline{1-3}\cline{5-7}
\end{tabular}
\caption{Paradigm of the Verb \emph{vvaúríhe}.}\label{tab:vvorihe-paradigm}
\end{table}

\subsection{Tense and Aspect Marking}\label{subsec:tense-and-aspect-marking}
Tense in PF is marked by additional sets of affixes that are appended to the verb in addition to the active/passive affixes.
There are two broad groups of such affixes: suffixes, which are appended to the end of the verb and replace the \s{act 1pl, 2pl} suffixes
in those persons, as well as circumfixes and prefixes, which are inserted before the active/passive markers and replace the
replace the \s{act 1pl, 2pl} suffixes in some cases.

\subsubsection{Suffixed Tenses}\label{subsubsec:suffixed-tenses}
The present anterior and preterite are formed by appending a set of suffixes to the verb. Table~\ref{tab:present-anterior-and-preterite-suffixes}
below lists the suffixes for those tenses. The present anterior has a perfect or perfective aspect, while the preterite has an imperfective aspect. The
former is commonly used to describe events that are completed or extend to the present—particularly events that occurred recently, hence the name—while the latter
is used to describe events that are ongoing or habitual.


\begin{table}[H]
\centering
\noindent\begin{tabular}{@{}|>{}l|>{\it}l|>{\it}l|>{}l|>{}l|>{\it}l|>{\it}l|}\cline{1-3}\cline{5-7}
\nf Present Anterior&\nf Sg&\nf Pl&\nf &\nf Preterite&\nf Sg&\nf Pl \\\cline{1-3}\cline{5-7}
1st       & -\L é & -\L â &&1st    & -\L á  & -y’aû  \\\cline{1-3}\cline{5-7}
2nd       & -\L á & -\L áḍ &&2nd   & -\L é  & -y’ẹ́  \\\cline{1-3}\cline{5-7}
3rd       & -\L á & -\L ér &&3rd m & -\L é  & -\L é   \\\cline{1-3}\cline{5-7}
Infinitive& \multicolumn{2}{c|}{\it -á }  && Infinitive & \multicolumn{2}{c|}{\it -é } \\\cline{1-3}\cline{5-7}
Participle& \multicolumn{2}{c|}{\it -ér }  && Participle & \multicolumn{2}{c|}{\it -ár } \\\cline{1-3}\cline{5-7}
\end{tabular}
\caption{Present Anterior and Preterite Affixes.}\label{tab:present-anterior-and-preterite-suffixes}
\end{table}

\noindent Neither tense distinguishes gender in the third person. All suffixes, except for the infinitive and \s{1pl, 2pl pret},
lenite any consonant *before* them, e.g. *ḅárḍáḍ* ‘to be willing’ to *jḅárḍát’hé* ‘I was willing’ but *dẹḅárḍáḍá*
‘to have been willing’.

Diachronically, the \s{1sg pret} is an interesting case; in EUF, it was originally \**-é*, but it later changed to *-á*
to distinguish it from the \s{2sg, 3sg pres ant}. The remaining forms—save the infinitives, which are derived from the
tenses’ definite endings by analogy—originated from the PF simple past tenses.

The table below lists the example paradigm of the verb *ad’hór* in the present anterior and preterite tenses.
Observe that there is no difference between the \s{1pl, 2pl} active and passive.

The participle suffixes coalesce with present participle affixes to form *êr*/*ệr* in the present anterior and *âr* in the preterite,
where applicable, e.g. present *ad’hórâ* ‘loving’ becomes *ad’hórêr* ‘having loved’.

In both tenses, the suffixes coalesce with vowels before them, replacing them and nasalising them if
they are already nasal, e.g. *jvvaúríé* ‘I remembered’.

If a verb takes both and active and a passive person affix, the suffix aligns with the active affix; thus
‘she loved me’ is *llavad’hórá* and not \**llavád’hóré*.

\begin{table}[H]
\centering
\noindent\begin{tabular}{@{}|>{}l|>{\it}l|>{\it}l|>{}l|>{}l|>{\it}l|>{\it}l|}\cline{1-3}\cline{5-7}
\nf Active & \nf Sg   & \nf Pl     & \nf & \nf Passive & \nf Sg   & \nf Pl    \\\cline{1-3}\cline{5-7}
1st        & jad’hóré  & rad’hórâ     &     & 1st         & vad’hóré  & rad’hórâ   \\\cline{1-3}\cline{5-7}
2nd        & ḍad’hórá  & b’had’hóráḍ  &     & 2nd         & ḍad’hórá  & b’had’hóráḍ \\\cline{1-3}\cline{5-7}
3rd m      & lad’hórá  & lad’hórér    &     & 3rd m       & y’ad’hórá & lýad’hórér  \\\cline{1-3}\cline{5-7}
3rd f      & llad’hórá & llad’hórér   &     & 3rd f       & y’ad’hórá & lýad’hórér  \\\cline{1-3}\cline{5-7}
3rd n      & ý’ad’hórá & lad’hórér    &     & 3rd n       & ý’ad’hórá & lýad’hórér  \\\cline{1-3}\cline{5-7}
Infinitive & \multicolumn{2}{c|}{\it dad’hórá} & & Infinitive & \multicolumn{2}{c|}{\it ád’hórá} \\\cline{1-3}\cline{5-7}
Participle&\multicolumn{2}{c|}{\it ad’hórêr}&&Participle&\multicolumn{2}{c|}{\it âd’hórér}\\\cline{1-3}\cline{5-7}
\end{tabular}
\caption{Present Anterior Paradigm of the Verb \emph{ad’hór}.}\label{tab:adhor-paradigm-pres-ant}
\end{table}

\begin{table}[H]
\centering
\noindent\begin{tabular}{@{}|>{}l|>{\it}l|>{\it}l|>{}l|>{}l|>{\it}l|>{\it}l|}\cline{1-3}\cline{5-7}
\nf Active & \nf Sg   & \nf Pl     & \nf & \nf Passive & \nf Sg   & \nf Pl    \\\cline{1-3}\cline{5-7}
1st        & jad’hórá  & rad’hóry’aû     &     & 1st     & vad’hórá  & rad’hóry’aû   \\\cline{1-3}\cline{5-7}
2nd        & ḍad’hóré  & b’had’hóry’ẹ́  &     & 2nd      & ḍad’hóré  & b’had’hóry’ẹ́ \\\cline{1-3}\cline{5-7}
3rd m      & lad’hóré  & lad’hóré    &     & 3rd m      & y’ad’hóré & lýad’hóré  \\\cline{1-3}\cline{5-7}
3rd f      & llad’hóré & llad’hóré   &     & 3rd f      & y’ad’hóré & lýad’hóré  \\\cline{1-3}\cline{5-7}
3rd n      & ý’ad’hóré & lad’hóré   &     & 3rd n       & ý’ad’hóré & lýad’hóré  \\\cline{1-3}\cline{5-7}
Infinitive & \multicolumn{2}{c|}{\it dad’hóré} & & Infinitive & \multicolumn{2}{c|}{\it ád’hóré} \\\cline{1-3}\cline{5-7}
Participle&\multicolumn{2}{c|}{\it ad’hórâr}&&Participle&\multicolumn{2}{c|}{\it âd’hórár}\\\cline{1-3}\cline{5-7}
\end{tabular}
\caption{Preterite Paradigm of the Verb \emph{ad’hór}.}\label{tab:adhor-paradigm-pret}
\end{table}

\subsubsection{Future I}\label{subsubsec:future-i}
The future tenses, that is, the Future, Future Anterior (a tense similar to the future perfect), as well
as the Conditional, are formed by adding prefixes to the present forms. The prefix is the same in all persons and numbers,
except that there is a separate prefix for the infinitive.

In the Future, much to the UF learner’s dismay, this prefix can go in two separate positions: either before the person marker(s) or
inbetween the person marker(s) and the stem. The former case is more common in speech, while the later is more literary
and strongly preferred in writing and poetry as well as in formal speech. But even in informal speech, the Future I alone
will still not be enough to get by, as the Conditional, a *very* common tense, is formed using the Future II.

First, let us examine the former, simpler case, commonly called the Future I. The prefix is *aú-* if the verb form
after it starts with a consonant (except glides), *aúr-* in all other cases; e.g. *aújad’hór* ‘I shall love’, but
*aúrý’ad’hór* ‘it will love’. In the infinitive passive, it
contracts with the initial *à-* or *á-* to *aú* or *aû*, e.g. *aûd’hór* ‘to be about to be loved’.\footnote{This form
has no direct equivalent in English and is fairly hard to translate on its own.} No contraction happens
if the infinitive starts with *â*, e.g. *aúrânvé* ‘to be about to be animated’. Since
there is little point in writing a table for just the prefixes, Table~\ref{tab:adhor-paradigm-future-1} instead shows the Future I paradigm
of the verb \emph{ad’hór}.

\begin{table}[H]
\centering
\noindent\begin{tabular}{@{}|>{}l|>{\it}l|>{\it}l|>{}l|>{}l|>{\it}l|>{\it}l|}\cline{1-3}\cline{5-7}
\nf Active&\nf Sg&\nf Pl&\nf &\nf Passive&\nf Sg&\nf Pl\\\cline{1-3}\cline{5-7}
1st&aújad’hór&aúrad’hóró&&1st&aúvad’hór&aúrad’hór\\\cline{1-3}\cline{5-7}
2nd&aú\d{}ad’hór&aúb’had’hóré&&2nd&aú\d{}ad’hór&aúb’had’hór\\\cline{1-3}\cline{5-7}
3rd m&aúlad’hór&aúlad’hór&&3rd m&aúry’ad’hór&aúlýad’hór\\\cline{1-3}\cline{5-7}
3rd f&aúllad’hór&aúllad’hór&&3rd f&aúry’ad’hór &aúlýad’hór\\\cline{1-3}\cline{5-7}
3rd n&aúrý’ad’hór&aúlad’hór&&3rd n&aúrý’ad’hór&aúlýad’hór\\\cline{1-3}\cline{5-7}
Infinitive&\multicolumn{2}{c|}{\it aúdad’hór}&&Infinitive&\multicolumn{2}{c|}{\it aûd’hór}\\\cline{1-3}\cline{5-7}
Participle&\multicolumn{2}{c|}{\it aúrad’hórâ}&&Participle&\multicolumn{2}{c|}{\it aúrâd’hór}\\\cline{1-3}\cline{5-7}
\end{tabular}
\caption{Future I Paradigm of the Verb \emph{ad’hór}.}\label{tab:adhor-paradigm-future-1}
\end{table}

\subsubsection{Future II}\label{subsubsec:future-ii}
The Future I paradigm is fairly straight-forward; unfortunately, the Future II is a lot worse: not only do the affixes
vary a lot more, but they are different depending on whether verb form following them starts with a vowel or a consonant.\footnote{This is
not a problem in the Future I, since the prefix is never adjacent to the stem.}
The vocalic and consonantal Future II affixes are shown in Tables~\ref{tab:future-2-vocalic}~and~\ref{tab:future-2-consonantal} below, respectively.

The diachrony of these forms is somewhat unclear—especially that of the participles. It would appear, however, that they result from a coalescence
of the personal pronouns with forms of some auxiliary (likely PF *avoir* and *aller*) as well as the PF future. It appears that
the \s{2sg} is derived from the formal PF \s{2pl} pronoun, which is in line with the fact that the Future II is generally
considered more formal than the almost colloquial Future I. The *v́* in the \s{2pl act} seems to be the result of metathesis.

\begin{table}[H]
\centering
\noindent\begin{tabular}{@{}|>{}l|>{\it}l|>{\it}l|>{}l|>{}l|>{\it}l|>{\it}l|}\cline{1-3}\cline{5-7}
Active&\nf Sg&\nf Pl& & Passive&\nf Sg&\nf Pl\\\cline{1-3}\cline{5-7}
1st   &b’h- -(ẹ)  &náý’- -aú      &&1st    &v- -é    &náý’-       \\\cline{1-3}\cline{5-7}
2nd   &ḍír- -(ẹ)  &b’haý’- -(r)ẹ́  &&2nd    &ḍír-     &b’haý’-     \\\cline{1-3}\cline{5-7}
3rd m &ł-  -(ẹ)   &lb’h- -aú      &&3rd m  &l-       &lb’h- -(r)e \\\cline{1-3}\cline{5-7}
3rd f &èł-  -(ẹ)  &lb’h- -aú      &&3rd f  &l-       &lb’h- -(r)e \\\cline{1-3}\cline{5-7}
3rd n &aúł-  -(ẹ) &lb’h- -aú      &&3rd n  &s-       &lb’h- -(r)e \\\cline{1-3}\cline{5-7}
Infinitive&\multicolumn{2}{c|}{\it d- -è}&&Infinitive&\multicolumn{2}{c|}{\it h-}\\\cline{1-3}\cline{5-7}
Participle&\multicolumn{2}{c|}{\it -ŷr}&&Participle&\multicolumn{2}{c|}{\it á- -ýr}\\\cline{1-3}\cline{5-7}
\end{tabular}
\caption{Vocalic Future II Affixes.}\label{tab:future-2-vocalic}
\end{table}

\begin{table}[H]
\centering
\noindent\begin{tabular}{@{}|>{}l|>{\it}l|>{\it}l|>{}l|>{}l|>{\it}l|>{\it}l|}\cline{1-3}\cline{5-7}
Active&\nf Sg&\nf Pl& & Passive&\nf Sg&\nf Pl\\\cline{1-3}\cline{5-7}
1st   &jaú- -ẹ́  &aúnraû- -aú &&1st   &vaú- -é  &naú-       \\\cline{1-3}\cline{5-7}
2nd   &b’há- -(ẹ) &v́aú- -e     &&2nd   &\d{}á-  &b’haú-     \\\cline{1-3}\cline{5-7}
3rd m &aúr-  -(ẹ) &laú- -aú    &&3rd m &y’aúr-  &laú- -(r)e \\\cline{1-3}\cline{5-7}
3rd f &aúr-  -(ẹ) &laú- -aú    &&3rd f &y’aúr-  &laú- -(r)e \\\cline{1-3}\cline{5-7}
3rd n &aúr-  -(ẹ) &laú- -aú    &&3rd n &saúr-   &laú- -(r)e \\\cline{1-3}\cline{5-7}
Infinitive&\multicolumn{2}{c|}{\it dẹ- -è}&&Infinitive&\multicolumn{2}{c|}{\it haú-}\\\cline{1-3}\cline{5-7}
Participle&\multicolumn{2}{c|}{\it -(r)ŷ}&&Participle&\multicolumn{2}{c|}{\it á- -(r)ý}\\\cline{1-3}\cline{5-7}
\end{tabular}
\caption{Consonantal Future II Affixes.}\label{tab:future-2-consonantal}
\end{table}

\Paragraph{Future Stem}
Many verbs have a different future stem that is used in all future tenses (except the Future I); for example, the future
stem of *vvaúríhe* ‘to remember’, is *vvaúríźe*; thus, we have
*jvvaúríhe* ‘to remember’ but *jaúvvaúríźẹ́* ‘I shall remember’. Note also that these forms already include the
active/passive affixes, which is why it’s *jaúvvaúríźẹ́* and not \**jaújvvaúríźẹ́* or \**jjaúvvaúríźẹ́*.
As in the present, the dictionary form of the future
stem is a verbal noun; thus, *vvaúríźe* roughly means ‘the act of being about to remember’.\footnote{As noted before, infinitive
and gerund forms of future tenses are difficult to translate into English.}

\Paragraph{Stem-final vowel elision and *-(ẹ)*}
The future stem usually ends with a vowel, which is dropped if any future suffix or a suffix that starts with a vowel is added, e.g.
*laúvvaúríźaú* ‘they will remember’, not \**laúvvaúríźeaú*. Note that in the case of future suffixes, even those that start
with a consonant cause the vowel to be dropped. The only exception to this is the suffix *-(ẹ)*, which is found in a number of
Future II forms; that suffix is dropped instead, e.g. *aúrvvaúríźe* ‘she will remember’, not \**aúrvvaúríźẹ*.

\Paragraph{Nasal Stems}
Some future stems are nasalising, which is the case if the final vowel is a nasal vowel; in such cases, that vowel
is still dropped if a suffix is added, but if that suffix starts with a vowel, nasalisation is applied to it, e.g.
in the case of *dír*, whose future stem is *dírẹ́*, we have *aúnraûdíraû* ‘we shall say’: the *-aú* suffix merges
with the nasalisation of the final vowel to become *aû*. Unlike with regular stems, the Future II *-(ẹ)* *does*
replace the final vowel and becomes *-ẹ́* for such verbs, e.g. *aúrdírẹ́* ‘he will say’, and \s{1sg fut pass}
vocalic *-é* becomes *-ê*.

\Paragraph{*r-* Dropping}
Initial *r* in Future II suffixes is dropped if the
last consonant before the final vowel of the future stem is *w*, or an ʁ-coloured consonant such as *ź*, e.g.
*laúvvaúríźe* ‘they will be remembered’, not \**laúvvaúríźre*. If the last consonant of the future stem is *r*, since
any following vowel (whether nasalised or not) is deleted when a Future II suffix is added, the final *r* of the stem and
the initial *-r* of the Future II suffixes that have one coalesce to *rr*, e.g. *b’haý’ad’hórérre* ‘you (\s{pl}) will
love’.

\Paragraph{Affix Stacking}
Note that when more than one affix is used, at most one can be a future affix, e.g. *jaúsyvvaúríźẹ́* ‘I shall remember it’
and not \**jaúsaúrvvaúríźẹ́*. Generally, the active prefix will be the future affix, but it is possible to use the
passive future affixes instead for emphasis e.g. *jy’aúrvvaúríźe* roughly ‘him, I shall remember’; often, this is
also used to aid in establishing a contrast to some other part of the sentence that does not have this inversion.

Since some of the passive future affixes also have suffix parts—unlike the present affixes, where the passive forms are
all prefixes—we can end up with multiple suffixes in addition to multiple prefixes, in which case active prefixes, instead
of simply preceding the passive ones, can be thought of as effectively ‘wrapping’ them, e.g. *aúlaúvvaúríźey’ó* ‘we shall
remember them’, which contains *laúvvaúríźe* ‘they will be remembered’.

Finally, as always, infinitive prefixes come first. If combined with other affixes, it will generally be the future affix,
e.g. *haúlývvaúríźe* roughly ‘to be about to remember them’ but, as with passive affixes, variations are possible for emphasis
or contrastive power, e.g. *dẹlaúvvaúríźe*, which puts more emphasis on ‘them’.

\Paragraph{Examples}
Table~\ref{tab:future-2-adhor} below shows the complete (vocalic) Future II paradigm of the verb *ad’hór* ‘to love’, and
Table~\ref{tab:future-2-vvaurihe} the complete (consonantal) Future II paradigm of II *vvaúríhe* ‘to remember’; recall
that the future stems of these verbs are *ad’hórérẹ́* and *vvaúríźe*.

\begin{table}[H]
\centering
\noindent\begin{tabular}{@{}|>{}l|>{\it}l|>{\it}l|>{}l|>{}l|>{\it}l|>{\it}l|}\cline{1-3}\cline{5-7}
Active&\nf Sg&\nf Pl& & Passive&\nf Sg&\nf Pl\\\cline{1-3}\cline{5-7}
1st   &b’had’hórérẹ́  &náý’ad’hóréraû      &&1st    &vad’hórérệ    &náý’ad’hórérẹ́   \\\cline{1-3}\cline{5-7}
2nd   &ḍírad’hórérẹ́  &b’haý’ad’hórérrẹ́    &&2nd    &ḍírad’hórérẹ́  &b’haý’ad’hórérẹ́ \\\cline{1-3}\cline{5-7}
3rd m &ład’hórérẹ́    &lb’had’hóréraû      &&3rd m  &lad’hórérẹ́    &lb’had’hórérre  \\\cline{1-3}\cline{5-7}
3rd f &èład’hórérẹ́   &lb’had’hóréraû      &&3rd f  &lad’hórérẹ́    &lb’had’hórérre  \\\cline{1-3}\cline{5-7}
3rd n &aúład’hórérẹ́  &lb’had’hóréraû      &&3rd n  &sad’hórérẹ́    &lb’had’hórérre  \\\cline{1-3}\cline{5-7}
Infinitive&\multicolumn{2}{c|}{\it dad’hóréré}&&Infinitive&\multicolumn{2}{c|}{\it had’hórérẹ́}\\\cline{1-3}\cline{5-7}
Participle&\multicolumn{2}{c|}{\it ad’hórérŷr}&&Participle&\multicolumn{2}{c|}{\it ád’hórérýr}\\\cline{1-3}\cline{5-7}
\end{tabular}
\caption{Vocalic Future II Paradigm of *ad’hór*.}\label{tab:future-2-adhor}
\end{table}

\begin{table}[H]
\centering
\noindent\begin{tabular}{@{}|>{}l|>{\it}l|>{\it}l|>{}l|>{}l|>{\it}l|>{\it}l|}\cline{1-3}\cline{5-7}
Active&\nf Sg&\nf Pl& & Passive&\nf Sg&\nf Pl\\\cline{1-3}\cline{5-7}
1st   &jaúvvaúríźẹ́  &aúnraûvvaúríźaú &&1st   &vaúvvaúríźé    &naúvvaúríźe       \\\cline{1-3}\cline{5-7}
2nd   &b’hávvaúríźẹ &v́aúvvaúríźe     &&2nd   &\d{}ávvaúríźe  &b’haúvvaúríźe     \\\cline{1-3}\cline{5-7}
3rd m &aúrvvaúríźẹ  &laúvvaúríźaú    &&3rd m &y’aúrvvaúríźe  &laúvvaúríźe \\\cline{1-3}\cline{5-7}
3rd f &aúrvvaúríźẹ  &laúvvaúríźaú    &&3rd f &y’aúrvvaúríźe  &laúvvaúríźe \\\cline{1-3}\cline{5-7}
3rd n &aúrvvaúríźẹ  &laúvvaúríźaú    &&3rd n &saúrvvaúríźe   &laúvvaúríźe \\\cline{1-3}\cline{5-7}
Infinitive&\multicolumn{2}{c|}{\it dẹvvaúríźè}&&Infinitive&\multicolumn{2}{c|}{\it haúvvaúríźe}\\\cline{1-3}\cline{5-7}
Infinitive&\multicolumn{2}{c|}{\it vvaúríźŷ}&&Infinitive&\multicolumn{2}{c|}{\it ávvaúríźý}\\\cline{1-3}\cline{5-7}
\end{tabular}
\caption{Consonantal Future II Paradigm of *vvaúríhe*.}\label{tab:future-2-vvaurihe}
\end{table}

\subsubsection{Future Anterior}
The Future Anterior tense is formed by combining the Future II and the Present Anterior affixes. The \s{pres ant} suffixes
are applied after the \s{fut ii} affixes. The vocalic and consonantal affixes are shown in
Tables~\ref{tab:future-anterior-vocalic}~and~\ref{tab:future-anterior-consonantal}.

\begin{table}[H]
\centering
\noindent\begin{tabular}{@{}|>{}l|>{\it}l|>{\it}l|>{}l|>{}l|>{\it}l|>{\it}l|}\cline{1-3}\cline{5-7}
Active&\nf Sg&\nf Pl& & Passive&\nf Sg&\nf Pl\\\cline{1-3}\cline{5-7}
1st   &b’h- -\L é  &náý’- -aúrâ      &&1st    &v- -\L ê    &náý’- -\L â      \\\cline{1-3}\cline{5-7}
2nd   &ḍír- -\L á  &b’haý’- -(r)ệḍ   &&2nd    &ḍír- -\L á  &b’haý’- -\L áḍ    \\\cline{1-3}\cline{5-7}
3rd m &ł-   -\L á  &lb’h- -aûr       &&3rd m  &l- -\L á    &lb’h- -(r)ér \\\cline{1-3}\cline{5-7}
3rd f &èł-  -\L á  &lb’h- -aûr       &&3rd f  &l- -\L á    &lb’h- -(r)ér \\\cline{1-3}\cline{5-7}
3rd n &aúł- -\L á  &lb’h- -aûr       &&3rd n  &s- -\L á    &lb’h- -(r)ér \\\cline{1-3}\cline{5-7}
Infinitive&\multicolumn{2}{c|}{\it d- -á}&&Infinitive&\multicolumn{2}{c|}{\it h- -á}\\\cline{1-3}\cline{5-7}
Participle&\multicolumn{2}{c|}{\it -ŷrér}&&Participle&\multicolumn{2}{c|}{\it á- -ýrér}\\\cline{1-3}\cline{5-7}
\end{tabular}
\caption{Vocalic Future Anterior Affixes.}\label{tab:future-anterior-vocalic}
\end{table}

\begin{table}[H]
\centering
\noindent\begin{tabular}{@{}|>{}l|>{\it}l|>{\it}l|>{}l|>{}l|>{\it}l|>{\it}l|}\cline{1-3}\cline{5-7}
Active&\nf Sg&\nf Pl& & Passive&\nf Sg&\nf Pl\\\cline{1-3}\cline{5-7}
1st   &jaú-  -\L ệ  &aúnraû- -aúrâ  &&1st   &vaú- -\L ê    &naú- -\L â      \\\cline{1-3}\cline{5-7}
2nd   &b’há- -\L á  &v́aú- -éḍ       &&2nd   &\d{}á- -\L á  &b’haú- -\L áḍ    \\\cline{1-3}\cline{5-7}
3rd m &aúr-  -\L á  &laú- -aûr      &&3rd m &y’aúr- -\L á  &laú- -(r)ér \\\cline{1-3}\cline{5-7}
3rd f &aúr-  -\L á  &laú- -aûr      &&3rd f &y’aúr- -\L á  &laú- -(r)ér \\\cline{1-3}\cline{5-7}
3rd n &aúr-  -\L á  &laú- -aûr      &&3rd n &saúr-  -\L á  &laú- -(r)ér \\\cline{1-3}\cline{5-7}
Infinitive&\multicolumn{2}{c|}{\it dẹ- -á}&&Infinitive&\multicolumn{2}{c|}{\it haú- -á}\\\cline{1-3}\cline{5-7}
Participle&\multicolumn{2}{c|}{\it -(r)ŷr}&&Participle&\multicolumn{2}{c|}{\it á- -(r)ýr}\\\cline{1-3}\cline{5-7}
\end{tabular}
\caption{Consonantal Future Anterior Affixes.}\label{tab:future-anterior-consonantal}
\end{table}

\noindent
Note that again, nasalised stems add another level of nasalisation, and vowel-dropping still applies, but
this time, there is no *-ẹ* dropping, since none of the affixes end with *ẹ* anymore.

\Paragraph{Coalescence}
All vowel suffixes coalesce with the final vowel of the stem; if the suffix vowel is nasal, a level of nasalisation is
added, e.g. *aúrvvaúrízá* ‘he will have remembered’ from the future stem *vvaúríźe*. Note also that the *ź* is lenited
to *z*; the quality of the suffix vowel overrides that of the stem vowel. *r* contraction still happens as in the
Future II.

Tables~\ref{tab:future-ant-adhor}~and~\ref{tab:future-ant-vvaurihe} below show
the paradigm of the verbs *ad’hór* ‘to love’ and *vvaúríhe* ‘to remember’ in the Future Anterior tense. Note that
both the rules for the Future Anterior tense as well as the Present Anterior tense apply here.

\begin{table}[H]
\centering
\noindent\begin{tabular}{@{}|>{}l|>{\it}l|>{\it}l|>{}l|>{}l|>{\it}l|>{\it}l|}\cline{1-3}\cline{5-7}
Active&\nf Sg&\nf Pl& & Passive&\nf Sg&\nf Pl\\\cline{1-3}\cline{5-7}
1st   &b’had’hórérệ  &náý’ad’hóréraûrâ     &&1st    &vad’hórérệ    &náý’ad’hórérậ   \\\cline{1-3}\cline{5-7}
2nd   &ḍírad’hórérậ  &b’haý’ad’hórérrệḍ    &&2nd    &ḍírad’hórérậ  &b’haý’ad’hórérậḍ \\\cline{1-3}\cline{5-7}
3rd m &ład’hórérậ    &lb’had’hóréraûr      &&3rd m  &lad’hórérậ    &lb’had’hórérrér  \\\cline{1-3}\cline{5-7}
3rd f &èład’hórérậ   &lb’had’hóréraûr      &&3rd f  &lad’hórérậ    &lb’had’hórérrér  \\\cline{1-3}\cline{5-7}
3rd n &aúład’hórérậ  &lb’had’hóréraûr      &&3rd n  &sad’hórérậ    &lb’had’hórérrér  \\\cline{1-3}\cline{5-7}
Infinitive&\multicolumn{2}{c|}{\it dad’hórérâ}&&Infinitive&\multicolumn{2}{c|}{\it had’hórérậ}\\\cline{1-3}\cline{5-7}
Participle&\multicolumn{2}{c|}{\it ad’hórérŷrér}&&Participle&\multicolumn{2}{c|}{\it ád’hórérýrér}\\\cline{1-3}\cline{5-7}
\end{tabular}
\caption{Vocalic Future Anterior Paradigm of *ad’hór*.}\label{tab:future-ant-adhor}
\end{table}

\begin{table}[H]
\centering
\noindent\begin{tabular}{@{}|>{}l|>{\it}l|>{\it}l|>{}l|>{}l|>{\it}l|>{\it}l|}\cline{1-3}\cline{5-7}
Active&\nf Sg&\nf Pl& & Passive&\nf Sg&\nf Pl\\\cline{1-3}\cline{5-7}
1st   &jaúvvaúrízệ  &aúnraûvvaúríźaúrâ &&1st   &vaúvvaúrízê    &naúvvaúrízâ       \\\cline{1-3}\cline{5-7}
2nd   &b’hávvaúrízá &v́aúvvaúríźéḍ     &&2nd   &\d{}ávvaúrízá  &b’haúvvaúrízáḍ     \\\cline{1-3}\cline{5-7}
3rd m &aúrvvaúrízá  &laúvvaúríźaûr    &&3rd m &y’aúrvvaúrízá  &laúvvaúríźér \\\cline{1-3}\cline{5-7}
3rd f &aúrvvaúrízá  &laúvvaúríźaûr    &&3rd f &y’aúrvvaúrízá  &laúvvaúríźér \\\cline{1-3}\cline{5-7}
3rd n &aúrvvaúrízá  &laúvvaúríźaûr    &&3rd n &saúrvvaúrízá   &laúvvaúríźér \\\cline{1-3}\cline{5-7}
Infinitive&\multicolumn{2}{c|}{\it dẹvvaúríźá}&&Infinitive&\multicolumn{2}{c|}{\it haúvvaúríźá}\\\cline{1-3}\cline{5-7}
Infinitive&\multicolumn{2}{c|}{\it vvaúríźŷr}&&Infinitive&\multicolumn{2}{c|}{\it ávvaúríźýr}\\\cline{1-3}\cline{5-7}
\end{tabular}
\caption{Consonantal Future Anterior Paradigm of *vvaúríhe*.}\label{tab:future-ant-vvaurihe}
\end{table}

\subsubsection{Conditional I and II}\label{subsubsec:conditional}
The Conditional tenses are fairly simple—so long as you know the Future II and Future Anterior, that is. Both Conditionals
are formed by adding the *-ss(a)-* infix between the Future II stem and any suffixes. The Conditional I is formed from
the Future II, and the Conditional II from the Future Anterior. As always, the vowel is omitted if
the suffix after the infix starts with a vowel, except for *ẹ*, which it replaces. For instance, Table~\ref{tab:cond-ii-vvaurihe}
below shows the consonantal Conditional II paradigm of *vvaúríhe* ‘to remember’. Note that the *ss* in this form
are *never* lenited:

\begin{table}[H]
\tabcolsep4pt
\centering
\noindent\begin{tabular}{@{}|>{}l|>{\it}l|>{\it}l|>{}l|>{}l|>{\it}l|>{\it}l|}\cline{1-3}\cline{5-7}
Active&\nf Sg&\nf Pl& & Passive&\nf Sg&\nf Pl\\\cline{1-3}\cline{5-7}
1st   &jaúvvaúríźessệ  &aúnraûvvaúríźessaúrâ &&1st   &vaúvvaúríźessê    &naúvvaúríźessâ       \\\cline{1-3}\cline{5-7}
2nd   &b’hávvaúríźessá &v́aúvvaúríźesséḍ      &&2nd   &\d{}ávvaúríźessá  &b’haúvvaúríźessáḍ     \\\cline{1-3}\cline{5-7}
3rd m &aúrvvaúríźessá  &laúvvaúríźessaûr     &&3rd m &y’aúrvvaúríźessá  &laúvvaúríźessrér \\\cline{1-3}\cline{5-7}
3rd f &aúrvvaúríźessá  &laúvvaúríźessaûr     &&3rd f &y’aúrvvaúríźessá  &laúvvaúríźessrér \\\cline{1-3}\cline{5-7}
3rd n &aúrvvaúríźessá  &laúvvaúríźessaûr     &&3rd n &saúrvvaúríźessá   &laúvvaúríźessrér \\\cline{1-3}\cline{5-7}
Inf&\multicolumn{2}{c|}{\it dẹvvaúríźessá}&&Inf&\multicolumn{2}{c|}{\it haúvvaúríźesse}\\\cline{1-3}\cline{5-7}
Inf&\multicolumn{2}{c|}{\it vvaúríźessŷr}&&Inf&\multicolumn{2}{c|}{\it ávvaúríźessý}\\\cline{1-3}\cline{5-7}
\end{tabular}
\caption{Consonantal Conditional II Paradigm of *vvaúríhe*.}\label{tab:cond-ii-vvaurihe}
\end{table}

\noindent The conditional tenses are mainly used in the apodoses of conditional clauses. On its own, their meaning
is similar to that of the English ‘would’ or ‘could’, e.g. *jaúvvaúríźessẹ́* ‘I would love’; the Conditional II,
even though it is morphologically a future tense, is used to express a hypothetical past, e.g. *jaúvvaúríźessệ*
‘I would have loved’.

\subsection{Miscellaneous Forms}
\subsubsection{The Gnomic}\label{subsubsec:gnomic}
The gnomic tense is marked by the infix *-j(ú)-* after the stem: *ad’hór* ‘to love’ to *rad’hórjô* ‘We love (for ever)’.
The *ú* is omitted if the infix is followed by the vowel, in which case it causes nasalisation.

\subsubsection{Dative Affixes}
The dative affixes *-vé* ‘me, us’, *-b’hẹ* ‘you’, and *-ḷẹ* ‘him, her, it, them’ are only used in conjunction with
ditransitive verbs and invariant to
tense, gender, number, and mood. A verb can only have one dative affix, and the dative affix is always placed last after all other
affixes and does not coalesce, lenite, or otherwise modify the rest of the verb, e.g. *dedónẹ́* ‘to bestow’ to
*dedónẹ́ḷẹ* ‘to bestow upon him’.

\subsubsection{Imperative}
The imperative mood exists only in the present tense, and only in the second and third person. It is formed by
affixing the following suffixes to the stem.
\begin{table}[H]
\centering
\noindent\begin{tabular}{@{}|>{}l|>{\it}l|>{\it}l|>{}l|>{}l|>{\it}l|>{\it}l|}\cline{1-3}\cline{5-7}
 Active&\nf Sg&\nf Pl& & Passive&\nf Sg&\nf Pl\\\cline{1-3}\cline{5-7}
2nd &c’h(e)-     &c’heb’h(y)- &&2nd& -rá   &-nú\\\cline{1-3}\cline{5-7}
3rd &\multicolumn{2}{c|}{\it c’hel(ẹ)-} &&3rd& -ḷẹ   &-b’hẹ\\\cline{1-3}\cline{5-7}
\end{tabular}
\caption{Imperative affixes.}\label{tab:imperative-affixes}
\end{table}

\noindent The diachrony of these forms is likely from subjunctive constructions with \s{pf} \**que* in the active
and from suffixed pronouns in the passive. Note that imperative affixes are added *in place* of
present active/passive affixes, e.g. *c’hedír!* ‘speak!’, not \**c’heḍẹdír*. As usual, the parenthesised
vowels are omitted if the verb form starts with a vowel, e.g. *c’had’hór!* ‘love!’.

Imperative affixes can be combined with active/passive affixes, though, as usual, an active imperative prefix
can only be paired with a passive present affix, and vice versa. Active imperative prefixes are always placed
first, e.g. *c’hevad’hór!* ‘love me!’, and passive affixes are placed last, e.g. *b’had’hórérá* ‘be loved by
us!’. The negation of the imperative uses the subjunctive and is explained in §~\ref{subsubsec:negated-subjunctive}.

\subsection{Subjunctive}\label{subsec:subjunctive}
The UF subjunctive forms are fortunately fairly simple: they use the same affixes as the present, past, and future
forms, except that each verb has a different subjunctive stem as well as a future subjunctive stem; the subjunctive
stem is typically formed by adding an *-s* to the end of the corresponding indicative stem, e.g. *ad’hór*
‘to love’ to *ad’hórs*; thus we have, e.g. *jad’hórs* ‘I may love’, and *rád’hórsó* ‘We may love’.

The future subjunctive stem is formed by adding the desinence *-śe* to the end of the future stem. For example,
the future stem of *ad’hór* is *ad’hórérẹ́*, so the future subjunctive stem is *ad’hórérẹ́śe*; similarly, the future
stem of *vvaúríhe* is *vvaúríźe*, so the future subjunctive stem is *vvaúríźeśe*.
There are several main uses of the UF subjunctive, each of which we shall examine in more detail below:
\begin{enum}
\item in reported speech, e.g. *lladírá vad’hórhé* ‘she said she loved me’;
\item with certain subordinating conjunctions, such as *b’he* ‘so that’;
\item to express deontic modality, e.g. *ḍẹḅars* ‘you may leave’;
\item as a jussive, e.g. *rad’hesó* ‘let’s go’;
\item as a negative imperative, e.g. *sá ḍẹḅars* ‘don’t leave’;
\item irrealis conditionals (see §~\ref{subsec:conditionals}).
\item in \s{aci}s and \s{pci}s.
\end{enum}

\subsubsection{Reported Speech}
UF does not use backshifting in reported speech, but rather, the corresponding subjunctive form is used. For instance,
*jḍad’hór* ‘I love you’ becomes *jdíré jḍad’hórs* ‘I said I love you’. Note that the tense stays the same in this
example: present indicative becomes present subjunctive. Accordingly, *jḍad’hóré* ‘I loved you’ becomes *jdíré
jḍad’hórsé* ‘I said I loved you’.

Consequently, the tense of the verb in reported speech is independent of the tense of the matrix clause, e.g.
*b’had’hrệ* ‘I shall go’ becomes *jdíré b’had’hrẹ́sé* ‘I said I would go’,\footnote{Note the lenition here because
of the present anterior suffix: *b’had’hrẹ́sé*, not \**b’had’hrẹ́śé*.} with *b’had’hrẹ́sé* being the Future II
subjunctive form of *b’had’hrẹ́*.

\subsubsection{Dependent Clauses}
The following subordinating conjunctions take the subjunctive:
\TwoCols[.45\hsize][.45\hsize] {
\begin{dlist}[\bfseries\itshape]
    \item[áhaúr] ‘even though’
    \item[ḅas] ‘because’
    \item[b’he] ‘so that’
    \item[c’haúr] ‘as’ (viz. ‘because’)
    \item[daúc’h] ‘therefore’
    \item[de] ‘once’
\end{dlist}
} {
\begin{dlist}[\bfseries\itshape]
    \item[ráhẹ] ‘though’
    \item[rê] ‘although’
    \item[s] ‘if’ (see §~\ref{subsec:conditionals})
    \item[sá] ‘without’
    \item[sauc’h] ‘except that’
    \item[váłé] ‘despite that’
\end{dlist}
}

\noindent Note that not all subordinating conjunctions take the subjunctive. For instance, the conjunction *y’is*
‘because’ takes the indicative: *jḍad’hórs c’haúr* ‘as I love you’, but *jḍad’hór y’ís* ‘because I love you’.

\subsubsection{Deontic Modality}
The subjunctive can also be used on its own, in which case it assumes a deontic or jussive meaning;
in the first person, it is generally a jussive, e.g. *rad’hesó* ‘let’s go’, but the jussive sense is not restricted
to the first person, e.g. *lẹsyrét’hes* ‘he take care of it’ (in the sense of ‘let him take care of it’).

The deontic sense is also apparent from that last example: *lẹsyrét’hes* can also be interpreted to mean ‘he
may take care of it’, which can either be a statement of permission or a condescending order. Note that even
though UF also has a word for ‘let’ (namely *le*), it is mostly used in questions or commands, while the
deontic subjunctive is used to grant permission.

\subsubsection{Negation}\label{subsubsec:negated-subjunctive}
The subjunctive is negated with the particle *sá*, rather than with *asý’ýâ*. The particle *sá* is placed
immediately before the verb form it negates, e.g. *sá jḍad’hórs c’haúr* ‘as I don’t love you’. It is reduced
to *s’* before vowels, but interestingly, it does not cause nasalisation in that case, e.g. *s’aúsydíssâ c’haúr*
‘as we didn’t say it’.

On its own, the negated subjunctive is used to express a negative imperative in the second and third person,
e.g. *sá ḍẹḅars* ‘don’t leave’, and a negative jussive in the first person e.g. *sá rad’hesó*, ‘let’s not go’.

\subsubsection{Infinitive}
Most curiously, UF has a *subjunctive infinitive*. This form is almost exclusively used to express deontic modality
in \s{aci}s and \s{pci}s. For example, the form *dad’hórs*, the subjunctive infinitive of *ad’hór*, while defying any attempt
at translation on its own,\footnote{The best attempt one could make to translate this would be something along the
lines of ‘to should love’, but that is not exactly grammatical in English.} can be translated as ‘should’ when combined
with an \s{acc} or \s{part}, e.g. *sráhó dad’hórs* roughly means ‘that fish should love’, though this form can only
occur as the complement of a verb.

\subsection{Optative}\label{subsec:optative}
The UF optative is used to express wishes, hopes, as well as in certain conditional constructions. It is formed
by affixing *y’(ẹ)\L* to the verb stem, e.g. *dẹvy’ẹvvaúríhe* ‘may you remember me’. Some prefixes in the future end with
*ý’*; this is dropped in the optative: e.g. *náý’ad’hóraú* ‘we shall love’ becomes *náy’ad’hóraú* ‘may we love’. Note
that the bare optative is difficult to translate into English; a more precise explanation of what these forms actually
mean will be given below. Uses of the optative include:
\begin{enum}
\item wishes, hopes, dreams, and aspirations;
\item with certain subordinating conjunctions, such as *auha* ‘in case’;
\item talking about fears;
\item counterfactual conditionals (see §~\ref{subsec:conditionals}).
\end{enum}

\subsubsection{Wishes and Hopes}
The most traditional use of the optative is to express wishes and hopes, e.g. *dẹvy’ẹvvaúríhe* ‘may you remember me’. In
the present or future tense, this use indicates a wish for something to happen; in the present tense, its meaning is
that of a wish for a condition to be true in the present in the face of uncertainty or lack of knowledge; thus, the
actual meaning of *dẹvy’ẹvvaúríhe* is roughly ‘I hope that you remember me’.\footnote{The context of this utterance could be
meeting someone again after a long time apart and hoping that they still remember you.} In the future tense, it indicates a wish
that a situation will be true in the future, e.g. *b’hávy’ẹvvaúríźe* ‘may you remember me’.

In the past tenses, the optative indicates dismay, regret, or disappointment that something did not happen, e.g.
\s{pres ant} *dẹvy’ẹvvaúríhá* ‘if only you had remembered me’. The optative can also be combined with the Conditional I
to convey uncertainty about a future wish, as well as with the Conditional II to express extreme regret over a past event.

\subsubsection{Dependent Clauses}
The following subordinating conjunctions take the optative:
\TwoCols[.45\hsize][.45\hsize] {
\begin{dlist}[\bfseries\itshape]
    \item[auha] ‘in case’
    \item[ab’há] ‘before’
    \item[ávrê] ‘unless’
    \item[ḅré] ‘after’
\end{dlist}
}{
\begin{dlist}[\bfseries\itshape]
    \item[fahaú] ‘in such a way that’
    \item[jys] ‘until’
    \item[sit’há] ‘supposing that’
    \item[úrbh] ‘provided that’
\end{dlist}
}

\subsubsection{Negation and Verbs of Fearing}
As with the negated subjunctive, the negated optative also has a separate negation particle, namely *t’hé*
(spelt *t’h’\N* before vowels). Note that a negated optative indicates that the speaker wishes that something does
or had not happened, e.g. *t’hé dẹvy’ẹvvaúríhá* ‘if only you had not remembered me’. The negation thus negates
the wish, and not the act of wishing; for the latter, the indicative or subjunctive together with a verb such
as *sḅé* ‘to wish’ are used instead.

Verbs of fearing are typically construed with a dependent clause in the negated optative, e.g. *jréd’hé
t’hé b’háy’ẹbharẹ́* ‘I was afraid lest you might leave’.

\subsection{Irregular Verbs}\label{subsec:irregular-verbs}
\subsubsection{The Conjugation of \textit{eḍ} ‘to be’}


\begin{table}[H]
\centering
\noindent\begin{tabular}{@{}|>{}l|>{\it}l|>{\it}l|l|l|>{\it}l|>{\it}l|l|l|>{\it}l|>{\it}l|}\cline{1-3}\cline{5-7}\cline{9-11}
Present&\nf Sg&\nf Pl    && Pres. Ant.&\nf Sg&\nf Pl    && Preterite&\nf Sg&\nf Pl      \\\cline{1-3}\cline{5-7}\cline{9-11}
1st       & vy’í & aúsó   && 1st      & vẹ     & aúfý   && 1st       & vet’h & weḍy’ó   \\\cline{1-3}\cline{5-7}\cline{9-11}
2nd       & ḍe   & b’heḍ && 2nd       & ḍyf    & b’hu   && 2nd       & ḍet’h & b’heḍy’é \\\cline{1-3}\cline{5-7}\cline{9-11}
3rd m     & le   & lẹsó  && 3rd m     & leb’h  & lẹfýr  && 3rd m     & let’h & let’he   \\\cline{1-3}\cline{5-7}\cline{9-11}
3rd f     & lle  & llẹsó && 3rd f     & lle’bh & llẹfýr && 3rd f     & llet’h & llet’he \\\cline{1-3}\cline{5-7}\cline{9-11}
3rd n     & se    & lasó  && 3rd n     & seb’h  & lafýr  && 3rd n     & set’h & laet’h   \\\cline{1-3}\cline{5-7}\cline{9-11}
Infinitive& \multicolumn{2}{c|}{\it éḍ} && Infinitive& \multicolumn{2}{c|}{\it éfyḍ} && Infinitive& \multicolumn{2}{c|}{\it ét’hẹd} \\\cline{1-3}\cline{5-7}\cline{9-11}
\end{tabular}
\caption{Paradigm of the verb \emph{eḍ}.}\label{tab:ed-paradigm}
\end{table}

\noindent The etymology of these forms is mostly from a gradual simplification of coalesced forms of the personal
pronouns with the PF copula. To compensate for the fact that PF lacks certain forms that are present in UF, some
of the forms were coined by analogy. For instance, the \s{pres ant inf} *éfyḍ* is derived from the \s{pres ant}
stem \**fy* and the \s{pres inf} *éḍ*, and the same is true for the \s{pret inf} *ét’hẹd*.

For obvious reasons, the copula lacks passive forms. At the same time, the first person forms are manifestly
derived from the first person passive pronoun, for unknown reasons.

Unlike nearly every other word in the language, all forms of the copula are summarily stressed on the first
syllable.


\subsection{Summary of Coalescence Rules}
When vowels collide at morpheme boundaries, though chiefly in suffixes, they often coalesce into a
single vowel that depends on the qualities and nasality of the two vowels. How exactly this coalescence
works depends on the morphemes in question, but generally, there are 4 overarching principles to be
aware of:
\begin{enumerate}
\item Vowels at the end of a suffix or at the beginning of a prefix may simply be omitted instead;
      this is particularly common in verb forms.
\item If one of the vowels is *ẹ*, it is dropped; the resulting vowel is the other vowel.
\item If one of the vowels is nasalised, the resulting vowel is generally also nasalised; if both
      vowels are nasalised or nasal, the resulting vowel will be nasal.
\item If the first vowel is part of a verb stem, it is often simply deleted or at least overridden
      by the second vowel in terms of quality.
\end{enumerate}

\noindent The following table lists all coalescence rules in the language. For more information, see the corresponding
sections in which the forms in question are introduced. Note that trivial cases of vowels being dropped entirely
are not listed in this table.

Unless otherwise indicated, vowel letters, e.g. *o*, represent any variant of that vowel, whether oral, nasalised, or nasal.
*o* also includes variants of *au* (e.g. *aú*, but not *áu*, of course, as those are two vowels and not a digraph).
Subscripts may be used to track nasalisation, and a + sign indicates a level of nasalisation is added. Since glides also
influence contractions in some cases, they are included in this table. In the case of the abbreviations V and G, if there
is no +, nasalisation is preserved.

Rules are matched top-down: the first matching rule is applied, all others are ignored. The affix in the 2nd column next to the
vowel(s) in the 3rd column in the same row are replaced with the letters of the 4th column in the indicated forms at the
indicated morpheme boundary in the 1st column. The position in the hyphen in the 2nd column indicates whether it coalesces
with vowels before or after it. The letters V (as well as V′) stands for ‘any vowel’. The letter G stands for
‘any glide’. Abbreviations may be used where applicable (e.g. *-e(r)* for *-e* and *-er* if there is no single *-e(r)*
suffix in the paradigm in question). The abbreviation -? means ‘any other suffix in this paradigm, even if they
start with a consonant’.

\medskip
{\centering
\noindent
\begin{tabular}{@{}>{\scshape}l|>{\it}p{5em}|>{\it}l|>{\it}l|l}
\nf Form    & \nf Affix & \nf Phonemes & \nf Result & Reference \\\hline\hline
pres 1pl          & aú-     & o\Sub α  & wo\Sub α & §~\ref{subsubsec:active-passive-affixes} \\
                  &         &\nf V & r\nf V \\\hline
pres 1pl act      & -y’ó    & o\Sub α  & o\Sub{α+} &ibid. \\\hline
pres 2pl          & b’h(y)- &\nf V & b’h\nf V &ibid. \\
                  &         &\nf G & b’h\nf G \\\hline
pres 2pl act      & -y’é    & e\Sub α  & e\Sub{α+} &ibid.\\\hline
pres inf pass     & à-      & a\Sub α& a\Sub {α+}&ibid. \\
                  &         &\nf V & h\nf V \\\hline
pres part         & -â, â-  & ẹ & â &ibid. \\
                  &         &\nf V̂ & {\nf V̂}â, â{\nf V̂} \\
                  &         &\nf V &\nf V̂ \\\hline\hline
pres ant, pret    & -é(r)   & ẹ & é(r) &§~\ref{subsubsec:suffixed-tenses}\\
                  &         &{\nf V}\Sub α & e\Sub{α+}(r) \\
                  & -á(r), -áḍ  & ẹ & á(r), áḍ &ibid. \\
                  &         &{\nf V}\Sub α & a\Sub{α+}(r), a\Sub{α+}ḍ \\\hline
pres ant 1pl act  & -â      &\nf V & â & ibid.\\\hline\hline
fut i inf pass    & aú(r)-  & â & aúrâ & §~\ref{subsubsec:future-i} \\
                  &         &\nf G & aúr\kern1pt\nf G \\
                  &         & à & aú \\
                  &         & á & aû \\\hline
fut ii, fut ant,  & -ẹ      &\nf V &\nf V &§§~\ref{subsubsec:future-ii}–\ref{subsubsec:conditional}\\
cond i, cond ii   &{\nf -V}\Sub α\nf & ẹ & {\nf -V}\Sub α\nf \\
                  &         & {\nf V′}\Sub{β} & {\nf -V}\Sub{α+β} \\
                  &\nf -?   &{\nf stem} ẹ, ẹ́ &\nf -? \\
gnomic            & j(ú)-   & {\nf V}\Sub α& j\kern1pt{\nf V}\Sub{α+} &§~\ref{subsubsec:gnomic} \\\hline
\end{tabular}\par}
\medskip

\noindent Lastly, note that the 4 principles mentioned earlier are guidelines, not rules. There are cases of affixes that do not coalesce
at all, e.g. the comparative prefix *lẹ* (see §~\ref{subsubsec:comparison}). If a form is not listed in this table, then,
unless explicitly stated where that form is introduced (in which case case we simply forgot to include it in the table),
it does not coalesce at all. Furthermore, this table only handles coalescence rules between vowels and some vowel elision rules; other
elision rules are either very regular or have nothing to do with adjacent vowels. This table exists only because coalescence rules are
very similar, but sometimes subtly different.

\section{Syntax}\label{sec:syntax}
UF syntax is unfortunately complicated in what morphological constructs are used in what situations, and
the rules are not always clear. The following is a list of the most common constructions.

\subsection{Independent Clauses}
The UF independent clause typically consists of a finite verb together with a subject perhaps several
objects. The verb is conjugated to agree with the subject in person, number, and gender in some cases.

\gloss {
    Rab’hadó iárb.
    r-ab’haḍ-ó| i-árb
    \s{1pl}-fell-\s{1pl}| \s{acc}-tree
    ‘We are felling the tree.’
}

The unmarked tense in UF is the present tense, which can generally be translated as either a present or
present continuous tense in English. For general truths and facts, the gnomic tense is generally used
instead.

\gloss {
    Rab’hadjô sárb.
    r-ab’haḍ-jô| s-árb
    \s{1pl}-fell-\s{gnomic}\Sl\s{1pl}| \s{acc.indef}-tree
    ‘We fell trees.’
}

The object is incorporated into the verb if it is a personal pronoun, in which case there are rules for
the order in which these affixes occur (see Section~\ref{subsec:verbal-morphology}).

\gloss {
    Lerab’hat’há.
    lẹ-r-ab’ha\Sl t’há.
    \s{3sgm}-\s{1pl.pass}-fell\Sl\s{3sg.pres.ant}
    ‘He felled us.’
}

Word order is rather lax due to the presence of case marking, and any constituent can be fronted for
emphasis, but the default word order is SVO or SOV.

\gloss {
    B’hehýnác aúlýab’hat’hâ.
    b’hehýn-ác| aú-lý-ab’ha\Sl t’hâ.
    \s{instr.indef}-axe|\s{1pl}-\s{3pl.pass}-fell\Sl\s{1pl.pres.ant}
    ‘With an axe, we have felled them.’
}

Note that words belonging to the same phrase are typically juxtaposed as adjectives are not inflected. However,
this rule may sometimes be broken, particularly in poetry. Consider, for example, the following passage in alexandrine
metre, written by the renowned poet \s{J.\,Y.\,B.\,Snet’h}, where we can find the verb positioned between a possessive
pronoun and its associated noun:

\gloss {
    Au lýr náý’acḍaúrâ sýeċ asvaúr sýárb.
    Au |lýr |náý’-acḍ-aúrâ|sý-eċ|as-vaúr|sý-árb
    And|their|\s{1pl.fut.ant}-cleave-\s{circ}|\s{acc.pl}-sin|\s{dat}-world|\s{acc}-tree
    ‘And we shall indeed have revealed their sins to the world’\footnotemark
}

\footnotetext{See the dictionary entry for *act’he*, sense 4, for more information about the use of this word here,
which normally means ‘cleave’. The literal meaning of this sentence is roughly: ‘And we shall have brought
down the trees upon their sins, to (= for the benefit of) the world’.}


\subsection{Negated Clauses}
Negation in the indicative is expressed using the particle *asý’ýâ* ‘not’, which is typically appended to verbs
as *’sý’ýâ*. For a discussion of negation in the subjunctive and optative, see
Sections~\ref{subsec:subjunctive}~and~\ref{subsec:optative}.
By default, the particle is placed right after the verb:
\gloss {
    Aúlýab’hat’hâ’sý’ýâ b’hehýnác.
    aú-lý-ab’ha\Sl t’hâ|’sý’ýâ|b’hehýn-ác.
    \s{1pl}-\s{3pl.pass}-fell\Sl\s{1pl.pres.ant}|not|\s{instr.indef}-axe
    ‘We have not felled them with an axe.’
}

In case of a fronted constituent in an independent clause (but not in dependent clauses), the particle is
placed after that constituent:
\gloss {
    B’hehýnác asý’ýâ aúlýab’hat’hâ.
    b’hehýn-ác|asý’ýâ|aú-lý-ab’ha\Sl t’hâ.
    \s{instr.indef}-axe|not|\s{1pl}-\s{3pl.pass}-fell\Sl\s{1pl.pres.ant}
    ‘It is not with an axe that we have felled them.’
}

Note that it is not valid to both front a constituent and not move the negation. For example,
the following sentence sounds very awkward and no UF speaker would ever say or write this,
save perhaps to sound extremely ironic.
\gloss {
    \#B’hehýnác aúlýab’hat’hâ’sý’ýâ.
    b’hehýn-ác|aú-lý-ab’ha\Sl t’hâ|’sý’ýâ.
    \s{instr.indef}-axe|\s{1pl}-\s{3pl.pass}-fell\Sl\s{1pl.pres.ant}|not
    *Roughly:* ‘With an axe, we have not-felled them.’
}

UF makes frequent use of double negation in conjunction with words that create a negative context
such as *jávé* ‘never’, *y’ê* ‘nothing’, or *ráv́â* ‘seldom’. Typically, such words are frontend,
and consequently, the negation particle then appears appended to them, e.g.:
\gloss {
    Ráv́â’sý’ýâ st’halẹ jaċt'hé.
    Ráv́â|’sý’ýâ |s\Sl t’halẹ |j-aċt'h\Sl é
    seldom|not |\s{acc.indef}\Sl table|\s{1sg}-buy\Sl\s{3sg.pres.ant}
    ‘Rarely have I ever bought a table.’
}

Note that double negation is required for the sentence to make sense; UF learners often forget
about that, which can lead to rather awkward constructs such as:
\gloss {
    \#Ráv́â st’halẹ jaċt'hé.
    Ráv́â|s\Sl t’halẹ |j-aċt'h\Sl é
    seldom|\s{acc.indef}\Sl table|\s{1sg}-buy\Sl\s{3sg.pres.ant}
    *Roughly:* ‘I rarely-bought a table.’
}

Still, if a fronted constituent is present, the negation particle is placed after that constituent:
\gloss {
    St’halẹ’sý’ýâ ráv́â  jaċt'hé.
    s\Sl t’halẹ|’sý’ýâ | ráv́â |j-aċt'h\Sl é
    \s{acc.indef}\Sl table|not| seldom |\s{1sg}-buy\Sl\s{3sg.pres.ant}
    ‘A table I have bought rarely.’
}

Foreigners often make the mistake of assuming that the negation particle is part of a word,
e.g. that *ráv́â’sý’ýâ* means ‘seldom’. As such, UF speakers, when imitating a foreigner, may
sometimes use more than one negation particle in a single sentence. Note that this is very
much not proper language; such constructions are summarily comedic and best compared to phrases
such as ‘it do be like that’ in English:
\gloss {
    \*Ráv́â’sý’ýâ st’halẹ jaċt'hé’sý’ýâ
    Ráv́â|’sý’ýâ |s\Sl t’halẹ |j-aċt'h\Sl é|’sý’ýâ
    seldom|not |\s{acc.indef}\Sl table|\s{1sg}-buy\Sl\s{3sg.pres.ant}|not
    *Roughly:* ‘Rarely-not I bought a table.’
}

\subsection{Interrogative Clauses}
In UF, questions are generally marked by one or more particles. Unlike in many other languages, the verb generally
does not move, except perhaps for emphasis. The most fundamental kind of question is a yes-no question, which is
marked by the interrogative particle *c’hes*. The particle typically occurs in second position in the sentence (that
is, after the first *constituent*, not after the first word):
\gloss {
    St’halẹ c’hes jaċt'hé?
    s\Sl t’halẹ |c’hes |j-aċt'h\Sl é
    \s{acc.indef}\Sl table|\s{q}|\s{1sg}-buy\Sl\s{3sg.pres.ant}
    ‘Did I buy a table?’
}

The exception to this is with forms of *eḍ* ‘to be’, which are typically immediately preceded by
the question particle, the two forming a single word, placed at the very end of the sentence:
\gloss {
    Raúl baú c’hesse?
    raúl|baú|c’hes|se
    \s{abs}-language|good|\s{q}|\s{3n}.be
    ‘Is it a good language?’
}

Negation is placed in the usual position. A negated question is marked by the negation particle *sý’ýâ*,
and the expected answer is ‘yes’:
\gloss {
    St’halẹ c’hes jaċt'hé’sý’ýâ?
    s\Sl t’halẹ |c’hes |j-aċt'h\Sl é|’sý’ýâ
    \s{acc.indef}\Sl table|\s{q}|\s{1sg}-buy\Sl\s{3sg.pres.ant}|not
    ‘Did I not buy a table?’
}

Alternatively, the particle *(r)vá* can be used to indicate that the speaker expects the answer to be ‘no’
or to indicate disbelief, surprise, or amazement. Note that this particle *replaces* the question particle.
Attempting to use both particles in the same sentence is ungrammatical and will likely be interpreted as
stuttering.
\gloss {
    St’halẹvá jaċt'hé?
    s\Sl t’halẹ |vá |j-aċt'h\Sl é
    \s{acc.indef}\Sl table|\s{q}|\s{1sg}-buy\Sl\s{3sg.pres.ant}
    ‘I bought a table?’
}

Unlike *c’hes*, this particle remains there even if the verb is *eḍ* ‘to be’:
\gloss {
    Raúlvá baú se?
    raúl|vá|baú|se
    \s{abs}-language|\s{q}|good|\s{3n}.be
    ‘It is a good language?’
}

Of course, these questions can also be negated:
\gloss {
    St’halẹvá jaċt'hé’sý’ýâ?
    s\Sl t’halẹ |vá |j-aċt'h\Sl é|’sý’ýâ
    \s{acc.indef}\Sl table|\s{q}|\s{1sg}-buy\Sl\s{3sg.pres.ant}|not
    ‘I didn’t buy a table?’
}

The precise meaning of these questions is as follows: In *St’halẹ c’hes jaċt'hé?* (‘Did I buy a table?’),
the speaker is asking whether they themselves bought a table; a plausible situation would be that they
simply forgot whether they did. Its negation, *St’halẹ c’hes jaċt'hé’sý’ýâ?* (‘Did I not buy a table?’),
could be used if the speaker is sure they bought a table sometime ago, but they can’t seem to find it and
are starting to doubt themselves (‘Did I not buy a table? I’m sure I did.’).

By contrast, the question *St’halẹvá jaċt'hé?*) would be an assertion of disbelief; maybe the speaker
found a table in their loft, and they can’t seem to remember buying it, but the price tag is still there.
Finally, its negation *St’halẹvá jaċt'hé’sý’ýâ?* would most likely be the speaker expressing their frustration
over the fact that they can’t seem to find their table and asserting that, in fact, they know for sure that
they did indeed buy a table (‘Did I not buy a table? I know I did!’).

Fronting of the verb in the last two cases generally indicates confusion rather than amazement or anger and
is most commonly used in response to someone else’s statement so as to ask for clarification (‘What do you mean
“I bought a table”; what are you talking about?’).
\gloss {
    Jaċt'hévá st’halẹ?
    j-aċt'h\Sl é |vá|s\Sl t’halẹ
    \s{acc.indef}\Sl table|\s{q}|\s{1sg}-buy\Sl\s{3sg.pres.ant}
    ‘I \textit{bought} a \textit{table}?!’
}

The same applies to the negated version of such a question:
\gloss {
    Jaċt'hé’sý’ýâvá st’halẹ?
    j-aċt'h\Sl é |’sý’ýâ|vá|s\Sl t’halẹ
    \s{acc.indef}\Sl table|not|\s{q}|\s{1sg}-buy\Sl\s{3sg.pres.ant}
    ‘I \textit{didn’t} buy a \textit{table}?!’
}

Note the order of particles: negation precedes the question particle. Placing them the other way around
makes it sound like you’re trying to correct yourself from *Jaċt'hévá* to *Jaċt'hé’sý’ýâ*.

\subsection{ACI and PCI}
The term \s{aci} is Latin for *accūsātīvus cum īnfīnītīvō* ‘accusative with infinitive’. As the name would suggest, this
grammatical construction consists of a dependent clause formed by an \s{acc} noun together with an infinitive; the
noun is the subject or object of the clause, and the infinitive the predicate. This construction is most well-known
from Classical languages such as Latin or Ancient Greek, but it is also found in various other languages, including
English and, of course, UF:
\gloss {
    Lácár sbhaú àfér láȷ́éd’há.
    lá\Sl c̣ár| s\Sl bhaú| à-fér| l-áȷ́éd’h\Sl á
    \s{nom}\Sl Charles|\s{acc.indef}\Sl bridge| \s{inf.pass}-build|\s{3m}-order\Sl \s{pres.ant}
    ‘Charles ordered a bridge to be built.’
}

In this sentence, the matrix clause is *Lácár láȷ́éd’há* ‘Charles ordered’, and the dependent clause is formed by
the \s{aci} *sbhaú àfér* ‘a bridge to be built’. Since ‘a bridge’ is the object in this case, the passive infinitive
is used. Observe how this sentence’s translation also uses an \s{aci} with a passive infinitive in both English (‘Charles
ordered a bridge to be built’) as well as Latin (*Carolus pontem fierī iussit*).

UF does not have a word for ‘that’ as in ‘I think that \ldots’ or ‘I know that \ldots’; instead, it uses
\s{aci}s in these cases. Just how multiple ‘that’ clauses can be chained in English, so can multiple \s{aci}s in UF.
\gloss {
    Icár sbhaú àfér dáȷ́édá jsav́á.
    i\Sl c̣ár| s\Sl bhaú| à-fér| d-áȷ́éd-á |j-sav́á
    \s{acc}\Sl Charles|\s{acc.indef}\Sl bridge| \s{inf.pass}-build|\s{inf}-order-\s{pres.ant}|\s{1sg}-know
    ‘I know that Charles ordered a bridge to be built.’
}

When multiple \s{aci}s are chained together, they are nested such that \s{acc} comes first and the infinitive
last or vice versa, and any nested \s{aci}s are placed inbetween; observe that, in the sentence above, the \s{aci}
*sbhaú àfér* ‘a bridge to be built’ is nested inside *Icár dáȷ́édá* ‘Charles to have ordered’. The literal
translation of this sentence would thus be ‘I know Charles to have ordered a bridge to be built’.

Furthermore, note that the finite verb of the matrix clause of an \s{aci} receives only a subject marker if the
\s{aci} is the object and vice versa. Thus, we have *jsav́á* ‘I know’ in the example above instead of e.g.
*jssav́á* ‘I know it’. It *would* be possible to add the object marker in the example above, but it would
sound a bit strange, roughly ‘I know it: that Charles ordered a bridge to be built’, and the verb would
probably have to be fronted for the sentence to make sense that way.

In addition to \s{aci}s, UF also has \s{pci}s, which use the *\s{part}* case instead. The \s{part} generally indicates
that an action is incomplete (see §~\ref{subsubsec:declension}), and thus \s{pci}s can be used to express something similar;
for instance:
\gloss {
    Lácár dŷnbaú àfér láȷ́éd’há.
    lá\Sl c̣ár| dŷn-ḅaú| à-fér| l-áȷ́éd’h\Sl á
    \s{nom}\Sl Charles|\s{part.indef}-bridge| \s{inf.pass}-build|\s{3m}-order\Sl \s{pres.ant}
    ‘Charles ordered to start building a bridge.’
}

The translation of the sentence above isn’t the best, but we start to run into a problem here, since UF uses
\s{aci}s and \s{pci}s much more prolifically than English does. A somewhat literal translation of this sentence would be
something along the lines of ‘Charles ordered the building of a bridge to be started’, but it isn’t perfect
either since ‘building’ is a gerund but in the sentence above, it’s an infinitive. In modern English, there simply
is no good literal translation for this sentence that preserves the passive infinitive.

When dealing with \s{aci}s and \s{pci}s that involve verbs that also take \s{acc} and \s{part} arguments, respectively, or
other infinitives which do, one must be careful not to construct garden-path sentences. For instance, take *\textbf{sḅáłýr}
sýċahý dýbháhẹ dylí \textbf{dub’hrá}*. Here, the \s{pci} is marked in bold, and the intended meaning is ‘for speakers to be
able to read each other’s thoughts’. Unfortunately, however, ‘read’ also takes a \s{part} here, and thus, it is
possible to construct a different \s{pci}, namely *\textbf{sḅáłýr} sýċahý dýbháhẹ \textbf{dylí} dub’hrá* ‘for speakers
to read each other’s thoughts’, and *dub’hrá* ‘to be able to’ is awkwardly left hanging at the end of the sentence.

To fix this problem, rearrange the sentence so the infinitive of the \s{aci} or \s{pci} is placed first and put the verbs of any
enclosed verb phrases first in those phrases to indicate that any immediately following \s{acc} or \s{part} nouns are part
of that verb rather than of the \s{aci} or \s{pci}: *\textbf{dub’hrá} dylí sýċahý dýbháhẹ \textbf{sḅáłýr}*.
This rule is sometimes intentionally subverted in cases where the double meaning is desirable, or in poetry, where word order
is a lot looser, but it would be very awkward to do so in prose.

In speech, this problem is more readily solved via intonation by placing emphasis and separating the ‘contents’ of the
\s{aci} or \s{pci} from the infinitive and noun with short pauses, e.g. *\textbf{sḅáłýr} {\nf ‖} sýċahý dýbháhẹ dylí
{\nf ‖} \textbf{dub’hrá}*.

Whenever a word is marked as taking an \s{aci} in the dictionary, it may also take a \s{pci} instead if
that makes sense semantically; there are no words that syntactically may take an \s{aci}, but not a \s{pci}.
Finally, note that ‘that’ is not always expressed with an \s{aci} or \s{pci}. Certain verbs, e.g. verbs of fearing, may
take a dependent clause in the subjunctive or optative instead (see §§~\ref{subsec:subjunctive},~\ref{subsec:optative}).

\subsection{Conditionals}\label{subsec:conditionals}
UF conditionals can broadly be divided into four categories: Simple, potential, irrealis, and counterfactual. Simple
conditionals indicate basic implications and logical truths. These conditionals use the indicative in both the protasis
and apodosis, in the appropriate tense. The protasis is generally introduced by the particle *s* ‘if’.
\gloss {
    S {\bf r} sré, aû-{\bf r} sfe.
    s|{\nf r}| s-ré| aû-|{\nf r}| s-fẹ
    if|*r*|\s{3n}-be.true|non-|*r*|\s{3n}-be.false
    ‘If *r* is true, then not-*r* is false.’\footnotemark
}

\footnotetext{UF does not use the letters *p* or *q*, and thus, discussions of propositional logic in UF tend to
use *r* and *t* instead. *s* is not used either so as to not confuse it with *s* ‘if’.}

Potential conditionals indicate that something is possible or could happen in the present or future (but *not* in
the past), provided some condition is met. These conditionals use the optative in the protasis and the Conditional
I in the apodosis:

- Irrealis conditionals (conditionals that describe a situation that is not true, and could never be true)
  use the subjunctive ‘If it were raining right now, we would be wet’.

- Potential conditionals, which describe a situation that could happen, and which the
  speaker considers plausible use the optative ‘If we were to go left now, we’d fall off a cliff’. These
  conditionals are only possible in the present and future.

- Counterfactual conditionals, which describe a situation that could be true, but isn’t. These conditionals
  exist only in the present and past and also use the optative ‘If we had gone left, we would have fallen off
  a cliff’.

\section{Examples}\label{sec:examples}
\subsection{Simple Examples}
\subsubsection{Simple Glossing Example}

\gloss {
    Cárvá, sráhó dwávaût’há dact’heá?
    Ċár |vá |s-ráhó |dwá-vaût’há |ḍ-aċt’he-á
    ˈj̊ɑ̃ːˠ|ʋ̃ɑ̃|ˌsɰɑ̃ˈhɔ̃|dɰɑ̃ˌʋ̃ɔ̃̃ˈθɑ̃|da̯j̊ˈθe.ɑ̃
    Charles.\s{voc}|\s{particle}|\s{indef.acc}-fish|\s{def.iness}-mountain|\s{2sg}-buy-\s{pres.ant.2sg}
    ‘Charles, you bought a fish on the mountain?’
}


\subsubsection{I Don’t Think This Warrants Explaining}
{\itshape\bfseries
Słérá de c’hóný áb’hásy’ô, ráy’ê y’aúhý dís dyb’hóy’e sab’héy’. Ez lé-el lalebet’he z’ihór bet’hê rêsol daudé.
Ýab’héy’ rêd’hes lab’hóy’ejú, dŷna c’haúr debauhib sá lasusy’és ýrâhe lasyrrájú.
}

\gloss {
    słé-rá| ḍẹ| c’hóný| áb’hásy’ô| ráy’ê| y’aúhý| ḍ-ís | dy-b’hóy’ẹ
    \s{cons.pl}-law|all|well.known|\s{gen}\Sl aviation|way|there.is.no|\s{inf-subj}\Sl can| \s{part}-to.fly
}

\gloss {
   s-ab’héy’ |ez |lé-el | la-lẹ-bet’hẹ|z’ | ihór | bet’hê
   \s{acc.indef}-bee|its|\s{nom.pl}-wing|\s{3pl-aff.comp}-be.small|its| \s{acc}\Sl body| be.small\Sl\s{part}
}

\gloss {
    rê-sol | ḍ-auḍé |ý-ab’héy’|rêd’hes|la-b’hóy’ẹ-jú|dŷn-a|c’haúr
    \s{abl}-soil|\s{inf}-obtain |\s{nom.pl.indef}-bee|of.course|\s{3n.pl}-fly-\s{gn}|\s{part}-what|as
}

\gloss {
    dẹ-ḅauhib|sá|la-susy’é\Sl s|ý-râhẹ|la-sy-rrá-jú
    \s{inf}-be.impossible|not|\s{3n.pl}-care.about\Sl \s{subj}|\s{nom.pl.indef}-human|\s{3n.pl}-\s{3n.pass}-believe-\s{gn}
}


\medskip\noindent
‘According to all known laws of aviation, there is no way a bee should be able to fly. Its wings are too
small to get its fat little body off the ground. The bee, of course, flies anyway because bees don't care
what humans think is impossible.’

\medskip\noindent
Literal translation: ‘According to all known laws of aviation, there is no way that a bee should be capable of flight.\footnote{
Note that UF here uses the verbal noun *b’hóy’ẹ* ‘to fly’ as a noun to mean ‘flight’.}
Its wings are too small for its little body to obtain [distance] from the ground. Of course, bees fly [anyway], as
they do not care about what humans believe to be impossible.

\subsection{Copypasta Translation}
{\itshape%\bfseries
Rub’hráy’ó rát’he au sré au sfèhe laut’hâ adŷbáłýr Át’hebhaú Raúl dedesle, s aút’hiy’ey’ó sývéhýr dýhisdé sérdé laúây’êr;
aúc’haúbrâdy’ó’sý’ýâ vé dúr dyhaúbhausy’ô sehabhvísy’ô. Sýlývy’ér saúr c’hesse? Lec’hdr\-aúv\-nét’hic’hâ nérje c’hesse?
Árdihyl c’hesse? Sauz-aud de c’hesse? Jávé’sý’ýâ jrét’hádé dedónéle dýha\-bha\-hit’he deý’ebhat’hic’hâ Áraúl dybháł.
Aúrsáheressá. Jdír jys dub’hrá au dylí sýcahý dýbháhe au dylýáv́áy’é b’hýcahý sbáłýr Áraúl.

Lásásc’hríd raúl révéy’ýr c’hessejú? Léraúb’he lasydír, lavâhe vé sbhárde sásy’élâ Áraúl. Sráhis’sý’ýâ id’hír deb’hýlnér
b’hesaúr rêvú u aû-át’heý’ebhat’he u B’helfaúr sraúb’he. Jav́ár sáví lyzy’ýr ádróid. Sy’u\-b’h\-rá dahaúr isásc’hríd
dwáníb’he araúl sébâ âc’hrír ‘dèc’hníc’hvâ’ Át’hebhaú Raúl ‘desybhérýr’, sjys vé delýc’hóbhár, lásásc’hríd c’haúr sýraúl
âc’hrír sc’hóváhá, lévás nórâ jys ‘desybáł’ dyhéy’é lay’ehóvâhér. Aúc’hóhid’héy’ó laúrvé Áraúl dynát’hýr rêâ, srâsírá,
dwác’hóvníc’h âbáł dývrê b’hehbár\-di\-hi\-bhá aûádróid, It’hebhaú Raúl abhraúl dérésdâ derâdvâvéy’ýr.
}

%{\centering\Large
%[ɰu̯βˈɰɑ̃ˑ.ɥɔ̃ ɰɑ̃ˈθə̥ o̯ˈsɰɛ̃ˑ o.sɸɛ̯ˈhə̥ ˈɮ̃o.θɑ̃̃ adʏ̃̃.bɑ̃ˈɮ̃ʶʏ̠̃ˑɰ ɑ̃ˈɰɔ̃ˑɮ̃ də.de̯ˈsɮ̃ə̥ sɔ̃.θi̯ˈɥe.ɥɔ̃ sʏ̃.ʋ̃ɛ̃ˈhʏ̠̃ˑɰ dʏ̃.hi̯sˈdɛ̃ˑ sɜ̃ɰˈdɛ̃ˑ ɮ̃ɔ̃.ɑ̃̃ˈɥɘ̃̃ˑɰ ɔ̃.χɔ̃ˈbɰɑ̃̃ˑ.dɥɔ̃.sɥ̃ʏ̃ɑ̃̃ ʋ̃ɛ̃ˈdũˑɰ dy.hɔ̃.bʱo̯ˈsɥɔ̃̃ˑ səh.abʱ.ʋ̃ĩˈsɥɔ̃̃ˑ]
%\par
%}

\subsubsection{Gloss}
\multigloss {
    r-ub’hrá-y’ó|rát’hẹ|au|s-ré|au|s-fèhẹ|laut’h-â
    \s{1pl}-can-\s{1pl}|you.see|and|\s{acc.pl.indef}-ray|\s{and}|\s{acc.pl.indef}-beam|float-\s{ptcp}

    aḍŷ-ḅáłýr|á-t’hebhaú raúl|dẹ-deslẹ|s|aú-t’hiy’e-y’ó|sý-véhýr
    \s{interess.pl.indef}-speaker|\s{gen}-Ultrafrench.language|\s{inf}-detect|if|\s{1pl}-use-\s{1pl}|\s{gen.pl.indef}-measure

    dý\Sl hisḍé|sérḍé|laú|â-y’\Sl ệr|aú-c’haúḅrâd-y’ó
    \s{part.pl.indef}\Sl system|certain|long|\s{ptcp.pass}-forbid\Sl \s{ptcp.pres.ant}|\s{1pl}-understand-\s{1pl}

    ’sý’ýâ|vé|ḍúr|dy\Sl haúbhausy’ô|sẹh|abh-vísy’ô|sý-lývy’ér|saúr|c’hes
    not|but|still|\s{part}\Sl composition|this|\s{gen.pl}-emission|\s{gen.indef}-light|\s{abs}.kind|\s{q}

    se|lec’hḍraúvnẹ́t’hic’h-â|nérjẹ|c’hes|se|árḍihyl|c’hes|se|sauz
    \s{3n}.be|electromagnetic-\s{ptcp}|\s{energy}.\s{abs}|\s{q}|\s{3n}.be|particle.\s{abs}|\s{q}|\s{3n}.be|\s{abs}.thing

    aud|ḍẹ|c’hes|se|jávé|’sý’ýâ|j-rét’hád-é|dẹ-dónẹ́-ḷẹ
    other|entire|\s{q}|\s{3n}.be|never|not|\s{1sg}-claim-\s{pres.ant}|\s{inf}-endow-\s{3.dat}

    dý\Sl habhahit’hẹ|ḍeý’ebhat’hic’h-â|á-raúl|dy\Sl bháł
    \s{part.pl.indef}\Sl ability|be.telepathic-\s{ptcp}|\s{gen}-language|\s{part}\Sl speak

    aúr-sáhere-ss\Sl a|j-dír|jys|d-ub’hrá|au|dy-lí
    \s{3n.fut.ii}-be.preposterous.\s{fut}-\s{cond}\Sl \s{circ}|\s{1sg}-say|only|\s{inf}-can|and|\s{part}-read

    sý\Sl ċahý|dý\Sl bháhẹ|au|dy-lý-áv́áy’é
    \s{gen.pl.indef}-each.other|\s{part.pl.indef}-thought|and|\s{part}-\s{3pl.pass}-send

    b’hý\Sl ċahý|s-ḅáłýr|á-raúl
    \s{dat.pl.indef}-each.other|\s{acc.pl.indef}-speaker|\s{gen}-language

    lá-sásc’hríḍ|raúl|ré-véy’ýr|c’hes|se-jú
    \s{nom}-Sanskrit|\s{abs}.language|\s{sup}-better|\s{q}|\s{3n}.be-\s{gn}

    lé-raúb’hẹ|la-sy-dír|la-vâhẹ|vé|s\Sl bhárḍẹ|sásy’él-â|á-raúl
    \s{nom.pl}-robot|\s{3pl}-\s{3n.pass}-say|\s{3pl}-miss.out|but|\s{acc.indef}\Sl \s{part}|be.essential-\s{ptcp}|\s{gen}-language

    s-ráhis|’sý’ýâ|i\Sl d’hír|dẹ-b’hýlnẹ́r|b’hel-saúr|rê-vú|u|aû-|á\Sl t’heý’ebhat’hẹ
    \s{3n}-be.racist|not|\s{acc}\Sl say|\s{inf}-be.unaffected|\s{instr.pl}-form|\s{sup}-many|or|non-|\s{gen}-telepathy

    u|b’he-faúr|s-raúb’he|j-av́ár|s-áví|lyzy’ýr|ádróid
    or|\s{instr}-Force|\s{acc.pl.indef}-robot|\s{1sg}-have|\s{acc.pl.indef}-friend|several|\s{abs}.android

    s-y’-ub’hrá|dahaúr|i-sásc’hríd|dwá-níb’hẹ|a-raúl|séḅ-â|â-c’hrír
    \s{3n-opt}-can|sure|\s{acc}-Sanskrit|\s{iness}-level|\s{gen}-language|be.plain-\s{ptcp}|\s{ptcp.pass}.write

    ḍèc’hníc’hvâ|á-t’hebhaú raúl|dẹ-sybhẹ́rýr|s-jys|vé|dẹ-lý-c’hóbhár
    technically|\s{gen}-Ultrafrench.language|\s{inf}-be.superior|\s{3n}-be.unfair|but|\s{inf}-\s{3pl.pass}-compare

    lá-sásc’hríd|c’haúr|sý-raúl|â-c’hrír|s-c’hóváh\Sl á
    \s{nom}-Sanskrit|as|\s{gen.indef}-language|\s{ptcp.pass}-write|\s{3n}-start.out.as.\s{subj}\Sl \s{pres.ant}

    lé-vás|nór-â|jys|dẹ-sy-ḅáł|dy\Sl héy’ẹ́|la-y’ẹ\Sl hóvâh\Sl ér
    \s{nom.pl}-masses|be.ignorant-\s{ptcp}|until|\s{inf}-\s{3n.pass}-speak|\s{part}\Sl attempt|\s{3pl}-\s{opt}\Sl start\Sl \s{pres.ant}

    aú-c’hóhid’hẹ́-y’ó|laúrvé|á-raúl|dy-nát’hýr|rê-â
    \s{1pl}-consider-\s{1pl}|but.when|\s{gen}-language|\s{part}-nature|be.triune-\s{ptcp}

    s-râsír-á|dwá-c’hóvníc’h|â-ḅáł|dývrê|b’heh-ḅárḍihibhá|aû-|ádróid
    \s{3n}-transpire-\s{pres.ant}|\s{iness}-communication|\s{ptcp.pass}-speak|\s{instr.pl.indef}-participant|non|\s{abs}.android

    i-t’hebhaú raúl|abh-raúl|ḍérésḍ-â|dẹ-râdvâ-véy’ýr
    \s{acc}-Ultrafrench.language|\s{gen.pl}-language|be.terrestrial-\s{ptcp}|\s{inf}-\s{superl}-be.good.\s{comp}
}

\subsubsection{Translation}
‘You see, we can detect rays and beams of energy floating between ULTRAFRENCH speakers if we use certain long-forbidden
measurement systems, but we still don’t understand the composition of these emissions. Are they some kind of light?
Electromagnetic energy? A particle? Something else entirely?

‘I’ve never claimed that speaking ULTRAFRENCH endows you with telepathic abilities. That would be preposterous. I’m just
saying that ULTRAFRENCH speakers can read each others minds and send thoughts to each other.

‘Is Sanskrit the best language? The robots tell me so.  But they are missing out on an essential part of ULTRAFRENCH.
It’s not racist to say robots are immune to most forms of not-telepathy and the Force. I have several android friends

‘Sanskrit might be “technically” “superior” to ULTRAFRENCH on the level of the plain written language. Sure, but it’s
unfair to compare them because Sanskrit started out as a written language until the ignorant masses started attempting
to “speak” it.

‘But when you consider the triune nature of ULTRAFRENCH, I think it’s clear that, at least in spoken communication with
non-android participants, ULTRAFRENCH is the best earth-based language.’

\subsubsection{Literal Translation}
We can, you see, detect both rays and beams of energy floating between speakers of The UF Language
if we use certain systems of measurement long-forbidden; we still don’t understand, however, the composition of these
emissions. Is it some kind of light? Is it electromagnetic energy? Is it a particle? Is it something else entirely?
I’ve never claimed that [the mere act of]\footnote{The
speaker uses a \s{pci} (*dybháł*) instead of an \s{aci} (*ibháł*) for ‘speaking’ here; had they used an \s{aci}, the meaning would be closer to ‘the act
of “fully speaking” the language’, as in, speaking and understanding it in its entirety. Thus, the speaker implicates that
it is not the mere act of making utterances in UF (*Áraúl dybháł*), but rather speaking and comprehending it in its entirety (*Áraúl ibháł*)
that gives rise to telepathic abilities.} the speaking of The Language endows them with telepathic abilities.
It would be preposterous. I’m only saying that speakers of The Language can both read each other’s thoughts\footnote{In UF, ‘to
read someone’s mind’ is expressed as ‘to read someone’s thoughts’.} and send them to each other.

Is Sanskrit the best language? The robots are saying it, but they miss out on an essential part of The Language. The act
of saying that robots are incapable of being affected by most forms of non-telepathy or\footnote{The UF text uses *u* \ldots\ *u*
\ldots\ ‘\ldots\ or \ldots (inclusive)’. This is for semantic reasons: the original text had a positive context (‘immune to’), whereas
the UF translation uses a negative context (‘incapable of being affected by’); thus, by De Morgan, we have to switch from ‘and’
to ‘or’ here.} by the Force is not racist. I have several android friends. Sure, Sanskrit might,\footnote{‘might be X’ is
generally expressed using the optative of *ub’hrá* + an \s{aci} with ‘to be X’.} on the level of the plain written
language, be ‘technically’ ‘superior’ to The UF Language, but it is unfair to compare them, as Sanskrit started out as
a written language, until the ignorant masses started attempting to ‘speak’ it. But when we consider the triune nature
of The Language, it has transpired that,\footnote{‘To become clear’ is expressed with the \s{pres ant} form of ‘transpire’.}
at least in spoken communication with non-android participants, UF is the best of the terrestrial languages.

%% NOTES:
%% ‘ez’: its, her, his. Before a word that starts w/ a vowel, this becomes *z* (e.g. **ez ihór* > *z’ihór*).
%%       No, the apostrophe doesn’t make sense in that position.
%%       Otherwise, after a word ending w/ a vowel, it becomes *’z*
%%
%%
%% Coalescence rules!
%%
%% Translate the title into UF and put it below the English title similar
%% to the Arodjun book.
%%
%% UF does not use the 2nd person for things like ‘when you consider that...’,
%% instead preferring the 1pl (lit ‘when we consider that...’).
%%
%% The dative is used to mark the subject of a sentence in the passive.
%%
%% The absolutive is used to turn nouns into modifiers, which act like adjectives.
%%
%% Genitives can precede or follow the possessee. Typically, they follow it, but
%% if the possessee is qualified with adjectives, then the adjectives must follow
%% the possessee immediately, lest they qualify the genitive instead, and thus, the
%% genitive is placed first.
%%
%% A verb can only have a passive affix if there is no explicit direct object; a verb always has
%% an active affix, unless there is no subject at all; a verb has a dative suffix,
%% iff there is no explicit indirect object.
%%
%% Remaining forms of to be.
%%
%% Numerals! (Make ‘10’ mean ‘half of 20’, 15 ‘half of 20 plus 5’ and so on)
%%
%% Participles go after the first word of the sentence
%%
%% Tous les 36 du mois -> Include the ‘36’ in the word!
%%
%% The PF infinitive endings -ir, -er, etc. are often dropped to form the base form (e.g. auḍé < *auḅḍénír).
%%
%% Adjectives are not declined and follow the noun they modify.
%%
%% ‘-t’he’, FUT ‘-ḍe’, SUBJ ‘t’hes’ is a productive word formation suffix that can be used to
%% turn a noun into a verb that roughly means ‘to use that thing’. E.g. ac ‘axe’ -> act’he  ‘roughly: to use
%% an axe’.




%% Dictionary.
\clearpage
\def\leftmark{\firstmark\ | \botmark}
\let\rightmark\leftmark

\twocolumn[\section{Dictionary}]

\ExplSyntaxOn

\cs_new:Npn \start_entry: {
    \hangindent = 6pt
    \hangafter = 1
    \noindent
}

\def \pfabbr {{\normalfont\scshape  pf \space }}

%% Word, part of speech, etymology, definition, (forms)
\long \def \entry #1 #2 #3 #4 #5 {
    \start_entry:

    %% Typeset word and part of speech.
    \mark { #1 }
    \textbf { \ignorespaces #1 } \space
    \textit { \ignorespaces #2 }

    %% Typeset etymology.
    \tl_set:Nn \l_tmpa_tl {#3}
    \tl_if_empty:NTF \l_tmpa_tl { } {
        \space [
            \ignorespaces \textit { \tl_use:N \l_tmpa_tl }
        ]
    }

    %% Typeset forms, if any.
    \tl_set:Nn \l_tmpa_tl {#5}
    \tl_if_empty:NTF \l_tmpa_tl { } {
        %\space {\nf\scshape{forms}}:
        \space
        \ignorespaces \tl_use:N \l_tmpa_tl
        .
    }

    %% Typeset definition.
    \space \ignorespaces #4 .
    \par
}

%% Reference to another entry.
\long \def \refentry #1 #2 {
    \start_entry:

    \textbf { \ignorespaces #1 } \space
    \(\to\) \space
    \textit { \ignorespaces #2 }
    .
    \par
}

\ExplSyntaxOff

%%%%%%%%%%%%%%%%%%%%%%%%%%%%%%%%%%%%%%%%%%%%%%%%%%%%%%%%%%%%%%%%%%%%%%%%
%%            This file was generated from DICTIONARY.txt             %%
%%                                                                    %%
%%                         DO NOT EDIT                                %%
%%%%%%%%%%%%%%%%%%%%%%%%%%%%%%%%%%%%%%%%%%%%%%%%%%%%%%%%%%%%%%%%%%%%%%%%

\entry{a}{pron.}{\pf{quoi}}{\textit{Interrogative and relative}.\\\s{indef} What?\\\s{def} Who? Whom?\\\s{indef} \textit{or} \s{def} Which, who, that \textit{(see grammar)}.}{}
\entry{á}{n.}{\pf{âme}}{Spirit.}{}
\entry{aḅ}{v.}{\pf{appeler}}{To call (+\s{acc} sbd./sth.) (+\s{abs} sbd./sth.). \textit{In PF, this verb used to take a double accusative, but this usage disappeared early on in UF, with the second accusative naturally being replaced by the absolutive, likely to avoid ambiguity that was starting to manifest as a result of UF’s increasingly free word order.} \ex \s{Snet’h v.2} \w{jdap rác’hsaý’adâ} ‘I call you a liar’; even in the writings of \s{Snet’h}, the double accusative is no longer attested.}{}
\entry{ábhec}{v.}{\pf{empêcher}}{+\s{acc} To prevent, stop (sth. from happening).}{\s{fut} ábhece, \s{subj} ábhecs}
\entry{abhérś}{v.}{\pf{apercevoir}}{To behold, descry (+\s{part}).}{}
\entry{aḅrâ}{v.}{\pf{apprendre}}{To learn.}{\s{fut} aḅrâdé, \s{subj} aḅrâs}
\entry{aḅraúc̣}{v.}{\pf{approcher}}{To approach, come near, walk up to (+\s{all} sbd./sth.).}{\s{fut} aḅraúc̣é, \s{subj} aḅraúc̣s}
\entry{aḅrdvê}{adv.}{\pf{après-demain}}{The day after tomorrow. \textit{The prefix \w{aḅr} can be prepended as often as necessary, e.g. \w{aḅraḅraḅrdvê} would be ‘in four days’}.}{}
\entry{ab’há}{conj.}{\pf{avant que}}{+\s{opt} Before.}{}
\entry{áb’há}{n.}{\pf{enfant}}{Child.}{}
\entry{ab’haḍ}{v.}{\pf{abattre}}{\\To cut down, fell, knock down, shoot down.\\To butcher, cut apart violently.}{\s{fut} ab’haḍrẹ́, \s{subj} ab’has}
\entry{ab’hásy’ô}{n.}{\pf{aviation}}{Aviation.}{}
\entry{ab’hèc’h}{v.}{\pf{affecter}}{+\s{acc} To affect, influence.}{\s{fut} ab’hèc’hre, \s{subj} ab’hè\-c’hes}
\entry{áb’hẹḍ}{v.}{\pf{embêtter}}{\\+\s{acc} To disturb, inconvenience sbd.\\+\s{part} To harass, bother sbd.}{}
\entry{ab’héy’}{n.}{\pf{abeille}}{Bee.}{}
\entry{ab’hínéb’heḅaý’évrâ}{v.}{\pf{habit ne fait pas le moi\-ne}}{To judge based on appearances.}{\s{fut} ab’hínéb’heḅaý’év́ẹ́, \s{subj} ab’hínéb’heḅaý’\-év́\-ás}
\entry{áb’hóhẹ}{v.}{\pf{enfoncer}}{To push, press, shove, drive (+\s{ill} into).}{}
\entry{ac}{n.}{\pf{hache}}{Axe, hatchet.}{}
\refentry{ach’es}{\w{a} + \w{c’hes}}
\entry{act’he}{v. tr.}{from \w{ac}}{\\To cut or cleave with an axe.\\+\s{acc} To bring an end to.\\+\s{acc def} \textit{of \w{árb} intr. (other than literal)} To get to the point, cut to the chase.\\+\s{acc def} \textit{of \w{árb} and \s{acc}} To bring to light, reveal. \textit{Originally, this idiom did not take a double \s{acc}, but was instead formed with the \s{acc} of ‘tree’ and the \s{ill} of the object, meaning something along the lines of ‘to bring down the tree(s) on sth’—the image here being that of cutting down trees in a wood until only a clearing remains or is ‘brought to light’}.}{\s{fut} acḍe, \s{subj} act’hes}
\entry{ac̣t’he}{v. tr.}{\pf{acheter}}{To buy.}{\s{fut} ac̣ḍrẹ́, \s{subj} ac̣t’hes}
\entry{aḍrá}{v.}{\pf{attraper}}{\\+\s{acc} \textit{or} \s{part} To take.\\\w{aḍrá faúr} \textit{intr.} To take shape, take form.}{}
\entry{ádróid}{n.}{\pf{androïde}}{Android.}{}
\entry{ady’ŷ}{v. or interj.}{\pf{adieu}}{\\Goodbye, farewell.\\+\s{gen} To say goodbye to sbd., bid sbd. farewell.}{}
\entry{ad’he}{v.}{\pf{vader}}{To go.}{\s{fut} í, \s{subj} al}
\entry{ad’hór}{v. tr.}{\pf{adore}}{\\To love, adore.\\+\s{part} To be in love with, have a crush on.\\+\s{gen} To desire, yearn for sbd./sth.\ex \s{Snet’h iv.17} \w{jad’hóré ávvaúríhe} ‘I yearned to remember’ \textit{(compare \w{jad’hóré devvaúríhe} ‘I loved to remember’)}.}{\s{fut} ad’hórérẹ́, \s{subj} ad’hórs}
\entry{ad’hyl}{v.}{\pf{adulte}}{To be adult, grown-up.}{\s{fut} ad’hyle, \s{subj} ad’hyls}
\entry{ád’hýr}{v.}{\pf{endure}}{To resist, endure, withstand (+\s{acc} sth.).}{}
\entry{áẹ}{n.}{\pf{en-haut}}{Sky. \textit{Often plural, especially in a religious sense.}.}{}
\entry{ah}{n.}{\pf{assez}}{\textit{sufficient comparative prefix; see §~\ref{subsubsec:comparison}}.}{}
\entry{áhaúr}{conj.}{\pf{encore}}{+\s{subj} Even though.}{}
\entry{áhaúr}{adv.}{\pf{encore}}{\\Still, yet \textit{(positive context)}.\\Again \textit{(negative context)}.}{}
\entry{áhâłát’hẹ}{v.}{\pf{ensanglanté}}{To be (very) bloody, bloodstained.}{}
\entry{ahúr}{v.}{\pf{assurer}}{To ensure.}{\s{fut} ahúré, \s{subj} ahúrs}
\entry{aír}{v.}{\pf{hair}}{To hate, abhor, detest, loathe, despise (+\s{acc} sbd./sth.).}{}
\entry{ânb’hé}{adv.}{\pf{en effet}, via metathesis from *\w{âné\-b’he}}{Verily, indeed, in fact.}{}
\entry{ánvé}{v. tr.}{\pf{animer}}{To bring to life, animate.}{}
\entry{árb}{n.}{\pf{arbre}}{Tree.}{}
\entry{árḍihyl}{n.}{\pf{particule}}{Particle.}{}
\entry{áríb’h}{v.}{\pf{arriver}}{To arrive.}{}
\entry{ársl}{v.}{\pf{harceler}}{To attack, assail, beset, bully (+\s{acc} sbd.).}{}
\entry{ârýý’}{v.}{\pf{enrouler}}{To wrap (+\s{acc} around sth.).}{}
\entry{ásy’ê}{v.}{\pf{ancien}}{To be ancient.}{\s{fut} ásy’êr, \s{subj} ásy’ês}
\entry{asý’ýâ}{particle}{\pf{pas absolument}}{Not, no. \textit{Commonly \w{’sý’ýâ} after vowels and verbs. This particle is used only in the indicative; see also \w{sá}, \w{t’hé}}.}{}
\entry{át’hád}{v.}{\pf{entendre}}{To hear, perceive (+\s{part} sbd./sth.).}{\s{fut} át’hádé, \s{subj} át’hás}
\entry{át’has}{v.}{\pf{entasser}}{\\\s{+acc} To heap, accumulate.\\\textit{refl.} To pile up, heap.}{}
\entry{át’hér}{v.}{\pf{enterrer}}{+\s{acc} To bury, inter.}{}
\entry{au}{conj.}{\pf{aussi}}{\\And, also, as well, too.\\\w{au} \ldots{} \w{au} \ldots{} ‘both \ldots{} and \ldots’}{}
\entry{aú}{n.}{\pf{homme}}{Man, human.}{}
\entry{aû}{particle}{\pf{non}}{Not-. \textit{Used to negate nouns, adjectives, and adverbs; see §~\ref{subsubsec:noun-negation}}.}{}
\refentry{aubhaus}{ní}
\entry{aublit’hér}{v.}{\pf{oblitérer}}{\\To defeat, vanquish, obliterate (+\s{acc} sbd./sth.).\\To be better than, ‘beat’ (+\s{inf} sbd/sth.).}{}
\entry{aub’heír}{v. (in)tr.}{\pf{obéir}}{To obey.}{}
\entry{auḍ}{adj.}{\pf{autre}}{Other, another.}{}
\entry{auḍé}{v.}{\pf{obtenir}}{\\To obtain, get, acquire.\\+\s{abl} To gain purchase on or hei\-ght or distance from.}{\s{fut} auḍy’édrẹ́}
\entry{auha}{conj.}{\pf{au cas où}}{+\s{opt} In case.}{}
\entry{aujúrdy’í}{adv.}{\pf{aujourd’hui}}{Today. \textit{Archaic, see also \w{júrdy’í}}.}{}
\entry{aúráj}{n.}{\pf{orage}}{\\\textit{(usually pl.)} Storm, tempest, thunderstorm.\ex \s{Snet’h ii.7} \w{phárýaúráj téríbâ} ‘like a terrible storm’.\\\textit{fig.} Upheaval, turmoil, crisis.}{}
\entry{ausc’hýr}{v.}{\pf{obscur}}{To be dark.}{}
\entry{av́ár}{v. irreg.}{\pf{avoir}}{+\s{acc} To have \textit{(usually inalienably)}.}{\s{pres ant} and \s{pret} y, \textit{obsolete} \s{pret} ab’hẹ, \s{fut} aúrẹ́, \s{subj} ès}
\entry{av́árḷý}{v.}{\pf{avoir lieu}}{To take place, happen.}{\s{fut} lav́árḷýé, \s{subj} lav́árḷýs}
\entry{áv́áy’é}{v.}{\pf{envoyer}}{To send.}{\s{fut} áv́áy’érẹ́, \s{subj} áv́áy’és}
\entry{áví}{n.}{\pf{ami}}{Friend.}{}
\entry{ávrê}{conj.}{\pf{à moins que}}{+\s{opt} Unless.}{}
\entry{aý’aúr}{conj.}{\pf{alors}}{While, as (temporal), because.}{}
\entry{Aý’èc’hsád}{n.}{\pf{Alexandre}}{\textit{Male given name}.}{}
\entry{áł}{v.}{from earlier *\w{ḅał} < \pf{parler}}{To speak.}{}
\entry{áȷ́éd}{v.}{\pf{enjoindre}}{To order, enjoin, command.}{}
\entry{ba}{v.}{\pf{baser}}{To base on, found on.}{\s{fut} bare, \s{subj} bas}
\entry{ḅá nórávíc’h}{n. archaic}{\pf{Panoramix}}{Druid. \textit{Only the \w{nórávíc’h} is inflected; infixing of adj. is attested.} \ex \s{Snet’h}, \s{iii.2}: \w{derúb’h phá ráinórávíc’h} ‘to find the great druid’, with infixed \w{rá}.}{}
\entry{baḍ}{v.}{\pf{battre}}{To beat, strike, hit (+\s{acc} sbd./sth.).}{}
\entry{ḅáhẹ}{n.}{\pf{pensée}}{Thought, reflection, meditation, faculty of thinking.}{}
\entry{ḅaj}{n.}{\pf{page}}{Page.}{}
\entry{ḅará}{n.}{\pf{parent}}{Parent.}{}
\entry{ḅarḍ}{v.}{\pf{partir}}{To leave, go away, depart.}{\s{fut} ḅarẹ́, \s{subj} ḅars}
\entry{ḅárḍáḍ}{v.}{\pf{partante}}{(+ \s{aci}) To be interested in, willing to, ready to, prepared for.}{}
\entry{ḅárḍẹ}{n.}{\pf{partie}}{Part, portion, piece, faction of a whole.}{}
\entry{ḅárḍihibhá}{n.}{\pf{participant}}{Participant.}{}
\entry{ḅáréḍ}{v.}{\pf{parraitre}; future stem from \pf{sembler}}{To seem, appear.}{\s{fut} sáb}
\entry{ḅas}{conj.}{\pf{parce que}}{+\s{subj} Because. \textit{Often used to explain motivation rather than cause, as in e.g. ‘We did that because\ldots’}.}{}
\entry{baú}{v. irreg.}{\pf{bon}}{\\To be good, well, healthy.\\To be right, correct, appropriate.\\\textit{usually intr.} To satisfy, fullfill, gratify.}{\s{fut} baúré, \s{subj} véy’ýrs; \s{comp} lẹvéy’ýr, y’ŷvéy’ýr, rêvéy’ýr; \s{sup} révéy’ýr, râdvâv\-éy’\-ýr}
\entry{ḅaú}{n.}{\pf{pont}}{Bridge.}{}
\entry{ḅauheŷnlabhé}{v.}{\pf{poser un lapin}}{To forsake, abandon.}{\s{fut} ḅauheŷnlabhére, \s{subj} ḅauheŷnlabhés}
\entry{ḅauhib}{v.}{\pf{impossible}}{To be impossible, unfeasible.}{\s{fut} ḅauhibre, \s{subj} ḅauh\-ibes}
\entry{Baúré}{n.}{\pf{Borée}}{Boreas, the North Wind.}{}
\entry{ḅáł}{v.}{\pf{parler}}{To speak, talk.}{\s{fut} báłérẹ́}
\entry{ḅáłýr}{n.}{\pf{parleur}}{Speaker, interlocutor.}{}
\entry{ḅelbec}{n.}{\pf{pelle} + \pf{bêche}}{Shovel.}{}
\entry{Bèrḍrá}{n.}{\pf{Bertrand}}{\textit{Male given name}.}{}
\entry{ḅẹt’hẹ}{v. irreg.}{\pf{petit}}{To be small, little.}{\s{fut} rêdẹ́, \s{subj} ḅẹt’hes; \s{comp} lẹrêd, y’ŷrêd, rêrêd; \s{sup} rérêd, râdvârêd}
\entry{ḅét’hýr}{v.}{\pf{peinture}}{To paint.}{}
\entry{ḅéy’í}{n.}{\pf{pays}}{Country, land, region, nation.}{}
\entry{ḅínár}{n.}{\pf{pinard}}{Wine.}{}
\refentry{bír}{vaúb’hẹ}
\entry{biwaú}{n.}{\pf{billion}}{\textit{(obsolete)} Billion (long scale, i.e. $10^{12}$). \textit{Replaced with modern \w{dýwaú})}.}{}
\entry{ḅré}{conj.}{\pf{après que}}{+\s{opt} After.}{}
\entry{ḅusy’ér}{n.}{\pf{poussière}}{Dust.}{}
\entry{bźé}{v.}{\pf{besoin}}{+\s{acc} \textit{or} \s{part} To need, require.}{}
\entry{b’há}{n.}{\pf{vent}}{Wind, breeze.}{}
\entry{b’hár}{n.}{\pf{vague}}{\\Wave.\\\textit{pl.} Ripples, undulations.}{}
\entry{b’hát’hiý’at’hýr}{v.}{\pf{ventilateur}}{To blow.}{}
\entry{b’hauḍ}{v.}{\pf{vôtre}}{To be yours (\s{pl}).}{\s{fut} b’hauḍre, \s{subj} b’haus}
\entry{b’haul}{v.}{\pf{voler}}{To hover, float.}{}
\entry{b’hây’ér}{adv.}{\pf{avant-hier}}{The day before yesterday. \textit{The prefix \w{b’hâ} can be prepended as often as necessary, e.g. \w{b’hâb’hâb’hây’ér} would be ‘four days ago’}.}{}
\entry{b’he}{conj.}{\pf{envers}}{+\s{subj} So that, so as to, to, in order to. \textit{Commonly enclitic \w{’b’h} after vowels}.}{}
\entry{b’hé}{n.}{\pf{vin}}{Grape.}{}
\entry{b’hénvâ}{n.}{\pf{évènement}}{Event, occurrence.}{}
\entry{b’hérḍy’ŷ}{v.}{\pf{vertueux}}{To be virtuous.}{}
\entry{b’hért’he}{n.}{from \pf{vérité}; the /i/ disappeared during Early Middle UF}{Truth.}{}
\entry{b’héy’}{v.}{\pf{veiller}}{\\\textit{intr.} To keep watch, keep guard.\\+\s{spress} To watch over, guard, keep an eye on.}{}
\entry{b’heý’au}{n.}{from archaic \w{b’heý’auhic’h}}{Bicycle.}{}
\entry{b’heý’auhic’h}{n. archaic}{\pf{vélo-cycle}}{Bicycle.}{}
\entry{b’heý’o}{v.}{back-formation from *\w{b’heý’os}, reanalysed as a subjunctive; from \pf{véloce}}{To be quick, fast.}{\s{fut} b’heý’o\-se, \s{subj} b’heý’os}
\entry{b’hí}{n.}{\pf{vigne}}{Vine.}{}
\entry{b’hic’hḍrár}{n.}{\pf{victoire}}{Victory.}{}
\entry{b’hid}{v.}{\pf{vide}}{To be empty.}{}
\entry{b’hizy’ô}{n.}{\pf{vision}}{Vision.}{}
\entry{b’hóy’ẹ}{v.}{\pf{voler}}{To fly. Flight.}{}
\entry{b’huḍ}{n.}{\pf{voûte}}{Vault, arched ceiling.}{}
\entry{b’hýlnẹ́r}{v.}{\pf{invulnérable}}{+\s{instr} To be incapable of being af\-fec\-ted by, invulnerable to.}{\s{fut} b’hýlnẹ́rẹ́, \s{subj} b’hýlnẹ́rs}
\entry{b’hŷnnúb’hâ}{adv.}{old \s{all} of \w{núb’hâ}}{Anew.}{}
\entry{caḍráy’ẹ́}{v.}{\pf{chatoyer}}{To shimmer, iridesce.}{}
\entry{cah}{v.}{\pf{chasser}}{To hunt.}{\s{fut} cahe, \s{subj} cas}
\entry{cahý}{pron. pl. indef.}{\pf{chacun}}{Each other, one another.}{}
\entry{Cár}{n.}{}{\textit{Male given name, equivalent to English ‘Kyle’ or ‘Charles’. Often declined like a regular noun, e.g. \s{nom} \w{Lác̣ár}}.}{}
\entry{cẹ}{v.}{\pf{chaud}}{To be hot.}{}
\entry{će}{v.}{\pf{échouer}}{\\+\s{part} To stumble, do a bad job at.\\+\s{acc} \textit{or} \s{aci} To fail, flunk, not pass.}{\s{fut} ćere, \s{subj} ćes}
\entry{cèc}{n.}{phonetic respelling of \w{cèc’h}}{\textit{(chess)} Check.}{}
\entry{cèc’h}{n.}{\pf{échec}}{Failure, defeat.}{}
\entry{cér}{v.}{\pf{cher}}{\\To be dear, important (+\s{dat} to sbd.). \textit{Possession of a noun qualified with this adjective verb is generally construed with the dative rather than the genitive, e.g. \w{asvẹ áví cérâ} or \w{áví cérâvé} ‘my dear friend’, rather than *\w{vaú áví cérâ}}.\\\textit{with \w{áví} ‘friend’} To be friends with.\ex \w{áví lẹcérvé} ‘he is a (dear) friend of mine’.}{\s{fut} céré, \s{subj} cés}
\entry{cévê}{n.}{\pf{chemin}}{Street.}{}
\entry{c’habhahit’hẹ}{n.}{\pf{capacité}}{Skill, capacity, ability.}{}
\entry{c’hánár}{n.}{\pf{canard}}{\\Ship, boat.\\\s{instr indef} By boat.}{}
\entry{c’hánaú}{n.}{\pf{canot}}{Duck (bir).}{}
\entry{c’háraúciḍ}{v.}{\pf{les carrotes sont cuites}}{To end for good, put to a permanent end.}{\s{fut} c’hár\-aúc\-re, \s{subj} c’háraúc}
\entry{c’hasḅesy’ál}{n.}{cas spécial}{Exception.}{}
\entry{c’haú}{adj.}{see sense 2}{\\Holy.\\\w{c’haú-}\L{} \textit{‘religious prefix’, prepended in derivation to nouns that have a religous connotation; this is historically a back-formation from \w{c’haúfrér} and \w{c’haúhýr} which happen to both start with this ‘prefix’}.}{}
\entry{c’haúáł}{n.}{\w{c’haú} + \w{áł}}{Prophecy.}{}
\entry{c’haúbhárrás}{n.}{\w{c’haú} + \pf{paroisse}}{Parish.}{}
\entry{c’haúbhausy’ô}{n.}{\pf{composition}}{Composition, arrangement, structure.}{}
\entry{c’haúbhèłínáj}{n.}{\w{c’haú} + \pf{pèlerinage}}{Pilgrimage.}{}
\entry{c’haúbhýríf}{n.}{\w{c’haú} + \pf{purifier}}{To purify (+\s{acc} sbd./sth.).}{}
\entry{c’haúḅlér}{.v}{\pf{complaire}}{To be complacent; to be accepting in the presence of +\s{gen} sbd./sth. perceived as negative.}{\s{fut} c’haúḅlére, \s{subj} c’haúḅlés}
\entry{c’haúḅrâd}{v.}{\pf{comprendre}}{+\s{part} To comprehend, understand, gr\-asp.}{\s{fut} c’haúḅrâdrẹ́, \s{subj} c’haúḅrâs}
\entry{c’haúb’héc’h}{v.}{\pf{convaincre}}{To persuade.}{}
\refentry{c’haúḍé}{ní}
\entry{c’haúḍrêd’hẹ}{n.}{\pf{compte-rendu}}{Account, rec\-ord.}{}
\entry{c’haúfí}{v.}{\pf{confiner}}{To contain.}{}
\entry{c’haúfrér}{n.}{\pf{confrère}}{Brother (religious). \textit{Masc. or pl. only, see also \w{c’haúhýr}}.}{}
\entry{c’haúhaúvnaút’hẹ}{n.}{\w{c’haú} + \pf{com\-mu\-nau\-té}}{Mo\-nastery.}{}
\entry{c’haúhýr}{n.}{\pf{consœur}}{Sister (religious). \textit{Fem. only, see also \w{c’haúfrér}}.}{}
\entry{c’haúnéhás}{.n}{\pf{connaissance}}{Knowledge.}{}
\entry{c’haúr}{conj.}{\pf{car} + \pf{comme}}{+\s{subj} As, because, since.}{}
\entry{c’haúv́ájẹ}{n.}{\w{c’haú} + \pf{magie}}{Magic.}{}
\entry{c’haúvnaút’hẹ}{n.}{\pf{communauté}}{Community.}{}
\entry{c’hd’hal}{adv.}{\pf{que dalle}}{Naught, absolutely no\-thing.}{}
\entry{C’hebèc’h}{n.}{\pf{Québec}}{The Promised Land.}{}
\entry{c’hèl}{det. postpos.}{\pf{quelques}}{Some, a few, a couple of.}{}
\entry{c’hèlc’hý}{pron.}{\pf{quelqu’un}}{Someone, somebody, anyone, anybody.}{}
\entry{c’hes}{quest. part.}{\pf{qu’est-ce que}}{\textit{see grammar; often \w{c’h’s} in older texts}.}{}
\entry{c’hesse}{}{contraction of \w{c’hes} + \w{se}}{Is it? \textit{Also substituted for other forms of ‘to be’ in questions, particularly for the plural neuter; stressed on the first syllable}.}{}
\entry{c’hlýr}{v.}{\pf{inclure}}{\\+\s{part} To include.\\+\s{acc} To possess, have \textit{(alienably)}.\\+\s{gen} \textit{usually} \s{indef} To sell, offer, have in stock.}{\s{fut} c’hlýré, \s{subj} c’hlýrs}
\entry{c’hóbhár}{v.}{\pf{comparer}}{To compare.}{\s{fut} c’hóbhárre, \s{subj} c’hó\-bhárs}
\entry{c’hóhid’hẹ́}{v.}{\pf{considérer}}{\\+\s{part} To consider, think ab\-out, ponder.\\+\s{acc} To think through.}{\s{fut} c’hóhid’hẹ́rẹ́, \s{subj} c’hóhid’hés}
\entry{c’hóný}{adj.}{\pf{connu}}{Known, familiar, well-kn\-own.}{}
\entry{c’hór}{n.}{\pf{corps}}{Body.}{}
\entry{c’hóvâ}{v.}{\pf{commencer}}{\\(\s{+ part}) To start, commence, begin.\\\s{+ gen} To start out as.\\\w{âc’hóvâ} \textit{def.} Beginning, start. \textit{Lit. ‘that which is being begun’}.}{\s{fut} c’hóvârẹ́, \s{subj} c’hóv\-ás}
\entry{c’hóvníc’h}{v.}{\pf{communiquer}}{\\To communicate (+\s{instr} with sbd.).\\Communication.}{\s{fut} c’hóvníc’hre, \s{subj} c’hóvníc’hes}
\entry{c’hrír}{v.}{\pf{écrire}}{To write.}{\s{fut} c’hrírẹ́, \s{subj} c’hrís}
\entry{c’hulvâ}{n.}{\pf{écoulement}}{Flow.}{}
\entry{c’húr}{v.}{\pf{court}}{To shrink, reduce in size, narrow.}{}
\entry{c’húr}{v.}{\pf{courrir}}{To run.}{}
\entry{c’hýr}{n.}{\pf{cœur}}{Heart.}{}
\refentry{c’h’s}{c’hes}
\entry{dá}{n.}{\pf{dent}}{Tooth.}{}
\entry{ḍá}{conj.}{\pf{tandis}}{Whereas.}{}
\entry{ḍád}{n.}{\pf{stand}}{Stand, stall, booth.}{}
\entry{dahaúr}{particle}{\pf{d’accord}}{Sure, ok, agreed, fine.}{}
\entry{ḍalẹ}{n.}{\pf{tableau}}{Table.}{}
\entry{ḍalisvâ}{n.}{\pf{établissement}}{Establishment, institution.}{}
\entry{dár}{v.}{\pf{darder}}{To throw, cast, yeet (+\s{acc} sth.).}{}
\entry{ḍaú}{n.}{\pf{tonne}}{Weight.}{}
\entry{daú(c’h)}{particle}{\pf{donc}}{Therefore, then, thus.}{}
\entry{ḍaúb’h}{v. intr.}{\pf{tomber}}{To fall, drop.}{}
\entry{daúb’hedwébhó}{v.}{\pf{tomber dans les pommes}}{To faint.}{\s{fut} daúb’hedwébhóre, \s{subj} daúb’hedwébhós}
\entry{ḍauḍ}{def. pron.}{from earlier \w{ḍẹ auḍ}}{Everything else, any other (one).}{}
\entry{Daúvníc’h}{n.}{}{\textit{male or female given name, equivalent to English ‘Dominic’}.}{}
\refentry{daú’b’h}{daú(c’h) \textnf{+} b’he}
\entry{db’hid’h}{n.}{\pf{individu}}{Person, individual.}{}
\entry{de}{conj.}{\pf{dès que}}{+\s{subj} Once, when once, as soon as.}{}
\entry{dẹ́}{particle}{from \w{Provençal} \textit{den}}{Then (sequential), next.}{}
\entry{ḍẹ}{adj.}{\pf{tout}}{\\All, every, whole, entire.\\\w{ḍẹ auḍ} Obsolete form of \w{ḍauḍ}.}{}
\entry{deḅlér}{v.}{\pf{déplaire}}{To displease (+\s{acc} sbd.), be displeasing.}{}
\entry{dẹb’hní}{v.}{\pf{devenir}}{To become, turn into (+\s{transl} sth./sbd.). \textit{The subject is in the \s{abs} case}.}{}
\entry{dec̣ír}{v.}{\pf{déchirer}}{\\+\s{part} To tear, rip, rend.\\+\s{acc} To rend asunder, tear to pieces.}{\s{fut} dec̣irrẹ, \s{subj} dec̣írs}
\entry{dèc’h}{adj.}{\pf{dextre}}{Right (side), right-handed.}{}
\entry{ḍèc’hníc’hvâ}{adv.}{\pf{techniquement}}{Technically.}{}
\entry{ḍédv́ér}{interj.}{\pf{putain de merde}}{Fuck. \textit{Generic expletive}.}{}
\entry{dẹh}{v.}{\pf{dessous}}{To be below, beneath.}{}
\entry{dehab’híy’}{v. tr.}{\pf{déshabiller}}{To undress +\s{acc} sbd.}{}
\entry{dẹhẹ}{n.}{\pf{dessus}}{\\Top, upper side.\\Surface of a body of water.}{}
\entry{dehid}{v.}{\pf{décider}}{To decide (+\s{inf} to do sth.).}{}
\entry{dej}{particle}{from \w{dejẹ}}{\textit{Emphatic particle; only used in the preterite}.\\\s{pret} + \w{dej} \textit{roughly} To have ever done sth.}{}
\entry{dejẹ}{adv.}{\pf{déjà}}{Already.}{}
\entry{ḍèl}{particle}{\pf{tel}}{\textit{Emphatic particle, used as an intensifier, often postpositive after the verb, but not so much intensifying the verb directly as it does the entire clause.} \ex \s{Snet’h}, \s{ii.34}: \w{lá-árb srýlé dèl} ‘so it was that the tree was burning’ or ‘the tree was burning fiercely’, or ‘indeed, the tree was burning’.}{}
\entry{ḍénéb}{n. pl.}{\pf{ténèbres}}{\textit{exclusively plural \w{lḍénéb}} Darkness.}{}
\entry{ḍèr}{v.}{\pf{taire}}{To silence, shut up.}{\s{fut} ḍérẹ́}
\entry{ḍéraúj}{v.}{\pf{interroger}}{To demand.}{}
\entry{ḍérésḍ}{v.}{\pf{terrestre}}{To be terrestrial, earth-based.}{\s{fut} ḍérésḍrẹ́, \s{subj} ḍérésḍs}
\entry{ḍẹ́ríb}{v.}{\pf{terrible}}{To be terrible (all senses).}{\s{fut} ḍẹ́ríre, \s{subj} ḍẹ́rís}
\entry{dérny’é}{adj.}{\pf{dernier}}{Last, final, ultimate.}{}
\entry{dérny’ẹ́huf}{n.}{\pf{dernier} + \pf{souffle}}{Death.}{}
\entry{ḍérsèd}{v.}{\pf{intercéder}}{To intercede.}{}
\entry{ḍèrvíc’h}{n.}{\pf{thermique}}{Heat, warmth.}{}
\entry{deslẹ}{v.}{\pf{déceler}}{To detect, discover, uncover, reveal.}{\s{fut} deslẹre, \s{subj} deslẹs}
\entry{dévýr}{v.}{\pf{demeurer}}{\\To remain, stay.\\To live, dwell (+\s{iness} somewhere).}{}
\entry{ḍeý’ebhat’hẹ}{n.}{\pf{télépathie}}{Telepathy.}{}
\entry{ḍeý’ebhat’hic’h}{v.}{\pf{télépathique}}{To be telepathic.}{\s{fut} ḍeý’ebh\-at’hic’hre, \s{subj} ḍeý’ebhat’hic’hes}
\entry{dír}{v. tr.}{\pf{dire}}{+\s{acc} To say, tell (+\s{dat} someone).}{\s{fut} dírẹ́, \s{subj} díss}
\entry{díríj}{v.}{\pf{diriger}}{+\s{acc} To direct, run, oversee, operate (a business or establishment).}{\s{fut} díríje, \s{subj} díríjs}
\entry{dónẹ́}{v.}{\pf{donner}}{\s{+ dat \& acc/part} To endow, bestow, give. \textit{The \s{acc} is used when talking about concrete, measurable, and finite objects or sums; the partitive to talk about abstract concepts or parts of objects; the \s{dat} is the person being endowed with}.}{\s{fut} dónrẹ́, \s{subj} dónés}
\entry{ḍúr}{adv.}{\pf{toujours}}{\\\textit{(positive context)} Always.\\\textit{(negative context)} Still.}{}
\entry{ḍúr}{n.}{\pf{tour}}{Tower.}{}
\entry{duý’ýr}{v.}{\pf{douleur}}{To suffer, be in pain.}{}
\entry{dývrê}{particle}{\pf{du moins}}{At least. \textit{As in e.g. ‘At least, I think that \ldots’}.}{}
\entry{dy’ê}{v.}{\pf{tien}}{To be yours (\s{sg}).}{\s{fut} dy’êrẹ́, \s{subj} dy’ês}
\entry{e}{n.}{\pf{eau}}{Water.}{}
\entry{ebhẹ}{v.}{\pf{épais}}{To be thick.}{\s{fut} ebhrẹ, \s{subj} ebhes}
\entry{ec̣}{n.}{\pf{péché}}{Sin, transgression, wrongdoing.}{}
\entry{ec’hlér}{v.}{\pf{éclairer}}{To shine.}{}
\entry{ed}{particle}{\pf{et}}{\textit{Used in numbers, see §~\ref{subsec:numerals}}.}{}
\entry{eḍ}{v. irreg.}{\pf{être}}{To be.}{\s{forms} \textit{active only, see §~\ref{subsec:ed-paradigm}}}
\entry{eḍrrá}{v.}{\pf{étroit}}{Pointy.}{}
\entry{Eḍy’ê}{n.}{}{\textit{male given name, equivalent to English ‘Ste\-phen’}.}{}
\entry{ehyó}{n.}{\pf{écusson}}{Shield.}{}
\entry{el}{n.}{\pf{ailles}}{Wing, blade, fin.}{}
\entry{ez-}{pron.}{\pf{ses}}{Its, her, his, their.}{}
\entry{F}{adj.}{from \pf{fẹ}}{\textit{Logic.} False, $\bot$. \textit{Always capitalised}.}{}
\entry{fahaú}{conj.}{\pf{de façon que}}{+\s{opt} In such a way that, such that, so much so.}{}
\entry{faú}{adv.}{\pf{fort}}{Very, right, really. \textit{Postpositive intensifier placed after adjectives, particularly in the comparative or superlative degrees.}.}{}
\entry{faúr}{adj.}{\pf{fort}}{\textit{obsolete, except in proverbs} Strong, mighty.}{}
\entry{faúr₁}{n.}{\pf{force}}{\\Force, strength, power.\\\w{Faúr} \s{def} \textit{(science fiction, Star Wars)} The Force.}{}
\entry{faúr₂}{n.}{\pf{forme}}{Shape, form. \textit{Sometimes also spelt \w{fór}}.}{}
\entry{fé}{n.}{\pf{fin}}{End.}{}
\entry{fẹ}{v.}{\pf{faux}}{To be false, incorrect, wrong.}{\s{fut} faure, \s{subj} faus}
\entry{fẹhab}{v.}{\pf{faisable}}{To be possible, feasible.}{\s{fut} fẹhabre, \s{subj} fẹhas}
\entry{fèhẹ}{n.}{\pf{faisceau}}{\\Bundle, bunch, cluster.\\Beam, ray.}{}
\entry{fér}{v.}{\pf{faire}}{\\To do, make, build, construct, erect.\\\textit{Expletive; see §~\ref{subsubsec:personal-pronouns}}.}{\s{fut} fẹ́, \s{subj} fés}
\entry{férḍufraú}{v.}{\pf{en faire tout un fromage}}{To make a big fuss a\-bout something.}{\s{fut} fér\-ḍu\-fraúrẹ́, \s{subj} férḍufraús}
\entry{férr-rásvát’h}{n.}{\pf{faire la grasse mat’}}{A long, deep sleep.}{}
\entry{fic’h}{v.}{back-formation from *\w{fic’hs}, reinterpreted as a subjunctive stem; from \pf{fixer}}{To fix, set, establish.}{\s{fut} fic’hre, \s{subj} fic’hs}
\entry{fihas}{v.}{\pf{efficace}}{To be efficient.}{}
\refentry{fór}{faúr₂}
\entry{fórvẹ́}{v.}{\pf{informer}}{To inform (+\s{acc} sbd.) (+\s{aci} of sth.).}{\s{fut} fórv́ẹ́, \s{subj} fórvẹ́s}
\entry{fúr}{v.}{\pf{fournir}}{To deliver, provide (+\s{dat} sbd.) (+ \s{acc} with sth.).}{}
\entry{fý}{n.}{\pf{feu}}{Fire.}{}
\entry{hab’híy’}{v. tr.}{\pf{habiller}}{To dress +\s{acc} sbd.}{}
\entry{í}{n.}{\pf{hymne}}{Legend, myth.}{}
\refentry{ís}{ub’hrá}
\entry{isḍrár}{n.}{\pf{histoire}}{Story, tale.}{}
\entry{íváj}{n. archaic}{\pf{image}}{Image, picture.}{}
\entry{iý’ývî}{v.}{\pf{illuminer}}{To light up, illuminate.}{}
\entry{Já}{n.}{}{\textit{male or female given name, equivalent to English ‘John’ or ‘Joan’}.}{}
\entry{Jac’h}{n.}{\pf{Jacques}}{\textit{Male given name}.}{}
\entry{jávé}{adv.}{\pf{jamais}}{\textit{neg. only} Never, at no time.}{}
\entry{jaý’aú}{n.}{\pf{jalon}}{\\Nail.\\\textit{obsolete} Stake, pole.}{}
\entry{Jed’háy’}{n.}{\pf{Jedi}}{Jedi (Star Wars).}{}
\entry{júrdy’í}{adv.}{from archaic \pf{aujúrdy’í}}{Today.}{}
\entry{jys}{conj.}{\pf{jusqu’à ce que}}{+\s{opt} Until.}{}
\entry{jys}{adv.}{\pf{juste}}{Just, only, merely.}{}
\entry{jys}{v.}{\pf{injuste}}{To be unjust, unfair.}{\s{fut} jysre, \s{subj} jyss}
\entry{lá}{v.}{\pf{planer}}{To fly.}{}
\entry{lab’h}{v.}{\pf{laver}}{\\To wash, clean (+\s{acc} sth.).\\\textit{refl} To wash oneself, take a bath, have a shower.}{}
\entry{Lác}{n.}{}{\textit{female given name, equivalent to English ‘Bi\-anca’}.}{}
\entry{láḍ}{n.}{\pf{plante}}{\\Blade of grass.\\\textit{pl.} Grass.}{}
\entry{lánẹ́}{v.}{\pf{flâner}}{To meander.}{}
\entry{lár}{v.}{\pf{large}}{Wide, broad.}{}
\entry{lârdávrá}{n.}{\pf{langue de bois}}{Evasive, unclear, or overly formal speech.}{}
\entry{las}{v.}{\pf{placer}}{To place, put, set (+\s{acc} sth.).}{}
\entry{laú}{v.}{\pf{long}}{Long. \textit{Often in compounds \w{laú-} ‘long-’}.}{}
\entry{laúrs}{conj.}{\pf{lorsque}}{When (temporal only).}{}
\entry{laúrvé}{conj.}{from \w{laúrs} + \w{vé}}{\textit{(contraction)} But, when. \textit{Stressed on the first syllable}.}{}
\entry{laut’h}{v.}{\pf{flotter}}{Fl\-oat, hover, levitate.}{\s{fut} laut’hre, \s{subj} laut’hes}
\entry{le}{v.}{\pf{laisser} > *\w{lehe}}{\textit{Chiefly in questions or imperative.} To let, allow, permit.}{\s{fut} lere, \s{subj} les}
\entry{lé}{n.}{\pf{plaine}}{Plain, plains.}{}
\entry{lẹ-}{prefix}{\pf{plus}}{\textit{Affirming comparative prefix. See grammar}.}{}
\entry{lec’hḍraúvnẹ́t’hic’h}{v.}{\pf{électromagnétique}}{To be electromagnetic.}{\s{fut} lec’hḍraúvnẹ́t’hic’hre, \s{subj} lec’hḍraúvnẹ́t’hic’hes}
\entry{leḍ}{n.}{\pf{lettre}}{\\Letter (of the alphabet).\\\w{lý’aúleḍ} By the book.}{}
\entry{lèheb’h}{v.}{\pf{laisser-faire}}{+\s{aci} To let happen.}{}
\entry{lẹhuvud}{n.}{\pf{coup de foudre}}{Love at first sight.}{}
\entry{lér}{v.}{\pf{clair}}{To be evident, obvious, frank, clear.}{}
\entry{lér}{v.}{\pf{plaire}}{To please (+\s{acc} sbd.), be pleasing.}{}
\entry{lí}{v.}{\pf{lire}}{\\+\s{part} To read from.\\+\s{acc} To peruse, read entirely.}{\s{fut} lírẹ́, \s{subj} lís}
\entry{lit’hijy’}{v.}{\pf{litigier}}{To litigate, be at law with (+\s{dat} sbd.).}{}
\entry{lívnád}{n.}{\pf{limonade}}{Lemonade.}{}
\entry{liv́uhé}{n.}{\pf{livre} + \pf{bouquin}}{Book.}{}
\entry{lúr}{v.}{\pf{lourd}}{To be bulky, oversized, heavy.}{}
\entry{ly}{particle}{\pf{plus}}{\textit{obsolete variant of \w{lẹ}, sometimes leniting}.}{}
\entry{lý}{n.}{\pf{plume}}{Pen, quill.}{}
\entry{ḷý}{n.}{\pf{lieu}}{\textit{Base of the spatial correlatives. In senses 2–5, case affixes are attached before \this, e.g. sense 2 \s{all} \w{sẹb’héḷý} ‘hither’}.\\Place, location.\\\w{sẹ}\ldots\this \s{def} [from \w{sẹh}] Here, hither, hence, \&c. \textit{Proximal demonstrative (all cases)}.\\\w{sý’\L}\ldots\this \s{def} [from \w{sý’ẹ}] There, thither, thence, \&c. \textit{Distal demonstrative (all cases)}.\\\this\w{hes} \s{indef} [from \w{c’hes}] Where, whither, when\-ce, \&c. \textit{Interrogative (locative cases only)}.\\\w{s’}/\w{sá\L}\ldots\this \s{indef} [from \w{sá}] No\-where, from no\-where, \&c. \textit{Negative (locative cases only)}.}{}
\entry{lybhárdyt’há}{adv.}{pluspart du temps}{Often.}{}
\entry{lýr}{pron.}{\pf{leur}}{Their.}{}
\entry{lýrḍ}{v.}{from \pf{leur}; the ‘ḍ’ was added in analogy with \w{naúḍ} and \w{b’hauḍ}}{To be theirs.}{\s{fut} lýrḍre, \s{subj} lýrs}
\entry{lys}{adv.}{\pf{plus} /plys/}{\textit{neg. only} No longer, not any more. \textit{The meaning of this and \w{lẹ} swapped at some point for unknown reasons}.}{}
\entry{lýv́á}{v. \s{3rd} person only}{\pf{pleuvoir}}{To rain.}{\s{fut} lýv́áre, \s{subj} lýv́ás}
\entry{lývy’ér}{n.}{\pf{lumière}}{Light.}{}
\entry{lyzy’ýr}{adj.}{\pf{plusieurs}}{Several.}{}
\entry{n}{n.}{\pf{haine}}{Hate, hatred, loathing.}{}
\entry{nájẹ}{v.}{\pf{nager}}{To swim.}{\s{fut} náȷ́ẹ, \s{subj} nájes}
\entry{nárrahóḍ}{v.}{\pf{raconter} + \pf{narrer}; subj. from \pf{filer}}{To narrate, recount \s{+part sth.}, tell (\s{+dat} sbd.) a story.}{\s{fut} nárrahóḍe, \s{subj} fils}
\entry{nát’hýr}{n.}{\pf{nature}}{\\\textit{(chiefly)} \s{indef} Nature, the natural world.\\\s{def} The way something is.}{}
\entry{naúḍ}{v.}{\pf{nôtre}}{To be ours.}{\s{fut} naúḍre, \s{subj} naús}
\entry{néḍ}{v. dep.}{\pf{naître}}{To be born. \s{This is a deponent verb whose subject takes the \s{acc} and which only takes passive affixes.}.}{\s{sub} néhs}
\entry{nérjẹ}{n.}{\pf{énergie}}{Energy.}{}
\entry{nés}{adj.}{from earlier \w{nésḍ}}{Left (side), left-handed.}{}
\entry{nésḍ}{adj. archaic}{\pf{senestre}}{Left (side), left-handed.}{}
\entry{ní}{v.}{\pf{nier}, \s{fut} from \pf{contrer}, \s{subj} from \pf{oposer}}{To deny, ref\-use, reject, rebut (+\s{acc} sbd./sth.).}{\s{fut} c’haúḍé, \s{subj} aubhaus}
\entry{ní}{conj.}{\pf{ni}}{Neither, nor.}{}
\entry{níb’hẹ}{n.}{\pf{niveau}}{\\Level, degree.\\\s{def iness + gen} On the level of.}{}
\entry{nór}{v.}{back-formation from *\w{nórâ} from \pf{ignorant}}{To be ignorant, unaware, oblivious.}{}
\entry{nóráv}{n.}{from archaic \w{ḅá nórávíc’h}}{Druid.}{}
\entry{núb’h}{v.}{\pf{nouveau}}{To be new.}{\s{fut} núb’he, \s{subj} núb’hs}
\refentry{p-}{ḅ-}
\refentry{ph-}{ḅ-}
\refentry{p’h-}{bh-}
\entry{R}{adj.}{from \pf{ré}}{\textit{Logic.} True, $\top$. \textit{Always capitalised}.}{}
\entry{r}{n.}{\pf{air}}{Air. \textit{Frequently plural}.}{}
\entry{ra}{conj.}{\pf{swa} > *\w{rá}}{\\Or. \textit{exclusive, see also~\w{u}}.\\\w{u}/\w{ra} \ldots\ \w{ra} \ldots\ ‘either \ldots\ or \ldots’ \textit{(exclusive)}.}{}
\entry{râ}{v.}{\pf{gagner}}{To win, gain, earn (+\s{acc} sth.).}{}
\entry{rá₁}{n.}{\pf{loi}}{Law, rule, regulation.}{}
\entry{rá₂}{adj.}{\pf{grand}}{Big, large, great.}{}
\entry{rá₃}{n.}{\pf{mois}}{Month.}{}
\entry{rá₄}{n.}{\pf{voix}}{Voice.}{}
\entry{rá₅}{n.}{\pf{bras}}{Arm.}{}
\entry{Ráb’h}{n.}{unknown; presumably the name of some celebrity or local deity}{\\\textit{indecl.} \s{def sg} \textit{always} \s{nom} \textit{or} \s{voc} Ráb’h. \textit{Main god of the ULTRAFRENCH pantheon; usually male. Old-fashioned also often all-caps \w{RÁB’H}.}.\ex \s{Snet’h}, \s{i.17}: \w{au lebálá daú RÁB’H} ‘and thus spake Ráb’h’.\\\w{Ráb’h sénýr} \s{def sg} Lord Ráb’h. \textit{Used for sense~{\bf 1} in all other cases; as with all names, only \w{sénýr} is inflected. Old-fashioned often \w{RÁB’H Sénýr}}.\ex \s{Snet’h}, \s{8.1}: \w{au labraúc RÁB’H naút B’héhénýr} ‘and they came to our Lord Ráb’h’.\\\textit{(rarely)} The main god of another culture. \textit{Only attested figuratively. Not capitalised in this sense, and declined like a regular word.}.\ex \s{Snet’h}, \s{ii.3}: \w{ledéraújá’z derévôt’he láráb’h} ‘their god demanded they return’.}{}
\refentry{ráb’h}{v́ár}
\entry{ráb’háy’}{v.}{\pf{travailler}, \s{fut} and \s{subj} from \pf{bos\-ser}}{To work.}{\s{fut} bohér, \s{subj} bos}
\entry{rác’hánár}{n.}{from \w{ráhe} + \w{c’hánár}}{Airship, dirigible.}{}
\entry{rác’hsaý’ad}{v.}{\pf{raconter des salades}}{To lie, tell tall tales, overexaggerate.}{\s{fut} rác’h\-sa\-ý’e, \s{subj} rác’hsaýs}
\entry{rád}{v. tr.}{\pf{rendre}}{To surrender +\s{acc} sth. (\s{dat} to sbd.).}{}
\entry{râd}{v.}{\pf{prendre}}{+\s{acc} \textit{or} \s{part} To grab. \textit{The \s{part} usually implies that only a part or some of a larger whole is grabbed, e.g. a handful of sand)}.}{}
\entry{râdrásôn}{v.}{\pf{prendre ses jambe à son cou}}{To run.}{\s{fut} râdrásônre, \s{subj} râdrásôns}
\entry{rádrénẹ́}{v. + \s{aci}}{\pf{les doigts dans le nez}}{To put no effort into.}{\s{fut} rádrénrẹ́, \s{subj} rádrénẹ́s}
\entry{râdvâ-}{prefix}{\pf{grandement}}{\textit{Superlative prefix. See grammar}.}{}
\entry{rád’hérn}{n.}{from \w{rá} + \w{dérny’é}}{\textit{(always definite)} Last month.}{}
\entry{rád’hsy’ô}{n.}{\pf{traditon}}{Tradition, custom.}{}
\entry{rád’hyc’hsy’ô}{n.}{\pf{traduction}}{Translation.}{}
\entry{râhaúḍ}{v.}{\pf{recontrer}}{To meet, encounter, come face to face (+\s{all} with sbd.).}{\s{fut} râhaúḍre, \s{subj} râhaús}
\entry{ráhe}{n.}{\pf{oiseau}}{Bird.}{}
\entry{ráhé}{n.}{from \w{ráhe} + \w{ráhó}}{Flying fish.}{}
\entry{ráhé}{n.}{\pf{voisin}}{Neighbour.}{}
\entry{ráhẹ}{conj.}{\pf{quoique}}{+\s{subj} Although, though.}{}
\entry{râhẹ}{n.}{\pf{français}}{Human, person.}{}
\entry{ráhis}{v.}{\pf{raciste}}{To be racist.}{\s{fut} ráhise, \s{subj} ráhiss}
\entry{ráhó}{n.}{\pf{poisson}}{Fish.}{}
\entry{ráhó}{n.}{\pf{gazon}}{Grassland, grassy field, meadow.}{}
\entry{ráhut’h}{n.}{\pf{grand} + \pf{couteau}}{Sword, blade \ex \w{áráhut’h’t ilý ly b’haúr} ‘the pen is mightier than the sword’ \textit{(originally a fossilised, obsolete ACI: \w{á\-hut’h\-rá éḍ ilý lẹb’haúr})}.}{}
\entry{rál}{n.}{\pf{toile}}{Canvas.}{}
\entry{rár}{v.}{\pf{voir}}{To see (+\s{part} sbd./sth.).}{\s{fut} b’hérẹ́, \s{subj} rárs}
\entry{rárd}{v.}{\pf{regarder}}{\\+\s{acc} To watch.\\+\s{part} To look at.}{\s{fut} rárdre, \s{subj} rárds}
\entry{râsír}{v.}{\pf{transpirer}}{+\s{aci}.\\To come to light, become known, transpire.\\\s{pres ant} For it to be clear, apparent, evident that \ldots \textit{Lit. ‘it has come to light that \ldots’}.}{\s{fut} râsírẹ́, \s{subj} râsírs}
\entry{rát’hẹ}{particle}{\pf{vois-tu}}{You see, you know.}{}
\entry{raû}{n. archaic}{\pf{tronc}}{Log (of a tree).}{}
\entry{raû}{interj.}{\pf{gône}}{Kid. \textit{This is grammatically a vocative—not that one could tell since it looks identical to the absolutive}.}{}
\entry{raú(b’hc’h)-}{prefix}{from \w{rób’hoc’h}}{\textit{Causative prefix, see §~\ref{subsec:diachrony-and-derivation}}.}{}
\entry{raúb’hẹ}{n.}{\pf{robot}}{Robot.}{}
\entry{raûc}{n.}{\pf{tronche}}{Head.}{}
\entry{raûd’hárb}{n.}{\pf{tronc d’arbre}}{Log (of a tree).}{}
\entry{raúhérẹ́}{v.}{from \w{raú-} + \w{sérẹ́}}{To tighten, make tighter (+\s{acc}).}{\s{fut} raúhérrẹ́, \s{subj} raúhérẹ́s}
\entry{raúhy’b’h}{v.}{from \w{raú-} + \w{sy’b’h}}{To raise, lift up (+\s{acc} sth.) (+\s{ela} from sth.).}{}
\entry{raúl}{n.}{\pf{parole}}{\\Language, speech, word.\\\w{Raúl} \textit{(definite only)} Short for \w{T’hebhaú Raúl}. \textit{\s{nom sg} irreg. \w{Raúl}; all other forms are regular}.}{}
\entry{raúvá}{n.}{\pf{fromage}}{Moon.}{}
\entry{ráv́â}{adv.}{\pf{rarement}}{\textit{neg. only} Seldom, rarely (ever).}{}
\entry{rávér}{n.}{\pf{grammaire}}{\\Grammar, the grammatical rules of a language.\\A textbook describing the grammar of a language.}{}
\entry{ráy’á}{v.}{\pf{voyage}}{\\To travel, go on a journey.\\\textit{n.} Travel, voyage, journey.}{}
\entry{ráy’é}{v.}{\pf{noyer}}{To drown.}{}
\entry{ráy’ê}{n.}{\pf{moyen}}{\\Way, means, method.\\\w{ráy’ê y’aúhý} + \s{aci} There is no way, that \ldots{}.\\\s{instr pl} \w{b’hehráy’ê} How, by what means, in this way.}{}
\entry{ráý’ẹ}{v.}{\pf{râler}}{To complain, grumble.}{}
\entry{ré}{v.}{\pf{vrai}}{To be true, correct, right.}{\s{fut} rẹ́, \s{subj} rés}
\entry{ré}{n.}{\pf{rai}}{Ray, beam.}{}
\entry{ré}{v.}{\pf{créer}, \s{subj} from \pf{fabriquer}}{To create, make (\s{+acc} sth.).}{\s{fut} rẹ́éré, \s{subj} faríc’hs}
\entry{ré}{adj.}{\pf{près}}{Near, close, nearby.}{}
\entry{ré}{v. intr.}{\pf{errer}}{To wander, roam (+\s{perl} across sth.).}{}
\entry{ré}{adv.}{en vain}{In vain, for nothing.\textit{Usually preceded directly by the verb it applies to}.}{}
\entry{ré}{n.}{\pf{souhait}}{Wish.}{}
\entry{rê}{conj.}{\pf{bien que}}{+\s{subj} Although, though.}{}
\entry{rê}{v.}{\pf{trine}}{To be composed of three parts or people; triune.}{\s{fut} rêrẹ́, \s{subj} rês}
\entry{rê}{n.}{\pf{airain}}{Copper.}{}
\entry{rê}{n.}{\pf{point}}{Point (in a score).}{}
\entry{ré-}{prefix}{\pf{très}}{\textit{Superlative prefix. See grammar}.}{}
\entry{rê-}{prefix}{\pf{moins}}{\textit{Neutral comparative prefix. See grammar}.}{}
\entry{réaû}{n.}{from \w{ré}}{Creation, making.}{}
\entry{rébh}{v.}{\pf{préparer}}{To anticipate (+\s{acc} sth.).}{}
\entry{rébhós}{n.}{\pf{réponse}}{Answer, response, reply.}{}
\entry{rẹ́b’h}{v. or n.}{\pf{rêver}}{\\To dream (+\s{gen} of sth.).\\Dream, a dreaming.}{\s{fut} rẹ́v́e, \s{subj} rẹ́b’hs}
\entry{réb’hní}{v.}{\pf{prévenir}}{\\To prevent, stop (+\s{acc} sth. from happening).\\To forewarn (+\s{part} of sth.).}{\s{fut} réb’hníre, \s{subj} réb’hnís}
\entry{réḍ}{v.}{\pf{souhaiter}}{To wish (+\s{acc/aci} for sth.).}{}
\entry{rêd}{v.}{\pf{craindre}}{+s{opt} To fear, lest \ldots \textit{Construed with the negated optative}.}{\s{fut} rêdrẹ́, \s{subj} rês}
\refentry{rêd}{ḅẹt’hẹ}
\entry{rêdrsýrśẹ}{v.}{\pf{prendre sur soi}}{\\+\s{aci} To take upon onself to do sth.\\+\s{pci} To take upon oneself to start doing sth.}{}
\entry{rẹ́dy’í}{v.}{\pf{réduire}}{To reduce (+\s{acc} \textit{or pass.} sbd./sth.) (+\s{all} to sth.).}{\s{fut} rẹ́dy’ré, \s{subj} rẹ́dy’ís}
\entry{rêd’hes}{particle}{\pf{bien sûr}}{Of course, certainly, surely.}{}
\entry{rẹ́flec̣}{v.}{\pf{réfléchir}}{To think (+\s{part} sth.).}{}
\entry{réhẹv́}{v.}{\pf{recevoir}}{To receive.}{\s{fut} réhẹv́é, \s{subj} rẹsy}
\entry{rêr}{n.}{\pf{fringues}}{\\An article of clothing, garment, piece of clothing.\\\textit{pl.} Clothes, garments.}{}
\entry{rés}{n.}{\pf{reste}}{Rest, remainder.}{}
\entry{rét’hád}{v.}{\pf{prétendre}}{To claim, allege.}{\s{fut} rét’hádrẹ́, \s{subj} rét’h\-ádes}
\entry{rét’hẹ}{v.}{\pf{traiter}}{To handle, take care of, deal with.}{\s{fut} rét’hẹre, \s{subj} rét’hes}
\entry{rét’hír}{v.}{\pf{retirer}}{\\(+\s{acc}) To pull, draw, withdraw.\\+\s{part} To pull on sth. without actually moving it; to try to pull sth.}{\s{fut} rét’hírẹ́, \s{subj} rét’hírs}
\entry{révôt’hẹ}{v.}{\pf{remonter}}{To return, come back.}{}
\entry{ríb’hy’ér}{n.}{\pf{rivière}}{River.}{}
\entry{rívnél}{n.}{\pf{criminel}}{Scoundrel, someone without virtue.}{}
\entry{ríy’ŷrệ}{n.}{\pf{prieuré}}{Priory.}{}
\entry{rjẹ}{n.}{\pf{Hergé}}{Comic book.}{}
\entry{rób’hoc’h}{v.}{\pf{provoquer}; future from \pf{infliger}}{\s{+acc} To cause, make happen.}{\s{fut} flijé, \s{subj} rób’hoc’hs}
\entry{rrá}{v.}{\pf{croire}}{Believe (something or someone).}{\s{fut} rrẹ́, \s{subj} rrás}
\entry{rráḍraúc}{n.}{\pf{droit} + \pf{gauche}}{Side.}{}
\entry{rrád’hahánár}{n.}{\pf{froid de canard}}{Extreme cold, coldness.}{}
\entry{rúb’h}{v.}{\pf{trouver}}{To find, discover.}{}
\entry{rvá}{interj.}{of unknown origin}{Alas, woe, oh. \textit{Exclamation of distress, surprise, sadness, or regret}.}{\textit{after words that end with ‘r’, this is spelt \w{-vá} instead}}
\entry{rýc̣ér}{v.}{\pf{requerir}}{To ask, question.}{}
\entry{rýd}{v.}{\pf{rude}}{To be uneven, rough, rugged.}{}
\entry{rýl}{v.}{\pf{brûler}}{\\\s{+acc} To burn.\\\s{+part} To scorch, singe.}{}
\entry{rýl}{n.}{\pf{gueule}}{Face.}{}
\entry{rýrŷ}{v.}{\pf{rugueux}}{To be rough, rugged.}{}
\entry{rýsḍ}{v.}{\pf{frustrer}}{To frustrate, vex, annoy.}{}
\entry{rývýr}{.n}{\pf{rumeur}}{History.}{}
\entry{rýý’ẹ́}{v.}{\pf{céruléen}}{To be cerulean, sky-blue.}{}
\entry{rzaúsḍ}{v.}{\pf{exhaustif}}{\\To be exhaustive, comprehensive, complete.\\To be finished, completed.}{\s{fut} rzaúsḍre, \s{subj} rzaúsḍs}
\entry{s}{conj.}{\pf{si}}{If, when, whenever.}{}
\entry{sá}{particle}{\pf{sans}}{Not, no. \textit{Always enclitic \w{s’} before vowels. This particle is used only in the subjunctive; see also \w{asý’ýâ}, \w{t’hé}}.}{}
\entry{sá}{conj.}{\pf{sans que}}{+\s{subj} Without (doing sth.).}{}
\entry{sáḍy’ér}{n.}{\pf{sanctuaire}}{Sanctuary, shrine.}{}
\entry{sáhẹ}{v.}{\pf{insensé}}{To be preposterous, absurd, nonsensical.}{\s{fut} sáhere, \s{subj} sáhes}
\entry{saj}{v.}{\pf{sage}}{To be wise, prudent.}{}
\entry{sajès}{n.}{\pf{sagesse}}{Wisdom.}{}
\entry{Sásc’hríḍ}{n. never lenited}{\pf{sanskrit}}{The Sanskrit language.}{}
\entry{sásy’él}{v.}{\pf{essentiel}}{To be essential.}{\s{fut} sásy’élẹ́, \s{subj} sásy’éls}
\entry{sauc’h}{conj.}{\pf{sauf que}}{+\s{subj} Except that.}{}
\entry{saul}{n.}{\pf{sol}}{Sun.}{}
\entry{saúr}{n.}{\pf{sorte}}{\\Kind, sort, type, form.\\\s{def + gen} (some) kind(s) of.}{}
\entry{saut’h}{v. intr. or tr.}{\pf{sauter}}{To teleport, translocate, warp (+\s{acc} sth.).}{}
\entry{sauz}{n.}{\pf{chose}}{Thing, object.}{}
\entry{sauz-aud}{adj.}{\pf{autre chose}}{Something else, another thing.}{}
\refentry{sauzaud}{sauz-aud}
\entry{sav́á}{v.}{\pf{savoir}}{To know (+\s{part/acc} sth. \textit{case depends on the depth of the speaker’s understanding}).}{\s{fut} saúr, \s{subj} sac}
\entry{Sávýy’él}{n.}{\pf{Samuel}}{\textit{Male given name}.}{}
\entry{sḅé}{v.}{\pf{espérer}}{\\To want (+\s{acc/inf} sth.).\\+\s{opt} To wish, want, desire.}{\s{fut} sḅérẹ́, \s{subj} sḅés}
\entry{sḅrí}{n.}{\pf{espirit}}{Soul.}{}
\entry{sb’hé}{v.}{\pf{se baigner}}{To bathe.}{}
\entry{sẹ}{particle}{\pf{ainsi}}{So, thus, as a result.}{}
\entry{séḅ}{v.}{\pf{simple}}{To be plain, simple.}{\s{fut} séḅrẹ́, \s{subj} séḅs}
\entry{seb’haúd}{v. intr.}{\pf{s’effondrer}}{To cave in, collapse.}{}
\entry{sèd’h}{part.}{from \pf{c’est du}}{It is due to (+\s{gen} sth. / +\s{aci} the fact that...).}{}
\entry{sẹh}{det.}{\pf{ceci}}{+\s{def} \textit{noun} This, these. \textit{Precedes and is attached to nouns}.}{}
\entry{sẹhérél}{v.}{\pf{se quereller}}{To quarrel, argue, fight about (+\s{part}).}{}
\entry{sehul}{v.}{\pf{s’écouler}}{To flow.}{}
\entry{sẹhúr}{v.}{\pf{secourir}}{To help, succour, give aid (+\s{dat} to sb.) (+\s{aci}/\s{acc} with sth.).}{\s{fut} sẹhúrre, \s{subj} sẹhús}
\entry{sénýr}{n.}{\pf{seigneur}}{\\Lord.\\\textit{Short for \w{Ráb’h sénýr}}.}{}
\entry{sẹrád}{v. intr.}{\pf{se rendre}}{To surrender.}{}
\entry{sérḍé}{det.}{\pf{certain}}{Certain, particular but not specified.}{}
\entry{sérẹ́}{v.}{\pf{serré}}{\\To be tight, close-fitting, snug.\\\s{indef} \textit{usually} \s{instr} \w{c’hýr sérệ} A heavy heart.}{\s{fut} sérrẹ́, \s{subj} sérẹ́s}
\entry{sèt’h}{v.}{\pf{sentir}}{To feel.}{\s{fut} sèt’he, \s{subj} sès}
\entry{séy’ẹ́}{v.}{\pf{essayer}}{+\s{part} \textit{or} \s{inf} To try, attempt.}{\s{fut} séy’ẹ́rẹ́, \s{subj} séy’ẹ́s}
\entry{siḍ}{n.}{\pf{site}}{Facility, site.}{}
\entry{sisḍé}{n.}{\pf{système}}{System.}{}
\entry{Sit’h}{n.}{\pf{Sith}}{Sith (Star Wars).}{}
\entry{sit’há}{conj.}{\pf{si tant est que}}{+\s{opt} Supposing that; if, assuming that.}{}
\entry{sívý’ér}{v.}{\pf{similaire}}{To be similar, alike (+\s{gen} to sth.).}{}
\entry{Snet’h}{n.}{}{\textit{Family name, equivalent to English ‘Smyth’}.}{}
\entry{sol}{n.}{\pf{sol}}{Ground, floor, earth, soil.\textit{The plural may be used to indicate a large quantity of soil}.}{}
\entry{suḍ}{v.}{\pf{soutenir}}{\\+\s{acc} To support, hold up.\\+\s{part} To help support, hold up part of.}{}
\entry{sud’hénvâ}{adv.}{\pf{soudainement}}{Suddenly.}{}
\entry{suf}{n.}{\pf{souffre}}{Pain.}{}
\entry{sufb’h}{n.}{\pf{souffle} + \pf{vie}}{Life.}{}
\entry{susy’é}{v.}{\pf{soucier}}{+\s{part, pci} To care about, worry about.}{\s{fut} susy’ére, \s{subj} susy’és}
\entry{swi}{det.}{\pf{celui}}{The one, that one, this one.}{}
\entry{sybhẹ́rýr}{v.}{\pf{supérieur}}{\textit{intr. or} +\s{gen} To be superior to, better than, higher than.}{\s{fut} sybhẹ́rýrẹ́, \s{subj} sybhẹ́rýrs}
\entry{syḅlẹ}{v.}{\pf{suppléer}}{\\To supplement (\s{acc} sth.) (+\s{instr} with sth.). \textit{If no \s{instr} is present, the subject is implied to be the supplement}.\\\w{syḅlâ} \textit{adj.} Additional, extra.}{}
\entry{syhyý’á}{v.}{\pf{succulent}}{To be succulent, delicious.}{\s{fut} syhyý’áré, \s{subj} syhyý’ás}
\entry{syl}{v.}{\pf{seul}}{\\To be the only one.\\To be lone, alone.}{\s{fut} syle, \s{subj} syls}
\entry{sy’b’h}{v. intr.}{\pf{se lever}}{To rise (+\s{ela} from sth.).}{}
\entry{sy’ê}{v.}{\pf{sien}}{To be his, hers, its.}{\s{fut} sy’êrẹ́, \s{subj} sy’ês}
\entry{sý’ẹ}{det.}{\pf{cela}}{+\s{def} \textit{noun} That, those. \textit{Precedes and is attached to nouns; often \w{sý’} before vowels, with one apostrophe, not two}.}{}
\refentry{s’}{sá}
\refentry{t-}{ḍ-}
\entry{t’hé}{conj.}{\pf{de peur que} > *\w{dbhýrc’h} > *\w{dýrc’h} > *\w{dc’hý} > \this}{Not, no. \textit{Always \w{t’h’\N} before vowels, but does not nasalise if the ‘é’ is still present. This particle is used only in the optative; see also \w{asý’ýâ}, \w{sá}}.}{}
\entry{T’hebhaú}{n. or adj.}{from \w{t’hebhaúz}}{(ULTRA-) France, (ULTRA-)French.}{}
\entry{T’hebhaú Raúl}{n. def. sg.}{from \w{t’hebhaúz} + \w{raúl}}{The ULTRAFR\-ENCH language. \textit{Only \w{T’hebhaú} is declined as though the entire phrase were one word. In informal speech and writing, this is typically shortened to \w{Raúl}}.}{\s{nom sg} \textit{irreg.} \w{T’hebhaú Raúl}}
\entry{t’hebhaúz}{v.}{\pf{jeter l’éponge}}{To be (ULTRA-)French.}{\s{fut} t’hebhaúźe, \s{subj} t’hebhaúś}
\entry{t’hiy’e}{v.}{from \w{yt’hiy’ihẹ}; \s{subj} via ba\-ck-formation from the \s{fut}}{+\s{part} To use, make use of.}{\s{fut} t’hiźe, \s{subj} \s{t’hizes}}
\entry{u}{conj.}{\pf{ou}}{\\Or. \textit{Inclusive, see also \w{ra}}.\\\w{u} \ldots\ \w{u} \ldots\ ‘\ldots\ or \ldots’ \textit{(inclusive)}.}{}
\entry{ub’h}{v.}{\pf{ouvrir}}{To open.}{\s{fut} uv́, \s{subj} ub’hs}
\entry{ub’hrá}{v.}{\pf{pouvoir}}{\\+\s{inf/aci} To be able to, can. \textit{Never construed with an \s{inf} if it in and of itself is the infinitive of an \s{aci} or \s{pci}, in which case the variant with the \s{part} (\senseref{2}) is used instead}.\\+\s{part} To be capable of \ldots\\\s{opt cond i + aci} To be possible; may. \textit{Dynamic or epistemic, never deontic; this and sense 4 are essentially a more emphatic optative}.\\\s{opt cond ii + aci} Might. \textit{Dynamic or epistemic, never deontic}.}{\s{fut} úrẹ́, \s{subj} ís}
\entry{ulíy’ẹ́}{v.}{\pf{oublier}}{To forget.}{\s{fut} ulíy’ẹ́rẹ́, \s{subj} ulíy’ẹ́s}
\entry{úrbh}{conj.}{\pf{pour peu que}}{+\s{opt} Provided that, so long as.}{}
\entry{urdálbhaúrḍ}{n.}{\pf{avoir un oursin dans le portefeuille}}{A very rich person; billionaire.}{}
\refentry{úrẹ́}{ub’hrá}
\entry{uy’ed’háb’hrí}{v.}{\pf{rouler dans la farine}}{To scam, cheat, swindle.}{\s{fut} uy’e\-d’háv́e, \s{subj} uy’ed’háb’hrís}
\entry{vá}{n.}{\pf{mât}}{Mast.}{}
\refentry{vá}{rvá}
\entry{vádłabhaud’hávúrsab’hád’háváb’hrárḍuẹ}{v. literary}{\pf{vendre la peau de ours avant de avoir tué}}{To depend on predictions of the future. \textit{Of disputed origin; first attested in the works of the Early UF comedian \s{J. A. B. Snet’h}}.}{\s{fut} vád\-ła\-bhau\-d’há\-vúr\-sa\-b’há\-d’há\-vá\-b’hrár\-ḍu\-re, \s{s} vád\-ła\-bhau\-d’há\-vúr\-sa\-b’há\-d’há\-vá\-b’hrár\-ḍus}
\entry{vâhẹ}{v.}{\pf{manquer}}{\\+\s{gen} To lack, want.\\+\s{part} \textit{or} \s{pass} To miss. \textit{The object and subject of this verb are swapped compared to English ‘to miss’, e.g. \w{b’hývvâhé} (\s{2pl.act} + \s{1sg.pass}) ‘I miss you (\s{pl})’, lit. roughly ‘you (\s{pl}) are wanting to me’)}.\\+\s{acc} To miss out on.}{\s{fut} vâhérẹ́, \s{subj} vâhés}
\entry{váj}{n.}{from \w{íváj}}{Image, picture.}{}
\entry{válḍrét’hás}{n.}{\pf{maltraitance}}{Torture.}{}
\entry{válfèz}{v.}{\pf{malfaisant}}{To be malfeasant, evil, malevolent.}{\s{fut} válfèź, \s{subj} válfès}
\entry{válv́áy’}{v.}{\pf{malvoyant}}{To be blind.}{}
\entry{válvê}{v.}{\pf{malmener}}{To mistreat, torture.}{\s{fut} válv́e, \s{subj} válvês}
\entry{v́ár}{v. irreg.}{\pf{devoir}}{\\\s{pass} +\s{aci} Must, have to, be obliged to. \textit{The subject is always in the passive in this sense only}.\\+\s{dat} To owe sbd. (+\s{acc} sth.).\\\s{cond i + aci} Even if; \textit{e.g.} \w{aúrdyssa dẹće} \textit{‘even if he should fail’}.}{\s{cond i, ii} dy, \s{fut} dv́e, \s{subj} ráb’h}
\entry{vás}{n. \s{pl def}}{\pf{masses}}{The masses, the people.}{}
\entry{vaúb’hẹ}{v. irreg.}{\pf{mauvais}}{\\To be bad.\\To be wrong, incorrect, inappropriate.}{\s{fut} bíré, \s{subj} bíres; \s{comp} lẹbír, y’ŷbír, rêbír; \s{sup} réb’hír, râdvâbír}
\entry{vaûd}{n.}{\pf{monde}}{World.}{}
\entry{vaûḍ}{v.}{\pf{montrer}}{To show, display (+\s{acc} sth.).}{}
\entry{vaúd’hér}{v.}{\pf{modérer}}{To be moderate.}{}
\entry{vaût’há}{n.}{\pf{montagne}}{Mountain.}{}
\entry{váý’eb’his}{n.}{\pf{maléfice}}{Vice.}{}
\entry{váý’ýr}{n.}{\pf{malheur}}{Tragedy, misfortune.}{}
\entry{váłé}{conj.}{\pf{malgré que}}{+\s{subj} Despite that, in spite of.}{}
\entry{vé}{conj.}{\pf{mais}}{But, however, although.}{}
\entry{vê₁}{adv.}{\pf{demain}}{Tomorrow.}{}
\entry{vê₂}{n.}{\pf{main}}{Hand.}{}
\entry{véc}{n.}{\pf{mèche}}{\\A strand of hair.\\\s{pl.} Hair.}{}
\entry{véḍ}{v.}{\pf{mettre}}{To lay, put, place (+\s{acc} sth.).}{}
\entry{véhýr}{conj.}{\pf{dans la mesure où}}{Insofar as.}{}
\entry{véhýr}{v/n.}{\pf{mesure}}{\\To measure.\\Measurement.}{\s{fut} véhýrẹ́, \s{subj} véhýrs}
\entry{vér}{n.}{\pf{mère}}{\textit{(informal)} Mum, mom.}{}
\entry{vérjet’hic’h}{v.}{\pf{énergétique}}{To be vigorous, energetic.}{}
\entry{vérr}{n.}{\pf{mer}}{Sea, ocean.}{}
\entry{vérs}{interj.}{\pf{merci}}{\\Thank you. (+\s{gen} for sth.).\\\w{dyvérs fér} To thank (+\s{dat} sbd.) (+\s{gen} for sth.).}{}
\refentry{vérvá}{vér + vá}
\entry{vêt’hnâ}{adv.}{from \pf{maintenant}, lenited for unknown reasons}{Now.}{}
\refentry{véy’ýr}{baú}
\entry{víd’hẹ}{n.}{\pf{midi}}{Noon, midday.}{}
\entry{Víd’hic’hlaúry’ê}{n.}{\pf{Midichlorien}}{Midichlorian (Star Wars).}{}
\entry{vísy’ô}{n.}{\pf{émission}}{\\Emission.\\Programme, broadcast, show.}{}
\entry{vnásḍér}{n.}{\pf{monastère}}{Castle.}{}
\entry{vú}{adj.}{\pf{moult}}{Many, much, a lot of.}{}
\entry{vúb’hvâ}{n.}{\pf{movement}}{Movement, motion.}{}
\entry{vúslihé}{n.}{\pf{mousse} + \pf{lichen}}{Moss.}{}
\entry{vvâ}{n.}{\pf{maman}}{Mother.}{}
\entry{vvâ}{n.}{\pf{moment}}{Moment, instant.}{}
\entry{vvaúríhe}{v. (in)tr.}{\pf{mémoriser}}{To remember.}{\s{fut} vvaúríźe, \s{subj} vvaúríhes}
\entry{vŷ}{v.}{\pf{mener}}{To lead.}{\s{fut} menre, \s{subj} mens}
\entry{w}{v.}{\pf{enlever}}{To remove (+\s{acc} sth.).}{}
\entry{ýr}{v.}{\pf{heurter}}{To hit, strike.}{\s{fut} ýrḍ, \s{subj} ýrs}
\entry{yt’hiy’ihẹ}{v.}{\pf{utiliser}}{+\s{part} \textit{Archaic}. To use, make use of.}{\s{fut} yt’hiy’iźe, \s{subj} yt’hiy’\-ihẹs}
\entry{Yý’is}{n.}{\pf{Ulysse}}{\textit{Male given name}.}{}
\entry{y’ác’hraúníc’h}{v.}{\pf{diachronique}}{To be diachronic.}{\s{fut} y’ác’hraú\-níc’hre, \s{subj} y’ác’hraúníc’hes}
\entry{y’aúhý}{part.}{\pf{il n’y a aucun}}{There is no, there are no, there is none.}{}
\entry{ý’aúhý}{part.}{\pf{il y a aucun}}{There is, there are.}{}
\entry{y’aúý’}{v.}{back-formation from \w{y’aúý’vâ}, displacing earlier \w{y’aúý’á}}{To be violent, vehement.}{}
\entry{y’aúý’á}{v. archaic}{back-formation from \w{y’aúý’ávâ}}{To be violent, vehement.}{}
\entry{y’aúý’ávâ}{adv. archaic}{\pf{violament}}{Violently, vehemently.}{}
\entry{y’aúý’vâ}{adv.}{back-formation from \w{y’aúý’á}, displacing earlier \w{y’aúý’ávâ}}{Violently, vehemently.}{}
\entry{y’é}{pron.}{\pf{rien}}{Nothing. \textit{Like most negative polarity items, this induces negation of the verb}.}{}
\entry{y’ẹ́}{v.}{\pf{nier}}{To forbid, deny.}{\s{fut} y’ẹ́rẹ́, \s{subj} y’ẹ́s}
\entry{y’ê}{v.}{\pf{mien}}{To be mine.}{\s{fut} y’êrẹ́, \s{subj} y’ês}
\entry{y’éjúré}{n.}{\pf{siège} + \pf{tabouret}}{Chair, seat.}{}
\entry{y’ér}{adv.}{\pf{hier}}{Yesterday.}{}
\entry{y’í}{n.}{\pf{nuit}}{Night.}{}
\entry{y’í}{n.}{\pf{puits}}{Well (water source).}{}
\entry{y’íhá}{v.}{\pf{puissant}}{To be powerful, mighty, puissant.}{}
\entry{y’ír}{v. (in)tr.}{\pf{ouïr}}{To understand, listen, \textit{(rarely)} hear.}{\s{fut} aúré, \s{subj} rás}
\entry{y’ís}{conj.}{\pf{puisque}}{Considering that, since, because. \textit{Unlike \w{c’haúr}, this does not take the subjunctive; it is used to indicate the (potential) cause of something}.}{}
\entry{y’úr}{n.}{\pf{jour}}{\\Day.\\\w{órdy’úr ád’y’úr} Day after day. \textit{Contracted \s{ela} and \s{ill}}.}{}
\entry{y’ŷ}{n.}{from \w{y’ŷvéłáfrí}}{Eye.}{}
\entry{y’ŷ-}{prefix}{\pf{mieux}}{\textit{Denying comparative prefix. See grammar}.}{}
\entry{y’ŷvéłáfrí}{n. pl. archaic}{\pf{yeux de merlan frit}}{Eyes.}{}
\entry{Zauḅ}{n.}{\pf{Ésope}}{Aesop.}{}
\refentry{’sý’ýâ}{asý’ýâ}






\end{document}
