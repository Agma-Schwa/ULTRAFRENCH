\documentclass[a4paper, 12pt, twoside, openright, final]{book}
\usepackage[margin=2cm]{geometry}
\usepackage{fontspec}
\usepackage{unicode-math}
\usepackage[english]{babel}
\usepackage{csquotes}
\usepackage{array, tabularx, multirow}
\usepackage{longtable}
\usepackage{float}
\usepackage{tabularray}
\usepackage{graphicx}
\usepackage{wasysym}
\usepackage{xcolor}
\usepackage{amsmath}
\usepackage[hidelinks]{hyperref}
\usepackage{tikz}

%\definecolor{bgcolor}{HTML}{2D2A2E}
%\definecolor{fgcolor}{HTML}{FAFCFC}
%\pagecolor{bgcolor}
%\color{fgcolor}

\setmainfont[Numbers=OldStyle]{Minion 3}
\setmathfont{latinmodern-math.otf}
\setmathfont[range=\mathit]{Minion 3 Italic}

%% TODO: Remove this and fix overfull boxes.
%% TODO: Switch to the book class at this point.
\hfuzz=10000pt
%\geometry{showframe}

%% Normalise to NFC.
\XeTeXinputnormalization=1

%% %%%%%%%%%%%%%%%%%%%%%%%%%%%%%%%%%%%%%%%%%%%%%%%%%%%%%%%%%%%%%%%%%%%%%%%%%%%%%
%%  Environment and Layout
%% %%%%%%%%%%%%%%%%%%%%%%%%%%%%%%%%%%%%%%%%%%%%%%%%%%%%%%%%%%%%%%%%%%%%%%%%%%%%%
\ExplSyntaxOn
\makeatletter

\edef \tableofcontents {
    \noexpand \chapter \noexpand* {
        Contents
        \noexpand\@mkboth { Contents } { Contents }
    }

    \noexpand\@starttoc{toc}
}

\cs_generate_variant:Nn \seq_set_split_keep_spaces:Nnn { Nnx }

\def \dlabelstyle #1 {
    \def\descriptionlabel ##1 {\hspace\labelsep \normalfont #1 ##1}
}

\cs_new:Npn \__two_cols:nnnnn #1 #2 #3 #4 #5 {
    \ifvmode\else\unskip\par\fi
    \noindent\leavevmode
    \hbox to \hsize {
        \hbox to #3 { \vtop {#1} }
        \hskip   #4
        \hbox to #5 { \vtop {#2} }
    } \par
}

\NewDocumentCommand \TwoCols {
    D[]{.475\hsize}
    D[]{.475\hsize}
    D[]{0pt plus 1fill}
    +m
    +m
} {
    \__two_cols:nnnnn{#4}{#5}{#1}{#3}{#2}
}

\cs_new:Npn \__gloss_insert_table_header: {
    %% Generating columns doesn’t seem to work, so this hack will do. If you
    %% need a gloss with more columns than this, I suggest you pause and take
    %% a moment to reflect on your life choices.
    \begin {tabular} { @{} *{100}l }
}

\cs_new:Npn \__gloss_table_start: {
    \ifvmode\else\unskip\par\fi
    \addvspace { 8pt }
    %\vtop{\iffalse}\fi
    \noindent
}

\cs_new:Npn \__gloss_table_end: {
    \end {tabular}
    \ifvmode\else\par\fi
    \addvspace { 8pt }
    %\iffalse{\fi}
    \everypar { \setbox\z@\lastbox \everypar{} }
}

%% Rescan a token list.
\cs_new:Npn \__gloss_rescan:n #1 {
    { \tl_rescan:nn {} { #1 } }
}

%% Format a single line of a gloss.
%%
%% The line number must be stored in \g_tempb_int.
%%
%% This takes the line #1 and formats it, inserting the contents of #2
%% at the start of each line, and #3 at the beginning of each column.
%% #3 can be a macro that takes one argument, in which case it will be
%% passed the contents of the column.
\cs_new:Npn \__gloss_format_line:nnn #1 #2 #3 {
    #2

    \seq_set_split:Nnn \l_tmpa_seq { | } { #1 }
    \bool_set_true:N \l_tmpa_bool

    \seq_map_inline:Nn \l_tmpa_seq {
        \bool_if:NF \l_tmpa_bool { & }
        \bool_set_false:N \l_tmpa_bool
        #3 { \__gloss_rescan:n {##1} }
    }
}

%% Format a gloss.
%%
%% This iterates over all lines in #1 and calls #2 on it.
\cs_new:Npn \__gloss_format:nn #1 #2 {
    \int_gset:Nn \g_tmpb_int { 1 }
    \tl_map_inline:nn { #1 } {
        \tl_if_blank:nF { ##1 } {
            %% Insert \cr or table header if this is the first line. We
            %% need to emit this here, as otherwise, TeX will encounter
            %% unexpandable tokens before the formatter (#2) is executed,
            %% which will start the cell before any actual content has
            %% been inserted; this means we can no longer tell TeX to
            %% \omit the cell header, which causes \multicolumn to break
            %% horribly.
            %%
            %% By emitting the table header here, we ensure that instead,
            %% the formatter is the first thing TeX gets to see after the
            %% initial table header and after every \cr.
            \int_compare:nNnTF { \g_tmpb_int } > { 1 } { \\ } { \__gloss_insert_table_header: }
            #2 { ##1 }
            \int_gincr:N \g_tmpb_int
        }
    }
}

%% Two lines means object language + gloss.
\cs_new:Npn \__gloss_two_lines:n #1 {
    \cs_gset:Npn \__formatter:n ##1 {
        \__gloss_format_line:nnn { ##1 } {} {
            \int_compare:nNnT { \g_tmpb_int } = { 1 } { \itshape }
        }
    }

    \__gloss_format:nn { #1 } { \__formatter:n }
}

%% Full gloss (text, object language, pronunciation, gloss, and translation).
\cs_new:Npn \__gloss_five_lines:n #1 {
    \cs_gset:Npn \__formatter_i:n ##1 { \multicolumn {100} {@{}l} {\itshape\bfseries\__gloss_rescan:n {##1}} }
    \cs_gset:Npn \__formatter_ii:n ##1 { \__gloss_format_line:nnn {##1} {} {\itshape} }
    \cs_gset:Npn \__formatter_iii:n ##1 { \__gloss_format_line:nnn {##1} {} {} }
    \cs_gset:Npn \__formatter_iv:n ##1 { \__gloss_format_line:nnn {##1} {} {} }
    \cs_gset:Npn \__formatter_v:n ##1 { \multicolumn {100} {@{}l} {\__gloss_rescan:n {##1}} }

    \__gloss_format:nn { #1 } {
        \cs:w __formatter_ \int_to_roman:n \g_tmpb_int :n \cs_end:
    }
}

%% Full gloss w/o IPA (text, object language, gloss, and translation).
\cs_new:Npn \__gloss_four_lines:n #1 {
    \cs_gset:Npn \__formatter_i:n ##1 { \multicolumn {100} {@{}l} {\itshape\bfseries \__gloss_rescan:n {##1} } }
    \cs_gset:Npn \__formatter_ii:n ##1 { \__gloss_format_line:nnn {##1} {} {\itshape} }
    \cs_gset:Npn \__formatter_iii:n ##1 { \__gloss_format_line:nnn {##1} {} {} }
    \cs_gset:Npn \__formatter_iv:n ##1 { \multicolumn {100} {@{}l} {\__gloss_rescan:n {##1}} }

    \__gloss_format:nn { #1 } {
        \cs:w __formatter_ \int_to_roman:n \g_tmpb_int :n \cs_end:
    }
}

\NewDocumentCommand \gloss {
    > { \exp_args:Nx \SplitList { \iow_char:N \^^M } } +v
} {
    \__gloss_table_start:

    %% Count lines.
    \int_gset:Nn \g_tmpa_int { 0 }
    \tl_map_inline:nn { #1 } {
        \tl_if_blank:nF { ##1 } {
            \int_gincr:N \g_tmpa_int
        }
    }

    %% Dispatch the appropriate number of lines.
    \int_case:nnF { \g_tmpa_int } {
        2 { \__gloss_two_lines:n { #1 } }
        4 { \__gloss_four_lines:n { #1 } }
        5 { \__gloss_five_lines:n { #1 } }
    }

    %% Any other line count is an error.
    {
        \msg_new:nnn { gloss } { too-many-lines } {
            Too~many~lines~in~gloss:~expected~2,~got~\int_use:N \g_tmpa_int
        }

        \msg_error:nn { gloss } { too-many-lines }
    }

    %% Close the table.
    \__gloss_table_end:
}

%% Process two words. Used by \multigloss.
\cs_new:Npn \__uf_multigloss_word:nn #1 #2 {
    \allowbreak

    \hbox {
        \begin{tabular}{@{}l}
            \itshape \tl_rescan:nn {} {#1} \\
            \noalign{\vskip-6pt}
            \tl_rescan:nn {} {#2}          \\
        \end{tabular}
    }

    \space
}

%% Process two lines. Used by \multigloss.
\cs_new:Npn \__uf_multigloss:NN #1 #2 {
    \ifvmode\noindent\leavevmode\fi

    %% Split the lines into words.
    \seq_set_split_keep_spaces:Nnx \l_tmpa_seq { | } { #1 }
    \seq_set_split_keep_spaces:Nnx \l_tmpb_seq { | } { #2 }

    %% Iterate over each word in the two lines.
    \seq_mapthread_function:NNN \l_tmpa_seq \l_tmpb_seq \__uf_multigloss_word:nn
}

%% Typeset two-line glosses across multiple lines.
\NewDocumentCommand \multigloss {
    > { \exp_args:Nx \SplitList { \iow_char:N \^^M } } +v
} {
    \ifvmode\else\unskip\par\fi
    \begingroup
    \linespread { 1.5 } \selectfont
    \raggedright
    \begin{sloppypar}

    %% Iterate over each line, two lines at a time.
    \bool_gset_true:N \g_tmpa_bool
    \tl_map_inline:nn { #1 } {
        %% Ignore empty lines entirely.
        \tl_if_blank:nF { ##1 } {
            \bool_if:NTF \g_tmpa_bool {
                \tl_gset:Nn \g_tmpa_tl { ##1 }
            } {
                \tl_gset:Nn \g_tmpb_tl { ##1 }
                \__uf_multigloss:NN \g_tmpa_tl \g_tmpb_tl
            }

            %% Flip.
            \bool_gset_inverse:N \g_tmpa_bool
        }
    }

    \end{sloppypar}
    \endgroup
}

\renewenvironment {verse} {
    \ifvmode\else\unskip\par\fi
    \begingroup
        \obeylines
        \obeyspaces
        \parindent0pt
        \parskip0pt
} {
        \par
    \endgroup
}

\def \footnoterule {
    \kern -3\p@
    \hrule \@width .4\columnwidth
    \kern 2.6\p@
}

\def \@makefntext #1 {
    \setlength \parindent { 1em }
    \noindent {
        \mbox {
            \llap { {}\textsuperscript{\@thefnmark} \kern.5pt }
        } { #1 }
    }
}

\newlength{\EnumItemSep} \EnumItemSep-3pt

\newenvironment { enum } [1] [0] {
    \vspace { -.5em }
    \settowidth \leftmargini { 99.\hskip\labelsep }
    \begin { enumerate }
    \setcounter { enumi } { #1 }
    \itemsep \EnumItemSep
} {
    \end { enumerate }
    \vspace { -.5em }
}

\newenvironment { dlist } [1] [{}] {
    \vspace { -.5em }
    \begingroup
    \def\descriptionlabel ##1 {\hspace\labelsep \normalfont #1 ##1}
    \settowidth \leftmargini { 99.\hskip\labelsep }
    \begin { description }
    \itemsep \EnumItemSep
} {
    \end { description }
    \endgroup
    \vspace { -.5em }
}

\def\pfabbr{{\normalfont\scshape pf\space}}
\def\pf#1{\pfabbr\textit{#1}}

\cs_new:Npn \items {
    \ifvmode\else\unskip\par\fi
    \addvspace\medskipamount
    \begingroup
    \itemize\kern-\topsep
    \itemsep0pt
}

\cs_new:Npn \enditems {
    \enditemize\kern-\topsep
    \endgroup
    \addvspace\medskipamount
}

\cs_new:Npn \__uf_no_indent: {
    \global \everypar { \setbox\z@\lastbox \everypar{} }
}

\cs_new:Npn \examples {
    \ifvmode\else\unskip\par\fi
    \begingroup
    \def \labelitemi { \(\diamond\) }
    \items
}

\cs_new:Npn \endexamples {
    \enditems
    \endgroup
    \__uf_no_indent:
}

%% %%%%%%%%%%%%%%%%%%%%%%%%%%%%%%%%%%%%%%%%%%%%%%%%%%%%%%%%%%%%%%%%%%%%%%%%%%%%%
%%  Settings and Utility
%% %%%%%%%%%%%%%%%%%%%%%%%%%%%%%%%%%%%%%%%%%%%%%%%%%%%%%%%%%%%%%%%%%%%%%%%%%%%%%
\def \UF { \bfseries \itshape }
\let \nf \normalfont
\def \textnf #1 {{\nf #1}}

\def \d {ḍ}
\def \D {Ḍ}
\def \b {ḅ}
\def \B {Ḅ}
\def \L {\textsuperscript{L}}
\def \N {\textsuperscript{N}}
\long \def \s #1 {{\normalfont\scshape #1 }}

\let \w \textit
\let \b \textbf
\let \Sl \textbackslash
\let \Sub \textsubscript
\let \Sup \textsuperscript
\def \parheading #1 { \noindent \textbf{#1} }
\let\MC\multicolumn

\newbox\BoxA
\newbox\BoxB

\def \Item #1 {
    \item [ \textsc{\textbf{#1}} ]
}

\def \Paragraph #1 {
    \ifvmode\else\unskip\par\fi
    \addvspace \bigskipamount
    \noindent \leavevmode \ignorespaces \textbf{#1} \par
    \everypar { \setbox 0 \lastbox \everypar {} }
    \nobreak
}

\frenchspacing
\raggedbottom

\AtBeginDocument {
    \def \today {
        \int_value:w \day \space
        \int_case:nn { \month } {
            { 1 }  { January }
            { 2 }  { February }
            { 3 }  { March }
            { 4 }  { April }
            { 5 }  { May }
            { 6 }  { June }
            { 7 }  { July }
            { 8 }  { August }
            { 9 }  { September }
            { 10 } { October }
            { 11 } { November }
            { 12 } { December }
        } \space
        \int_value:w \year
    }
}

\def \ps@ultrafrench {
    \let \@oddfoot \@empty
    \let \@evenfoot \@empty
    \let \@mkboth \markboth

    \def \@evenhead { \thepage   \hfil \leftmark }
    \def \@oddhead  { \rightmark \hfil \thepage  }

    \def \chaptermark ##1 {
        \markboth {
            \thechapter \quad ##1
        } { }
    }

    \def \sectionmark ##1 {
        \markright {
            \thesection \quad ##1
        }
    }
}

%% Override this to set the pagestyle to 'empty'.
\def \chapter {
    \cleardoublepage
    \thispagestyle { empty }
    \global \@topnum \z@
    \@afterindentfalse
    \secdef \@chapter \@schapter
}

\pagestyle{ultrafrench}

\makeatother
\ExplSyntaxOff

%% %%%%%%%%%%%%%%%%%%%%%%%%%%%%%%%%%%%%%%%%%%%%%%%%%%%%%%%%%%%%%%%%%%%%%%%%%%%%%
%%  Document
%% %%%%%%%%%%%%%%%%%%%%%%%%%%%%%%%%%%%%%%%%%%%%%%%%%%%%%%%%%%%%%%%%%%%%%%%%%%%%%
\title{A Comprehensive Diachronic Grammar of Modern ULTRAFRENCH}
\author{Ætérnal, Annwan, Agma Schwa}
\date{\today}

\begin{document}
\makeatletter
\thispagestyle{empty}
\begin{titlepage}
    \null\vfil
    \vskip 60\p@
    \centering
    \large
    {\LARGE \@title \par}%
    \vskip1em
    {\LARGE \w{Ŷrávér Réy’ác’hraúníc’hâ Rzaúsdâ Át’hebhaú Raúl} \par}%
    \vskip 1em
    {\Large \@author \par}%
    \vskip .5em
    \today \par
    \vfil\null
\end{titlepage}
\makeatother


\setcounter{page}{1}
\thispagestyle{empty}
\tableofcontents

\chapter{Phonology and Evolution from Modern Pseudo-French}\label{sec:phonology}
\thispagestyle{empty}
{\def\arraystretch{1.25}\setlength{\tabcolsep}{.4em}
\noindent\begin{tabular}{@{}l|l|l|l|l|ll@{\quad}l|l|l}
                       & Labial & Coronal   & Palatal  & Velar & Glottal &&            & Front        & Back        \\ \cline{1-6} \cline{8-10}
    Stop               & b, bʱ  & d         &          &       &         && Close      & i ĩ ĩ̃ i̥ y ẙ  & u ũ ũ̃ u̥ \\ \cline{1-6} \cline{8-10}
    Nasal              &        & n         &          &       &         && Near-close & ʏ̃ ʏ̃̃      &             \\ \cline{1-6} \cline{8-10}
    Fricative          & ɸ β    & s z, θ ð  & ɕ ʑ, (ç) & x     & h       && Close-mid  & e ẽ ẽ̃ e̥      & o o̥         \\ \cline{1-6} \cline{8-10}
    Fric. (ʁ-coloured) & βʶ     & sʶ zʶ, ɮ̃ʶ & ɕʶ ʑʶ    &       &         && Mid        & \multicolumn{2}{c}{ə ə̥} \\\cline{1-6} \cline{8-10}
    Trill              &        &           &          & ʀ     &         && Open-mid   & ɛ ɛ̃ ɛ̃̃      & ɔ̃ ɔ̃̃         \\ \cline{1-6} \cline{8-10}
    Approximant        & ʋ̃      &           & ɥ ɥ̃, j̊   & ɰ ɰ̃   &         && Near-open  & \multicolumn{2}{c}{ɐ ɐ̥} \\ \cline{1-6} \cline{8-10}
    Lateral Fricative  &        & ɮ̃         & ʎ̝̃        &       &         && Open       &              & ɑ̃ ɑ̃̃         \\
\end{tabular}}\bigskip

\parheading{Legend}\par\noindent
Ṽ = nasalised vowel, Ṽ̃ = nasal vowel, V = any vowel (or, in conjunction with Ṽ/Ṽ̃, oral vowel)\\
N = nasal consonant, C̃ = nasalised consonant (e.g. /ɰ̃/, but not true nasals), C = any consonant.\medskip
\def\scalpha{\kern-2pt\raisebox{2pt}{\Sub α}}

%% NOTE: In case the changes below and the ones listed
%% in the Lexurgy file differ, the latter are authoritative,
%% as I may forget to update these here sometimes.

\TwoCols[.45\hsize][.45\hsize][0pt]{
\parheading{Preliminary Changes}
\begin{enum}
    \item g, w > ɰ ⟨r⟩
    \item œ, œ̃, ø > y, ʏ̃, ʏ̃
    \item ɔ > o
    \item u > v / \_o
    \item y > j / \_(\#)V
    \item V\scalpha > $\emptyset$ / \_\#V\scalpha
    \item lj, lɥ > ʎ
    \item j > ɥ ⟨y’⟩
    \item ɰ > ɥ / \_i
    \item ʁʁ > ʀ
    \item sʁ, ʃʁ, zʁ, ʒʁ > sʶ, ʃʶ, zʶ, ʒʶ
    \item vʁ > vʶ
    \item ʁ > ɰ
    \item C > $\emptyset$ / \#\_C
    \item C > $\emptyset$ / C\_\#
    \item k > x ⟨c’h⟩
    \item ʃ, ʃʶ, ʒ, ʒʶ > ɕ, ɕʶ, ʑ, ʑʶ
    \item nt > nθ
    \item t > ḍ [d] (‘hard /d/’)
    \item p > \b{} [b] (‘hard /b/’)
    \item f, v, vʶ > ɸ ⟨f⟩, β ⟨b’h⟩, βʶ ⟨v́⟩
\end{enum}
\parheading{Simplification}
\begin{enum}[21]
    \item d, ḍ, b, ḅ > $\emptyset$ / \_s
\end{enum}
}{
\parheading{Great Nasal Shift}
\begin{enum}[22]
    \item Ṽl > ɰ̃ ⟨w⟩
    \item V > Ṽ̃ / [NC̃ɥɰ]\_N\#
    \item V, Ṽ > Ṽ, Ṽ̃ / \_[NC̃ɥɰ], [NC̃ɥɰ]\_
    \item ə̃, ə̃̃, ã, ã̃, õ, õ̃ > ɛ̃, ɛ̃̃, ɑ̃, ɑ̃̃, ɔ̃, ɔ̃̃
    \item N, C̃ > $\emptyset$ / V\_\#
    \item ɲ, ŋ > n
    \item V, Ṽ > $\emptyset$ / N \_ N
    \item m, l, ʎ > ʋ̃ ⟨v⟩, ɮ̃ ⟨l⟩, ʎ̝̃ ⟨ḷ⟩
    \item ɮ̃ɰ, ɰɮ̃ > ɮ̃ʶ ⟨ł⟩
\end{enum}

\parheading{Intervocalic Lenition (/ V\_V is implied)}
\begin{enum}[31]
    \item x, s, z > h
    \item ɕ, ɮ̃, ʎ̝̃ > j̊ ⟨c̣⟩, ɥ̃, ɰ̃
    \item nθ > n
    \item d, ḍ, b, \b{} > ð ⟨d’h⟩, θ ⟨t’h⟩, β, bʱ ⟨bh⟩
    \item ɸ > β / V\_V
\end{enum}

\parheading{Late Changes}
\begin{enum}[36]
    \item C[+stop, -alveolar]C\scalpha > C\scalpha
    \item C[+stop]C\scalpha[+stop] > C\scalpha
    \item h > $\emptyset$ / hV\_
    \item ə > $\emptyset$ / C\_C
    \item V[-nasalised, -nasal] > ə̥ / \_\#
    \item ɰɰ > ʀ
    \item eɛ̃ > ẽ
\end{enum}
}\medskip

\section{Pronunciation, Allophony, and Stress}\label{subsec:pronunciation-allophony-and-stress}
There is not a lot of allophony in UF, save that /x/ is realised as [χ] around back vowels and [ɕ] elsewhere, e.g.
\w{c’húr} /xũɰ/ ‘to shrink’ is pronounced [χũˑˠ]. Furthermore, /h/ is [ç] before variants of /i/ and /y/, and [h] elsewhere.

The vast majority PF words are stressed on the last syllable of the root, e.g. \w{ad’hór} ‘to love’ /aˈðɔ̃ɰ/, but \w{b’had’hóré}
‘you (\s{pl}) love’ /βaˈðɔ̃.ɰɛ̃/. The stress is not indicated in writing, neither in actual texts, nor in this
grammar or in dictionaries. The main exception to this are names, which are generally stressed on the first syllable,
and receive secondary stress on the last syllable,\footnote{That is, unless the name ends in an obvious suffix, in which case the last
syllable before any such suffixes receives secondary stress; however, this is generally quite rare.} e.g. \w{Daúvníc’h} /ˈdɔ̃ʋ̃ˌnĩx/.

The only exception to this rule are certain particles and irregular verbs, some of which have irregular stress; for instance,
the forms of \w{eḍ} ‘to be’ are all stressed on the first syllable. Any such words that deviate from the norm will be pointed
out in this grammar and in dictionaries.

Oral vowels before the stressed syllable are often somewhat muted or reduced, albeit still audible, and stressed vowels are lengthened if they
are nasalised, e.g. the pronunciation of \w{ad’hór}, which we just transcribed as /aˈðɔ̃ɰ/, is actually closer to [ɐ̯ˈðɔ̃ˑɰ].
Word-final voiceless \w{ẹ} is always /ə̥/. Finally, non-back vowels that are followed by /ɰ/ or /ɰ̃/ are retracted, e.g. \w{y’ẹ́rẹ́}, the future
stem of \w{y’ẹ́} ‘forbid’, is phonemically /ɥẽˈɰẽ/, but pronounced [ɥɘ̃ˈɰẽ].

Oral vowels have a nasalised and nasal counterpart. /i/ and /u/ do not vary in quality when na\-sa\-lis\-ed.
/a/ is normally [ɐ], but becomes [ɑ] when nasalised or nasal. Similarly, /e/ becomes [ɛ],
/y/ becomes [ʏ], and /o/ becomes [ɔ]. Note that nasalised [ẽ] exists, but it’s
rare. The quality never changes when going from nasalised to nasal. The schwa has no nasal(lised) counterpart. Lastly, oral vowel
also have voiceless counterparts, whose quality is the same as that of the base vowel.

The difference between nasalised vowels and nasal vowels is that the former are merely coarticulated with nasalisation, whereas
the latter are completely and utterly \textit{in the nose}—no air escapes through the mouth when a nasal vowel is articulated, and all
the air flows just through the nose. Middle UF and some modern dialects also distinguish between sinistral and dextral nasal
vowels,\footnote{Sinistral nasal vowels are articulated with the left nostril, and dextral nasal vowels with the right nostril.}
but this distinction is no longer present in the modern standard language.

Initial /ɰ/ is sometimes elided after words that end with /ɰ/.

\section{Orthography}
The spelling of most UF sounds is indicated above; the less exotic consonants are spelt as
one might expect. In addition, UF employs a variety of diacritics—though some only in grammatical
material—to differentiate its many sounds with an otherwise unsatisfactory array of symbols.

\subsection{Consonants}
As one might expect, /b, d, n, ɸ, s, z, h/ are spelt ⟨b, d, n, f, s, z, h⟩, respectively.

Several fricatives are spelt with an apostrophe followed by a ⟨h⟩, viz. /x/ ⟨c’h⟩, /θ/ ⟨t’h⟩, /ð/ ⟨d’h⟩,
and /β/ ⟨b’h⟩. Apostrophes are also often used to mark shortened forms, e.g. \w{t’hé},
the optative negation particle, is shortened to \w{t’h’} before vowels.

Conventional letters are used for rather unconventional sounds, mostly for diachronic reasons:
/l/ does not exist in UF, so ⟨l⟩ is either /ɮ̃/ or /ʎ̝̃/, ⟨v⟩ is /ʋ̃/, ⟨j⟩ is /ʑ/, ⟨c⟩ is /ɕ/, ⟨r⟩ is /ɰ/, ⟨w⟩ is /ɰ̃/. The vowel
/y/ is spelt ⟨y⟩, and its consonantal equivalent /ɥ/ as well as nasalised /ɥ̃/ are spelt with an apostrophe, that is
⟨y’⟩ and ⟨ý’⟩. The ʁ-fricated fricatives /βʶ, ɮ̃ʶ, sʶ, ɕʶ, ʑʶ, zʶ/
are spelt ⟨v́, ł, ś, ć, ȷ́, ź⟩, respectively.

Double consonant letters indicate a lengthened consonant; these are rare, but they can occur in any position. The only
exception to this is ⟨rr⟩, which is not /ɰː/, but rather /ʀ/. UF does not have phonemic vowel length (though recall
that phonetic lengthening occurs in some situations), so a double vowel letter is always pronounced as two separate vowels.

\subsection{Hard \textit{ḅ} and \textit{ḍ}}
The ‘hard’ voiced \w{ḅ}, \w{ḍ} which are pronounced exactly like their regular counterparts, are normally also spelt ⟨b⟩ and
⟨d⟩. However, a dot below is commonly used in dictionaries and grammatical material to distinguish between the two
as they differ from one another in how they mutate.

In Early Modern UF (and Middle UF before it), such as in the writings of renowned poet and writer \s{Jac’h Yý’is Bèrtrá (J.\,Y.\,B.)
Snet’h}, \w{ḅ} and \w{ḍ} sometimes retain their diachronic spellings of ⟨p(h)⟩ and ⟨t⟩—and \w{bh} is sometimes spelt
⟨p’h⟩ instead—though this is not consistent and often not applied word-internally or between vowels in general—even across multiple
words—where these sounds were already voiced even at the time.

For instance, Snet’h commonly writes e.g. \w{naut} ‘our’ for \w{nauḍ}, but e.g. \w{labraúc} ‘they came up to’ for \w{laḅraúc}, and not
\w{lapraúc} or \w{laphraúc}, is found in the very same passage. This style is often imitated by writers who want to seem archaic, but failing to
understand the pronunciation of the time, they tend to use ⟨t⟩ and ⟨p(h)⟩ everywhere, even word-internally.

\subsection{\textit{rrr}}
The sequence ⟨rrr⟩ could be /ʀɰ/ or /ɰʀ/. In grammatical material, this is disambiguated by writing either \w{rr-r}, e.g.
\w{férr-rásvát’h} ‘a long, deep sleep’, or \w{r-rr}, respectively, but in actual text, both are written \w{rrr}.

\subsection{Vowels}
The vowels are mostly spelt as one might expect; nasalised vowels are indicated by an acute, and nasal vowels by a circumflex.
The variants of /i, y, u, a, e/ are spelt with ⟨i, y, u, a, e⟩ as their base letters. Nasal /ẽ/ and /ẽ̃/ as well as Schwa are
indicated by adding a dot below the ⟨e⟩ in grammars and dictionaries only.

Oral /ɛ/ is rare and is spelt ⟨è⟩. Word-initially and word-finally, a grave indicates that the vowel is voiceless. Word-final
voiceless /ə/ is always voiceless.\footnote{Thus, a word-final ⟨e⟩
can be /e/, such as in \w{vvaúríhe} /ʋ̃ːɔ̃ɰĩˈhe/ ‘to remember’, or /ə̥/, such as in \w{dale} /daɮ̃ə̥/ ‘table’. As a rule of thumb, it is
usually /e/ at the end of verb stems—but not verb forms in general—and /ə̥/ elsewhere. Fortunately they are differentiated by a
dot below in dictionaries and in this grammar: \w{vvaúríhe} vs \w{ḍalẹ}.}

\subsection{/o/}

The vowel /o/ is spelt ⟨au⟩ or ⟨o⟩ for diachronic reasons; when ⟨au⟩ is accented, the acute or circumflex is added only to the
⟨u⟩.\footnote{The diphthong /au/ is spelt ⟨äu⟩, ⟨aü⟩, or with accents on both vowels.} Generally speaking, there is no consistent
rule as to which one is used in what circumstances, though ⟨au⟩ usually preferred (especially word-initially)—even if the
PF root was spelt with ⟨o⟩—except word-finally and after ⟨w⟩. As an exception to the exception, in verb affixes, \w{au} is quite
common word-finally. This notwithstanding, the sequence ⟨wau⟩ does not exist in UF.

The distinction between ⟨au⟩ and ⟨o⟩ sometimes used contrastively: e.g. \w{faúr} may mean ‘force’ or ‘form’; thus, when the
intended meaning is not obvious from context, the latter is usually spelt \w{fór} instead, e.g. \w{av́ár sb’haúr} ‘to have force’
as opposed to \w{av́ár sb’hór} ‘to have form’. However, \w{aḍrá faúr} ‘to take shape’, an idiom, is never spelt *\w{aḍrá fór}.

Lastly, note that ⟨áu⟩, which is rare, but occurs e.g. in the superlative case, is pronounced /ɑ̃u/.

\subsection{Dot Below}
A dot below or above a letter is commonly to indicate
a variety of different things, depending on the letter:
\begin{items}\itemsep .5ex plus .1ex minus .1ex\relax
\item a dot below in \w{ḅ}, \w{ḍ} indicates that they are the ‘hard’ variants of the letter, which are pronounced
      the same, but lenited differently;
\item a dot below in \w{ḷ} indicates that it is palatal /ʎ̝̃/ instead of alveolar /ɮ̃/;
\item a dot below in \w{ẹ} indicates that it is a schwa;
\item a dot below in \w{ẹ̀} indicates that it is /e̥/;
\item a dot below nasalised \w{ẹ́}, \w{ệ} indicates that they are /ẽ/, /ẽ̃/ instead of /ɛ̃/, /ɛ̃̃/;
\item a dot below in \w{c̣} indicates that it is lenited /j̊/.
\end{items}

\noindent Thus, in non-grammatical writing, the following are indistinguishable:
\begin{items}\itemsep .5ex plus .1ex minus .1ex\relax
\item \w{l} can be palatal /ʎ̝̃/ or alveolar /ɮ̃/;
\item \w{e} can be a schwa, or /e/;
\item \w{é}, \w{ê} can be /ɛ̃/, /ɛ̃̃/ or /ẽ/, /ẽ̃/;
\item \w{c} can be /ɕ/ or /j̊/.
\end{items}

\subsection{Other Punctuation Marks}
\noindent Elided initial /ɰ/ is indicated by omitting the \w{r} in writing and attaching the word to the previous one with a hyphen,
e.g. \w{-vá} ‘alas’.

UF seldom uses hyphens to separate or join words and instead prefers to spell them as one word instead; an exception
to this is that, starting in the late Early Modern UF period, affixes that end with a vowel are typically separated
from the word they are attached to with a hyphen if that word starts with (a variant of) the same vowel. For example,
the \s{def nom sg} of \w{el} ‘wing’ is \w{láel}, but the plural is \w{lé-el}, with \w{léel} only found in archaic
writing.

\section{Lenition and Nasalisation}
Certain morphological elements subject surrounding context to lenition or nasalisation. Nasalisation affects vowels,
which become more nasal (that is, (voiceless) oral vowels become nasalised, and nasalised vowels become nasal; nasal
vowels are unaffected), as well as \w{ḍ}, which becomes \w{n}, unless it would directly be preceded by another \w{n}, in
which case it is simply deleted.

Lenition is more complicated; it affects only consonants and causes a softening similar to what happened diachronically
between vowels. All ʁ-fricated consonants simply lose their ʁ-frication, and a number of other consonants are also
affected by lenition (note the difference between \w{ḅ, ḍ} and \w{b, d} here):

\begin{table}[H]
\centering
\itshape
\begin{tabular}{r|lll|l|l|ll|l|l|l|l|l|l|l|l|l}
\bf Consonant & c’h & s & z & sw     & c & b & f                        & ḅ   & d   & ḍ   & v́ & ł & ś & ć & ȷ́ & ź  \\\hline
\bf Lenited & \multicolumn{3}{c|}{h} & ź & c̣ & \multicolumn{2}{c|}{b’h} & bh  & d’h & t’h & v & l & s & c & j & z \\
\end{tabular}
\nf
\caption{Consonants Affected by Lenition}\label{tab:lenition}
\end{table}

\noindent Note that double consonants are typically unaffected by morphological lenition, e.g. \w{dír} ‘to say’,
whose subjunctive stem is \w{díss}, forms \w{aúdíssâ} (rougly ‘we should have said’), not *\w{aúdíhhâ}.

\subsection{Glossing}
To simplify glosses, cases are assumed to be definite and singular unless otherwise stated, and verb forms are
assumed to be indicative, present tense, and active, unless otherwise stated.

\section{Diachrony and Derivation}\label{subsec:diachrony-and-derivation}
The PF infinitive endings (\w{-ir} etc.) became progressively more vestigial in Middle UF and were eventually
often dropped completely in derivation, e.g. \w{auḍé} ‘obtain’ from earlier *\w{auḅḍénír}, later resulting in
a need for new infinitive affixes to be formed to distinguish infinitives from the base form, thus giving
rise to e.g. \w{dauḍé} ‘to obtain’.

The suffix \w{-t’he}, \s{fut} \w{-ḍe}, \s{subj} \w{-t’hes}, is a productive derivational suffix that can be used to turn a
noun ‘X’ into a verb that roughly means ‘to use X’, e.g. \w{ac} ‘axe’ $\to$ \w{act’he} ‘to cut with an axe’.

The prefix \w{raú(b’hc’h)-} can be prepended to the stem of a verb to turn it into a causative, e.g. \w{raúb’hc’had’hór}
‘cause someone to love’. The \w{-b’hc’h-} is dropped if the word starts with a consonant, in which case it also
lenites, e.g. \w{raúd’hír} ‘make someone say’.

The suffix \w{-aû} can be appended to any verb that expresses an action with a result to denote that result, e.g.
\w{iý’ývî} ‘illuminate’ $\to$ \w{iý’ývîaû} ‘illumination’.

The suffix \w{-(é)raû} can be appended to a noun to denote a profession, specifically someone who regularly engages in
creating or constructing the noun in question, e.g. \w{ḍalẹ} ‘table’ $\to$ \w{ḍaléraû} ‘carpenter’. The \w{é} is retained
if the noun ends with a consonant; it replaces word-final \w{ẹ}, and is dropped itself if the word ends with any other vowel,
to which a level if nasalisation is added if possible.

\chapter{Nouns}\label{sec:accidence}
This chapter covers nouns in the broad sense, i.e. nouns substantive, adjectives, adverbs, numerals, and pronouns.

\section{Declension}\label{subsubsec:declension}
UF has 2 declensions. A definite declension and an indefinite declension.
UF has no morphologically separate articles; rather, the old PF articles have been incorporated
into the declensions. Furthermore, UF no longer has a gender distinction in nouns.

The table below shows the most common cases of the definite and indefinite declensions. UF pretty has no
adpositions and instead uses cases instead. Cases are divided into primary, secondary, and locative cases.

Many of the case prefixes cause lenition in the initial consonant of the noun, e.g. \w{ḍalẹ} ‘table’ to
\s{def acc sg} \w{s’thalẹ}; this lenition is blocked in the \s{indef acc pl} due to the presence of a hypercorrected ‘s’
in \pf{ces}, e.g. \w{s’ḍalẹ} ‘the tables (\s{acc})’ (not \w{s’thalẹ}, which is the singular), as well as in
less commonly used forms such as the \s{def iness} \w{dwáḍalẹ} ‘on the table’.

%% TODO: longtable or sth.
\noindent\begin{longtable}{l|>{\it}l|>{\it}lll|>{\it}l|>{\it}l}
Definite    &\nf Sg&\nf Pl && Indefinite       &\nf Sg&\nf Pl\\\cline{1-3}\cline{5-7}
\endhead
Absolutive    & $\emptyset$            & l-       && Absolutive    & $\emptyset$-\N       & $\emptyset$-\L  \\
Nominative    & lá-\L          & lé-\L    && Nominative    & ŷn-\N        & ý-\L    \\
Vocative      & $\emptyset$-\L         & $\emptyset$-\N   && Vocative      & /            & /       \\
Partitive     & dy-\L          & dẹ-\L    && Partitive     & dŷn-\N       & dý-\L   \\
Accusative    & i-\L           & sý-\L    && Accusative    & s-\L         & s-      \\
Genitive      & á-\L           & abh-\L   && Genitive      & sý-\N        & sý-\L   \\
Dative        & as-\L          & a-\L     && Dative        & an-\N        & an-\L   \\
Instructive   & ráh-\L         & ráh-     && Instructive   & rút’hýn-\N   & rút’hýz- \\\cline{1-3}\cline{5-7}

Essive        & ḅáł-           & ḅá-      && Essive        & ḅárýn-       & ḅárý-    \\
Abessive      & sá-\L          & sá-\N    && Abessive      & sáhýn-       & sáhý-    \\
Translative   & cáj-           & cájvâ-   && Translative   & cájŷn-       & cájvý-   \\ % change(ments)
Considerative & słá-           & słé-     && Considerative & sý’óýn-      & sý’óý-   \\
Causal        & ah-\N          & áh-\N    && Causal        & ahýn-        & áhý-     \\ % ‘à cause de’ / ‘ensemble’ both give ah-/áh-
Instrumental  & b’hel-         & b’he-    && Instrumental  & b’he(hý)(n)- & b’heh-   \\
Contrative    & c’haú-\L       & c’haú-   && Contrative    & c’haút’hýn-  & c’haút’hý- \\\cline{1-3}\cline{5-7} % contre

Elative       & órd-           & aúr-     && Elative       & órdŷn-       & aúrŷ-    \\
Inessive      & dwá-           & dwé-     && Inessive      & dáhŷn-       & dáhŷ-    \\
Illative      & ádá-           & ádé-     && Illative      & ádŷn-        & ádŷ-     \\
Ablative      & rê(d)-         & rês-     && Ablative      & rêdýn-       & rêdý-    \\
Allative      & b’hé-\L        & b’hér-   && Allative      & b’hŷn-\N     & b’hý-\L  \\
Postessive    & déry’          & dér-     && Postessive    & déry’ýn-     & déry’ý-  \\
Postlative    & réh-\L         & réh-     && Postlative    & réhýn-\N     & réhyl-   \\ % reculer
Interessive   & aḍá-           & aḍé-     && Interessive   & aḍŷn-        & aḍŷ-     \\
Superessive   & sýr-           & dẹ(h)-\L && Superessive   & dẹhýn-       & sýrŷ-    \\
Superlative   & áu-            & áud’h-   && Superessive   & án-          & ád’hý-   \\ % en haut de (in UF, /ɑ̃o/ > /ɑ̃u/)
Circumessive  & auḍúr-\L(d’h)- & auḍúr-   && Circumlative  & auḍúrýn-     & auḍúrŷ-  \\ % autour de
Circumlative  & ḍúr-\L(d’h)-   & ḍúr-     && Circumlative  & ḍúrýn-       & ḍúrŷ-    \\ % autour de
Antessive     & ab’há-         & ab’h-    && Antessive     & ab’hŷn-      & ab’hŷ-   \\ % avant
Perlative     & lý’aú-\L(d’h)- & lý’aú-   && Perlative     & lý’ýn-       & lý’ý-    \\ % le long de
\noalign{\medskip}
\caption{UF Declension.}\label{tab:table-uf-declension}
\end{longtable}

\noindent The table below shows the paradigm of \w{ḍale} ‘table’ in its definite and indefinite forms; this, of course,
is a rather extreme example, as the initial \w{ḍ} means that it is subject to both lenition and nasalisation.

\noindent\begin{longtable}{l|>{\it}l|>{\it}lll|>{\it}l|>{\it}l}
Definite    &\nf Sg&\nf Pl && Indefinite&\nf Sg&\nf Pl\\\cline{1-3}\cline{5-7}
\endhead
Absolutive    & ḍalẹ         & lḍalẹ      && Absolutive    & ḍalẹ́            & ḍalẹ        \\
Nominative    & lát’halẹ     & lét’halẹ   && Nominative    & ŷnalẹ           & ýt’halẹ     \\
Vocative      & t’halẹ       & nalẹ       && Vocative      & /               & /           \\
Partitive     & dyt’halẹ     & dẹt’halẹ   && Partitive     & dŷnalẹ          & dýt’halẹ    \\
Accusative    & it’halẹ      & sýt’halẹ   && Accusative    & st’halẹ         & sḍalẹ       \\
Genitive      & át’halẹ      & abht’halẹ  && Genitive      & sýnalẹ          & sýt’halẹ    \\
Dative        & ast’halẹ     & at’halẹ    && Dative        & analẹ           & ant’halẹ    \\
Instructive   & ráht’halẹ    & ráhḍalẹ    && Instructive   & rút’hýnalẹ      & rút’hýzḍalẹ \\\cline{1-3}\cline{5-7}

Essive        & ḅáłḍalẹ      & ḅáḍalẹ     && Essive        & ḅárýnḍale       & ḅárýḍale    \\
Abessive      & sát’halẹ     & sánalẹ     && Abessive      & sáhýnḍale       & sáhýḍale    \\
Translative   & cájḍalẹ      & cájvâḍalẹ  && Translative   & cájŷnḍalẹ       & cájvýḍalẹ   \\
Considerative & słáḍalẹ      & słéḍalẹ    && Considerative & sý’óýnḍalẹ      & sý’óýḍalẹ   \\
Causal        & ahnalẹ       & áhnalẹ     && Causal        & ahýnḍalẹ        & áhýḍalẹ     \\
Instrumental  & b’helḍalẹ    & b’heḍalẹ   && Instrumental  & b’hehýḍalẹ      & b’hehḍalẹ   \\
Contrative    & c’haút’halẹ  & c’haúḍalẹ  && Contrative    & c’haút’hýnḍalẹ  & c’haút’hýḍalẹ \\\cline{1-3}\cline{5-7}

Elative       & órdḍalẹ      & aúrḍalẹ    && Elative       & órdŷnḍalẹ       & aúrŷḍalẹ    \\
Inessive      & dwáḍalẹ      & dwéḍalẹ    && Inessive      & dáhŷnḍalẹ       & dáhŷḍalẹ    \\
Illative      & ádáḍalẹ      & ádéḍalẹ    && Illative      & ádŷnḍalẹ        & ádŷḍalẹ     \\
Ablative      & rêḍalẹ       & rêsḍalẹ    && Ablative      & rêdýnḍalẹ       & rêdýḍalẹ    \\
Allative      & b’hét’halẹ   & b’hérḍalẹ  && Allative      & b’hŷnalẹ        & b’hýt’halẹ  \\
Postessive    & déry’ḍalẹ    & dérḍalẹ    && Postessive    & déry’ýnḍalẹ     & déry’ýḍalẹ  \\
Postlative    & réht’halẹ    & réhḍalẹ    && Postlative    & réhýnalẹ        & réhylḍalẹ   \\
Interessive   & aḍáḍalẹ      & aḍéḍalẹ    && Interessive   & aḍŷnḍalẹ        & aḍŷḍalẹ     \\
Superessive   & sýrḍalẹ      & dẹt’halẹ   && Superessive   & dẹhýnḍalẹ       & sýrŷḍalẹ    \\
Superlative   & áuḍalẹ       & áud’hḍalẹ  && Superessive   & ánḍalẹ          & ád’hýḍalẹ   \\
Circumessive  & auḍúrt’halẹ  & auḍúrḍalẹ  && Circumlative  & auḍúrýnḍalẹ     & auḍúrŷḍalẹ  \\
Circumlative  & ḍúrt’halẹ    & ḍúrḍalẹ    && Circumlative  & ḍúrýnḍalẹ       & ḍúrŷḍalẹ    \\
Antessive     & ab’háḍalẹ    & ab’hḍalẹ   && Antessive     & ab’hŷnḍalẹ      & ab’hŷḍalẹ   \\
Perlative     & lý’aút’halẹ  & lý’aúḍalẹ  && Perlative     & lý’ýnḍalẹ       & lý’ýḍalẹ    \\
\noalign{\medskip}
\caption{Paradigm of \w{ḍalẹ}.}\label{tab:vocalic-declension}
\end{longtable}


%% Add ALL THE LOCATIVES. E.g. Circumessive/lative/ablative (circumablative -> e.g. take off a jacket,
%% peel a fruit [from the PEEL’S point of view], e.g. unfurl). Antessive/lative/ablative

\section{Primary Cases}
In UF grammar, the term \w{cyḍ rêhibâ} ‘primary case’ refers to the cases that are commonly used to mark complements
of verbs, i.e. the \s{abs, nom, acc, part, gen, dat}.

\subsection{Absolutive}
The \s{abs} is the base form of the noun, with the \s{abs def} being identical to the uninflected root, which is also
used as the citation form. It is used mainly in contexts where no case marking is otherwise assigned to noun.

\Paragraph{Verbs of Being and Becoming}
The \s{abs} is used for the subject and predicate noun of predicative sentences,
i.e. ‘X’ and ‘Y’ in ‘X is Y’, the subject of a sentence whose predicate is an adjective verb, and the subject of a verb
that carries a sense of being or becoming:
\begin{examples}
    \item \w{Aúsó ḍe \b{ráhó}.} ‘We are all fish.’
    \item \w{\b{Ráhó} sausc’hýr.} ‘The fish is dark.’
    \item \w{\b{Aú} sdẹb’hní cájŷnárb.} ‘The man turned into a tree.’
    \item \w{\b{Aý’èc’hsád} lẹy’abhá.} ‘He called himself Alexandre.’
\end{examples}

Note that this use can lead to ambiguity, e.g. \w{vaût’há se ráhó} could mean ‘a mountain is a fish’ or ‘a fish is a
mountain’ (word order, is irrelevant here). However, if the nouns have different number or the subject is gendered,
this is disambiguated since the verb will agree with the subject, e.g. \w{\b{Cár} le ráhó} ‘Charles is a fish’, as opposed to
\w{Cár se \b{ráhó}} ‘a/the fish is Charles’.

\Paragraph{Modifiers}
Additionally, the \s{abs} turns nouns into modifier nouns. The modifier is generally placed after the noun it modifies,
though this is only mandatory if ambiguity would arise otherwise.
\begin{examples}
    \item \w{abhárḍihyl \b{lývy’ér}} ‘of light particles’
    \item \w{ḍaléraû \b{véḍ} âférér} ‘made by a master carpenter’
    \item \w{lásásc’hríḍ \b{raúl}} ‘the Sanskrit language’
\end{examples}

\subsection{Nominative}
The \s{nom} is one of the most common cases in UF; its main and really only use is to mark the subject of an active sentence
\begin{examples}
    \item \w{Seb’h dwásẹ \b{lá}raúb’hẹ.} ‘The robot was here.’
    \item \w{Lasẹhérélé au \b{lá}b’haúré au \b{lá}haul.} ‘The Sun and the North Wind were quarreling.’
\end{examples}

The \s{indef nom sg} \w{ŷn-} prefix and some other forms nasalise nouns; as a reminder, this means that in
nouns starting with \w{ḍ}, the \w{ḍ} is deleted, e.g. \w{ŷnalẹ} ‘a table’; it causes nasalisation in words
that start with a vowel e.g. \w{ehyó} ‘shield’ to \w{ŷnéhyó} ‘a shield’. As lenition, nasalisation too is
blocked in rarer forms, e.g. \s{indef iness} \w{dáhŷnḍalẹ} ‘on a table’.

\subsection{Vocative}
The vocative is a rare case that is used to address someone or something, e.g. \w{C̣ár!} ‘O Charles!’. The vocative
only occurs in the definite and usually with proper nouns.\footnote{The indefinite \s{voc}
does not exist, as that would be nonsense.} Note that the vocative is not marked by an affix, but
rather by mutation only, e.g. \w{C̣ár} from \w{Cár}.

\subsection{Accusative and Partitive}
These two cases, while often different in meaning, are almost identical in grammatical purpose: Their main use is to
mark the direct object of a verb. While some verbs, e.g. \w{ub’hrá} ‘to be able to’ always take a \s{part}, and others
always take an \s{acc}, the difference between the two, in most contexts, is one of semantics. There is hardly a grammatical
construction that admits the \s{acc} in which the \s{part} would be ungrammatical, and vice versa.

To elaborate, the \s{acc} indicates that an action is being or has been performed in its entirety or to completion. Consequently,
\s{pres ant} forms, which are mainly perfective, generally take the \s{acc}, whereas \s{pret} forms, which are mainly imperfective,
generally take the \s{part}. As most conversations generally concern matters that are relevant to the present and thus still incomplete,
the \s{part} ends up being the commoner of the two cases.
\begin{examples}
    \item \w{Jlí \b{s}liv́uhé.} ‘I peruse a book.’
    \item \w{Jlí \b{dŷn}liv́uhé.} ‘I read from a book.’ or ‘I am reading a book.’
    \item \w{Jlíé \b{i}liv́uhé.} ‘I’ve read the book (to completion).’
    \item \w{Jlíá \b{dy}liv́uhé.} ‘I was reading (from) the book.’
\end{examples}

Both cases are also used to express a ‘passive’, as UF has no traditional morphological passive in that the so-called
‘active’ and ‘passive’ affixes are really ‘agent’ and ‘patient’ affixes. As such, a ‘passive’ form is construed by
omitting the agent affix; the ‘subject’, if there is any, will then be in the \s{acc} or \s{part}.
\begin{examples}
    \item \w{Sylí \w{dý}liv́uhé.} ‘The book is being read.’
    \item \w{Dy-yl syc’hahé.} ‘The window has been broken.’
\end{examples}

Lastly, the \s{acc/part} are also used in a construction known as the \s{aci/pci} (see §~\ref{subsec:aci-pci}).

\subsection{Genitive}
The \s{gen} case is chiefly used to indicate possession or belonging, as well as with certain verbs.

\Paragraph{Possession}
The possessor can precede or follow its possessee. Typically,
the \s{gen} follows the possessee, but if the possessee is qualified with adjectives, then the adjectives must follow the
possessee immediately, lest they end up qualify the genitive instead, and thus, the genitive is placed first.
\begin{examples}
    \item \w{ḅárḍẹ sásy’élâ \b{á}raúl} ‘an essential part of the language’
    \item \w{c’haúnéhás \b{á}rráḍraúc ausc’hýrâ} ‘knowledge of the dark side’
    \item \w{\b{vaú} ḍalẹ} ‘my table’
\end{examples}

\Paragraph{Comparison}
In constructions using the comparative, the \s{gen} marks the standard of comparison (see §~\ref{subsec:comparison}).
\begin{examples}
    \item \w{aûlẹrá \b{á}raúvá} ‘no bigger than the moon’
    \item \w{Sẹh lá isḍrár svaût lybhárdyt’há ihaúb’héc’h \b{á}y’aúý’á dẹrêfihasjú.}\\‘This tale shows that often persuation
          is more efficient than violence.’
\end{examples}

\Paragraph{Objects}
The \s{gen} is also used to mark the object of certain verbs.
\begin{examples}
    \item \w{Lẹc’hlýr \b{sý}rjẹ.} ‘He sells comic books.’
    \item \w{\b{Daú} jady’ŷ.} ‘I bid you farewell.’
    \item \w{Ssívý’érá \b{sýn}árb.} ‘It was similar to a tree.’
    \item \w{vâhâ \b{sý}faúr} ‘lacking strength’
    \item \w{c’haúḅlérâ \b{sý}válfèzâ} ‘to be complacent in the presence of evildoers’
\end{examples}

\subsection{Dative}
The \s{dat} case is used to indicate the indirect object of some verbs. Some verbs, may instead govern a different case,
such as the allative. Unlike most other cases, there actually is
a dative affix (see §~\ref{subsubsec:dative-affixes}) that can be added to a verb in some cases.

\Paragraph{Indirect Objects}
The \s{dat} is used with a large number of ditransitive verbs.
\begin{examples}
    \item \w{Dyvérs jféré \b{as}aú.} ‘I thanked the man.’
    \item \w{Jv́ár\b{b’hẹ} sufb’h vaú.} ‘I owe you my life.’
    \item \w{Jdónẹ́\b{ḷẹ} iárb.} ‘I bestow upon him a tree.’
\end{examples}

\Paragraph{Beneficiary}
Another use of the dative is to indicate a beneficiary or recipient in a more general sense.
\begin{examples}
    \item \w{Jsyfér \b{as}vẹ.} ‘I did it for me.’
    \item \w{Jsydíré \b{as}lẹ.} ‘I said it for his sake.’
    \item \w{\b{As}a c’hes sydír?} ‘For whom is it said?’
\end{examples}

\subsection{Instructive}
The \s{insv} case is used for the ‘subject’\footnote{This can also be paired with a passive participle, in which case the
\s{insv} ends up denoting the patient.} of non-finite verbs—chiefly participles. Its main use is as part of the so-called \textit{instructive
absolute}, in which a noun without any thematic role is combined with a participle—similarly to the \textit{ablative absolute} in
Latin or the \textit{genitive absolute} in Ancient Greek. In some cases (particularly in idioms), the participle can even be
omitted if it is obvious from context.
\begin{examples}
    \item \w{\b{Ráh}t’halẹ âdárér, raḅraúcó b’héárb.} ‘The table thrown, we approach the tree.’
    \item \w{\b{ráh}ráhut’h b’helvê} ‘sword in hand’
\end{examples}

\section{Secondary Cases}
The ‘secondary cases’ (UF \w{lcyḍ dýzy’êâ}) are cases that eschew broader classification; this category comprises anything
that is not a primary or locative case.

\subsection{Essive}
The \s{ess} case is used to mean ‘as X’, ‘like X’, ‘in the form of X’, or ‘in the role of X’.
\begin{examples}
    \item \w{Lẹt’hiy’e dyt’halẹ \b{ḅárýn}y’éjúré.} ‘He uses the table as a chair.’
    \item \w{dáhŷnáẹ \b{ḅárýn}rê} ‘in a copper sky’
    \item \w{\b{ḅárýn}c’hánár âḅét’hýrér} ‘like a painted ship’
\end{examples}

In Middle UF, and sometimes still in poetry, the \s{ess} may be used interchangeably with the \s{iness} to indicate
specifically an ongoing temporal state.
\begin{examples}
    \item \w{\b{ḅáł}víd’hẹ} ‘at noon’
\end{examples}

\subsection{Abessive}
The \s{abess} case denotes the lack or absence of something.
\begin{examples}
    \item \w{\b{sá}-árb} ‘without the tree’
    \item \w{\b{sáhýn}vúb’hvâ} ‘without movement’
\end{examples}

\subsection{Translative}
The \s{trans} case indicates the end or target state of a process or transformation.
\begin{examples}
    \item \w{Aú sdẹb’hní \b{cájŷn}árb.} ‘The man turned into a tree.’
\end{examples}

\subsection{Considerative}
The \s{cons} case is a bit of a weird one and can be translated as ‘according to’, or ‘in the opinion of’, and is used to
express the opinion of the speaker or point out something as an opinion, belief, or hypothesis of someone or something.
\begin{examples}
    \item \w{\b{słé}rá ḍẹ c’hóný áb’hásy’ô} ‘according to all known laws of aviation’
\end{examples}

\subsection{Causal}
The causal case indicates that something is the cause for something else.
\begin{examples}
    \item \w{\b{ah}ârb} ‘because of the tree’
\end{examples}

\subsection{Instrumental}
The \s{instr} case is used to indicate the instrument or means using or by which an action is performed.
\begin{examples}
    \item \w{\b{b’he}ac} ‘with an axe’
    \item \w{\b{b’hen}c’hánár} ‘by boat’
    \item \w{\b{b’he}faúr lẹvú} ‘with much force’
    \item \w{\b{b’hel}c’hýr sérệ} ‘with a heavy heart’
    \item \w{Derâd’há sbhelbec daú \b{b’he}vê.} ‘You grabbed a shovel with your hands.’
\end{examples}

The \w{hý} in the \s{indef instr sg} may be omitted arbitrarily, particularly in literary language; there
isn’t really any rule as to when this happens, but it is most common in words that start with a fricative. However, the \w{n}
is always retained if the \w{hý} is omitted, e.g. we have \w{b’henc’hánár} or sometimes \w{b’hehýc’hánár} ‘by boat’, but never
*\w{b’hehýnc’hánár} or *\w{b’hec’hánár}. The \w{n} in the \s{indef instr sg} is omitted according to the same rules
as the \w{d} in the \s{def abl sg}.

\subsection{Contrative}
The \s{contr} case has the meaning of ‘against’.
\begin{examples}
    \item \w{n \b{c’hau}vẹ} ‘hate against me’
\end{examples}


\section{Tertiary Cases}
The latives and locatives, often also called \w{lcyḍ rrázy’êâ} ‘tertiary cases’, are cases that indicate the position of
or movement away from, towards, or in relation to an object.


\subsection{Illative}

\w{B’hét’hebhaú} ‘to France’

\subsection{Inessive}
The \s{iness} indicates that an object is inside another, e.g. \w{dwávnásḍér} ‘in the castle’. For objects which you tend
to be on top of rather than inside of, the \s{iness} typically means ‘on (top of)’, e.g. \w{dwáḍalẹ} ‘on the table’, but
can still mean ‘in’ depending on the context, e.g. \w{se jaý’aú dwáḍalẹ} ‘the nail is in the table’.

Furthermore, the \s{iness} is used not only for places, but also time, e.g. \w{dwávíd’hẹ} ‘at noon’. It can also mean
‘while’ in this context, e.g. \w{dwádír « jad’hór »} ‘while saying “I love you”’ (lit. ‘in the saying (of) \ldots’).

\subsection{Superessive}
The \s{spress} case means ‘above’, e.g. \w{sýrḍalẹ} ‘above the table’. Note that it normally does \textit{not} mean
‘on’; for that, the \s{iness} is generally used instead. In rare cases, however, the \s{spress} \textit{is} used to mean
‘on’; this is mainly the case for nouns that refer to images, pictures, statues, and other forms of (artistic) imagery,
e.g. \w{dẹhýnrál} ‘on a canvas’, \w{dẹhýnváj} ‘in a picture’.

The \w{h} in the \s{spress def pl} is omitted if the following sound is a consonant, e.g. \w{dẹhárb} ‘above the trees’
but \w{dẹt’halẹ} ‘above the tables’.

\subsection{Superlative}
The \s{super} case—not to be confused with the superlative degree of comparision—indicates motion to the top of or on
top of an object, e.g. \w{áuḍalẹ} ‘atop the table’. Note also that the \w{áu} is \textit{not} a typo.

\subsection{Perlative}
The \s{perl} case denotes motion along a path, e.g. \w{lý’aúc̣évê} ‘along the street’; across a plane, e.g. \w{lý’aúd’háẹ}
‘across the sky’; or through some medium or object, e.g. \w{lý’aúvérr} ‘through the sea’. It can also be used figuratively,
e.g. \w{lý’aúleḍ} ‘by the book’.

The formation \s{perl def sg} deserves some explanation: the \w{d’h} is dropped if the word starts with a consonant, in
which case the prefix causes lenition, e.g. \w{lý’aút’halẹ} ‘across the table’; conversely, it is retained if the word
starts with a vowel, e.g. \w{lý’aúd’háẹ} ‘across the sky’.

\subsection{Ablative}
The \s{abl} signifies motion away from or off an object, e.g. \w{rêvá} ‘away from the mast’.

The \w{d} in the \s{def abl sg} and is omitted if the noun starts with a consonant, e.g. \w{rêḍalẹ} ‘from the table’; be
careful especially with words that start with \w{s}, whose \s{abl sg} is often mistaken for a plural, e.g. \w{rêsol}
‘from the floor’, but \w{rêssol} ‘from the floors’.

\subsection{Elative}
The \s{ela} expresses motion out of or down from something—whether the sense is ‘out of’ or ‘down from’ depends on the
same criteria as the \s{iness}: if the latter means ‘in’, the former means ‘out of’, and vice versa. E.g.
\w{jrét’hír ijaý’au órdḍalẹ} ‘I pull the nail out of the table’.

When paired with the \s{ill}, the meaning of the two is often ‘from X to Y’, or, if the word is the same, ‘X after X’,
or ‘X and X again’, signifying repetition, e.g. \w{órḍraúvá ádásaul} ‘from moon to sun’; \w{órdy’úr ádáy’úr} ‘day after
day’.

\subsection{Postessive and Postlative}
The \s{pstess} and \s{pstlat} cases are used to indicate that an object is positioned or moving behind another object,
respectively.

%% [TODO: Postelative (idea: ‘out of hiding’ from lit. ‘away from behind a bush’)]


\subsection{Diachronic Comments}
The diachrony of these forms is mostly from the PF definite and indefinite pronouns as well as from PF prepositions, though
some forms, such as the
accusative, are borrowed from demonstratives instead (\s{def} from \pf{celui} and \s{indef} from PF \pf{ce}); the definite
partitive forms are from the PF partitive article, and
the indefinite forms are formed with an additional \w{d-} by analogy to the definite forms. The locative cases are combinations
of the articles and PF prepositions. The ablative is from \pf{loin de} ‘away from’. The diachrony of the genitive singular
is unclear.

\section{Negation}\label{subsubsec:noun-negation}
Nouns, as well as proper adjectives and adverbs (i.e. those which are not formed from an adjective verb) are negated
using the particle \w{aû}, which may be separated from the noun by a hyphen for clarity, e.g. \w{aûádróid} or \w{aû-ádróid}
‘non-android’. Improper adjectives and adverbs are negated just like any other verb.

\section{Adjectives}
UF does not have many actual adjectives. Most words in UF are either nouns or verbs, and most ‘adjectives’ are just
participles, which can always be used like adjectives. Indeed, there are a lot of verbs whose meaning is something
along the lines of ‘to be X’, whose present participle behaves like the adjective ‘X’, e.\,g. \w{ḅẹt’hẹ} ‘to be small’
to \w{ḅẹt’hâ} ‘small’ (literally ‘being small’).

Adjectives generally—but not always—follow the noun they modify and are never inflected, e.g. \w{át’halẹ ḅẹt’hâ} ‘of the small table’.
There is no established order of adjectives.

\subsection{Tense of Adjective-Participles}
First, this section is about the tense of the participle form of an adjective verb; if the adjective verb occurs as a
finite form, it behaves just like any other verb.

Like most of the time in UF, the tense of the participle is relative to the frame of the conversation, not
the event described: Even if the event takes place in the past or future, e.g. \w{jráré slé lárâ} ‘I saw a large plain’,
the participle will still be present tense, provided the thing described continues to hold that quality. Thus, a past
participle, e.g. \w{jráré slé lárêr} ‘I saw a formerly-large plain’, indicates that the quality no longer applies to
the referent at the time of speaking.

Future participles, however, are a bit more complicated. There are generally two classes of adjective verbs: those that
describe properties that
could reasonably be known in advance, and those that describe properties that could not. This is a purely semantic distinction: e.g. \w{ad’hyl}
‘adult’ can refer to a future situation, since if you know how old someone is and that they are not an adult yet, you
can reasonably conclude that they will be an adult at some future point in time and when; thus, e.g. \w{vaú áb’há ad’hylŷr}
‘my soon-to-be adult child’ is relative to the time of the utterance.

By contrast, a future participle of e.g. \w{cér} ‘dear’ cannot be relative to the time of utterance, as there is no way of
knowing whether something will be dear to you in advance. Thus, such a participle is only valid if the context is in
the past, in which case its sense takes on that of a future-in-the-past, e.g. \w{jrâhaút’hé b’hŷnâví cérrŷvé} ‘I met a dear
friend of mine’ more literally means ‘I met someone who would become (lit. “will be”) a dear friend to me’, whereas
\#\w{jrâhaúḍ b’hŷnâví cérrŷvé} ‘I meet a friend that will be to me’ is semantically invalid.

Such future participles specifically indicate that the property only started applying to the referent at some point in the
past after the context that the utterance refers to. The above example \w{jrâhaút’hé b’hŷnâví cérrŷvé} indicates that the
person I met was not dear to me, whereas a present participle, \w{jrâhaút’hé b’hŷnâví cérâvé}, would indicate that
they were already dear to me, and that we simply came together at that point in time.

Lastly, in hypothetical scenarios, future adjective-participles of the latter class can often be relative to the time of
utterance, e.g. consider optative \w{jaúy’ẹrâhaúḍrẹ́ vê b’hŷnâví cérrŷvé} ‘I hope that tomorrow, I will meet a friend that
will become dear to me’.

\subsection{Adverbs}
Adverbs are formed from adjectives in one of two ways. For adjectives that are not adjective verbs, \w{-(é)vâ} is added,
e.g. \w{c’haú} ‘holy’ becomes \w{c’haúvâ} ‘holily’; the \w{e} is only present if the adjective ends with a consonant. Adjectives that
are derived from adjective verbs instead replace the \w{-â} affix with \w{-vâ}, e.g. \w{réâ} ‘true’ becomes \w{révâ} ‘truly’. This
form \textit{never} inserts an \textit{e}.

\subsection{Agent and Patient Nouns}% agent noun, patient noun
The active and passive participles can be inflected as though they were nouns to form agent and
patient nouns, e.g. \w{lá-ad’hórâ} ‘the lover’ and \w{láhad’hór} ‘the loved one’. Be mindful of the fact
that this usage may be indistinguishable from a normal participle in the absolutive case.

Agent nouns often carry with them an innate constancy in that the quality they denote
is primarily understood to be inherently gnomic, for which reason they generally do not take the gnomic affix,
e.g. \w{válvêâ} ‘torturer’ first and foremost describes someone who is a torturer by profession. This notwithstanding,
in a context such as \w{ez válvêâ} ‘his torturer’, it may instead refer to a person that is merely doing the torturing in the
situation in question, but does not torture on a regular basis. Forms such as \w{válvêjâ}, though attested, are rare in
literary language, and chiefly serve to emphasise the fact that the quality in question should be understood as gnomic
in a context where that may not be immediately obvious.

Notably, this is generally \textit{not} the case for patient nouns! E.g. \w{âválvê} ‘torturee’ would never be interpreted
as gnomic, as ‘being someone who is tortured frequently’ is not a particularly common property—and the same applies to
most other verbs.

%% ‘lẹ-’: from PF ‘plus’. Comparative prefix.
\subsection{Comparison}\label{subsec:comparison}
Unlike in many other languages, there are 4 comparatives in UF: The affirming comparative, so called
because it affirms the positive (‘better, and also good’); the denying comparative, which denies the positive
(‘better, but not good’), the sufficient comparative, which indicates that there is just enough of something,
and the neutral comparative, which does not make any statement about the positive (‘better’).

To illustrate the difference between the three: We might say that an ant is ‘bigger’ than a grain of sand, but
an ant is still not big, all things considered. By contrast, an elephant may be ‘smaller’ than a mountain,
but that doesn’t mean that an elephant is small. Finally, a human is small enough to fit through a regular-sized
door.

In UF, the comparatives are expressed by four infixes, which are prefixed directly to the stem. The affirming
comparative prefix is \w{lẹ}, the denying comparative prefix is \w{y’ŷ}, the sufficient comparative prefix is \w{ah},
and the neutral comparative prefix is \w{rê}. Thus, we have \w{ḅẹt’hâ} ‘small’, \w{lẹḅẹt’hâ} ‘smaller, and also small’,
\w{y’ŷḅẹt’hâ} ‘smaller, but not small’, \w{ahḅẹt’hâ} ‘small enough’, and \w{rêḅẹt’hâ} ‘smaller’. Unlike other degrees
of comparison, the sufficient comparative is always regular.

The comparative prefixes can also be applied to verbs, though they usually only make sense for the aforementioned
‘adjective verbs’, e.g. \w{jy’ŷḅẹt’hẹ} ‘I am smaller, but still big’. Note that these prefixes
might cause a verb’s forms to change from vocalic to consonantal, e.g. \w{ebhẹ} ‘to be thick’ (future stem \w{ebhrẹ})
is vocalic \w{náy’ebhraú} ‘we shall be thick’ in the positive, but consonantal \w{aúnraûy’ŷebhraû} ‘we shall be
thicker, but not thick’ in the negative comparative.

The affirming comparative can also be used absolutely, with the meaning of ‘to a large degree’. Thus,
we have \w{ḅẹt’hâ} ‘small’, and \w{lẹḅẹt’hâ} ‘tiny’; sometimes, this also leads to a slight change in meaning
or perception, e.g. \w{ebhâ} ‘thick’, but \w{lẹ-ebhâ} ‘thicc’.

The affirming and denying comparative can also mean ‘too X’ and ‘not X enough’, respectively; thus, \w{lẹḅẹt’hâ}
can also mean ‘too small’, and \w{y’ŷḅẹt’hâ} can also mean ‘not small enough’, though this meaning is somewhat
uncommon in isolation and most commonly found in constructions (see below).

The sufficient comparative can be combined with a clause in the subjunctive to express something along the lines
of ‘so X that \ldots’, expressing that someone’s faculty in X is sufficient to do something, e.g. \w{ahrá le lísá
\ldots} ‘he was so big that he could \ldots’. \w{ah} can also be infixed between the case prefix and stem of a
noun, in which case it has the sense of ‘such’, e.g. \w{lyá sahc’haúnéhás} ‘he had such knowledge (that he could \ldots)’.

The superlative is formed with one of two prefixes: \w{ré\L} and \w{râdvâ}. Be careful not to confuse the former
with the neutral comparative \w{rê}! The two prefixes are largely interchangeable, however, the former is more
literary and also older. The latter is a more recent development to reduce potential ambiguity with the
neutral comparative. Note that \w{ré} lenites, whereas \w{râdvâ} does not. Thus, we have \w{rébhẹt’hâ} or
\w{râdvâḅẹt’hâ} ‘smallest’.

The standard of comparison (viz. the thing being compared against) is marked with the \s{gen} case, e.g. \w{rêrá ábhárb}
‘bigger than the trees’.

\subsection{Constructions}
The comparative can be used together with an infinitive, \s{aci}, or \s{pci}. The affirming comparative here has the meaning
of ‘too X to \ldots’, and the denying comparative has the meaning of ‘not X enough to \ldots’. A good illustrative
example of this is the following UF proverb:

\gloss {
    Láráhó slẹlúrá b’héd’hẹhẹ dẹnájẹ.
    Lá-ráhó|s-lẹ-lúr-á|b’hé\Sl d’hẹhẹ|dẹ-nájẹ
    \s{nom}-fish|\s{3n}-\s{aff.comp}-bulky-\s{3sg.pres.ant}|\s{all}\Sl surface|\s{inf}-swim
    ‘The fish was too bulky to swim to the surface’\footnotemark
}

\footnotetext{This is a very common proverb (often also just \textit{láráhó slẹlúr} ‘The fish is too bulky’)
and roughly means that something has gone too far or gone on for too long (‘Now you’ve done it’ or ‘Now
it’s too late’). Variations of it exists; in the optative, for instance, this proverb means ‘Let’s not overdo this’.}

The comparative forms can also be prefixed to verbs, in which case they precede the stem directly and have the meaning
of ‘to X more’, etc., e.g. \w{jrêdír} ‘I say more’. In this sense, the affirming comparative is generally construed
as continuing an action, e.g. \w{jlẹdíré} ‘I continued saying’/‘I continued’/‘I said further’ (lit. ‘I said more, and a lot’
$\approx$ ‘I said more after already having said a lot’), the denying as resuming or commencing an action after some of
inaction has passed, e.g. \w{jy’ŷdíré} ‘I finally said’ (literally ‘I said more, but not much’ $\approx$ ‘I said (more)
after not having said much’). Finally, the sufficient comparative has the expected meaning, e.g. \w{ḍahférá} ‘You have done enough’.

\section{Pronouns}
Pronouns in UF are a rather complicated subject matter since they are becoming increasingly vestigial. UF has
a set of personal pronouns which only exist in oblique cases; a set of simple possessive pronouns, a set of demonstrative
pronouns, as well as interrogative and relative pronouns.

\subsection{Personal pronouns}\label{subsubsec:personal-pronouns}
Table~\ref{tab:personal-pronouns} below lists all forms of the UF personal pronouns.\enlargethispage{\baselineskip}
\begin{table}[H]
\centering\itshape
\begin{tabular}{@{}>{\nf}l|llll|lll}
              & \multicolumn{4}{c|}{\nf Sg} & \multicolumn{3}{c}{\nf Pl} \\\cline{2-8}
              & \s{1st} & \s{2nd}  & \s{3c} & \s{ 3n} & \s{1st} & \s{2nd} & \s{3rd} \\\hline
Absolutive    & vè  & t’hè  & lè  & sè                      & aú    & vaú  & y   \\
Vocative      & /   & et’hè & /   & /                       & /     & evaú & /   \\
Genitive      & vaú & daú   & \multicolumn{2}{l|}{ez/z’/’z} & naúḍ  & vaúd & lýr \\
Prepositional & vẹ  & t’hẹ  & lẹ  & sẹ                      & aun   & vau   & ly \\
\end{tabular}
\nf
\caption{UF Personal Pronouns}\label{tab:personal-pronouns}
\end{table}

\Paragraph{Nominative and Accusative}
There are a few things that need to be noted here: there are no \s{nom} and \s{acc} pronouns; those
forms have been incorporated into the verb and cannot be used without a verb. For instance, when answering a question,
typically, either the same verb that was used to ask the question
is repeated or an appropriate form of the verb \w{fér} ‘to do’ is used, e.g. if asked \w{U c’hes ḍẹvad’hór ra ḍẹy’ad’hór?}
‘Do you love me or him?’, an UF speaker might respond with \w{ḍad’hór} ‘I love you’ or \w{ḍẹfér} ‘you’ (lit. ‘I do you’).

On that note, there are several ways of shortening the question itself: In sentences that contain the same verb with the same
affix twice, the second occurrence of that affix may be omitted, e.g. \w{U c’hes ḍẹvad’hór ra y’ad’hór?}; the verb \w{fér} may be used
to avoid repetition, e.g. \w{U c’hes ḍẹvad’hór ra y’fér?}; and, finally, the entire first occurrence of the verb sans the person markers may
be omitted, leaving said affixes stranded in the sentence, e.g. \w{U c’hes ḍẹv- ra y’ad’hór?}. This last option is generally
preferred since it it is the shortest option, but, of course, it is only possible if the first verb form contains only prefixes.

\Paragraph{Partitive}
The \s{part} forms of the personal pronoun are rather strange; generally, verbs that govern the \s{part} simply take regular
passive affixes instead. However, verbs that can be formed with both the \s{acc} and \s{part} as well as \s{pci}s employ
special partitive forms of the passive affixes that are constructed by infixing \w{-dy-} after the prefix part
of the corresponding passive affix—or before the suffix part if there is no prefix part—e.g. \w{jsylí} ‘I peruse it’ vs
\w{jsydylí} ‘I read from it’ or \w{lírá} ‘be perused!’ vs \w{lídyrá} ‘be read from!’.\footnote{The only passive forms that do
not have prefix parts are imperatives.}

\Paragraph{Genitive}
The possessees of \s{gen} pronouns can be definite or indefinite, e.g. \w{vaú lát’halẹ} ‘my table’ vs \w{vaú ŷnalẹ} ‘a table
of mine’. Most \s{gen} pronouns are not particularly special and behave just like regular genitives; the exception is the
\s{3sg} pronoun that is used for all three genders: its base form is \w{ez}, e.g. \w{ez lát’halẹ} ‘his/her/its table’, but
after a word that ends with a vowel, the \w{e} is dropped, and it is instead written \w{’z}, e.g. \w{ḍevvaúríhe’z st’halẹ} ‘to
remember a table of his/hers/its’. If the following word starts with a vowel, it is somtimes written \w{z’},\footnote{The
apostrophe in \w{z’} makes no sense in that position,
but it probably came about in imitation of similar forms that affix to the following word, e.g. the \s{opt} negation particle
\w{t’hé}, which becomes \w{t’h’} before vowels.} e.g. \w{ḍevvaúríhe z’it’halẹ} ‘to remember his/her/its table’, though
\w{ḍevvaúríhe’z it’halẹ} is also common and preferred in traditional literature.

In Early Modern UF, \w{ez} was sometimes infixed between a case affix and the stem of its noun, e.g.
\w{dwá’zárb} ‘in his tree’.

\Paragraph{Prepositional}
The ‘prepositional’ form is not a case, but rather a form that case prefixes attach to to form all the other cases, e.g.
the \s{2sg instr} would be \w{b’helt’hẹ} ‘with you’. Note that personal pronouns use the \textit{definite} case prefixes of the
appropriate number. All remaining cases can be formed this way, but of course not the \s{abs, nom, acc, part, voc}, and
\s{gen}. The prepositional form is never used in isolation.

\subsection{Possessive Pronouns}
UF does not really have possessive pronouns; instead, it has a series of possessive adjectives, which—just like most other
‘adjectives’—are really just a series of adjective verbs: \w{y’ê} ‘to be mine’, \w{dy’ê} ’to be yours (\s{sg}),
\w{sy’ê} ‘to be his/hers/its’, \w{naúḍ} ‘to be ours’, \w{vauḍ} ‘to be yours (\s{pl})’, and \w{lýrḍ} ‘to be theirs’. These verbs
are chiefly used as verbs, e.g. \w{ŷnalẹ sy’ê} ‘it is a table of mine’; for just expressing ‘my’ etc., the \s{gen} of the
corresponding possessive pronoun is used instead, as indicated above.

\subsection{Demonstrative Pronouns}
UF has three main demonstrative pronouns: \w{swi} ‘the one, that one, this one’, \w{sẹh} ‘this’, and \w{sý’ẹ} ‘that’. All
three are normally indeclinable and precede whatever they qualify: the first generally occurs in isolation, in which
case it is declined as a definite noun, or indeclinably with an adjective or pronoun, e.g. \w{swi ḅẹt’hâ} ‘the small one’
or \w{swi a lẹḅẹt’hẹ} ‘the one who is small’. The latter two always precede a definite noun, e.g. \w{sẹh lát’halẹ} ‘this table’,
and are themselves never declined. It is not possible to combine demonstratives with one another.

\subsection{Relative Pronoun}
The UF relative pronoun is \w{a} ‘which, who, that’. Its most obvious and direct use is to form relative
clauses and agrees in definiteness and number with the noun it qualifies, e.g. \w{lát’halẹ, ia jad’hór} ‘the
table that I love’ or \w{ŷnalẹ, sa jad’hór} ‘a table that I love’.

If the antecedent is too far away from the relative clause, it may be repeated in the relative clause,
usually in the definite, typically at the very beginning, in which case the relative pronoun follows it and is not inflected at
all, e.g. \w{lát’halẹ, it’halẹ a jad’hór} ‘the table, which table I love’. In literary language, this
construction is generally preferred over inflecting the relative pronoun if the two are far apart.

If there is no single antecedent (e.g. because it is ‘A and B’), or no antecedent at all (e.g. ‘that which’) then the
relative pronoun may be used on its own, and is always inflected in that case, e.g. \s{dat} \w{asa jad’hór} ‘to the
one I love’.

\subsection{Interrogative Pronoun}
The interrogative pronoun is the same as the relative pronoun, except that it is also followed by the
question particle \w{c’hes}. Unlike the relative pronoun, it is always declined. On its own, it takes
indefinite case when it refers to a thing, e.g. \w{Sa c’hes ḍad’hór?} ‘What do you love?’, and definite
case when it refers to a person, e.g. \w{Ia c’hes ḍad’hór?} ‘Whom do you love?’

If the subject of the question is a noun phrase that contains more than just the interrogative pronoun,
pronoun and question particle are added after the entire phrase, and the pronoun is not declined, e.g.
\w{Ŷnalẹ a c’hes ḍad’hór?} ‘Which table do you love?’. In informal speech, the \w{a} is even omitted
sometimes.

A common variant spelling in older literature is to write the pronoun and question particle as one
word, e.g. \w{sac’hes} instead of \w{sa c’hes} or to contract the ‘e’, e.g. \w{sac’h’s}.

\section{Numerals}\label{subsec:numerals}
UF has four sets of numerals: cardinals, e.g. \w{dý} ‘two’; ordinals, e.g. \w{dýzy’ê} ‘second’; multipliers, which can
be adverbs, e.g. \w{dub} ‘twice’, or adjectives, e.g. \w{dubâ} ‘twofold’; and fractions, e.g. \w{déví} ‘half’. The numerals
are shown in the table below.

{
\itshape
\begin{longtable}{>{\nf}l|l|l|l|l}
    \nf № & \nf Cardinal & \nf Ordinal & \nf Multiplier & \nf Fractional \\\hline
    1      & ý               & révy’ẹ́             & séḅ               & áḍy’ẹ́           \\
    2      & dý              & dýzy’ê             & dub               & déví            \\
    3      & rrá             & rrázy’ê            & ríḅ               & y’ér            \\
    4      & c’haḍ           & c’haḍríy’ê         & c’hadrýḅ          & c’hár           \\
    5      & séc’h           & séc̣é               & c’hét’hyḅ         & c’hé            \\
    6      & sis             & sizy’ê             & sec’hsḍyḅ         & sic’hs          \\
    7      & sèḍ             & sèḍy’ê             & sèḍyḅ             & sè              \\
    8      & y’íḍ            & y’íḍy’ê            & auc’hḍyḅ          & auc’h           \\
    9      & nýt’h           & nýb’hy’ê           & nýḅ               & ny              \\\hline
    10     & dis             & dizy’ê             & dehyḅ             & deh             \\
    11     & aúz             & aúzy’ê             & aúzyḅ             & auz             \\
    12     & duz             & duzy’ê             & duzyḅ             & duz             \\
    13     & réz             & rézy’ê             & rézyḅ             & rez             \\
    14     & c’hat’haúr      & c’hat’haúrzy’ê     & c’hat’haúrzyḅ     & c’hat’haurz     \\
    15     & c’héz           & c’hézy’ê           & c’hézyḅ           & c’hez           \\
    16     & sez             & sezy’ê             & sezyḅ             & sez             \\
    17     & dihèḍ           & dihèḍy’ê           & dihèḍyḅ           & dihè            \\
    18     & dizy’íḍ         & dizy’íḍy’ê         & dizy’íḍyḅ         & dizy’i          \\
    19     & diznýt’h        & diznýb’hy’ê        & diznýt’hyḅ        & diznyb’h        \\\hline
    20     & b’hé            & b’héy’ê            & b’héḍyḅ           & b’he            \\
    21     & b’hé’d ý        & b’hé’d rév’yẹ́      & b’hé’d séḅ        & b’hé’d áḍy’ẹ́    \\
    30     & b’hé’d dis      & b’hé’d dizy’ê      & b’hé’d dehyḅ      & b’hé’d deh      \\
    31     & b’hé’d aúz      & b’hé’d aúzy’ê      & b’hé’d aúzyḅ      & b’hé’d auz      \\
    40     & dýb’hé          & dýb’héy’ê          & dýb’héḍyḅ         & dýb’he          \\
    50     & dýb’hé’d dis    & dýb’hé’d dizy’ê    & dýb’hé’d dehyḅ    & dýb’hé’d deh    \\
    60     & rráb’hé         & rráb’héy’ê         & rráb’héḍyḅ        & rráb’he         \\
    70     & rráb’hé’d dis   & rráb’hé’d dizy’ê   & rráb’hé’d dehyḅ   & rráb’hé’d deh   \\
    80     & c’haḍb’hé       & c’haḍb’héy’ê       & c’haḍb’héḍyḅ      & c’haḍb’he       \\
    90     & c’haḍb’hé’d dis & c’haḍb’hé’d dizy’ê & c’haḍb’hé’d dehyḅ & c’haḍb’hé’d deh \\\hline
    100    & sá              & sáḍy’ê             & sáḍyḅ             & sáḍ             \\
    101    & sá’d ý          & sá’d rév’yẹ́        & sá’d séḅ          & sá’d áḍy’ẹ́      \\
    200    & dýsá            & dýsáḍy’ê           & dýsáḍyḅ           & dýsáḍ           \\
    300    & rásá            & rásáḍy’ê           & rásáḍyḅ           & rásáḍ           \\
    400    & c’hasá          & c’hasáḍy’ê         & c’hasáḍyḅ         & c’hasáḍ         \\
    500    & sésá            & sésáḍy’ê           & sésáḍyḅ           & sésáḍ           \\
    600    & sisá            & sisáḍy’ê           & sisáḍyḅ           & sisáḍ           \\
    700    & sèsá            & sèsáḍy’ê           & sèsáḍyḅ           & sèsáḍ           \\
    800    & y’ísá           & y’ísáḍy’ê          & y’ísáḍyḅ          & y’ísáḍ          \\
    900    & nýsá            & nýsáḍy’ê           & nýsáḍyḅ           & nýsáḍ           \\\hline
    1~000  & víl             & víly’ê             & vílḍyḅ            & víláḍ           \\
    1~001  & víl ed ý        & víl ed rév’yẹ́      & víl ed séb        & vil ed áḍy’ẹ́    \\
    2~000  & dý víl          & dý víly’ê          & dý vílḍyḅ         & dý víláḍ        \\
    10~000 & dis víl         & dis víly’ê         & dis vílḍyḅ        & dis víláḍ       \\
    10$^6$ & víwaú           & víwaúy’ê           & víwaúḍyḅ          & víwaúḍ          \\
    10$^{12}$& dýwaú         & dýwaúy’ê           & dýwaúḍyḅ          & dýwaúḍ          \\
    10$^{18}$& ráwaú         & ráwaúy’ê           & ráwaúḍyḅ          & ráwaúḍ          \\
\end{longtable}
}

\noindent
The numbers 1–20 are irregular;
after that, ordinals are formed by adding \w{-y’ê} to the cardinal and multipliers by adding \w{-ḍyḅ} to the cardinal;
fractionals are more irregular: the tens lose nasalisation of the final vowel, e.g. \w{dýb’hé} ‘forty’ vs \w{dýb’he}
‘(a) fortieth’; in the hundreds and after, a final \w{-(á)ḍ} is added instead. Extra syllables added by non-cardinal
forms do not count as part of the stem for the purpose of stress.

Ordinals can be inflected like adjective verbs, e.g. \w{révy’ẹ́â} ‘primary’.

After 20, numbers of different orders of magnitude are combined with the particle \w{ed}, which is solely used for this
exact purpose. After a vowel, it is reduced to \w{’d}, e.g. \w{sá’d ý} ‘101’ or \w{sá’d b’hé’d ý} ‘121’ from \w{sá} ‘100’,
\w{b’hé} ‘20’ and \w{ý} ‘one’. In non-cardinals, only the last part is of ordinal, multiplier,
or fractional form, e.g. \w{sá’d b’hé’d séḅ} ‘121 times’.

In writing, non-cardinals are frequently abbreviated, preferably with superscripts if possible. Ordinals are abbreviated
with \w{\Sup{y’ê}}, e.g. \w{27\Sup{y’ê}} ‘27th’, except for numbers ending
with \w{révy’ẹ́} ‘first’, which are abbreviated with \w{\Sup{y’ẹ́}} instead, e.g. \w{21\Sup{y’ẹ́}} ‘21st’, as
well as numbers ending with \w{séc̣é} ‘fifth’, which are abbreviated with \w{\Sup{c̣é}} instead, e.g. \w{25\Sup{c̣é}}
‘25th’.

The adverbial multipliers (sometimes also called \textit{multiplicative} numerals) shown in the table above are abbreviated with
subscripts if possible; those ending with \w{séḅ} ‘once’ to \w{nýḅ} ‘nine times’ are abbreviated with the last two letters of that
word, e.g. \w{23\Sub{íḅ}} ‘23 times’. All other adverbial multipliers are abbreviated with \w{\Sub{yḅ}}, e.g. \w{31\Sub{yḅ}}
‘31 times’.\footnote{Note that ‘31’ in UF is not really ‘thirty-one’, but rather ‘twenty-eleven’ and thus doesn’t
end with ‘one’.}

The marker \w{-â}, presumably a fossilised form of the \s{pres ptcp} affix, is used to turn adverbial multipliers into
adjectives, e.g. \w{dehyḅ} ‘ten times’ becomes \w{dehyḅâ} ‘tenfold’. Adjective multipliers are abbreviated with a single
subscript \w{\Sub{â}}, e.g. \w{23\Sub{â}} ‘23-fold’.

‘Firstly’/‘at first’, ‘secondly’, etc. are constructed from the ordinals using the usual adverb suffix, e.g.
\w{révy’ẹ́vâ} ‘at first’.

Fractions are typically abbreviated with the usual notation, e.g. \w{½}, \w{¼}, etc.

\chapter{Verbs}\label{subsec:verbal-morphology}
Verbs in UF are inflected for person, number, tense, aspect, mood, and voice. Verbal inflexion is mainly done
by means of concatenating a vast set of affixes. This chapter details these affixes, their meanings, uses,
forms, and restrictions in their use.

\section{Fundamental Forms}
\subsection{Active/Passive Affixes}\label{subsubsec:active-passive-affixes}
The most fundamental affixes in UF are a set of active/subject and passive/object affixes (often referred to as the
‘active/passive affixes’) which can be used on their own or in combination
with one another, though at most one active and one passive affix may be combined in any one finite verb form.\footnote{Doubly
passive forms can occur in rare cases if infinitives are involved; see §~\ref{subsubsec:pronominal-aci}.}
Table~\ref{tab:active-passive-prefixes} below lists those affixes.

\begin{table}[H]
\centering
\noindent\begin{tabular}{l|>{\it}l|>{\it}lll|>{\it}l|>{\it}l}
Active&\nf Sg&\nf Pl& & Passive&\nf Sg&\nf Pl\\\cline{1-3}\cline{5-7}
\s{1st}  &j-     &aú-/r-/w- -(y’)ó      &&\s{1st}  &v-    &aú-/r-/w- \\
\s{2nd}  &ḍ(ẹ)-  &b’h(y)- -(y’)é        &&\s{2nd}  &ḍ(ẹ)- &b’h(y)-   \\
\s{3m}   &l(ẹ)-  &l(ẹ)-                 &&\s{3m}   &y’-   &lý-       \\
\s{3f}   &ll(a)- &ll(ẹ)-                &&\s{3f}   &y’-   &lý-       \\
\s{3n}   &s-     &l(a)-                 &&\s{3n}   &sy-   &lý-       \\\cline{1-3}\cline{5-7}
\s{inf}  &\multicolumn{2}{c}{\it d(ẹ)-} &&\s{inf}  &\multicolumn{2}{c}{\it à-/h-}\\
\s{ptcp} &\multicolumn{2}{c}{\it -â}    &&\s{ptcp} &\multicolumn{2}{c}{\it â-}\\
\end{tabular}
\caption{Active (left) and passive (right) verbal affixes.}\label{tab:active-passive-prefixes}
\end{table}

\noindent A great degree of syncretism can be observed in the third-person forms. The gender distinction in the
\s{3sg} that diachronically resulted from gendered personal pronouns is almost non-existent in the
plural; the reason for this development is that those forms are derived from the old dative form, which lacked
this distinction altogether.

The \s{act 1pl, 2pl} forms are only distinguished from their passive counterparts by
the presence of additional suffixes in the former. The \s{3sg n} in the active and passive is derived from the PF
demonstrative \pf{ce} and its variants; the \s{3pl n} is derived from the other \s{3pl} forms.

A verb can have a passive affix only if there is no other explicit direct object in the clause. In other words, while verbs
\textit{do} take active person marking even if there is an explicit subject e.g. \w{lávvâ llad’hór} ‘the mother loves’, they do
\w{not} take passive person marking if there is an explicit object (unless there is no subject), e.g. \w{lávvâ llvad’hór} ‘the
mother loves me’, vs \w{lávvâ llad’hór iáb’há} ‘the mother loves the child’, which has \w{llad’hór} ‘she loves’ instead of
\w{llsyad’hór} ‘she loves it’.

Every finite verb form requires at least one finite affix. A verb form without any active, passive, or dative affix whatsoever
would not be a finite verb form and could thus never be the predicate of a sentence.\footnote{Excluding of course the fact
that infinitives could be considered to function as predicates of \s{aci}s and \s{pci}s (see §~\ref{subsec:aci-pci}).}

Thus, if the active affix is omitted, the verb has to have at least a passive or dative marker.
Such a construction would be the closest equivalent to a passive in UF, since there is
no true distinct syntactical or morphological passive, e.g. \w{y’ad’hór ivvâ} ‘the mother is loved’.\footnote{Note the accusative
\textit{ivvâ} here, and recall also that, as mentioned above, the verb takes passive marking even though there is an explicit
object, simply because there is no subject at all.} Furthermore, it is impossible to express the agent
in the ‘passive’ by any
means other than reintroducing an active affix, which would render the form no longer a passive.\footnote{The closest
UF gets to an ‘agent in the passive’ is by forming a regular active, but placing the agent last in the clause.}

It is possible to combine both the active and passive infinitive marker to form a reflexive infinitive, e.g. \w{dẹhad’hór}
‘to love oneself’.

Lastly, ditransitive verbs and verbs governing the dative case generally take a dative affix (see
§~\ref{subsubsec:dative-affixes}) iff there is no explicit indirect object.

\Paragraph{Usage Notes}
\begin{items}
\Item{2sg}
Watch out for the \s{2sg act}, which in verbs that start with a vowel is indistinguishable from the \s{inf act} in
actual writing, e.g. \w{ḍad’hór} ‘you love’ vs \w{dad’hór} ‘to love’; since the dot is omitted in writing, both forms
look the same: \w{dad’hór}. Moreover, the \s{2sg pass} is identical to the \s{2sg act} in any case.

Which form is intended can often be inferred from context: if the clause already has a finite verb, especially one
that takes an infinitive or ACI, it is more likely to be an infinitive; by contrast, if it is the only (possibly finite)
verb in the clause, then it is probably a \s{2sg}. Whether it is active or passive can then be deduced based on whether
the verb is transitive and whether there is an explicit object in the clause.

\Item{1pl}
The \s{1pl} prefix varies if there is a vowel following it: if it is
any vowel that is \textit{not} a variant of ‘o’, the prefix is realised as \w{r-} instead, e.g. \w{ad’hór} ‘love’ to
\w{rad’hóró} ‘we love’. If the vowel a variant of ‘o’, the prefix is realised as \w{w-} instead, e.g. \w{aub’heír} ‘obey’
to \w{wob’heíró} ‘we obey’.\footnote{Diachronically, the base form of this prefix is *\w{o-}, whence e.g.
*\w{oad’hóró} > \w{rad’hóró} and *\w{oob’heíró} > \w{wob’heíró}.} Note that this also leads to a change in spelling: stem-initial
⟨au⟩ is changed to ⟨o⟩.

\Item{1,2 pl}
The \w{y’} in the suffix parts of the \s{1pl, 2pl act} are dropped if the verb ends with a consonant, e.g. \w{ad’hór}
to \w{b’hád’hóré}, or if it ends with a vowel that is a variant of ‘o’ in the case of the \s{1pl} and ‘e’ in the case
of the \s{2pl}, in which cases the vowels are contracted and a level of nasalisation is added, e.g. \w{vvaúríhe}
‘to remember’ to \w{b’hyvvaúríhé} ‘you (\s{pl}) remember’ (not *\w{b’hyvvaúríhy’é}). In all other cases, the \w{y’} is retained,
e.g. \w{aúvvaúríhey’ó} ‘we remember’.

\Item{inf}
The \s{inf pass} prefix \w{à-} coalesces with any vowel following it: it becomes \w{á} if it
is followed by a non-nasal variant of ‘a’, e.g. \w{ad’hór} to \w{ád’hór} ‘to be loved’; \w{â} if it is
followed by a nasal variant of ‘a’, e.g. \w{ánvé} ‘give life to’ to \w{ânvé} ‘to be animated’; and \w{h-} if it is
followed by any other vowel, e.g. \w{aub’heír} to \w{haub’heír} ‘to be obeyed’.

In the present tense, the base form—and not the \s{inf}—of the verb is inflected to form gerunds, e.g. \w{ŷnád’hór} ‘a
loving’, not *\w{ŷndad’hór}. However, the \s{inf} \textit{is} used as the base form for gerunds in other tenses, e.g. \w{ŷndad’hórá} ‘a having loved’.

\Item{part}
The participle affixes are commonly used to form adjectives since the vast majority of adjectives in UF are actually ‘adjective verbs’
with a meaning of ‘to be X’. The participle can be used to convert such a verb back into a regular adjective, e.g. \w{lár} ‘to be wide’
to \w{lárâ} ‘wide’. Like the passive infinitive affix, the participle affixes coalesce with vowels and always form a maximally
nasal vowel, e.g. \w{vvaúríhe} ‘to remember’ forms \w{vvaúríhê} ‘remembering’, and \w{ad’hór} forms \w{âd’hór} ‘being loved’. As with other
coalescence rules, the \w{-â} instead \textit{replaces} final or initial \w{ẹ}, and \w{ẹ} only: e.g. \w{ḅẹt’hẹ} ‘to be small’ becomes \w{ḅẹt’hâ} ‘being
small’. Note that if the word already ends with a maximally nasal vowel, no coalescence occurs, e.g. \w{rê} ‘to be triune’ becomes
\w{rêâ} ‘triune’.

\Item{{\upshape\textit{-ẹ-}}}
The parenthesised vowels are used if the prefix is followed by a consonant, e.g. \w{dír} ‘say’ to \w{llẹ{}dír}
‘they (\s{f}) say’ and \w{b’hydíré} ‘you (\s{pl}) say’, but \w{ad’hór} to \w{llad’hór} ‘they (\s{f}) love’ and \w{b’had’hóré} ‘you
(\s{pl}) love’. The prefixes \w{aú-} and \w{à-} retain their main forms if followed by a consonant,
e.g. \w{dír} ‘say’ to \w{aúdíró} ‘We say’ and \w{àdír} ‘to be said’.

\Item{{\upshape\textit{-y-}}}
The exception to this is that \s{2pl} \w{b’h(y)-}
drops the \w{y} if followed by a glide, e.g. \w{y’ír} ‘to hear’ to \w{b’hy’íré} ‘you (\s{pl}) hear’ (not *\w{b’hyy’íré}).
\end{items}

\Paragraph{Combining Prefixes}
When multiple prefixes are used together, active prefixes precede passive prefixes, except that infinitive and participle prefixes
always come first, e.g. \w{ad’hór} ‘love’ to \w{jvad’hór} ‘I love myself’ (not *\w{vjad’hór}) and \w{b’hy’ad’hóré} ‘you (\s{pl}) love him/her’,
but \w{dẹvad’hór} ‘to love me’ and \w{àb’had’hóré} ‘to be loved by you (\s{pl})’. Recall that at most one infinitive prefix
and at most one participle affix may be used.

\Paragraph{Impersonal Forms}
UF does not use the \s{2nd} person in sentences such as ‘when \textit{one} considers / when \textit{you} consider that\ldots’, instead preferring the \s{1pl} (lit.
‘when \textit{we} consider that\ldots’) to express such impersonal constructions. The \s{3n} is used as an expletive for verbs
that do not really have a subject, such as \w{lýv́á} ‘rain’, which forms \w{slýv́á} ‘it rains’.

\Paragraph{Example Paradigms}
By way of illustration, consider the paradigm of the verb \w{ad’hór} as shown in Table~\ref{tab:adhor-paradigm} below.
Since this word starts with a vowel, the parenthesised vowels in Table~\ref{tab:active-passive-prefixes} above
are not used. Furthermore, since it starts with a non-nasal ‘a’-like vowel, the \w{aú-} prefix is realised as \w{r-}
and the \w{à-} prefix coalesces with the initial ‘a’ of the stem to form \w{á}.

% TEMPLATE:
%\noindent\begin{tabular}{@{}|>{}l|>{\it}l|>{\it}l|>{}l|>{}l|>{\it}l|>{\it}l|}\cline{1-3}\cline{5-7}
%\nf Active&\nf Sg&\nf Pl&\nf &\nf Passive&\nf Sg&\nf Pl \\\cline{1-3}\cline{5-7}
%1st       &   &  &&1st   &   &   \\\cline{1-3}\cline{5-7}
%2nd       &   &  &&2nd   &   &   \\\cline{1-3}\cline{5-7}
%3rd m     &   &  &&3rd m &   &   \\\cline{1-3}\cline{5-7}
%3rd f     &   &  &&3rd f &   &   \\\cline{1-3}\cline{5-7}
%3rd n     &   &  &&3rd n &   &   \\\cline{1-3}\cline{5-7}
%\s{inf}& \multicolumn{2}{c|}{\it }  && Infinitive & \multicolumn{2}{c|}{\it } \\\cline{1-3}\cline{5-7}
%\end{tabular}

\begin{table}[H]
\centering
\noindent\begin{tabular}{l|>{\it}l|>{\it}lll|>{\it}l|>{\it}l}
\nf Active&\nf Sg&\nf Pl&\nf &\nf Passive&\nf Sg&\nf Pl\\\cline{1-3}\cline{5-7}
\scshape 1st&jad’hór&rad’hóró    &&\scshape 1st&vad’hór&rad’hór\\
\scshape 2nd&ḍad’hór&b’had’hóré  &&\scshape 2nd&ḍad’hór&b’had’hór\\
\scshape 3m&lad’hór&lad’hór      &&\scshape 3m&y’ad’hór&lýad’hór\\
\scshape 3f&llad’hór&llad’hór    &&\scshape 3f&y’ad’hór &lýad’hór\\
\scshape 3n&sad’hór&lad’hór      &&\scshape 3n&syad’hór&lýad’hór\\\cline{1-3}\cline{5-7}
\s{inf}&\multicolumn{2}{c}{\it dad’hór}&&\scshape inf&\multicolumn{2}{c}{\it ád’hór}\\
\s{ptcp}&\multicolumn{2}{c}{\it ad’hórâ}&&\s{ptcp}&\multicolumn{2}{c}{\it âd’hór}\\
\end{tabular}
\caption{Paradigm of the Verb \emph{ad’hór}.}\label{tab:adhor-paradigm}
\end{table}

\noindent For comparison, the paradigm of the verb \w{vvaúríhe} ‘remember’ is shown in Table~\ref{tab:vvorihe-paradigm} below.
Since it starts with a consonant, the parenthesised vowels in Table~\ref{tab:active-passive-prefixes} are used, and any
prefixes that end with a vowel remain unchanged.

\begin{table}[H]
\centering
\noindent\begin{tabular}{l|>{\it}l|>{\it}lll|>{\it}l|>{\it}l}
\nf Active&\nf Sg&\nf Pl&\nf &\nf Passive&\nf Sg&\nf Pl\\\cline{1-3}\cline{5-7}
\scshape 1st&jvvaúríhe&aúvvaúríhey’ó &&\scshape 1st&vvvaúríhe&aúvvaúríhe\\
\scshape 2nd&ḍẹvvaúríhe&b’hyvvóríhé  &&\scshape 2nd&ḍẹvvaúríhe&b’hyvvaúríhe\\
\scshape 3m&lẹvvaúríhe&lẹvvaúríhe    &&\scshape 3m&y’vvaúríhe&lývvaúríhe\\
\scshape 3f&llavvaúríhe&llẹvvaúríhe  &&\scshape 3f&y’vvaúríhe&lývvaúríhe\\
\scshape 3n&svvaúríhe&lavvaúríhe    &&\scshape 3n&syvvaúríhe&lývvaúríhe\\\cline{1-3}\cline{5-7}
\s{inf}&\multicolumn{2}{c}{\it dẹvvaúríhe}&&\scshape inf&\multicolumn{2}{c}{\it àvvaúríhe}\\
\s{ptcp}&\multicolumn{2}{c}{\it vvaúríhê}&&\s{ptcp}&\multicolumn{2}{c}{\it âvvaúríhe}\\
\end{tabular}
\caption{Paradigm of the Verb \emph{vvaúríhe}.}\label{tab:vvorihe-paradigm}
\end{table}

\subsection{Dative Affixes}\label{subsubsec:dative-affixes}
The dative affixes \w{-vé} ‘me, us’, \w{-b’hẹ} ‘you’, and \w{-ḷẹ} ‘him, her, it, them’ are used in conjunction with
ditransitive verbs and are invariant to tense, gender, number, and mood. A verb can only have one dative affix, and
the dative affix is always placed last after all other affixes and does not coalesce, lenite, or otherwise modify
the rest of the verb, e.g. \w{dedónẹ́} ‘to bestow’ to \w{dedónẹ́ḷẹ} ‘to bestow upon him’.

These affixes are generally not used if the \s{dat} assumes the sense of ‘for someone’, or ‘to someone’; for instance, while
\w{fúr} ‘to provide’ takes a \s{dat} as its indirect object, e.g. \w{jfúrb’hẹ} ‘I provide you (with something)’, the verb
\w{fér} ‘to do, make’ does not, and thus, it is not e.g. *\w{jsyférvé}, but rather \w{jsyfér asvẹ} ‘I did it for me/us’, where
\w{asvè} is the \s{dat} inflexion of the \s{1sg} pronoun.

Lastly, which one—the \s{dat} affixes or a \s{dat} pronoun—is ultimately used often depends on the verb in question. Some speakers prefer
one over the other with certain verbs, and some verbs regularly admit both, albeit with different meanings, e.g. \w{jsydíréḷẹ}
‘I said it to him’ vs \w{jsydíré aslẹ} ‘I said it for his sake’.

\subsection{The Gnomic}\label{subsubsec:gnomic}
The gnomic tense is marked by the infix \w{-j(ú)-} after the stem: \w{ad’hór} ‘to love’ to \w{rad’hórjô} ‘We love (for ever)’.
The \w{ú} is omitted if the infix is followed by the vowel, in which case it causes nasalisation. The presence of the gnomic
is does not affect how verbs are negated.

The gnomic is used to express general truths, habitual actions, or timeless statements. It is more common in
literary language than in speech, which prefers to substitute the present tense instead. Northern dialects
of UF also tend to not make use of the gnomic at all.

\subsection{Imperative}
The imperative mood exists only in the present tense, and only in the second and third person. It is formed by
affixing the following suffixes to the stem.
\begin{table}[H]
\centering
\noindent\begin{tabular}{l|>{\it}l|>{\it}lll|>{\it}l|>{\it}l}
 Active&\nf Sg&\nf Pl& & Passive&\nf Sg&\nf Pl\\\cline{1-3}\cline{5-7}
2nd &c’h(e)-     &c’heb’h(y)- &&2nd& -rá   &-nú\\\cline{1-3}\cline{5-7}
3rd &\multicolumn{2}{c}{\it c’hel(ẹ)-} &&3rd& -ḷẹ   &-b’hẹ\\
\end{tabular}
\caption{Imperative Affixes.}\label{tab:imperative-affixes}
\end{table}

\noindent The diachrony of these forms is likely from subjunctive constructions with \pf{que} in the active
and from suffixed pronouns in the passive. Note that imperative affixes are added \textit{in place} of
present active/passive affixes, e.g. \w{c’hedír!} ‘speak!’, not *\w{c’heḍẹdír}. As usual, the parenthesised
vowels are omitted if the verb form starts with a vowel, e.g. \w{c’had’hór!} ‘love!’.

Imperative affixes can be combined with active/passive affixes, though, as usual, an active imperative prefix
can only be paired with a passive present affix, and vice versa. Active imperative prefixes are always placed
first, e.g. \w{c’hevad’hór!} ‘love me!’, and passive affixes are placed last, e.g. \w{b’had’hórérá} ‘be loved by
us!’. The negation of the imperative uses the subjunctive and is explained in §~\ref{subsubsec:negated-subjunctive}.

\section{Past Tenses}\label{subsec:tense-and-aspect-marking}
Past tenses in UF are marked by additional sets of affixes that are appended to the verb in addition to the active/passive affixes.
There are two broad groups of such affixes. UF has three past tenses:
\begin{items}
\item the Present Anterior, which has a perfect or perfective aspect and is commonly used
      to describe events that are completed or extend to the present—particularly events that occurred recently, hence the name;
\item the Preterite, which has an imperfective aspect and is used to describe events that are ongoing or habitual;
\item the Preterite Anterior, which functions as a pluperfect.
\end{items}

\subsection{Present Anterior and Preterite}\label{subsubsec:suffixed-tenses}
The present anterior and preterite are formed by appending a set of suffixes to the verb, which replace or coalesce with the active/passive
suffixes, see Tables~\ref{tab:pres-ant-combined}~and~\ref{tab:pret-combined} below.

\begin{table}[H]
\centering
\noindent\begin{tabular}{l|>{\it}l|>{\it}lll|>{\it}l|>{\it}lll|>{\it}l|>{\it}l}
\nf Suffix &\nf Sg&\nf Pl&& Active&\nf Sg&\nf Pl& & Passive&\nf Sg&\nf Pl\\\cline{1-3}\cline{5-7}\cline{9-11}
\s{1st}  & -\L é & -\L â      && \s{1st}  &j- -\L é     &aú-/r-/w- -\L â  &&\s{1st}  &v- -\L é    &aú-/r-/w- -\L â \\
\s{2nd}  & -\L á & -\L áḍ     && \s{2nd}  &ḍ(ẹ)- -\L á  &b’h(y)- -\L áḍ   &&\s{2nd}  &ḍ(ẹ)- -\L á &b’h(y)- -\L áḍ  \\
\s{3m}   & -\L á & -\L ér     && \s{3m}   &l(ẹ)- -\L á  &l(ẹ)- -\L ér     &&\s{3m}   &y’- -\L á   &lý- -\L ér      \\
\s{3f}   & -\L á & -\L ér     && \s{3f}   &ll(a)- -\L á &ll(ẹ)- -\L ér    &&\s{3f}   &y’- -\L á   &lý- -\L ér      \\
\s{3n}   & -\L á & -\L ér     && \s{3n}   &s- -\L á     &l(a)- -\L ér     &&\s{3n}   &sy- -\L á   &lý- -\L ér      \\\cline{1-3}\cline{5-7}\cline{9-11}
\s{inf}  & \MC{2}{c}{\it -á}  && \s{inf}  &\MC{2}{c}{\it d(ẹ)- -á} &&\s{inf}  &\MC{2}{c}{\it à-/h- -á}\\
\s{ptcp} & \MC{2}{c}{\it -ér} && \s{ptcp} &\MC{2}{c}{\it -êr}      &&\s{ptcp} &\MC{2}{c}{\it â- -ér}\\
\end{tabular}
\caption{Present Anterior Affixes.}\label{tab:pres-ant-combined}
\end{table}

\begin{table}[H]
\centering
\noindent\begin{tabular}{l|>{\it}l|>{\it}lll|>{\it}l|>{\it}lll|>{\it}l|>{\it}l}
\nf Suffix & \nf Sg&\nf Pl&& Active&\nf Sg&\nf Pl& & Passive&\nf Sg&\nf Pl\\\cline{1-3}\cline{5-7}\cline{9-11}
\s{1st}  & -\L á  & -y’aû     && \s{1st}  &j- -\L á     &aú-/r-/w- -\L y’aû  &&\s{1st}  &v- -\L é    &aú-/r-/w- -\L y’aû \\
\s{2nd}  & -\L é  & -y’ẹ́      && \s{2nd}  &ḍ(ẹ)- -\L é  &b’h(y)- -\L y’ẹ́     &&\s{2nd}  &ḍ(ẹ)- -\L é &b’h(y)- -\L y’ẹ́    \\
\s{3m}   & -\L é  & -\L é     && \s{3m}   &l(ẹ)- -\L é  &l(ẹ)- -\L é         &&\s{3m}   &y’- -\L é   &lý- -\L é          \\
\s{3f}   & -\L é  & -\L é     && \s{3f}   &ll(a)- -\L é &ll(ẹ)- -\L é        &&\s{3f}   &y’- -\L é   &lý- -\L é          \\
\s{3n}   & -\L é  & -\L é     && \s{3n}   &s- -\L é     &l(a)- -\L é         &&\s{3n}   &sy- -\L é   &lý- -\L é          \\\cline{1-3}\cline{5-7}\cline{9-11}
\s{inf}  & \MC{2}{c}{\it -é}  && \s{inf}  &\MC{2}{c}{\it d(ẹ)- -é}           &&\s{inf}  &\MC{2}{c}{\it à-/h- -é}\\
\s{ptcp} & \MC{2}{c}{\it -ár} && \s{ptcp} &\MC{2}{c}{\it -âr}                &&\s{ptcp} &\MC{2}{c}{\it â- -ár}\\
\end{tabular}
\caption{Preterite Affixes.}\label{tab:pret-combined}
\end{table}

\Paragraph{Lenition}
All \s{pres ant} and \s{pret} suffixes, except for the infinitive and \s{1pl, 2pl pret}, lenite any consonant \textit{before} them, e.g.
\w{ḅárḍáḍ} ‘to be willing’ to \w{jḅárḍát’hé} ‘I was willing’ but \w{dẹḅárḍáḍá} ‘to have been willing’.

\Paragraph{Coalescence}
In both tenses, the initial vowel of suffixes coalesces with any preceding vowel according to the following rules; note
that all of these except the first describe distinct cases.
\begin{items}
    \item First, if the preceding vowel is \w{ẹ}, it is simply deleted, e.g. \w{jrévôt’hẹ} ‘I return’ becomes \w{jrévôt’hé}
          ‘I returned’. This case takes precedence over all other cases.
    \item If either vowel is fully nasal, no coalescence occurs, e.g. \w{jvŷ} ‘I lead’ becomes \w{jvŷé} ‘I led’.
    \item If the preceding vowel is \w{è} or \w{ẹ́} and the suffix vowel is \w{é}, they merge into \w{ẹ́} or \w{ệ}.
    \item If both vowels have the same quality (and neither is fully nasal), they merge into a vowel with that
          quality, and a level of nasalisation is added,\footnote{All suffixes are either nasalised and nasal, so there
          can never be a case where we’d end up with two oral vowels coalescing here.} e.g.
          \w{jvvaúríhe} ‘I remember’ becomes \w{jvvaúríhé} ‘I remembered’.
    \item In any other case (i.e. if the vowels differ in quality), hiatus is maintained, e.g. \w{ní} ‘to deny’ becomes
          \w{âníér} ‘having been denied’.
\end{items}

\Paragraph{Multiple Affixes}
If a verb takes both and active and a passive person affix, the suffix aligns with the active affix; thus
\s{pres ant} ‘she loved me’ is \w{llavad’hórá}. Note that \w{llavád’hóré}, while also grammatical, is the
corresponding \s{pret} form instead since the \w{-é} indicates a \s{pret} in the \s{3f}.

\Paragraph{Diachrony}
Diachronically, the \s{1sg pret} is an interesting case; in EUF, it was originally *\w{-é}, but it later changed to \w{-á}
to distinguish it from the \s{2sg, 3sg pres ant}. The remaining forms—save the infinitives, which are derived from the
tenses’ definite endings by analogy—originated from the PF simple past tenses.

\Paragraph{Examples}
The table below lists the example paradigm of the verb \w{ad’hór} in the present anterior and preterite tenses.
Observe that there is no difference between the \s{1pl, 2pl} active and passive.

\begin{table}[H]
\centering
\noindent\begin{tabular}{l|>{\it}l|>{\it}lll|>{\it}l|>{\it}l}
\nf Active & \nf Sg   & \nf Pl     & \nf & \nf Passive & \nf Sg   & \nf Pl    \\\cline{1-3}\cline{5-7}
\s{1st}        & jad’hóré  & rad’hórâ     &     & \s{1st}         & vad’hóré  & rad’hórâ   \\
\s{2nd}        & ḍad’hórá  & b’had’hóráḍ  &     & \s{2nd}         & ḍad’hórá  & b’had’hóráḍ \\
\s{3m}         & lad’hórá  & lad’hórér    &     & \s{3m}       & y’ad’hórá & lýad’hórér  \\
\s{3f}         & llad’hórá & llad’hórér   &     & \s{3f}       & y’ad’hórá & lýad’hórér  \\
\s{3n}         & sad’hórá & lad’hórér    &     & \s{3n}       & syad’hórá & lýad’hórér  \\\cline{1-3}\cline{5-7}
\s{inf} & \multicolumn{2}{c}{\it dad’hórá} & & \s{inf} & \multicolumn{2}{c}{\it ád’hórá} \\
\s{ptcp}&\multicolumn{2}{c}{\it ad’hórêr}&&\s{ptcp}&\multicolumn{2}{c}{\it âd’hórér}\\
\end{tabular}
\caption{Present Anterior Paradigm of the Verb \emph{ad’hór}.}\label{tab:adhor-paradigm-pres-ant}
\end{table}

\begin{table}[H]
\centering
\noindent\begin{tabular}{l|>{\it}l|>{\it}lll|>{\it}l|>{\it}l}
\nf Active & \nf Sg   & \nf Pl     & \nf & \nf Passive & \nf Sg   & \nf Pl    \\\cline{1-3}\cline{5-7}
\s{1st}        & jad’hórá  & rad’hóry’aû   && \s{1st}     & vad’hórá  & rad’hóry’aû   \\
\s{2nd}        & ḍad’hóré  & b’had’hóry’ẹ́  && \s{2nd}      & ḍad’hóré  & b’had’hóry’ẹ́ \\
\s{3m}         & lad’hóré  & lad’hóré      && \s{3m}      & y’ad’hóré & lýad’hóré  \\
\s{3f}         & llad’hóré & llad’hóré     && \s{3f}      & y’ad’hóré & lýad’hóré  \\
\s{3n}         & sad’hóré & lad’hóré      && \s{3n}       & syad’hóré & lýad’hóré  \\\cline{1-3}\cline{5-7}
\s{inf} & \multicolumn{2}{c}{\it dad’hóré} & & \s{inf} & \multicolumn{2}{c}{\it ád’hóré} \\
\s{ptcp}&\multicolumn{2}{c}{\it ad’hórâr}&&\s{ptcp}&\multicolumn{2}{c}{\it âd’hórár}\\
\end{tabular}
\caption{Preterite Paradigm of the Verb \emph{ad’hór}.}\label{tab:adhor-paradigm-pret}
\end{table}

\subsection{Preterite Anterior}
The preterite anterior tense, sometimes also called the ‘pluperfect’, is used to describe events that happened before
another event in the past, e.g. \w{jyád’hórâr} ‘I had loved’; it is formed using coalesced forms of the preterite participle and the preterite
form of the verb \w{av́ár} ‘to have’.\footnote{Note that the modern preterite stem of \w{av́ár} is \w{y}.}
The following table illustrates the underlying construction using \w{ad’hór}, though it is worth noting that these
forms are not actually grammatical:

\begin{table}[H]
\centering
\noindent\begin{tabular}{l|>{\it}l|>{\it}lll|>{\it}l|>{\it}l}
Active&\nf Sg&\nf Pl& & Passive&\nf Sg&\nf Pl\\\cline{1-3}\cline{5-7}
\s{1st} &*jyá ad’hórâr  &*ryy’aû ad’hórâr   &&\s{1st} &*vyá âd’hórár  &*ryy’aû âd’hórár\\
\s{2nd} &*ḍyé ad’hórâr  &*b’hyy’ẹ́ ad’hórâr  &&\s{2nd} &*ḍyé âd’hórár  &*b’hyy’ẹ́ âd’hórár\\
\s{3m}  &*lyé ad’hórâr  &*lyé ad’hórâr      &&\s{3m}  &*y’yé âd’hórár &*lýÿé âd’hórár\\
\s{3f}  &*llyé ad’hórâr &*llyé ad’hórâr     &&\s{3f}  &*y’yé âd’hórár &*lýÿé âd’hórár\\
\s{3n}  &*syé ad’hórâr  &*lyé ad’hórâr      &&\s{3n}  &*syÿé âd’hórár &*lýÿé âd’hórár\\\cline{1-3}\cline{5-7}
\s{inf}&\multicolumn{2}{c}{\it *dyé ad’hórâr}&&\scshape inf&\multicolumn{2}{c}{\it *hyé âd’hórár}\\
\s{ptcp}&\multicolumn{2}{c}{\it *yâr ad’hórâr}&&\s{ptcp}&\multicolumn{2}{c}{\it *âyár âd’hórár}\\
\end{tabular}
\caption{Preterite Anterior Construction.}\label{tab:preterite-ant}
\end{table}

\noindent
Based on this underlying principle, the actual preterite anterior forms can be constructed using a series of coalescence
rules: first, if the participle starts with a consonant (which is only possible in the active as the passive will always have
the passive participle prefix \w{â-} prepended to it), or the form of \w{av́ár} ends with a consonant (which is only the case
in the participle) the two verbs forms are simply written as one word, e.g. \w{jyávvaúríhê} ‘I had remembered’.

Otherwise, we have a collision of two vowels. The first vowel of the participle is erased. If it was nasal(ised),
a \textit{single} level of nasalisation is added to the last vowel of the form of \w{av́ár}, then, the two forms are concatenated as
by the first rule, e.g. \w{ḍyéd’hórâr} ‘you had loved’, and \w{ḍyêd’hórár} ‘you had been loved’. Thus, the actual paradigm
of \w{ad’hór} in the preterite anterior is as shown in Table~\ref{tab:adhor-paradigm-pret-ant} below.

\begin{table}[H]
\centering
\noindent\begin{tabular}{l|>{\it}l|>{\it}lll|>{\it}l|>{\it}l}
Active&\nf Sg&\nf Pl& & Passive&\nf Sg&\nf Pl\\\cline{1-3}\cline{5-7}
\s{1st} &jyád’hórâr  &ryy’aûd’hórâr   &&\s{1st} &vyâd’hórár  &ryy’aûd’hórár\\
\s{2nd} &ḍyéd’hórâr  &b’hyy’ẹ́d’hórâr  &&\s{2nd} &ḍyêd’hórár  &b’hyy’ệd’hórár\\
\s{3m}  &lyéd’hórâr  &lyéd’hórâr      &&\s{3m}  &y’yêd’hórár &lýÿêd’hórár\\
\s{3f}  &llyéd’hórâr &llyéd’hórâr     &&\s{3f}  &y’yêd’hórár &lýÿêd’hórár\\
\s{3n}  &syéd’hórâr  &lyéd’hórâr      &&\s{3n}  &syÿêd’hórár &lýÿêd’hórár\\\cline{1-3}\cline{5-7}
\s{inf}&\multicolumn{2}{c}{\it dyéd’hórâr}&&\scshape inf&\multicolumn{2}{c}{\it hyêd’hórár}\\
\s{ptcp}&\multicolumn{2}{c}{\it yârad’hórâr}&&\s{ptcp}&\multicolumn{2}{c}{\it âyárâd’hórár}\\
\end{tabular}
\caption{Preterite Anterior Paradigm of \emph{ad’hór}.}\label{tab:adhor-paradigm-pret-ant}
\end{table}

\noindent
Note that the active participle is used with active prefixes and the passive participle with passive prefixes. If both
are present, either may be used, depending on the dialect; for example, the passive participle is preferred in
literary language, whereas the active participle is more common in speech.

The subjunctive and optative paradigms can be obtained using the same construction and follow the same coalescence
rules: first, construct the appropriate form of \w{av́ár}, and the perform the merging with the appropriate \textit{indicative}
participle, e.g. *\w{jèsá ad’hórâr} > \w{jèsád’hórâr} (roughly ‘I should have had loved’\footnote{This is another of
those forms that has no real equivalent in English and is fairly untranslatable.}).

Finally, as always, these forms are stressed on the last syllable of the stem of the actual verb; the coalesced form
of \w{av́ár} is unstressed.

\section{Future Tenses}
UF has two paradigms of future tenses: The Future I is a more modern construction and is only used in spoken informal
language. The Future II is an older, more literary tense that uses a separate stem, which is also used to form other future
tenses such as the Future Anterior and the Conditionals.

\subsection{Future I}\label{subsubsec:future-i}
The future tenses, i.e. the Future I and II, Future Anterior (a tense similar to the future perfect), as well
as the Conditional I and II, are formed by adding prefixes to the present forms. The prefix is the same in all persons and numbers,
except that there is a separate prefix for the infinitive.

In the Future, much to the UF learner’s dismay, this prefix can go in two separate positions: either before the person marker(s) or
inbetween the person marker(s) and the stem. The former case is more common in speech, while the later is more literary
and strongly preferred in writing and poetry as well as in formal speech. But even in informal speech, the Future I alone
will still not be enough to get by, as the Conditional, a \textit{very} common tense, is formed using the Future II.

First, let us examine the former, simpler case, commonly called the Future I. The prefix is \w{aú-} if the verb form
after it starts with a consonant (except glides), \w{aúr-} in all other cases; e.g. \w{aújad’hór} ‘I shall love’, but
\w{aúrý’ad’hór} ‘it will love’. In the infinitive passive, it
contracts with the initial \w{à-} or \w{á-} to \w{aú} or \w{aû}, e.g. \w{aûd’hór} ‘to be about to be loved’.\footnote{This too is
hard to translate literally.} No contraction happens
if the infinitive starts with \w{â}, e.g. \w{aúrânvé} ‘to be about to be animated’. Since
there is little point in writing a table for just the prefixes, Table~\ref{tab:adhor-paradigm-future-1} instead shows the Future I paradigm
of the verb \emph{ad’hór}.

\begin{table}[H]
\centering
\noindent\begin{tabular}{l|>{\it}l|>{\it}lll|>{\it}l|>{\it}l}
\nf Active&\nf Sg&\nf Pl&\nf &\nf Passive&\nf Sg&\nf Pl\\\cline{1-3}\cline{5-7}
\s{1st}&aújad’hór&aúrad’hóró   &&\s{1st} &aúvad’hór&aúrad’hór\\
\s{2nd}&aúḍad’hór&aúb’had’hóré &&\s{2nd} &aúḍad’hór&aúb’had’hór\\
\s{3m}&aúlad’hór&aúlad’hór     &&\s{3m} &aúry’ad’hór&aúlýad’hór\\
\s{3f}&aúllad’hór&aúllad’hór   &&\s{3f} &aúry’ad’hór &aúlýad’hór\\
\s{3n}&aúrý’ad’hór&aúlad’hór   &&\s{3n} &aúrý’ad’hór&aúlýad’hór\\\cline{1-3}\cline{5-7}
\s{inf}&\multicolumn{2}{c}{\it aúdad’hór}&&\scshape inf&\multicolumn{2}{c}{\it aûd’hór}\\
\s{ptcp}&\multicolumn{2}{c}{\it aúrad’hórâ}&&\s{ptcp}&\multicolumn{2}{c}{\it aúrâd’hór}\\
\end{tabular}
\caption{Future I Paradigm of the Verb \emph{ad’hór}.}\label{tab:adhor-paradigm-future-1}
\end{table}

\subsection{Future II}\label{subsubsec:future-ii}
The Future I paradigm is fairly straight-forward; unfortunately, the Future II is a lot worse: not only do the affixes
vary a lot more, but they are different depending on whether verb form following them starts with a vowel or a consonant.\footnote{This is
not a problem in the Future I, since the prefix is never adjacent to the stem.}
The vocalic and consonantal Future II affixes are shown in Tables~\ref{tab:future-2-vocalic}~and~\ref{tab:future-2-consonantal} below, respectively.

The diachrony of these forms is somewhat unclear—especially that of the participles. It would appear, however, that they result from a coalescence
of the personal pronouns with forms of some auxiliary (likely PF \pf{avoir} and \pf{aller}) as well as the PF future. It appears that
the \s{2sg} is derived from the formal PF \s{2pl} pronoun, which is in line with the fact that the Future II is generally
considered more formal than the almost colloquial Future I. The \w{v́} in the \s{2pl act} seems to be the result of metathesis.

\begin{table}[H]
\centering
\noindent\begin{tabular}{l|>{\it}l|>{\it}lll|>{\it}l|>{\it}l}
Active&\nf Sg&\nf Pl& & Passive&\nf Sg&\nf Pl\\\cline{1-3}\cline{5-7}
\s{1st}   &b’h- -(ẹ)  &náý’- -aú      &&\s{1st}    &v- -é    &náý’-     \\
\s{2nd}   &ḍír- -(ẹ)  &b’haý’- -(r)ẹ́  &&\s{2nd}    &ḍír-     &b’haý’-   \\
\s{3m} &ł-  -(ẹ)   &lb’h- -aú         &&\s{3m}   &l-       &lb’h- -(r)e \\
\s{3f} &èł-  -(ẹ)  &lb’h- -aú         &&\s{3f}   &l-       &lb’h- -(r)e \\
\s{3n} &aúł-  -(ẹ) &lb’h- -aú         &&\s{3n}   &s-       &lb’h- -(r)e \\\cline{1-3}\cline{5-7}
\s{inf}&\multicolumn{2}{c}{\it d- -è}&&\scshape inf&\multicolumn{2}{c}{\it h-}\\
\s{ptcp}&\multicolumn{2}{c}{\it -ŷr}&&\s{ptcp}&\multicolumn{2}{c}{\it á- -ýr}\\
\end{tabular}
\caption{Vocalic Future II Affixes.}\label{tab:future-2-vocalic}
\end{table}

\begin{table}[H]
\centering
\noindent\begin{tabular}{l|>{\it}l|>{\it}lll|>{\it}l|>{\it}l}
Active&\nf Sg&\nf Pl& & Passive&\nf Sg&\nf Pl\\\cline{1-3}\cline{5-7}
\s{1st}   &jaú- -ẹ́  &aúnraû- -aú &&\s{1st}   &vaú- -é  &naú-    \\
\s{2nd}   &b’há- -(ẹ) &v́aú- -e   &&\s{2nd}   &ḍá-  &b’haú-      \\
\s{3m} &aúr-  -(ẹ) &laú- -aú     &&\s{3m}  &y’aúr-  &laú- -(r)e \\
\s{3f} &aúr-  -(ẹ) &laú- -aú     &&\s{3f}  &y’aúr-  &laú- -(r)e \\
\s{3n} &aúr-  -(ẹ) &laú- -aú     &&\s{3n}  &saúr-   &laú- -(r)e \\\cline{1-3}\cline{5-7}
\s{inf}&\multicolumn{2}{c}{\it dẹ- -è}&&\scshape inf&\multicolumn{2}{c}{\it haú-}\\
\s{ptcp}&\multicolumn{2}{c}{\it -(r)ŷ}&&\s{ptcp}&\multicolumn{2}{c}{\it á- -(r)ý}\\
\end{tabular}
\caption{Consonantal Future II Affixes.}\label{tab:future-2-consonantal}
\end{table}

\Paragraph{Future Stem}
Many verbs have a different future stem that is used in all future tenses (except the Future I); for example, the future
stem of \w{vvaúríhe} ‘to remember’, is \w{vvaúríźe}; thus, we have
\w{jvvaúríhe} ‘to remember’ but \w{jaúvvaúríźẹ́} ‘I shall remember’.

Note also that these forms already include the
active/passive affixes, which is why it’s \w{jaúvvaúríźẹ́} and not *\w{jaújvvaúríźẹ́} or *\w{jjaúvvaúríźẹ́}.
As in the present, the dictionary form of the future
stem is a verbal noun; thus, \w{vvaúríźe} roughly means ‘the act of being about to remember’.\footnote{As noted before, infinitive
and gerund forms of future tenses are difficult to translate into English.}

The future \textit{subjunctive} uses a different stem; for that, see §~\ref{subsec:subjunctive}.

\Paragraph{Stem-final vowel elision and \w{-(ẹ)}}
The future stem usually ends with a vowel, which is dropped if any future suffix or a suffix that starts with a vowel is added, e.g.
\w{laúvvaúríźaú} ‘they will remember’, not *\w{laúvvaúríźeaú}. Note that in the case of future suffixes, even those that start
with a consonant cause the vowel to be dropped. The only exception to this is the suffix \w{-(ẹ)}, which is found in a number of
Future II forms; that suffix is dropped instead, e.g. \w{aúrvvaúríźe} ‘she will remember’, not *\w{aúrvvaúríźẹ}.

\Paragraph{Nasal Stems}
Some future stems are nasalising, which is the case if the final vowel is a nasal vowel; in such cases, that vowel
is still dropped if a suffix is added, but if that suffix starts with a vowel, nasalisation is applied to it, e.g.
in the case of \w{dír}, whose future stem is \w{dírẹ́}, we have \w{aúnraûdíraû} ‘we shall say’: the \w{-aú} suffix merges
with the nasalisation of the final vowel to become \w{aû}. Unlike with regular stems, the Future II \w{-(ẹ)} \textit{does}
replace the final vowel and becomes \w{-ẹ́} for such verbs, e.g. \w{aúrdírẹ́} ‘he will say’, and \s{1sg fut pass}
vocalic \w{-é} becomes \w{-ê}.

\Paragraph{\w{r-} Dropping}
Initial \w{r} in Future II suffixes is dropped if the
last consonant before the final vowel of the future stem is \w{w}, or an ʁ-coloured consonant such as \w{ź}, e.g.
\w{laúvvaúríźe} ‘they will be remembered’, not *\w{laúvvaúríźre}. If the last consonant of the future stem is \w{r}, since
any following vowel (whether nasalised or not) is deleted when a Future II suffix is added, the final \w{r} of the stem and
the initial \w{-r} of the Future II suffixes that have one coalesce to \w{rr}, e.g. \w{b’haý’ad’hórérre} ‘you (\s{pl}) will
love’.

\Paragraph{Affix Stacking}
Note that when more than one affix is used, at most one can be a future affix, e.g. \w{jaúsyvvaúríźẹ́} ‘I shall remember it’
and not *\w{jaúsaúrvvaúríźẹ́}. Generally, the active prefix will be the future affix, but it is possible to use the
passive future affixes instead for emphasis e.g. \w{jy’aúrvvaúríźe} roughly ‘him, I shall remember’; often, this is
also used to aid in establishing a contrast to some other part of the sentence that does not have this inversion.

Finally, as always, infinitive prefixes come first. If combined with other affixes, it will generally be the future affix,
e.g. \w{haújvvaúríźe} roughly ‘them to be about to be remembered by me’ but, as with passive affixes, variations are possible for emphasis
or contrastive power, e.g. \w{dẹjaúvvaúríźẹ́}, which puts more emphasis on ‘me’.

\Paragraph{Examples}
Table~\ref{tab:future-2-adhor} below shows the complete (vocalic) Future II paradigm of the verb \w{ad’hór} ‘to love’, and
Table~\ref{tab:future-2-vvaurihe} the complete (consonantal) Future II paradigm of II \w{vvaúríhe} ‘to remember’; recall
that the future stems of these verbs are \w{ad’hórérẹ́} and \w{vvaúríźe}.

\begin{table}[H]
\centering
\noindent\begin{tabular}{l|>{\it}l|>{\it}lll|>{\it}l|>{\it}l}
Active&\nf Sg&\nf Pl& & Passive&\nf Sg&\nf Pl\\\cline{1-3}\cline{5-7}
\s{1st}   &b’had’hórérẹ́  &náý’ad’hóréraû   &&\s{1st}    &vad’hórérệ    &náý’ad’hórérẹ́   \\
\s{2nd}   &ḍírad’hórérẹ́  &b’haý’ad’hórérrẹ́ &&\s{2nd}    &ḍírad’hórérẹ́  &b’haý’ad’hórérẹ́ \\
\s{3m} &ład’hórérẹ́    &lb’had’hóréraû      &&\s{3m}   &lad’hórérẹ́    &lb’had’hórérre  \\
\s{3f} &èład’hórérẹ́   &lb’had’hóréraû      &&\s{3f}   &lad’hórérẹ́    &lb’had’hórérre  \\
\s{3n} &aúład’hórérẹ́  &lb’had’hóréraû      &&\s{3n}   &sad’hórérẹ́    &lb’had’hórérre  \\\cline{1-3}\cline{5-7}
\s{inf}&\multicolumn{2}{c}{\it dad’hóréré}&&\scshape inf&\multicolumn{2}{c}{\it had’hórérẹ́}\\
\s{ptcp}&\multicolumn{2}{c}{\it ad’hórérŷr}&&\s{ptcp}&\multicolumn{2}{c}{\it ád’hórérýr}\\
\end{tabular}
\caption{Vocalic Future II Paradigm of \w{ad’hór}.}\label{tab:future-2-adhor}
\end{table}

\begin{table}[H]
\centering
\noindent\begin{tabular}{l|>{\it}l|>{\it}lll|>{\it}l|>{\it}l}
Active&\nf Sg&\nf Pl& & Passive&\nf Sg&\nf Pl\\\cline{1-3}\cline{5-7}
\s{1st}   &jaúvvaúríźẹ́  &aúnraûvvaúríźaú &&\s{1st}   &vaúvvaúríźé    &naúvvaúríźe \\
\s{2nd}   &b’hávvaúríźe &v́aúvvaúríźe     &&\s{2nd}   &ḍávvaúríźe  &b’haúvvaúríźe  \\
\s{3m} &aúrvvaúríźe  &laúvvaúríźaú       &&\s{3m} &y’aúrvvaúríźe  &laúvvaúríźe \\
\s{3f} &aúrvvaúríźe  &laúvvaúríźaú       &&\s{3f} &y’aúrvvaúríźe  &laúvvaúríźe \\
\s{3n} &aúrvvaúríźe  &laúvvaúríźaú       &&\s{3n} &saúrvvaúríźe   &laúvvaúríźe \\\cline{1-3}\cline{5-7}
\s{inf}&\multicolumn{2}{c}{\it dẹvvaúríźè}&&\scshape inf&\multicolumn{2}{c}{\it haúvvaúríźe}\\
\s{ptcp}&\multicolumn{2}{c}{\it vvaúríźŷ}&&\scshape ptcp&\multicolumn{2}{c}{\it ávvaúríźý}\\
\end{tabular}
\caption{Consonantal Future II Paradigm of \w{vvaúríhe}.}\label{tab:future-2-vvaurihe}
\end{table}

\subsection{Future Anterior}
The Future Anterior tense is formed by combining the Future II and the Present Anterior affixes. The \s{pres ant} suffixes
are applied after the \s{fut ii} affixes. The vocalic and consonantal affixes are shown in
Tables~\ref{tab:future-anterior-vocalic}~and~\ref{tab:future-anterior-consonantal}.

\begin{table}[H]
\centering
\noindent\begin{tabular}{l|>{\it}l|>{\it}lll|>{\it}l|>{\it}l}
Active&\nf Sg&\nf Pl& & Passive&\nf Sg&\nf Pl\\\cline{1-3}\cline{5-7}
\s{1st}   &b’h- -\L é  &náý’- -aúrâ    &&\s{1st}    &v- -\L ê    &náý’- -\L â    \\
\s{2nd}   &ḍír- -\L á  &b’haý’- -(r)ệḍ &&\s{2nd}    &ḍír- -\L á  &b’haý’- -\L áḍ \\
\s{3m} &ł-   -\L á  &lb’h- -aûr        &&\s{3m}  &l- -\L á    &lb’h- -(r)ér   \\
\s{3f} &èł-  -\L á  &lb’h- -aûr        &&\s{3f}  &l- -\L á    &lb’h- -(r)ér   \\
\s{3n} &aúł- -\L á  &lb’h- -aûr        &&\s{3n}  &s- -\L á    &lb’h- -(r)ér   \\\cline{1-3}\cline{5-7}
\s{inf}&\multicolumn{2}{c}{\it d- -á}&&\scshape inf&\multicolumn{2}{c}{\it h- -á}\\
\s{ptcp}&\multicolumn{2}{c}{\it -ŷrér}&&\s{ptcp}&\multicolumn{2}{c}{\it á- -ýrér}\\
\end{tabular}
\caption{Vocalic Future Anterior Affixes.}\label{tab:future-anterior-vocalic}
\end{table}

\begin{table}[H]
\centering
\noindent\begin{tabular}{l|>{\it}l|>{\it}lll|>{\it}l|>{\it}l}
Active&\nf Sg&\nf Pl& & Passive&\nf Sg&\nf Pl\\\cline{1-3}\cline{5-7}
\s{1st}   &jaú-  -\L ệ  &aúnraû- -aúrâ  &&\s{1st}   &vaú- -\L ê    &naú- -\L â  \\
\s{2nd}   &b’há- -\L á  &v́aú- -éḍ       &&\s{2nd}   &ḍá- -\L á  &b’haú- -\L áḍ  \\
\s{3m} &aúr-  -\L á  &laú- -aûr         &&\s{3m} &y’aúr- -\L á  &laú- -(r)ér \\
\s{3f} &aúr-  -\L á  &laú- -aûr         &&\s{3f} &y’aúr- -\L á  &laú- -(r)ér \\
\s{3n} &aúr-  -\L á  &laú- -aûr         &&\s{3n} &saúr-  -\L á  &laú- -(r)ér \\\cline{1-3}\cline{5-7}
\s{inf}&\multicolumn{2}{c}{\it dẹ- -á}&&\scshape inf&\multicolumn{2}{c}{\it haú- -á}\\
\s{ptcp}&\multicolumn{2}{c}{\it -(r)ŷr}&&\s{ptcp}&\multicolumn{2}{c}{\it á- -(r)ýr}\\
\end{tabular}
\caption{Consonantal Future Anterior Affixes.}\label{tab:future-anterior-consonantal}
\end{table}

\noindent
Note that again, nasalised stems add another level of nasalisation, and vowel-dropping still applies, but
this time, there is no \w{-ẹ} dropping, since none of the affixes end with \w{ẹ} anymore.

\Paragraph{Coalescence}
All vowel suffixes coalesce with the final vowel of the stem; if the suffix vowel is nasal, a level of nasalisation is
added, e.g. \w{aúrvvaúrízá} ‘he will have remembered’ from the future stem \w{vvaúríźe}. Note also that the \w{ź} is lenited
to \w{z}; the quality of the suffix vowel overrides that of the stem vowel. \w{r} contraction still happens as in the
Future II.

Tables~\ref{tab:future-ant-adhor}~and~\ref{tab:future-ant-vvaurihe} below show
the paradigm of the verbs \w{ad’hór} ‘to love’ and \w{vvaúríhe} ‘to remember’ in the Future Anterior tense. Note that
both the rules for the Future Anterior tense as well as the Present Anterior tense apply here.

\begin{table}[H]
\centering
\noindent\begin{tabular}{l|>{\it}l|>{\it}lll|>{\it}l|>{\it}l}
Active&\nf Sg&\nf Pl& & Passive&\nf Sg&\nf Pl\\\cline{1-3}\cline{5-7}
\s{1st}   &b’had’hórérệ  &náý’ad’hóréraûrâ  &&\s{1st}    &vad’hórérệ    &náý’ad’hórérậ    \\
\s{2nd}   &ḍírad’hórérậ  &b’haý’ad’hórérrệḍ &&\s{2nd}    &ḍírad’hórérậ  &b’haý’ad’hórérậḍ \\
\s{3m} &ład’hórérậ    &lb’had’hóréraûr      &&\s{3m}  &lad’hórérậ    &lb’had’hórérrér  \\
\s{3f} &èład’hórérậ   &lb’had’hóréraûr      &&\s{3f}  &lad’hórérậ    &lb’had’hórérrér  \\
\s{3n} &aúład’hórérậ  &lb’had’hóréraûr      &&\s{3n}  &sad’hórérậ    &lb’had’hórérrér  \\\cline{1-3}\cline{5-7}
\s{inf}&\multicolumn{2}{c}{\it dad’hórérâ}&&\scshape inf&\multicolumn{2}{c}{\it had’hórérậ}\\
\s{ptcp}&\multicolumn{2}{c}{\it ad’hórérŷrér}&&\s{ptcp}&\multicolumn{2}{c}{\it ád’hórérýrér}\\
\end{tabular}
\caption{Vocalic Future Anterior Paradigm of \w{ad’hór}.}\label{tab:future-ant-adhor}
\end{table}

\begin{table}[H]
\centering
\noindent\begin{tabular}{l|>{\it}l|>{\it}lll|>{\it}l|>{\it}l}
Active&\nf Sg&\nf Pl& & Passive&\nf Sg&\nf Pl\\\cline{1-3}\cline{5-7}
\s{1st} &jaúvvaúrízệ  &aúnraûvvaúríźaúrâ &&\s{1st} &vaúvvaúrízê   &naúvvaúrízâ    \\
\s{2nd} &b’hávvaúrízá &v́aúvvaúríźéḍ      &&\s{2nd} &ḍávvaúrízá    &b’haúvvaúrízáḍ \\
\s{3m}  &aúrvvaúrízá  &laúvvaúríźaûr     &&\s{3m}  &y’aúrvvaúrízá &laúvvaúríźér   \\
\s{3f}  &aúrvvaúrízá  &laúvvaúríźaûr     &&\s{3f}  &y’aúrvvaúrízá &laúvvaúríźér   \\
\s{3n}  &aúrvvaúrízá  &laúvvaúríźaûr     &&\s{3n}  &saúrvvaúrízá  &laúvvaúríźér \\\cline{1-3}\cline{5-7}
\s{inf}&\multicolumn{2}{c}{\it dẹvvaúríźá}&&\scshape inf&\multicolumn{2}{c}{\it haúvvaúríźá}\\
\s{ptcp}&\multicolumn{2}{c}{\it vvaúríźŷr}&&\scshape ptcp&\multicolumn{2}{c}{\it ávvaúríźýr}\\
\end{tabular}
\caption{Consonantal Future Anterior Paradigm of \w{vvaúríhe}.}\label{tab:future-ant-vvaurihe}
\end{table}

\subsection{Conditional I and II}\label{subsubsec:conditional}
The Conditional tenses are fairly simple—so long as you know the Future II and Future Anterior, that is. Both Conditionals
are formed by adding the \w{-ss(a)-} infix between the Future II stem and any suffixes.

\begin{table}[H]
\tabcolsep4pt
\centering
\noindent\begin{tabular}{l|>{\it}l|>{\it}lll|>{\it}l|>{\it}l}
Active&\nf Sg&\nf Pl& & Passive&\nf Sg&\nf Pl\\\cline{1-3}\cline{5-7}
\s{1st}   &jaúvvaúríźessệ  &aúnraûvvaúríźessaúrâ &&\s{1st}   &vaúvvaúríźessê    &naúvvaúríźessâ   \\
\s{2nd}   &b’hávvaúríźessá &v́aúvvaúríźesséḍ      &&\s{2nd}   &ḍávvaúríźessá  &b’haúvvaúríźessáḍ   \\
\s{3m} &aúrvvaúríźessá  &laúvvaúríźessaûr        &&\s{3m}  &y’aúrvvaúríźessá  &laúvvaúríźessrér \\
\s{3f} &aúrvvaúríźessá  &laúvvaúríźessaûr        &&\s{3f}  &y’aúrvvaúríźessá  &laúvvaúríźessrér \\
\s{3n} &aúrvvaúríźessá  &laúvvaúríźessaûr        &&\s{3n}  &saúrvvaúríźessá   &laúvvaúríźessrér \\\cline{1-3}\cline{5-7}
\s{inf}&\multicolumn{2}{c}{\it dẹvvaúríźessá}&& \s{inf}&\multicolumn{2}{c}{\it haúvvaúríźesse}\\
\s{ptcp}&\multicolumn{2}{c}{\it vvaúríźessŷr}&&\s{ptcp}&\multicolumn{2}{c}{\it ávvaúríźessý}\\
\end{tabular}
\caption{Consonantal Conditional II Paradigm of \w{vvaúríhe}.}\label{tab:cond-ii-vvaurihe}
\end{table}

\noindent The Conditional I is formed from
the Future II, and the Conditional II from the Future Anterior. The \w{a} in \w{-ss(a)-} is omitted if
the suffix after the infix starts with a vowel, except for \w{ẹ}, which it replaces. Table~\ref{tab:cond-ii-vvaurihe}
shows the consonantal Conditional II paradigm of \w{vvaúríhe} ‘to remember’. Note that the \w{ss} in this form
is \textit{never} lenited.

The conditional tenses are mainly used in the apodoses of conditional clauses. On their own, their meaning
is similar to that of the English ‘would’ (I) or ‘could’ (II), e.g. \w{jaúvvaúríźessẹ́} ‘I would remember’. The Conditional I
can be combined with the gnomic to express a general observation of someone’s disposition, e.g.
\w{laúsynárrahódejússaub’he’sý’ýâ} ‘they wouldn’t narrate it to you (implied: because they just don’t do things like that)’.

The Conditional I can also be used to express a future-in-the-past, and the Conditional II, even though it is
morphologically a future tense, is used to express a hypothetical past, e.g. \w{jaúvvaúríźessệ} ‘I could have loved’.
In reported speech, this can lead to a subjunctive conditional construction.


\section{Subjunctive}\label{subsec:subjunctive}
The UF subjunctive forms are fortunately fairly simple: they use the same affixes as the present, past, and future
forms, except that each verb has a different, often irregular, subjunctive stem, which is generally formed
by adding an \w{-s} to the end of the corresponding indicative stem, e.g. \w{ad’hór} ‘to love’ to \w{ad’hórs};
thus we have, e.g. \w{jad’hórs} ‘I may love’, and \w{rád’hórsó} ‘We may love’.

The future subjunctive stem is always regular and formed by adding the desinence \w{-śe} to the end of the future stem. For example,
the future stem of \w{ad’hór} is \w{ad’hórérẹ́}, so the future subjunctive stem is \w{ad’hórérẹ́śe}; similarly, the future
stem of \w{vvaúríhe} is \w{vvaúríźe}, so the future subjunctive stem is \w{vvaúríźeśe}. The subjunctive stem coalesces like
a regular non-nasal future stem.

There are several main uses of the UF subjunctive, each of which we shall examine in more detail below:
\begin{enum}
\item in reported speech, e.g. \w{lladírá vad’hórhé} ‘she said she loved me’;
\item with certain subordinating conjunctions, such as \w{b’he} ‘so that’;
\item to express deontic modality, e.g. \w{ḍẹḅars} ‘you may leave’;
\item as a jussive, e.g. \w{rad’hesó} ‘let’s go’;
\item as a negative imperative, e.g. \w{sá ḍẹḅars} ‘don’t leave’;
\item irrealis conditionals (see §~\ref{subsec:conditionals});
\item in a serial verb construction in the future, expressing purpose;
\item in \s{aci}s and \s{pci}s.
\end{enum}

\subsection{Reported Speech}
UF does not use backshifting in reported speech, but rather, the corresponding subjunctive form is used. For instance,
\w{jḍad’hór} ‘I love you’ becomes \w{jdíré jḍad’hórs} ‘I said I love you’. Note that the tense stays the same in this
example: present indicative becomes present subjunctive. Accordingly, \w{jḍad’hóré} ‘I loved you’ becomes \w{jdíré
jḍad’hórsé} ‘I said I loved you’.

Consequently, the tense of the verb in reported speech is independent of the tense of the matrix clause, e.g.
\w{b’had’hrệ} ‘I shall go’ becomes \w{jdíré b’had’hrẹ́sé} ‘I said I would go’,\footnote{Note the lenition here because
of the present anterior suffix: \w{b’had’hrẹ́sé}, not *\w{b’had’hrẹ́śé}.} with \w{b’had’hrẹ́sé} being the Future II
subjunctive form of \w{b’had’hrẹ́}.

\subsection{Dependent Clauses}
The following subordinating conjunctions take the subjunctive:
\TwoCols[.45\hsize][.45\hsize] {
\begin{dlist}[\bfseries\itshape]
    \item[áhaúr] ‘even though’
    \item[ḅas] ‘because’
    \item[b’he] ‘so that’
    \item[c’haúr] ‘as’ (viz. ‘because’)
    \item[de] ‘once’
    \item[ráhẹ] ‘though’
\end{dlist}
} {
\begin{dlist}[\bfseries\itshape]
    \item[rê] ‘although’
    \item[s] ‘if’ (see §~\ref{subsec:conditionals})
    \item[sá] ‘without’
    \item[sauc’h] ‘except that’
    \item[váłé] ‘despite that’
\end{dlist}
}

\noindent Note that not all subordinating conjunctions take the subjunctive. For instance, the conjunction \w{y’is}
‘because’ takes the indicative: \w{jḍad’hórs c’haúr} ‘as I love you’, but \w{jḍad’hór y’ís} ‘because I love you’.

\subsection{Deontic Modality}
The subjunctive can also be used on its own, in which case it assumes a deontic or jussive meaning;
in the first person, it is generally a jussive, e.g. \w{rad’hesó} ‘let’s go’, but the jussive sense is not restricted
to the first person, e.g. \w{lẹsyrét’hes} ‘he take care of it’ (in the sense of ‘let him take care of it’).

The deontic sense is also apparent from that last example: \w{lẹsyrét’hes} can also be interpreted to mean ‘he
may take care of it’, which can either be a statement of permission or a condescending order. Note that even
though UF also has a word for ‘let’ (namely \w{le}), it is mostly used in questions or commands, while the
deontic subjunctive is used to grant permission.

\subsection{Negation}\label{subsubsec:negated-subjunctive}
The subjunctive is negated with the particle \w{sá}, rather than with \w{asý’ýâ}. The particle \w{sá} is placed
immediately before the verb form it negates, e.g. \w{sá jḍad’hórs c’haúr} ‘as I don’t love you’. It is reduced
to \w{s’} before vowels, but interestingly, it does not cause nasalisation in that case, e.g. \w{s’aúsydíssâ c’haúr}
‘as we didn’t say it’.

On its own, the negated subjunctive is used to express a negative imperative in the second and third person,
e.g. \w{sá ḍẹḅars} ‘don’t leave’, and a negative jussive in the first person e.g. \w{sá rad’hesó}, ‘let’s not go’.

\subsection{Infinitive}
Most curiously, UF has a \textit{subjunctive infinitive}. This form is almost exclusively used to express deontic modality
in \s{aci}s and \s{pci}s. For example, the form \w{dad’hórs}, the subjunctive infinitive of \w{ad’hór}, while defying any attempt
at translation on its own,\footnote{The best attempt one could make to translate this would be something along the
lines of ‘to should love’, but that is not exactly grammatical in English.} can be translated as ‘should’ when combined
with an \s{acc} or \s{part}, e.g. \w{sráhó dad’hórs} roughly means ‘that fish should love’, though this form can only
occur as the complement of a verb.

\subsection{Future Subjunctive of Intent or Purpose}
The future subjunctive is used in a serial verb construction with another finite verb to express purpose or intent: a
serial verb construction is a clause with two finite verbs; in this case, one combines any finite verb with
a finite subjunctive Future II, e.g. \w{jsyc’hrír jaúvvaúríźeśẹ́} ‘I’m writing
it down so I don’t forget’; the two needn’t agree in person, and word order, as ever with inflected forms, is not fixed,
e.g. \w{náý’aúréśaú sybźâ} ‘It was needed
for us to understand’.\footnote{Lit. ‘I write it [down]; I should-will remember’ and ‘It was needed, so that we
should-will understand’, respectively.}

The main semantic difference between this construction and \w{b’he} is that the latter strictly means ‘in order to’ or ‘so that’,
whereas this can be a bit broader in meaning; however, the future subjunctive of intent is also sometimes used to mean ‘in order to’
or ‘so that’.

\section{Optative}\label{subsec:optative}
The UF optative is used to express wishes, hopes, as well as in certain conditional constructions. It is formed
by prefixing \w{y’(ẹ)\L} to the indicative (or future) stem,\footnote{The use of the (future) subjunctive stem to form the optative, with
no change in overall meaning, is fairly archaic and only encountered in poetry in modern UF.} e.g. \w{dẹvy’ẹvvaúríhe} ‘may
you remember me’. As ever, the \w{(ẹ)} is omitted if the stem starts with a vowel.

In the future, this generally does \textit{not} change whether the consonantal or vocalic affixes are used! If the stem was
vocalic, the vocalic affixes are also used in the optative. This is because the optative is conceptually appended to the prefix rather than
prepended to the stem. Moreover, some prefixes in the future end with
\w{ý’}, which this is dropped in the optative: e.g. \w{náý’ad’hóraú} ‘we shall love’ becomes \w{náy’ad’hóraú} ‘may we love’ (the
difference is minor: \w{ý’} vs \w{y’}). A bare optative is difficult to translate into English; a more precise explanation of what
these forms actually mean will be given below. Uses of the optative include:

\begin{enum}
\item wishes, hopes, dreams, and aspirations;
\item with certain subordinating conjunctions, such as \w{auha} ‘in case’;
\item talking about fears;
\item counterfactual conditionals (see §~\ref{subsec:conditionals}).
\end{enum}

\subsection{Wishes and Hopes}
The most traditional use of the optative is to express wishes and hopes, e.g. \w{dẹvy’ẹvvaúríhe} ‘may you remember me’. In
the present or future tense, this use indicates a wish for something to happen; in the present tense, its meaning is
that of a wish for a condition to be true in the present in the face of uncertainty or lack of knowledge; thus, the
actual meaning of \w{dẹvy’ẹvvaúríhe} is roughly ‘I hope that you remember me’.\footnote{The context of this could be e.g.
meeting someone again after a long time apart and hoping that they still remember you.} In the future tense, it indicates a wish
that a situation will be true in the future, e.g. \w{b’hávy’ẹvvaúríźe} ‘may you remember me’.

In the past tenses, the optative indicates dismay, regret, or disappointment that something did not happen, e.g.
\s{pres ant} \w{dẹvy’ẹvvaúríhá} ‘if only you had remembered me’. The optative can also be combined with the Conditional I
to convey uncertainty about a future wish, as well as with the Conditional II to express extreme regret over a past event;
certain verbs, e.g. \w{ub’hrá} ‘can, may, might’, also have constructions with the optative.

\subsection{Dependent Clauses}
The following subordinating conjunctions take the optative:
\TwoCols[.45\hsize][.45\hsize] {
\begin{dlist}[\bfseries\itshape]
    \item[auha] ‘in case’
    \item[ab’há] ‘before’
    \item[ávrê] ‘unless’
    \item[ḅré] ‘after’
\end{dlist}
}{
\begin{dlist}[\bfseries\itshape]
    \item[fahaú] ‘in such a way that’
    \item[jys] ‘until’
    \item[sit’há] ‘supposing that’
    \item[úrbh] ‘provided that’
\end{dlist}
}

\subsection{Negation and Verbs of Fearing}\label{subsubsec:negated-optative}
As with the negated subjunctive, the negated optative also has a separate negation particle, namely \w{t’hé\N{}}
(\w{t’h’} with no nasalisation before vowels). Note that a negated optative indicates that the speaker wishes that something does
or had not happened, e.g. \w{t’hé dẹvy’ẹvvaúríhá} ‘if only you had not remembered me’. The negation thus negates
the wish, and not the act of wishing; for the latter, the indicative or subjunctive together with a verb such
as \w{sḅé} ‘to wish’ are used instead.

Verbs of fearing are typically construed with a dependent clause in the negated optative, e.g. \w{jréd’hé
t’hé b’háy’ẹbharẹ́} ‘I was afraid lest you might leave’.

\section{The Copula \textit{eḍ}}\label{subsec:ed-paradigm}
There is only one irregular verb in UF, namely the copula \w{eḍ}. All of its forms are highly irregular. The copula lacks
passive forms, a gnomic, as well as the Future I.  The preterite anterior is a periphrastic construction of the preterite participle
of \w{eḍ} and its present tense,\footnote{The original morphological preterite anterior tense of \textit{eḍ} was lost in
Late Middle UF.} e.g. \w{t’hẹdâ vy’í} ‘I had been’. Note that only the participle is
inflected for mood in this case, e.g. subjunctive \w{t’hẹrâ vy’í} ‘I should have been’.
%\footnote{At the same time, the \s{1sg} forms seem to be derived from the \s{acc} of the PF \s{1sg} pronoun,
%for unknown reasons.}


\begin{table}[H]
\centering
\let\M\multicolumn
\noindent\begin{tabular}{l|>{\it}l|>{\it}l|>{\it}l|>{\it}l|>{\it}l|>{\it}l|>{\it}l|>{\it}l|>{\it}l|>{\it}l}
&\M{2}{c|}{Present}&\M{2}{c|}{Pres. Ant.}&\M{2}{c|}{Preterite}&\M{2}{c|}{Future II}&\M{2}{c}{Fut. Ant.}\\\cline{2-11}
\s{ind} &\nf Sg&\nf Pl  & \nf Sg &\nf Pl   & \nf Sg &\nf Pl & \nf Sg &\nf Pl & \nf Sg &\nf Pl \\\hline
\s{1st} & vy’í  & aúsó   & vẹ     & aúfý   & vet’h  & weḍy’ó   & vẹ́hér  & aúhér   & vẹhér    & aúfêr \\
\s{2nd} & ḍe    & b’heḍ  & ḍyf    & b’hu   & ḍet’h  & b’heḍy’é & dyhér  & b’hehér & ḍyfér    & b’huhér \\
\s{3m}  & le    & lẹsó   & leb’h  & lẹfýr  & let’h  & let’he   & lehér  & lẹhér   & leb’hér  & lẹfêr \\
\s{3f}  & lle   & llẹsó  & lleb’h & llẹfýr & llet’h & llet’he  & llehér & llẹhér  & lleb’hér & llẹfêr \\
\s{3n}  & se    & lasó   & seb’h  & lafýr  & set’h  & laet’h   & sehér  & lahér   & seb’hér  & lafêr \\\hline
\s{inf}& \M{2}{c|}{\it éḍ} &\M{2}{c|}{\it éfyḍ} & \M{2}{c|}{\it ét’hẹd} & \M{2}{c|}{\it éhér} & \M{2}{c}{\it éfér} \\
\s{ptcp}& \M{2}{c|}{\it ḍâ} &\M{2}{c|}{\it fyḍâ} & \M{2}{c|}{\it t’hẹdâ} & \M{2}{c|}{\it hérâ} & \M{2}{c}{\it férâ} \\
\end{tabular}
\caption{Indicative Paradigm of \emph{eḍ}.}\label{tab:ed-paradigm-ind}
\end{table}

\begin{table}[H]
\centering
\let\M\multicolumn
\noindent\begin{tabular}{l|>{\it}l|>{\it}l|>{\it}l|>{\it}l|>{\it}l|>{\it}l|>{\it}l|>{\it}l|>{\it}l|>{\it}l}
&\M{2}{c|}{Present}&\M{2}{c|}{Pres. Ant.}&\M{2}{c|}{Preterite}&\M{2}{c|}{Future II}&\M{2}{c}{Fut. Ant.}\\\cline{2-11}
\s{subj} &\nf Sg&\nf Pl  & \nf Sg &\nf Pl   & \nf Sg &\nf Pl & \nf Sg &\nf Pl & \nf Sg &\nf Pl \\\hline
\s{1st} & vy’íra  & aúra   & vẹsa   & aúfýs  & veḍra  & weḍra      & vẹ́héra  & aúhéra   & vẹhéra    & aúfêra \\
\s{2nd} & ḍera    & b’hera & ḍys    & b’hus  & ḍeḍra  & b’heḍra    & dyhéra  & b’hehéra & ḍyféra    & b’huhéra \\
\s{3m}  & lera    & lẹra   & les    & lẹfýs  & leḍra  & le’thra    & lehéra  & lẹhéra   & leb’héra  & lẹfêra \\
\s{3f}  & llera   & llẹra  & lles   & llẹfýs & lleḍra & llet’hra   & llehéra & llẹhéra  & lleb’héra & llẹfêra \\
\s{3n}  & sera    & lara   & ses    & lafýs  & seḍra  & laet’hra   & sehéra  & lahéra   & seb’héra  & lafêra \\\hline
\s{inf}& \M{2}{c|}{\it éḍra} &\M{2}{c|}{\it éfysa} & \M{2}{c|}{\it ét’hẹra} & \M{2}{c|}{\it éhéra} & \M{2}{c}{\it éféra} \\
\s{ptcp}& \M{2}{c|}{\it ḍerâ} &\M{2}{c|}{\it fysâ} & \M{2}{c|}{\it t’hẹrâ} & \M{2}{c|}{\it hérarâ} & \M{2}{c}{\it férarâ} \\
\end{tabular}
\caption{Subjunctive Paradigm of \emph{eḍ}.}\label{tab:ed-paradigm-subj}
\end{table}

\begin{table}[H]
\centering
\let\M\multicolumn
\noindent\begin{tabular}{l|>{\it}l|>{\it}l|>{\it}l|>{\it}l|>{\it}l|>{\it}l|>{\it}l|>{\it}l|>{\it}l|>{\it}l}
&\M{2}{c|}{Present}&\M{2}{c|}{Pres. Ant.}&\M{2}{c|}{Preterite}&\M{2}{c|}{Future II}&\M{2}{c}{Fut. Ant.}\\\cline{2-11}
\s{opt} &\nf Sg&\nf Pl  & \nf Sg &\nf Pl   & \nf Sg &\nf Pl & \nf Sg &\nf Pl & \nf Sg &\nf Pl \\\hline
\s{1st} & víra      & aúry’a   & vẹsy’a    & aúfýy’a  & veḍraä  & weḍraä      & vẹ́ra  & aúra   & vẹra     & aúfrá \\
\s{2nd} & ḍy’era    & b’hery’a & ḍysy’a    & b’huy’a  & ḍeḍraä  & b’heḍraä    & dyra  & b’hera & ḍyra     & b’hura \\
\s{3m}  & ly’era    & lẹry’a   & lesy’a    & lẹfýy’a  & leḍraä  & le’thraä    & lera  & lẹra   & leb’hra  & lẹfrá \\
\s{3f}  & lly’era   & llẹry’a  & llesy’a   & llẹfýy’a & lleḍraä & llet’hraä   & lléra & llẹra  & lleb’hra & llẹfrá \\
\s{3n}  & sy’era    & lary’a   & sesy’a    & lafýy’a  & seḍraä  & laet’hraä   & sera  & lara   & seb’hra  & lafrá \\\hline
\s{inf}& \M{2}{c|}{\it éḍy’a} &\M{2}{c|}{\it éfyy’a} & \M{2}{c|}{\it ét’hẹä} & \M{2}{c|}{\it éhérá} & \M{2}{c}{\it éférá} \\
\s{ptcp}& \M{2}{c|}{\it ḍy’â} &\M{2}{c|}{\it fyy’â} & \M{2}{c|}{\it t’hẹáâ} & \M{2}{c|}{\it héráâ} & \M{2}{c}{\it féráâ} \\
\end{tabular}
\caption{Optative Paradigm of \emph{eḍ}.}\label{tab:ed-paradigm-opt}
\end{table}

\noindent All forms of the
copula are shown in Tables~\ref{tab:ed-paradigm-ind}–\ref{tab:ed-paradigm-opt}, except for the Conditional I and II,
which are formed by infixing \w{-ss-} before the \w{-ér}, \w{-êr} desinences and \w{-ssa-} before the \w{-ra} and \w{-rá} desinences of the
Future II and Future Anterior forms, respectively.

Unlike nearly every other word in the language, disyllabic forms of the copula are stressed on
the first syllable, and trisyllabic forms are stressed on the second syllable—except for \w{hérarâ}, \w{férarâ}, \w{héráâ},
and \w{féráâ}, which are stressed on the first syllable. All other participle forms are stressed on the last syllable.
In forms of the copula, \w{ae} is pronounced /ai̯/.

The etymology of these forms is mostly from a gradual simplification of coalesced forms of the personal
pronouns with the PF copula. To compensate for the fact that PF lacks certain forms that are present in UF, some
of the forms were coined by analogy. For instance, the \s{pres ant inf} \w{éfyḍ} is derived from the \s{pres ant}
stem *\w{fy} and the \s{pres inf} \w{éḍ}, and the same is true for the \s{pret inf} \w{ét’hẹd}.

\section{Summary of Coalescence Rules}
When vowels collide at morpheme boundaries, though chiefly in suffixes, they often coalesce into a
single vowel that depends on the qualities and nasality of the two vowels. How exactly this coalescence
works depends on the morphemes in question, but generally, there are 4 overarching principles to be
aware of:
\begin{enumerate}
\item Vowels at the end of a suffix or at the beginning of a prefix may simply be omitted instead;
      this is particularly common in verb forms.
\item If one of the vowels is \w{ẹ}, it is dropped; the resulting vowel is the other vowel.
\item If one of the vowels is nasalised, the resulting vowel is generally also nasalised; if both
      vowels are nasalised or nasal, the resulting vowel will be nasal.
\item If the first vowel is part of a verb stem, it is often simply deleted or at least overridden
      by the second vowel in terms of quality.
\end{enumerate}

\noindent The following table lists all coalescence rules in the language. For more information, see the corresponding
sections in which the forms in question are introduced. Note that trivial cases of vowels being dropped entirely
are not listed in this table.

Unless otherwise indicated, vowel letters, e.g. \w{o}, represent any variant of that vowel, whether oral, nasalised, or nasal.
\w{o} also includes variants of \w{au} (e.g. \w{aú}, but not \w{áu}, of course, as those are two vowels and not a digraph).
Subscripts may be used to track nasalisation, and a + sign indicates a level of nasalisation is added. Since glides also
influence contractions in some cases, they are included in this table. In the case of the abbreviations V and G, if there
is no +, nasalisation is preserved.

Rules are matched top-down: the first matching rule is applied, all others are ignored. The affix in the 2nd column next to the
vowel(s) in the 3rd column in the same row are replaced with the letters of the 4th column in the indicated forms at the
indicated morpheme boundary in the 1st column. The position in the hyphen in the 2nd column indicates whether it coalesces
with vowels before or after it. The letters V (as well as V′) stands for ‘any vowel’. The letter G stands for
‘any glide’. Abbreviations may be used where applicable (e.g. \w{-e(r)} for \w{-e} and \w{-er} if there is no single \w{-e(r)}
suffix in the paradigm in question). The abbreviation -? means ‘any other suffix in this paradigm, even if they
start with a consonant’.

\medskip
{\centering
\noindent
\begin{longtable}{@{}>{\scshape}l|>{\it}p{5em}|>{\it}l|>{\it}l|l}
\nf Form    & \nf Affix & \nf Phonemes & \nf Result & Reference \\\hline\hline
pres 1pl          & aú-     & o\Sub α  & wo\Sub α & §~\ref{subsubsec:active-passive-affixes} \\
                  &         &\nf V & r\nf V \\\hline
pres 1pl act      & -y’ó    & o\Sub α  & o\Sub{α+} &ibid. \\\hline
pres 2pl          & b’h(y)- &\nf V & b’h\nf V &ibid. \\
                  &         &\nf G & b’h\nf G \\\hline
pres 2pl act      & -y’é    & e\Sub α  & e\Sub{α+} &ibid.\\\hline
pres inf pass     & à-      & a\Sub α& a\Sub {α+}&ibid. \\
                  &         &\nf V & h\nf V \\\hline
pres part         & -â, â-  & ẹ & â &ibid. \\
                  &         &\nf V̂ & {\nf V̂}â, â{\nf V̂} \\
                  &         &\nf V &\nf V̂ \\\hline\hline
pres ant, pret    & -é(r)   & ẹ & é(r) &§~\ref{subsubsec:suffixed-tenses}\\
                  &         &e\Sub α & e\Sub{α+}(r) \\
                  & -á(r), -áḍ  & ẹ & á(r), áḍ &ibid. \\
                  &         &a\Sub α & a\Sub{α+}(r), a\Sub{α+}ḍ \\\hline
pres ant 1pl act  & -â      &à, a, á & â & ibid.\\\hline\hline
fut i inf pass    & aú(r)-  & â & aúrâ & §~\ref{subsubsec:future-i} \\
                  &         &\nf G & aúr\kern1pt\nf G \\
                  &         & à & aú \\
                  &         & á & aû \\\hline
fut ii, fut ant,  & -ẹ      &\nf V &\nf V &§§~\ref{subsubsec:future-ii}–\ref{subsubsec:conditional}\\
cond i, cond ii   &{\nf -V}\Sub α\nf & ẹ & {\nf -V}\Sub α\nf \\
                  &         & {\nf V′}\Sub{β} & {\nf -V}\Sub{α+β} \\
                  &\nf -?   &{\nf stem} ẹ, ẹ́ &\nf -? \\
gnomic            & j(ú)-   & {\nf V}\Sub α& j\kern1pt{\nf V}\Sub{α+} &§~\ref{subsubsec:gnomic} \\\hline
\end{longtable}\par}
\medskip

\noindent Lastly, note that the 4 principles mentioned earlier are guidelines, not rules. There are cases of affixes that do not coalesce
at all, e.g. the comparative prefix \w{lẹ} (see §~\ref{subsec:comparison}). If a form is not listed in this table, then,
unless explicitly stated where that form is introduced (in which case case we simply forgot to include it in the table),
it does not coalesce at all. Furthermore, this table only handles coalescence rules between vowels and some vowel elision rules; other
elision rules are either very regular or have nothing to do with adjacent vowels. This table exists only because coalescence rules are
very similar, but sometimes subtly different.

\chapter{Syntax}\label{sec:syntax}
UF syntax is unfortunately complicated in what morphological constructs are used in what situations, and
the rules are not always clear. The following is a list of the most common constructions.

\section{Noun Phrases}
\subsection{Names and Titles}\label{subsubsec:names-and-titles}
Proper nouns are declined in the definite form only. Unlike other nouns, the nominative is almost always
unmarked, e.g. \w{Daúvníc’h} ‘Dominic’, not *\w{Ládaúvníc’h}. Titles always follow the noun they qualify,
e.g. \w{Snet’h C’haúfrér} ‘Brother Smyth’. Titles attached to names also do not receive any marking in the
nominative, and are declined instead of the name in other cases, e.g. the accusative of ‘Brother Smyth’
would be \w{Snet’h Ihaúfrér}, not *\w{Ihnet’h Ihaúfrér} or *\w{Ihnet’h C’haúfrér}.

If two titles are attached to the same (proper) noun, both are declined, e.g. the accusative of
‘Brother Smyth the Wise’ \w{Snet’h Ihaúfrér Ihaj}.

\section{Independent Clauses}
The UF independent clause typically consists of a finite verb together with a subject perhaps several
objects. The verb is conjugated to agree with the subject in person, number, and gender in some cases.

\gloss {
    Rab’hadó iárb.
    r-ab’haḍ-ó| i-árb
    \s{1pl}-fell-\s{1pl}| \s{acc}-tree
    ‘We are felling the tree.’
}

The unmarked tense in UF is the present tense, which can generally be translated as either a present or
present continuous tense in English. For general truths and facts, the gnomic tense is generally used
instead.

\gloss {
    Rab’hadjô sárb.
    r-ab’haḍ-jô| s-árb
    \s{1pl}-fell-\s{gnomic}\Sl\s{1pl}| \s{acc.indef}-tree
    ‘We fell trees.’
}

The object is incorporated into the verb if it is a personal pronoun, in which case there are rules for
the order in which these affixes occur (see Section~\ref{subsec:verbal-morphology}).

\gloss {
    Lerab’hat’há.
    lẹ-r-ab’ha\Sl t’há.
    \s{3sgm}-\s{1pl.pass}-fell\Sl\s{3sg.pres.ant}
    ‘He felled us.’
}

Word order is rather lax due to the presence of case marking, and any constituent can be fronted for
emphasis, but the default word order is SVO or SOV.

\gloss {
    B’hehýnác aúlýab’hat’hâ.
    b’hehýn-ác| aú-lý-ab’ha\Sl t’hâ.
    \s{instr.indef}-axe|\s{1pl}-\s{3pl.pass}-fell\Sl\s{1pl.pres.ant}
    ‘With an axe, we have felled them.’
}

Note that words belonging to the same phrase are typically juxtaposed as adjectives are not inflected. However,
this rule may sometimes be broken, particularly in poetry. Consider, for example, the following passage in alexandrine
metre, written by the renowned poet \s{J.\,Y.\,B.\,Snet’h}, where we can find the verb positioned between a possessive
pronoun and its associated noun:

\gloss {
    Au lýr náý’acḍaúrâ sýec̣ asvaúr sýárb.
    Au |lýr |náý’-acḍ-aúrâ|sý-ec̣|as-vaúr|sý-árb
    And|their|\s{1pl.fut.ant}-cleave-\s{circ}|\s{acc.pl}-sin|\s{dat}-world|\s{acc}-tree
    ‘And we shall indeed have revealed their sins to the world’\footnotemark
}

\footnotetext{See the dictionary entry for \w{act’he}, sense 4, for more information about the use of this word here,
which normally means ‘cleave’. The literal meaning of this sentence is roughly: ‘And we shall have brought
down the trees upon their sins, to (= for the benefit of) the world’.}


\section{Negated Clauses}
Negation in the indicative is expressed using the particle \w{asý’ýâ} ‘not’, which is typically appended to verbs
as \w{’sý’ýâ}. For a discussion of negation in the subjunctive, optative, and \s{aci}s/\s{pci}s see
Sections~\ref{subsec:subjunctive},~\ref{subsec:optative},~and~\ref{subsec:aci-pci}.
By default, the particle is placed right after the verb:
\gloss {
    Aúlýab’hat’hâ’sý’ýâ b’hehýnác.
    aú-lý-ab’ha\Sl t’hâ|’sý’ýâ|b’hehýn-ác.
    \s{1pl}-\s{3pl.pass}-fell\Sl\s{1pl.pres.ant}|not|\s{instr.indef}-axe
    ‘We have not felled them with an axe.’
}

In case of a fronted constituent in an independent clause (but not in dependent clauses), the particle is
placed after that constituent:
\gloss {
    B’hehýnác asý’ýâ aúlýab’hat’hâ.
    b’hehýn-ác|asý’ýâ|aú-lý-ab’ha\Sl t’hâ.
    \s{instr.indef}-axe|not|\s{1pl}-\s{3pl.pass}-fell\Sl\s{1pl.pres.ant}
    ‘It is not with an axe that we have felled them.’
}

Note that it is not valid to both front a constituent and not move the negation. For example,
the following sentence sounds very awkward and no UF speaker would ever say or write this,
save perhaps to sound extremely ironic.
\gloss {
    \#B’hehýnác aúlýab’hat’hâ’sý’ýâ.
    b’hehýn-ác|aú-lý-ab’ha\Sl t’hâ|’sý’ýâ.
    \s{instr.indef}-axe|\s{1pl}-\s{3pl.pass}-fell\Sl\s{1pl.pres.ant}|not
    \textit{Roughly:} ‘With an axe, we have not-felled them.’
}

UF makes frequent use of double negation in conjunction with words that create a negative context
such as \w{jávé} ‘never’, \w{y’ê} ‘nothing’, or \w{ráv́â} ‘seldom’. Typically, such words are frontend,
and consequently, the negation particle then appears appended to them, e.g.:
\gloss {
    Ráv́â’sý’ýâ st’halẹ jac̣t'heá.
    Ráv́â|’sý’ýâ |s\Sl t’halẹ |j-ac̣t'he- á
    seldom|not |\s{acc.indef}\Sl table|\s{1sg}-buy-\s{3sg.pres.ant}
    ‘Rarely have I ever bought a table.’
}

Note that double negation is required for the sentence to make sense; UF learners often forget
about that, which can lead to rather awkward constructs such as:
\gloss {
    \#Ráv́â st’halẹ jac̣t'heá.
    Ráv́â|s\Sl t’halẹ |j-ac̣t'he-á
    seldom|\s{acc.indef}\Sl table|\s{1sg}-buy-\s{3sg.pres.ant}
    \textit{Roughly:} ‘I rarely-bought a table.’
}

Still, if a fronted constituent is present, the negation particle is placed after that constituent:
\gloss {
    St’halẹ’sý’ýâ ráv́â  jac̣t'heá.
    s\Sl t’halẹ|’sý’ýâ | ráv́â |j-ac̣t'he-á
    \s{acc.indef}\Sl table|not| seldom |\s{1sg}-buy-\s{3sg.pres.ant}
    ‘A table I have bought rarely.’
}

Foreigners often make the mistake of assuming that the negation particle is part of a word,
e.g. that \w{ráv́â’sý’ýâ} means ‘seldom’. As such, UF speakers, when imitating a foreigner, may
sometimes use more than one negation particle in a single sentence. Note that this is very
much not proper language; such constructions are summarily comedic and best compared to phrases
such as ‘it do be like that’ in English:
\gloss {
    *Ráv́â’sý’ýâ st’halẹ jac̣t'heá’sý’ýâ
    Ráv́â|’sý’ýâ |s\Sl t’halẹ |j-ac̣t'he-á|’sý’ýâ
    seldom|not |\s{acc.indef}\Sl table|\s{1sg}-buy-\s{3sg.pres.ant}|not
    \textit{Roughly:} ‘Rarely-not I bought a table.’
}

\section{Interrogative Clauses}
In UF, questions are generally marked by one or more particles. Unlike in many other languages, the verb generally
does not move, except perhaps for emphasis. The most fundamental kind of question is a yes-no question, which is
marked by the interrogative particle \w{c’hes}. The particle typically occurs in second position in the sentence (that
is, after the first \textit{constituent}, not after the first word):
\gloss {
    St’halẹ c’hes jac̣t'heá?
    s\Sl t’halẹ |c’hes |j-ac̣t'he-á
    \s{acc.indef}\Sl table|\s{q}|\s{1sg}-buy-\s{3sg.pres.ant}
    ‘Did I buy a table?’
}

The exception to this is with forms of \w{eḍ} ‘to be’, which are typically immediately preceded by
the question particle, the two forming a single word, placed at the very end of the sentence:
\gloss {
    Raúl baú c’hesse?
    raúl|baú|c’hes|se
    \s{abs}-language|good|\s{q}|\s{3n}.be
    ‘Is it a good language?’
}

Negation is placed in the usual position. A negated question is marked by the negation particle \w{sý’ýâ},
and the expected answer is ‘yes’:
\gloss {
    St’halẹ c’hes jac̣t'heá’sý’ýâ?
    s\Sl t’halẹ |c’hes |j-ac̣t'he-á|’sý’ýâ
    \s{acc.indef}\Sl table|\s{q}|\s{1sg}-buy-\s{3sg.pres.ant}|not
    ‘Did I not buy a table?’
}

Alternatively, the particle \w{(r)vá} can be used to indicate that the speaker expects the answer to be ‘no’
or to indicate disbelief, surprise, or amazement. Note that this particle \textit{replaces} the question particle.
Attempting to use both particles in the same sentence is ungrammatical and will likely be interpreted as
stuttering.
\gloss {
    St’halẹvá jac̣t'heá?
    s\Sl t’halẹ |vá |j-ac̣t'he-á
    \s{acc.indef}\Sl table|\s{q}|\s{1sg}-buy-\s{3sg.pres.ant}
    ‘I bought a table?’
}

Unlike \w{c’hes}, this particle remains there even if the verb is \w{eḍ} ‘to be’:
\gloss {
    Raúlvá baú se?
    raúl|vá|baú|se
    \s{abs}-language|\s{q}|good|\s{3n}.be
    ‘It is a good language?’
}

Of course, these questions can also be negated:
\gloss {
    St’halẹvá jac̣t'heá’sý’ýâ?
    s\Sl t’halẹ |vá |j-ac̣t'he-á|’sý’ýâ
    \s{acc.indef}\Sl table|\s{q}|\s{1sg}-buy-\s{3sg.pres.ant}|not
    ‘I didn’t buy a table?’
}

The precise meaning of these questions is as follows: In \w{St’halẹ c’hes jac̣t'heá?} (‘Did I buy a table?’),
the speaker is asking whether they themselves bought a table; a plausible situation would be that they
simply forgot whether they did. Its negation, \w{St’halẹ c’hes jac̣t'heá’sý’ýâ?} (‘Did I not buy a table?’),
could be used if the speaker is sure they bought a table sometime ago, but they can’t seem to find it and
are starting to doubt themselves (‘Did I not buy a table? I’m sure I did.’).

By contrast, the question \w{St’halẹvá jac̣t'heá?}) would be an assertion of disbelief; maybe the speaker
found a table in their loft, and they can’t seem to remember buying it, but the price tag is still there.
Finally, its negation \w{St’halẹvá jac̣t'heá’sý’ýâ?} would most likely be the speaker expressing their frustration
over the fact that they can’t seem to find their table and asserting that, in fact, they know for sure that
they did indeed buy a table (‘Did I not buy a table? I know I did!’).

Fronting of the verb in the last two cases generally indicates confusion rather than amazement or anger and
is most commonly used in response to someone else’s statement so as to ask for clarification (‘What do you mean
“I bought a table”; what are you talking about?’).
\gloss {
    Jac̣t'heává st’halẹ?
    j-ac̣t'he-á |vá|s\Sl t’halẹ
    \s{1sg}-buy-\s{3sg.pres.ant}|\s{q}|\s{acc.indef}\Sl table
    ‘I \textit{bought} a \textit{table}?!’
}

The same applies to the negated version of such a question:
\gloss {
    Jac̣t'heá’sý’ýâvá st’halẹ?
    j-ac̣t'he-á |’sý’ýâ|vá|s\Sl t’halẹ
    \s{1sg}-buy-\s{3sg.pres.ant}|not|\s{q}|\s{acc.indef}\Sl table
    ‘I \textit{didn’t} buy a \textit{table}?!’
}

Note the order of particles: negation precedes the question particle. Placing them the other way around
makes it sound like you’re trying to correct yourself from \w{Jac̣t'hévá} to \w{Jac̣t'hé’sý’ýâ}.

\section{Subordination and Coordination}
Subordinate dependent clauses clauses as well as coordinate independent clauses are typically introduced by conjunctions
or sentential particles; usually, these particles are placed in second position in the clause (i.e. after some other
constituent), e.g. observe the position of \w{vé} ‘but’ in the following sentence:
\gloss {
    St’halẹ vé jad’hór.
    s\Sl t’halẹ|vé|j-ad’hór
    \s{acc.indef}\Sl table|but|\s{1sg}-love
    ‘But I love a table.’
}

Note that particles may sometimes also be placed after the first word of a clause instead if the first constituent
is too long; this is particularly common with conjunctions that come in pairs such as \w{u \ldots\ u \ldots} ‘\ldots\
or \ldots (inclusive)’:
\gloss {
    U vé st’halẹ u sárb jad’hór.
    u|vé|s\Sl t’halẹ|u|s-árb|j-ad’hór
    or|but|\s{acc.indef}\Sl table|or|\s{acc.indef}-tree|\s{1sg}-love
    ‘But I love a table or a tree.’
}

\section{ACI and PCI}\label{subsec:aci-pci}
The term \s{aci} is Latin for \textit{accūsātīvus cum īnfīnītīvō} ‘accusative with infinitive’. As the name would suggest, this
grammatical construction consists of a dependent clause formed by an \s{acc} noun together with an infinitive; the
noun is the subject or object of the clause, and the infinitive the predicate. This construction is most well-known
from Classical languages such as Latin or Ancient Greek, but it is also found in various other languages, including
English and, of course, UF:
\gloss {
    Lácár sbhaú àfér láȷ́éd’há.
    lá\Sl c̣ár| s\Sl bhaú| à-fér| l-áȷ́éd’h\Sl á
    \s{nom}\Sl Charles|\s{acc.indef}\Sl bridge| \s{inf.pass}-build|\s{3m}-order\Sl \s{pres.ant}
    ‘Charles ordered a bridge to be built.’
}

In this sentence, the matrix clause is \w{Lácár láȷ́éd’há} ‘Charles ordered’, and the dependent clause is formed by
the \s{aci} \w{sbhaú àfér} ‘a bridge to be built’. Since ‘a bridge’ is the object in this case, the passive infinitive
is used. Observe how this sentence’s translation also uses an \s{aci} with a passive infinitive in both English (‘Charles
ordered a bridge to be built’) as well as Latin (\textit{Carolus pontem fierī iussit}).

UF does not have a word for ‘that’ as in ‘I think that \ldots’ or ‘I know that \ldots’; instead, it uses
\s{aci}s in these cases. Just how multiple ‘that’ clauses can be chained in English, so can multiple \s{aci}s in UF.
\gloss {
    Icár sbhaú àfér dáȷ́édá jsav́á.
    i\Sl c̣ár| s\Sl bhaú| à-fér| d-áȷ́éd-á |j-sav́á
    \s{acc}\Sl Charles|\s{acc.indef}\Sl bridge| \s{inf.pass}-build|\s{inf}-order-\s{pres.ant}|\s{1sg}-know
    ‘I know that Charles ordered a bridge to be built.’
}

Whenever a word is marked as taking an \s{aci} in the dictionary, it may also take a \s{pci} instead if
that makes sense semantically; there are no words that syntactically may take an \s{aci}, but not a \s{pci}.
Finally, note that ‘that’ is not always expressed with an \s{aci} or \s{pci}. Certain verbs, e.g. verbs of fearing, may
take a dependent clause in the subjunctive or optative instead (see §§~\ref{subsec:subjunctive},~\ref{subsec:optative}).

\subsection{Nested ACIs}
When multiple \s{aci}s are chained together, they are nested such that \s{acc} comes first and the infinitive
last or vice versa, and any nested \s{aci}s are placed inbetween; observe that, in the sentence above, the \s{aci}
\w{sbhaú àfér} ‘a bridge to be built’ is nested inside \w{Icár dáȷ́édá} ‘Charles to have ordered’. The literal
translation of this sentence would thus be ‘I know Charles to have ordered a bridge to be built’.

Furthermore, note that the finite verb of the matrix clause of an \s{aci} receives only a subject marker if the
\s{aci} is the object and vice versa. Thus, we have \w{jsav́á} ‘I know’ in the example above instead of e.g.
\w{jsysav́á} ‘I know it’. It \textit{would} be possible to add the object marker in the example above, but it would
sound a bit strange, roughly ‘I know it: that Charles ordered a bridge to be built’, and the verb would
probably have to be fronted for the sentence to make sense that way.

The exception, of course, is if the matrix clause is in the passive, in which case, as ever, the passive affix
is added regardless, seeing as the verb would not be finite otherwise, e.g. \w{sysav́á} ‘it is known that’.

\subsection{PCIs}
In addition to \s{aci}s, UF also has \s{pci}s, which use the \s{part} case instead. The \s{part} generally indicates
that an action is incomplete (see §~\ref{subsubsec:declension}), and thus \s{pci}s can be used to express something similar;
for instance:
\gloss {
    Lácár dŷnbaú àfér láȷ́éd’há.
    lá\Sl c̣ár| dŷn-ḅaú| à-fér| l-áȷ́éd’h\Sl á
    \s{nom}\Sl Charles|\s{part.indef}-bridge| \s{inf.pass}-build|\s{3m}-order\Sl \s{pres.ant}
    ‘Charles ordered to start building a bridge.’
}

The translation of the sentence above isn’t the best, but we start to run into a problem here, since UF uses
\s{aci}s and \s{pci}s much more prolifically than English does. A somewhat literal translation of this sentence would be
something along the lines of ‘Charles ordered the building of a bridge to be started’, but it isn’t perfect
either since ‘building’ is a gerund but in the sentence above, it’s an infinitive. In modern English, there simply
is no good literal translation for this sentence that preserves the passive infinitive.

\subsection{Resolving Ambiguity}
When dealing with \s{aci}s and \s{pci}s that involve verbs that also take \s{acc} and \s{part} arguments, respectively, or
other infinitives which do, one must be careful not to construct garden-path sentences. For instance, take \w{\textbf{sḅáłýr}
sýc̣ahý dýbháhẹ dylí \textbf{dub’hrá}}. Here, the \s{pci} is marked in bold, and the intended meaning is ‘for speakers to be
able to read each other’s thoughts’. Unfortunately, however, ‘read’ also takes a \s{part} here, and thus, it is
possible to construct a different \s{pci}, namely \w{\textbf{sḅáłýr} sýc̣ahý dýbháhẹ \textbf{dylí} dub’hrá} ‘for speakers
to read each other’s thoughts’, and \w{dub’hrá} ‘to be able to’ is awkwardly left hanging at the end of the sentence.

To fix this problem, rearrange the sentence so the infinitive of the \s{aci} or \s{pci} is placed first and put the verbs of any
enclosed verb phrases first in those phrases to indicate that any immediately following \s{acc} or \s{part} nouns are part
of that verb rather than of the \s{aci} or \s{pci}: \w{\textbf{dub’hrá} dylí sýc̣ahý dýbháhẹ \textbf{sḅáłýr}}.
This rule is sometimes intentionally subverted in cases where the double meaning is desirable, or in poetry, where word order
is a lot looser, but it would be very awkward to do so in prose.

In speech, this problem is more readily solved via intonation by placing emphasis and separating the ‘contents’ of the
\s{aci} or \s{pci} from the infinitive and noun with short pauses, e.g. \w{\textbf{sḅáłýr} {\nf ‖} sýc̣ahý dýbháhẹ dylí
{\nf ‖} \textbf{dub’hrá}}.

\subsection{Negation}
Negation of \s{aci}s and \s{pci}s uses the same particle as negation in the optative, viz. \w{t’hé} (see
Section~\ref{subsubsec:negated-optative}), attached to the verb of the \s{aci}. For example:
\gloss {
    Lácár sbhaú t’h’àfér láȷ́éd’há.
    lá\Sl c̣ár| s\Sl bhaú| t’h’-à-fér| l-áȷ́éd’h\Sl á
    \s{nom}\Sl Charles|\s{acc.indef}\Sl bridge| \s{inf.pass}-build|\s{3m}-order\Sl \s{pres.ant}
    ‘Charles ordered that no bridge be built.’
}

Note that this only applies to negating verb of the \s{aci} itself—the
verb of the matrix clause is negated normally. Where the meaning of the two is equivalent, negating
the verb of the \s{aci} is generally preferred.
\gloss {
    Lácár sbhaú àfér láȷ́éd’há’sý’ýâ.
    lá\Sl c̣ár| s\Sl bhaú| à-fér| l-áȷ́éd’h\Sl á|’sý’ýâ
    \s{nom}\Sl Charles|\s{acc.indef}\Sl bridge| \s{inf.pass}-build|\s{3m}-order\Sl \s{pres.ant}|not
    ‘Charles did not order a bridge to be built.’
}

\subsection{Pronominal ACIs and PCIs}\label{subsubsec:pronominal-aci}
One of the most counterintuitive constructions in UF is the pronominal \s{aci}, i.e an \s{aci} that is formed using
an infinitive and a pronoun. However, since separate pronouns do not exist in the \s{acc} or \s{part} case (see
§~\ref{subsubsec:personal-pronouns}), passive affixes are used instead, even if the form is intended to be active
in meaning:
\gloss {
    Lácár delýbard láȷ́éd’há.
    lá\Sl c̣ár| dẹ-lý-ḅarḍ| l-áȷ́éd’h\Sl á
    \s{nom}\Sl Charles|\s{inf}-\s{3pl.pass}-leave|\s{3m}-order\Sl \s{pres.ant}
    ‘Charles ordered them to leave.’
}

If the meaning of the sentence is intended to be passive, the passive infinitive is used instead. This is one of
the only cases where a verb can receive two markers of the same voice:
\gloss {
    Lácár àsyfér láȷ́éd’há.
    lá\Sl c̣ár| à-sy-fér| l-áȷ́éd’h\Sl á
    \s{nom}\Sl Charles|\s{inf.pass}-\s{3sg.pass}-build|\s{3m}-order\Sl \s{pres.ant}
    ‘Charles ordered it to be built.’
}

Thus, the voice of a pronominal \s{aci} or \s{pci} depends on the voice of the infinitive marker, and not that of
the finite marker. Finally, a pronominal \s{pci} is formed as expected, i.e. with the pronominal partitive infix
\w{-dy-} (see §~\ref{subsubsec:personal-pronouns})
\gloss {
    Lácár delýdybard láȷ́éd’há.
    lá\Sl c̣ár| dẹ-lý-dy-ḅarḍ| l-áȷ́éd’h\Sl á
    \s{nom}\Sl Charles|\s{inf}-\s{3pl.pass}-\s{part}-leave|\s{3m}-order\Sl \s{pres.ant}
    ‘Charles ordered them to get going.’
}

\section{Conditionals}\label{subsec:conditionals}
UF conditionals can broadly be divided into four categories: Simple (\s s), potential (\s p), counterfactual (\s c), and
irrealis (\s i). In the examples below, the letter in brackets indicates the type of conditional.

Unlike other languages, UF does not use any form of backshifting. Thus, a past tense is used in a conditional sentence if and
only if the action, from the speaker’s perspective, takes place in the past. Even counterfactual conditionals, if they
appertain to the present, still use present tense. There are, however, other restrictions on tense in that not all kinds of
conditionals appertain to all tenses. For instance, it is impossible to construct a potential conditional in the past tense—that
would have to be a counterfactual conditional instead.

\subsection{Simple Conditionals}
Simple conditionals indicate basic implications and logical truths. These conditionals use the indicative in both the protasis
and apodosis, in the appropriate tense. The protasis is generally introduced by the particle \w{s} ‘if’.
\gloss {
    S {\bf r} sré, aû-{\bf r} sfe. \nf (\s s)
    s|{\nf r}| s-ré| aû-|{\nf r}| s-fẹ
    if|\w{r}|\s{3n}-be.true|non-|\w{r}|\s{3n}-be.false
    ‘If \w{r} is true, then not-\w{r} is false.’\footnotemark
}

\footnotetext{UF does not use the letters \w{p} or \w{q}, and thus, discussions of propositional logic in UF tend to
use \w{r} and \w{t} instead. \w{s} is not used either so as to not confuse it with \w{s} ‘if’.}

\subsection{Potential Conditionals}
Potential conditionals indicate that something is possible or could happen in the present or future (but \textit{not} in
the past), provided some condition is met, but which is not currently the case. These conditionals use the present
indicative (\s p1) or the present (spoken) or Future II (literary) optative in the future (\s p2) in the protasis, and
the Conditional I in the apodosis.
\gloss {
    S desehúrvé, aúrrzaúsdressa júrdy’í. \nf (\s p1)
    s|ḍẹ-sẹhúr-vé|aúr-rzaúsḍre-ss\Sl a|júrdy’í
    if|\s{2sg}-help-\s{dat.1st}|\s{3n.fut.ii₁}-be.complete–\s{cond}\Sl \s{circ₁}|today
    ‘If you were to help me, it could be finished today.’
}

\gloss {
    S vê dey’ehehúrvé, aúrrzaúsdressa abrdvê. \nf (\s p2)
    s|vê|ḍẹ-y’ẹ\Sl hẹhúr-vé|aúr-rzaúsḍre-ss\Sl a|aḅrdvê
    if|tomorrow|\s{2sg}-\s{opt}\Sl help-\s{dat.1st}|\s{3n.fut.ii₁}-be.complete–\s{cond}\Sl \s{circ₁}|day.after.tomorrow
    ‘If you were to help me tomorrow, it could be finished the day after tomorrow.’
}

This sentence indicates that the speaker believes that, if the addressee helps them, there is a \textit{possibility} that they
could finish the task. If, by contrast, the speaker is certain that they will get the task done, a simple conditional is used
instead:
\gloss {
    S desehúrvé, aúrrzaúsdre júrdy’í. \nf (\s s)
    s|ḍẹ-sẹhúr-vé|aúr-rzaúsḍr\Sl e|júrdy’í
    if|\s{2sg}-help-\s{dat.1st}|\s{3n.fut.ii₁}-be.complete\Sl \s{circ₁}|today
    ‘If you help me, it will (with certainty) be finished today.’
}

\subsection{Counterfactual Conditionals}
Counterfactual conditionals are conditionals whose protasis is false. These conditionals exist only in the present
and past and use the subjunctive in the present or any past tense in the protasis, and the Conditional II in the apodosis:
\gloss {
    S desehúsvé, aúrrzaúsdressá. \nf (\s c)
    s|ḍẹ-sẹhús-vé|aúr-rzaúsḍre-ss\Sl á|
    if|\s{2sg}-help.\s{subj}-\s{dat.1st}|\s{3n.fut.ant₁}-be.complete–\s{cond}\Sl \s{circ₁}
    ‘If you were helping me, it would be finished.’
}

\gloss {
    S desehúhávé, aúrrzaúsdressá y’ér. \nf (\s c)
    s|ḍẹ-sẹhúh\Sl á-vé|aúr-rzaúsḍre-ss\Sl á|y’ér
    if|\s{2sg}-help.\s{subj}\Sl \s{3rd.pres.ant}-\s{dat.1st}|\s{3n.fut.ant₁}-be.complete–\s{cond}\Sl \s{circ₁}|yesterday
    ‘If you had helped me, it would have been finished yesterday.’
}

\subsection{Irrealis Conditionals}
Irrealis conditionals are conditionals that describe a situation that could never be true. They are distinct from
potential conditionals in that they cannot possibly happen, and from counterfactuals in that the apodosis is not ‘false’,
either because it is not a statement, but rather a wish etc. (\s i1), or because it hasn’t happened yet (\s i2). This also means that
irrealis conditionals are constrained to the present and future tense and are chiefly used to describe something that
the speaker knows won’t happen. In a sense, they are often the opposite of potential conditionals. They use the
optative in the protasis and the subjunctive in the apodosis.
\gloss {
    S dey’ehehúrvé, srzaúsdhá y’ér! \nf (\s i1)
    s|ḍẹ-y’ẹ\Sl hẹhúr-vé|s-rzaúsḍh\Sl á|y’ér
    if|\s{2sg}-\s{opt}\Sl help-\s{dat.1st}|\s{3n}-be.complete.\s{subj}\Sl \s{pres.ant.3sg}|yesterday
    ‘If only you were helping me—it would have been finished yesterday!’
}

{\tabcolsep.5\tabcolsep\gloss {
    S vê b’háy’ehehúrrevé, aúr-rzaúsdre-śe abrdvê. \nf (\s i2)
    s|vê|b’há-y’ẹ\Sl hẹhúrre-vé|aúr-rzaúsḍre-śe|aḅrdvê
    if|tomorrow|\s{2sg.fut.ii}-\s{opt}\Sl help.\s{fut}-\s{dat.1st}|\s{3n.fut.ii}-be.complete.\s{fut}-\s{fut.subj}|overmorrow
    \textit{Roughly:} ‘If you had been able to help me tomorrow, it would have been finished the day after.’
}}

The second example in particular is hard to translate since it communicates an irrealis in the future, at the same time
using a morphological future in both the apodosis and the protasis. The tenses used in the translation here thus do not
reflect the tense actually used in UF.

\chapter{Examples}\label{sec:examples}
\def\Author#1{{\centering—\space\ignorespaces#1\par}}

\section{Fully-Glossed Examples}
\subsection{Simple Glossing Example}

\gloss {
    Cárvá, sráhó dwávaût’há dact’heá?
    C̣ár |vá |s-ráhó |dwá-vaût’há |ḍ-ac̣t’he-á
    ˈj̊ɑ̃ːɰ|ʋ̃ɑ̃|ˌsɰɑ̃ˈhɔ̃ˑ|dɰɑ̃ˌʋ̃ɔ̃̃ˈθɑ̃ˑ|da̯j̊ˈθe.ɑ̃
    Charles.\s{voc}|\s{particle}|\s{indef.acc}-fish|\s{def.iness}-mountain|\s{2sg}-buy-\s{pres.ant.2sg}
    ‘Charles, you bought a fish on the mountain?’
}


\subsection{CCC 2 Text}
{\itshape
Słérá de c’hóný áb’hásy’ô, ráy’ê y’aúhý dís dyb’hóy’e sab’héy’. Ez lé-el lalebet’he z’ihór bet’hê rêsol daudé.
Ýab’héy’ rêd’hes lab’hóy’ejú, dŷna c’haúr debauhib sá lasusy’és ýrâhe lasyrrájú.
}

\multigloss {
    słé-rá| ḍẹ| c’hóný| áb’hásy’ô| ráy’ê| y’aúhý| ḍ-ís | dy-b’hóy’ẹ
    \s{cons.pl}-law|all|well.known|\s{gen}\Sl aviation|way|there.is.no|\s{inf-subj}\Sl can| \s{part}-to.fly

   s-ab’héy’ |ez |lé-el | la-lẹ-bet’hẹ|z’ | ihór | bet’hê
   \s{acc.indef}-bee|its|\s{nom.pl}-wing|\s{3pl-aff.comp}-be.small|its| \s{acc}\Sl body| be.small\Sl\s{part}

    rê-sol | ḍ-auḍé |ý-ab’héy’|rêd’hes|la-b’hóy’ẹ-jú|dŷn-a|c’haúr
    \s{abl}-soil|\s{inf}-obtain |\s{nom.pl.indef}-bee|of.course|\s{3n.pl}-fly-\s{gn}|\s{part}-what|as

    dẹ-ḅauhib|sá|la-susy’é\Sl s|ý-râhẹ|la-sy-rrá-jú
    \s{inf}-be.impossible|not|\s{3n.pl}-care.about\Sl \s{subj}|\s{nom.pl.indef}-human|\s{3n.pl}-\s{3n.pass}-believe-\s{gn}
}


\medskip\noindent
‘According to all known laws of aviation, there is no way a bee should be able to fly. Its wings are too
small to get its fat little body off the ground. The bee, of course, flies anyway because bees don't care
what humans think is impossible.’

\medskip\noindent
Literal translation: ‘According to all known laws of aviation, there is no way that a bee should be capable of flight.\footnote{
Note that UF here uses the verbal noun \w{b’hóy’ẹ} ‘to fly’ as a noun to mean ‘flight’.}
Its wings are too small for its little body to obtain [distance] from the ground. Of course, bees fly [anyway], as
they do not care about what humans believe to be impossible.

\subsection{Copypasta Translation}
{\itshape
Rub’hráy’ó rát’he au sré au sfèhe laut’hâ adŷbáłýr Át’hebhaú Raúl dedesle, s aút’hiy’ey’ó sývéhýr dýhisdé sérdé laúây’êr;
aúc’haúbrâdy’ó’sý’ýâ vé dúr dyhaúbhausy’ô sehabhvísy’ô. Sýlývy’ér saúr c’hesse? Lec’hdr\-aúv\-nét’hic’hâ nérje c’hesse?
Árdihyl c’hesse? Sauz-aud de c’hesse? Jávé’sý’ýâ jrét’hádé dedónéle dýha\-bha\-hit’he deý’ebhat’hic’hâ Áraúl dybháł.
Aúrsáheressá. Jdír jys dub’hrá au dylí sýcahý dýbháhe au dylýáv́áy’é b’hýcahý sbáłýr Áraúl.

Lásásc’hríd raúl révéy’ýr c’hessejú? Léraúb’he lasydír, lavâhe vé sbhárde sásy’élâ Áraúl. Sráhis’sý’ýâ id’hír deb’hýlnér
u b’hesaúr rêvú aû-át’heý’ebhat’he u B’helfaúr sraúb’he. Jav́ár sáví lyzy’ýr ádróid. Sy’u\-b’h\-rá dahaúr isásc’hríd
dwáníb’he araúl sébâ âc’hrír ‘dèc’hníc’hvâ’ Át’hebhaú Raúl ‘desybhérýr’, sjys vé delýc’hóbhár, lásásc’hríd c’haúr sýraúl
âc’hrír sc’hóváhá, lévás nórâ jys ‘desybáł’ dyhéy’é la\-y’e\-hó\-vâ\-hér. Aúc’hóhid’héy’ó laúrvé Áraúl dynát’hýr rêâ, srâsírá,
dwác’hóvníc’h âbáł dývrê b’hehbár\-di\-hi\-bhá aû-á\-dr\-ó\-id, It’hebhaú Raúl abhraúl dérésdâ derâdvâvéy’ýr.
}

%{\centering\Large
%[ɰu̯βˈɰɑ̃ˑ.ɥɔ̃ ɰɑ̃ˈθə̥ o̯ˈsɰɛ̃ˑ o.sɸɛ̯ˈhə̥ ˈɮ̃o.θɑ̃̃ adʏ̃̃.bɑ̃ˈɮ̃ʶʏ̠̃ˑɰ ɑ̃ˈɰɔ̃ˑɮ̃ də.de̯ˈsɮ̃ə̥ sɔ̃.θi̯ˈɥe.ɥɔ̃ sʏ̃.ʋ̃ɛ̃ˈhʏ̠̃ˑɰ dʏ̃.hi̯sˈdɛ̃ˑ sɜ̃ɰˈdɛ̃ˑ ɮ̃ɔ̃.ɑ̃̃ˈɥɘ̃̃ˑɰ ɔ̃.χɔ̃ˈbɰɑ̃̃ˑ.dɥɔ̃.sɥ̃ʏ̃ɑ̃̃ ʋ̃ɛ̃ˈdũˑɰ dy.hɔ̃.bʱo̯ˈsɥɔ̃̃ˑ səh.abʱ.ʋ̃ĩˈsɥɔ̃̃ˑ]
%\par
%}

\subsection{Gloss}
\multigloss {
    r-ub’hrá-y’ó|rát’hẹ|au|s-ré|au|s-fèhẹ|laut’h-â
    \s{1pl}-can-\s{1pl}|you.see|and|\s{acc.pl.indef}-ray|\s{and}|\s{acc.pl.indef}-beam|float-\s{ptcp}

    aḍŷ-ḅáłýr|á-t’hebhaú raúl|dẹ-deslẹ|s|aú-t’hiy’e-y’ó|sý-véhýr
    \s{interess.pl.indef}-speaker|\s{gen}-Ultrafrench.language|\s{inf}-detect|if|\s{1pl}-use-\s{1pl}|\s{gen.pl.indef}-measure

    dý\Sl hisḍé|sérḍé|laú|â-y’\Sl ệr|aú-c’haúḅrâd-y’ó
    \s{part.pl.indef}\Sl system|certain|long|\s{ptcp.pass}-forbid\Sl \s{ptcp.pres.ant}|\s{1pl}-understand-\s{1pl}

    ’sý’ýâ|vé|ḍúr|dy\Sl haúbhausy’ô|sẹh|abh-vísy’ô|sý-lývy’ér|saúr|c’hes
    not|but|still|\s{part}\Sl composition|this|\s{gen.pl}-emission|\s{gen.indef}-light|\s{abs}.kind|\s{q}

    se|lec’hḍraúvnẹ́t’hic’h-â|nérjẹ|c’hes|se|árḍihyl|c’hes|se|sauz
    \s{3n}.be|electromagnetic-\s{ptcp}|\s{energy}.\s{abs}|\s{q}|\s{3n}.be|particle.\s{abs}|\s{q}|\s{3n}.be|\s{abs}.thing

    aud|ḍẹ|c’hes|se|jávé|’sý’ýâ|j-rét’hád-é|dẹ-dónẹ́-ḷẹ
    other|entire|\s{q}|\s{3n}.be|never|not|\s{1sg}-claim-\s{pres.ant}|\s{inf}-endow-\s{3.dat}

    dý\Sl habhahit’hẹ|ḍeý’ebhat’hic’h-â|á-raúl|dy\Sl bháł
    \s{part.pl.indef}\Sl ability|be.telepathic-\s{ptcp}|\s{gen}-language|\s{part}\Sl speak

    aúr-sáhere-ss\Sl a|j-dír|jys|d-ub’hrá|au|dy-lí
    \s{3n.fut.ii}-be.preposterous.\s{fut}-\s{cond}\Sl \s{circ}|\s{1sg}-say|only|\s{inf}-can|and|\s{part}-read

    sý\Sl c̣ahý|dý\Sl bháhẹ|au|dy-lý-áv́áy’é
    \s{gen.pl.indef}-each.other|\s{part.pl.indef}-thought|and|\s{part}-\s{3pl.pass}-send

    b’hý\Sl c̣ahý|s-ḅáłýr|á-raúl
    \s{dat.pl.indef}-each.other|\s{acc.pl.indef}-speaker|\s{gen}-language

    lá-sásc’hríḍ|raúl|ré-véy’ýr|c’hes|se-jú
    \s{nom}-Sanskrit|\s{abs}.language|\s{sup}-better|\s{q}|\s{3n}.be-\s{gn}

    lé-raúb’hẹ|la-sy-dír|la-vâhẹ|vé|s\Sl bhárḍẹ|sásy’él-â|á-raúl
    \s{nom.pl}-robot|\s{3pl}-\s{3n.pass}-say|\s{3pl}-miss.out|but|\s{acc.indef}\Sl part|be.essential-\s{ptcp}|\s{gen}-language

    s-ráhis|’sý’ýâ|i\Sl d’hír|dẹ-b’hýlnẹ́r|u|b’hel-saúr|rê-vú|aû-|á\Sl t’heý’ebhat’hẹ
    \s{3n}-be.racist|not|\s{acc}\Sl say|\s{inf}-be.unaffected|or|\s{instr.pl}-form|\s{sup}-many|non-|\s{gen}-telepathy

    u|b’he-faúr|s-raúb’he|j-av́ár|s-áví|lyzy’ýr|ádróid
    or|\s{instr}-Force|\s{acc.pl.indef}-robot|\s{1sg}-have|\s{acc.pl.indef}-friend|several|\s{abs}.android

    s-y’-ub’hrá|dahaúr|i-sásc’hríd|dwá-níb’hẹ|a-raúl|séḅ-â|â-c’hrír
    \s{3n-opt}-can|sure|\s{acc}-Sanskrit|\s{iness}-level|\s{gen}-language|be.plain-\s{ptcp}|\s{ptcp.pass}.write

    ḍèc’hníc’hvâ|á-t’hebhaú raúl|dẹ-sybhẹ́rýr|s-jys|vé|dẹ-lý-c’hóbhár
    technically|\s{gen}-Ultrafrench.language|\s{inf}-be.superior|\s{3n}-be.unfair|but|\s{inf}-\s{3pl.pass}-compare

    lá-sásc’hríd|c’haúr|sý-raúl|â-c’hrír|s-c’hóváh\Sl á
    \s{nom}-Sanskrit|as|\s{gen.indef}-language|\s{ptcp.pass}-write|\s{3n}-start.out.as.\s{subj}\Sl \s{pres.ant}

    lé-vás|nór-â|jys|dẹ-sy-ḅáł|dy\Sl héy’ẹ́|la-y’ẹ\Sl hóvâh\Sl ér
    \s{nom.pl}-masses|be.ignorant-\s{ptcp}|until|\s{inf}-\s{3n.pass}-speak|\s{part}\Sl attempt|\s{3pl}-\s{opt}\Sl start\Sl \s{pres.ant}

    aú-c’hóhid’hẹ́-y’ó|laúrvé|á-raúl|dy-nát’hýr|rê-â
    \s{1pl}-consider-\s{1pl}|but.when|\s{gen}-language|\s{part}-nature|be.triune-\s{ptcp}

    s-râsír-á|dwá-c’hóvníc’h|â-ḅáł|dývrê|b’heh-ḅárḍihibhá|aû-|ádróid
    \s{3n}-transpire-\s{pres.ant}|\s{iness}-communication|\s{ptcp.pass}-speak|at least|\s{instr.pl.indef}-participant|non|\s{abs}.android

    i-t’hebhaú raúl|abh-raúl|ḍérésḍ-â|dẹ-râdvâ-véy’ýr
    \s{acc}-Ultrafrench.language|\s{gen.pl}-language|be.terrestrial-\s{ptcp}|\s{inf}-\s{superl}-be.better
}

\subsection{Translation}
‘You see, we can detect rays and beams of energy floating between ULTRAFRENCH speakers if we use certain long-forbidden
measurement systems, but we still don’t understand the composition of these emissions. Are they some kind of light?
Electromagnetic energy? A particle? Something else entirely?

‘I’ve never claimed that speaking ULTRAFRENCH endows you with telepathic abilities. That would be preposterous. I’m just
saying that ULTRAFRENCH speakers can read each others minds and send thoughts to each other.

‘Is Sanskrit the best language? The robots tell me so.  But they are missing out on an essential part of ULTRAFRENCH.
It’s not racist to say robots are immune to most forms of not-telepathy and the Force. I have several android friends

‘Sanskrit might be “technically” “superior” to ULTRAFRENCH on the level of the plain written language. Sure, but it’s
unfair to compare them because Sanskrit started out as a written language until the ignorant masses started attempting
to “speak” it.

‘But when you consider the triune nature of ULTRAFRENCH, I think it’s clear that, at least in spoken communication with
non-android participants, ULTRAFRENCH is the best earth-based language.’

\subsection{Literal Translation}
We can, you see, detect both rays and beams of energy floating between speakers of The UF Language
if we use certain systems of measurement long-forbidden; we still don’t understand, however, the composition of these
emissions. Is it some kind of light? Is it electromagnetic energy? Is it a particle? Is it something else entirely?
I’ve never claimed that [the mere act of]\footnote{The speaker uses a \s{pci} (\w{dybháł}) instead of an \s{aci}
(\w{ibháł}) for ‘speaking’ here; had they used an \s{aci}, the meaning would be closer to ‘the act
of “fully speaking” the language’, as in, speaking and understanding it in its entirety. Thus, the speaker implicates that
it is not the mere act of making utterances in UF (\w{Áraúl dybháł}), but rather speaking and comprehending it in its entirety (\w{Áraúl ibháł})
that gives rise to telepathic abilities.} the speaking of The Language endows them with telepathic abilities.
It would be preposterous. I’m only saying that speakers of The Language can both read each other’s thoughts\footnote{In UF, ‘to
read someone’s mind’ is expressed as ‘to read someone’s thoughts’.} and send them to each other.

Is Sanskrit the best language? The robots are saying it, but they miss out on an essential part of The Language. The act
of saying that robots are incapable of being affected by most forms of non-telepathy or\footnote{The UF text uses \w{u} \ldots\ \w{u}
\ldots\ ‘\ldots\ or \ldots (inclusive)’. This is for semantic reasons: the original text had a positive context (‘immune to’), whereas
the UF translation uses a negative context (‘incapable of being affected by’); thus, by De Morgan, we have to switch from ‘and’
to ‘or’ here.} by the Force is not racist. I have several android friends. Sure, Sanskrit might,\footnote{‘might be X’ is
generally expressed using the optative of \w{ub’hrá} + an \s{aci} with ‘to be X’.} on the level of the plain written
language, be ‘technically’ ‘superior’ to The UF Language, but it is unfair to compare them, as Sanskrit started out as
a written language, until the ignorant masses started attempting to ‘speak’ it. But when we consider the triune nature
of The Language, it has transpired that,\footnote{‘To become clear’ is expressed with the \s{pres ant} form of ‘transpire’.}
at least in spoken communication with non-android participants, UF is the best of the terrestrial languages.

\subsection{Two Stanzas from ‘The Rime of the Ancient Mariner’}

%% FIXME: This is dumb. Write a proper TwoCols macros that parses two
%% boxes first using the \afterassignment+\aftergroup trick.
%%
%% \unvbox or sth in the verse environment to get the longest line?
\makeatletter
\setbox\BoxA\hbox{\it Right up above the mast did stand,}
\@tempdima\wd\BoxA
\setbox\BoxA\vtop{\hsize\@tempdima
\begin{verse}
All in a hot and copper sky,
\quad The bloody Sun, at noon,
Right up above the mast did stand,
\quad No bigger than the Moon.\\

Day after day, day after day,
\quad We stuck, ne breath ne motion;
As idle as a painted ship
\quad Upon a painted ocean.\\
\end{verse}
}

\setbox\BoxB\hbox{\it Órdy’úr ád’y’úr, órdy’í ád’y’í,}
\@tempdima\wd\BoxB
\setbox\BoxB\vtop{\hsize\@tempdima
\begin{verse}\it
Dáhŷná’ câ, bárýnrê de,
Láhaul dwávíd’h’, áhâłát’hâ,
Sýrvá sb’haulá dèl sý’dwálý,
\quad Aûlerá áraúvá.\\

Órdy’úr ád’y’úr, órdy’í ád’y’í,
Aúrdévýry’aû, sáhýnvúb’hvâ,
Bárýnc’hánár âbét’hýrér,
\quad Dáhŷnvérr dehýnrál.
\end{verse}
}
\makeatother

{\centering
\noindent\leavevmode\usebox\BoxA\qquad\usebox\BoxB\par
}

\Author{\s{Sávy’él D. C’haulełij}, rád’hyc’hsy’ô \s{Áhnet’h}}


\Paragraph{Gloss}
\multigloss {
dáhŷn\Sl á(ẹ)|câ|ḅárýn-rê|ḍẹ
\s{iness.indef}\Sl sky|be.hot\Sl\s{ptcp}|\s{ess.indef}-copper|all

lá\Sl haul|dwá-víd’h(ẹ)|áhâłát’h\Sl â
\s{nom}-sun|\s{iness}-noon|be.bloody-\s{ptcp}

sýr-vá|s-b’haul-á|dèl|sý’-dwá-ḷý
\s{spress}-mast|\s{3n}-hover-\s{pres.ant}|{\it particle}|{\it distal}-\s{iness}-{\it sp. correl.}

aû-lẹ-rá|áraúvá
not-\s{aff.comp}-big|\s{gen}-moon

órd-y’úr|ád(á)-y’úr|órd-y’í|ád(á)-y’í
\s{ela}-day|\s{ill}-day|\s{ela}-night|\s{ill}-night %Definite to fit the metre.

aúr-dévýr-y’aû|sáhýn-vúb’hvâ
\s{1pl}-remain-\s{pret.1pl}|\s{abess}-movement

ḅárýn-c’hánár|â-ḅét’hýr-ér
\s{ess.indef}-ship|\s{ptcp.pass}-paint-\s{ptcp.pres.ant}

dáhŷn-vérr|dẹhýn-rál
\s{iness.indef}-sea|\s{spress.indef}-canvas
}

\subsection{The Tragedy of Darth Plagueis the Wise}
\Paragraph{Láváý’ýr Blac’his Ád’hart’h Áhaj}\medskip
{\it%
Dát’hád’hé dej dyváý’ýr Blac’his Ád’hart’h Áhaj. Jréflecé’sý’ýâ dyźi. Isdrár se a dyisdrár léjed’háy’
laúnárrahódejússaub’he’sý’ýâ. Î se Abhhit’h. Sénýr Abht’hénéb Abhhit’h le Blac’his Dart’h; leb’h au ahy’íhâ
au ahsajâ líhá dab’hèc’h àré shufb’h sývíd’hic’hlaúry’ê. Lyá sahc’haúnéhás árrádraúc ausc’hýrâ lihá
id’hérny’éhuf abhźi la lacérérle deréb’hní.
}

\Paragraph{Gloss}
\multigloss{
ḍ-át’hád’h\Sl é|dej|dy-váý’ýr|Ḅlac’his Á\Sl d’hart’h|á\Sl haj
\s{2sg}-hear\Sl\s{pret}|\s{particle}|\s{part}-tragedy|Plagueis \s{gen}\Sl Darth|\s{gen}\Sl be.wise

j-rẹ́flec̣-é|’sý’ýâ|dy\Sl źi
\s{1sg}-think-\s{pres.ant}|not|\s{part}-this.one

se|isḍrár|a|dy-isḍrár|lé-jed’háy’|laú-nárrahóḍe-jú-ss\Sl au-b’hẹ|’sý’ýâ
be.\s{3n}|story.\s{abs}|\s{rel}|\s{part}-story|\s{nom.pl}-Jedi|\s{3pl}-\s{narrate.fut}-\s{gnomic}-\s{cond}\Sl\s{circ}-\s{2.dat}|not

se|î|abh\Sl hit’h
be.\s{3n}|legend.\s{abs.indef}|\s{gen.pl}-Sith

Sénýr|abh\Sl t’hénéb|abh\Sl hit’h|le|Blac’his Dart’h
lord.\s{abs.indef}|\s{gen.pl}-darkness|\s{gen.pl}-Sith|be.\s{3m}|Darth Plagueis

leb’h|au|ah-y’íh\Sl â|au|ah-saj-â|l-ih\Sl á
be.\s{3m.pres.ant}|and|\s{suff.comp}-be.powerful\Sl \s{ptcp}|and|\s{suff.comp}-be.wise-\s{ptcp}|\s{3m}-be.able.to.\s{subj}\Sl \s{pres.ant}

d-ab’hèc’h|à-ré|s\Sl hufb’h|sý-víd’hic’hlaúry’ê
\s{inf}-influence|\s{inf.pass}-create|\s{acc.indef}-life|\s{acc.pl}-Midichlorian

l-y-á|s-ah-c’haúnéhás|á-rráḍraúc|ausc’hýr-â|l-ih\Sl á
\s{3m}-have-\s{pres.ant}|\s{acc.indef}-such-knowledge|\s{gen}-side|be.dark-\s{ptcp}|\s{3m}-be.able.to.\s{subj}\Sl \s{pres.ant}

i\Sl d’hérny’ẹ́huf|abh\Sl źi|l-a|la-cér-ér-ḷẹ|dẹ-réb’hní
\s{acc}\Sl death|\s{gen.pl}-the.one|\s{abs.pl}-\s{rel}|\s{3pl}-be.dear-\s{pres.ant}-\s{3dat}|\s{inf}-prevent
}

\subsection{The North Wind and the Sun}
{\it%
\noindent Lasehérélé au Láb’haúré au Láhaul dŷnfaúr. Ladehid’hér haúdóné dýb’hic’htrár asa aúrdehab’híy’á ráy’á sráy’â.
Sc’hóvâá láb’haúré, b’hát’hiý’at’hýrâ y’aúý’ávâ. Phas láaú’z sýrêr leraúhéréhá, sárslá b’hefaúr levú. Lárrád’hahánár
vé dèl y’áb’hedêr, srêr syplâ latrâ, fahaú dèl, âníér, Láb’haúré sy’y’erádá Ashaul. Sec’hlérá révy’évâ vaúd’hérvâ.
Ráhaú dé wêr irêr syplâ, Láhaul sdárá sré lerýlâ, Láaú jys, ub’hrâ’sý’ýâ lys dád’hýr it’hèrvíc’h, au ly’y’edehab’híy’á
au ly’ad’há desb’hé dáhŷnríb’hy’ér ré.

Seh láistrár svaût lyp’hárdyt’há ihaúb’héc’h áy’aúý’á derêfihasjú.
}\medskip

\Author{\s{Zaub},\footnote{\w{Zauḅ} is the UF name for Aesop.} rád’hyc’hsy’ô \s{Áhnet’h}}

\Paragraph{Original Text}\smallskip
\noindent Βορέας καὶ Ἥλιος περὶ δυνάμεως ἤριζον. ἔδοξε δὲ αὐτοῖς ἐκείνῳ τὴν νίκην ἀπονεῖμαι, ὃς ἂν αὐτῶν ἄνθρωπον ὁδοιπόρον ἀποδύσῃ. καὶ ὁ Βορέας ἀρξάμενος σφοδρὸς ἦν· τοῦ δὲ ἀνθρώπου ἀντεχομένου τῆς ἐσθῆτος μᾶλλον ἐπέκειτο. ὁ δὲ ὑπὸ τοῦ ψύχους καταπονούμενος ἔτι μᾶλλον, καὶ περιττοτέραν ἐσθῆτα προσελάμβανεν, ἕως ἀποκαμὼν ⟨ὁ Βορέας⟩ τῷ Ἡλίῳ αὐτὸν παρέδωκε. κἀκεῖνος τὸ μὲν πρῶτον μετρίως προσέλαμψε· τοῦ δὲ ἀνθρώπου τὰ περισσὰ τῶν ἱματίων ἀποτιθεμένου, σφοδρότερον τὸ καῦμα ἐπέτεινεν, ἕως οὗ πρὸς τὴν ἀλέαν ἀντέχειν μὴ δυνάμενος, ἀποδυσάμενος, ποταμοῦ παραρρέοντος ἐπὶ λουτρὸν ἀπῄει.

ὁ λόγος δηλοῖ ὅτι πολλάκις τὸ πείθειν τοῦ βιάζεσθαι ἀνυτικώτερόν ἐστι.

\Paragraph{Translation by Some Random Frenchman}\smallskip
\noindent Borée et le Soleil contestaient de leur force. Ils décidèrent d’attribuer la palme à celui d’entre eux qui dépouillerait un voyageur de ses vêtements. Borée commença ; il souffla avec violence. Comme l’homme serrait sur lui son vêtement, il l’assaillit avec plus de force. Mais l’homme incommodé encore davantage par le froid, prit un vêtement de plus, si bien que, rebuté, Borée le livra au Soleil. Celui-ci tout d’abord luisit modérément ; puis, l’homme ayant ôté son vêtement supplémentaire, le Soleil darda des rayons plus ardents, jusqu’à ce que l’homme, ne pouvant plus résister à la chaleur, ôta ses habits et s’en alla prendre un bain dans la rivière voisine.

Cette fable montre que souvent la persuasion est plus efficace que la violence.

\Paragraph{Gloss}
\multigloss{
la-sẹhérél-é|au|lá\Sl b’haúré|au|lá\Sl haul|dŷn-faúr|
\s{3pl}-quarrel-\s{pret}|and|\s{def}\Sl North.Wind|and|\s{def}-Sun|\s{indef.part}-strength.

La-dehid’h\Sl ér|haú-dónẹ́|dý-b’hic’hḍrár|as-a|
\s{3pl}-decide.\s{3pl.pres.ant}|\s{pass.inf.fut.ii}-give|\s{part.def}-victory|\s{dat-rel.pron}

aúr-dehab’híy’-á|s-ráy’\Sl â
\s{3n.fut.ant}-undress-\s{circ}|\s{acc.indef}-travel-\s{ptcp}

s-c’hóvâ-á|lá\Sl b’haúré|b’hát’hiý’at’hýr-â|y’aúý’ávâ
\s{3n}-begin-\s{pres.ant}|\s{def}\Sl North.Wind|blow-\s{ptcp}|violently

ḅas|lá-aú|’z|sý-rêr|lẹ-raúhérẹ́h\Sl á
because|\s{def}-man|his|\s{acc.pl}-clothing|\s{3m}-\s{tighten}-\s{pres.ant}

s-ársl-á|b’he-faúr|lẹ-vú
\s{3n}-\s{attack}-\s{pres.ant}|\s{instr.indef}-force|\s{aff.comp}-much

lá-rrád’hahánár|vé|ḍèl|y’-áb’hẹḍ-êr|s-rêr|syḅlâ|l-aḍr\Sl â|
\s{def}-cold|but|\s{particle}|\s{3m.pass}-inconvenience-\s{pres.ant.ptcp}|\s{acc.indef}-clothing|additional|\s{3m}-take-\s{pres.ant}|

fahaú|dèl|â-ní-ér|lá\Sl b’haúré|s-y’-y’ẹ-rád-á|as\Sl haul
so.much.so|emphatic|\s{ptcp.pass}-rebut-\s{pres.ant}|\s{def}\Sl North.Wind|\s{3n}-\s{3m.pass}-\s{opt}-surrender-\s{pres.ant}|\s{dat}-sun

s-ec’hlér-á|révy’ẹ́-vâ|vaúd’hér-vâ
\s{3n}-shine-\s{pres.ant}|first-\s{adv}|be.moderate-\s{adv}

ráh-aú|dẹ́|w-êr|i-rêr|syḅlâ|lá\Sl haul|s-dár-á|s-ré
lá-aú|then|remove-\s{pres.ant.ptcp}|\s{acc}-clothing|supplementary|\s{def}\Sl sun|\s{3n}-throw-\s{pres.ant}|\s{acc.indef.pl}-ray

lẹ-rýl-â|lá-aú|jys|ub’hr\Sl â|’sý’ýâ|lys|d-ád’hýr|i\Sl t’hèrvíc’h|and
\s{aff.comp}-burn-\s{ptcp}|\s{def}-man|until|be.able-\s{ptcp}|not|no.longer|\s{inf}-resist|\s{acc}-heat|

au|l-y’-y’ẹ-dehab’híy’-á|au|l-y’-ad’h\Sl á|dẹ-sb’hé|
and|\s{3m}-\s{3m.pass}-\s{opt}-undress-\s{pres.ant}|and|\s{3m}-\s{opt}-go\Sl \s{pres.ant}|\s{inf}-bathe|

dáhŷn-ríb’hy’ér|sẹh|lá-isḍrár|s-vaût|lybhárdyt’há|i\Sl haúb’héc’h|á-y’aúý’á|dẹ-rê-fihas-jú
\s{iness}-river|this|\s{nom}-tale|\s{3n}-show|often|\s{acc}-persuade|\s{gen}-be.violent|\s{inf}-\s{comp}-be.efficient-\s{gnomic}
}

\subsection{2024 YouTube Conlang Relay}
\Paragraph{Original English Text}\medskip
Suddenly, you (sg.) experienced a vision and felt like prey. In this vision, that (distal) one-toothed river-baby
was torn asunder by its parents.

Verily! Your dream appears to have included the parents. You ask: How can you hear the voice of the river? What
could have caused this to happen—to you, when you can dream? Must what the river hears be true?
For whom is it intended?

And in this moment, in order to hear the river again, you may rip apart the baby’s hands.

That druid would have said: In order to ensure their prophecies, the river quickly delivers its flow. How did that
(dist.) druid hear that you have learnt? The druids do not want prophecies. And in order to stop the act of tearing
apart the hands, I flow to torture this river. However, in order to see prophecies, druids are needed.

Thus, in order to prevent the prophecies and the suffering upon the blind druid, for his prophecies to happen, you
should have torn apart your hands. Hear me! You must have purified the place. Thus, you grabbed a shovel in all your
hands, and buried a log with much moss.

\Paragraph{Translation}{\it
Sb’hizy’ô sud’hénvâ deréhevá au desèt’há âcahár. Seh dwáb’hizy’ô sý’e iáb’há ríb’hy’ér, lá-áb’há a syá sd’há sylâ,
lasydecírér ez lébará.
Ânb’hé, daú láréb’h debhará dec’hlýrá sbáréd. Derýcér : B’hehráy’ê c’hes dub’hrá dyrá áríb’hy’ér dát’hád? Ŷna c’hesse
a aúrflijéssá iźi ast’he, dís váłé deréb’h? Syv́ár c’hes dŷná láríb’hy’ér sát’hád deré? Asa c’hes sydír?

Au seh dwávvâ, b’hŷnnúb’hâ’b’h dyríb’hy’ér syát’hás, á-áb’há sývê dedecír díry’úréssa.
Sý’e bá lá\-nó\-rá\-víc’h aúrdíréssá : Lýahúrs b’he lýr sýhaúáł, b’heý’ovâ’z ihulvâ láríb’hy’ér sfúr. B’hehráy’ê sý’e lánóráv
sát’hád’há dedabrâá. Lasbéjú’sý’ýâ lénóráv sýhaúáł. Syábhecs b’he id’hecír abhvê, seh iríb’hy’ér jaúválv́eśé jsehul.
Laúb’héréśe vé dýhaúáł dýnóráv lýbźéjú.

Lýábhecs daú’b’h au sýhaúáł au id’huý’ýr bá válv́áy’â ánórávíc’h, lav́árlýs b’he’z léhaúáł,
b’hádecirres\-sá daú sývê. C’hát’hád! Ilý dec’haúbhýrífá dev́ár. Derâd’há daúc’h sbhelbec daú b’hevê de, au dát’hérá
sraûd’hárb dáhŷnvúslihé vú.}

\Paragraph{Gloss}
\multigloss {
    Sud’hénvâ|s-b’hizy’ô|ḍẹ-réhẹv\Sl á|au
    Suddenly|\s{acc.indef}-vision|\s{2sg}-receive\Sl \s{pres.ant.2sg}|and

    ḍẹ-sèt’h\Sl á|â-cah-ár
    \s{2sg}-feel\Sl \s{pres.ant.2sg}|\s{ptcp.pass}-hunt-\s{pret}

    sẹh|dwá-b’hizy’ô|sý’ẹ|i-áb’há|ríb’hy’ér|lá-áb’há|a|s-y-á
    this|\s{iness}-vision|that|\s{acc}-child|river.\s{abs}|\s{nom}-child|\s{rel}|\s{3n}-have-\s{pres.ant}

    s\Sl d’há|syl-â|la-sy-dec̣ír-ér|ez|lé-ḅará
    tooth|\s{acc.indef}\Sl tooth|to.be.the.only.one-\s{part}|\s{3n.pl}-\s{3n.pass}-tear-\s{pres.ant.3pl}|its|\s{nom.pl}-parent

    ânb’hé|daú|lá-rẹ́b’h|dẹ\Sl bhará|dẹ-c’hlýr-á|s-ḅáréḍ
    verily|your|\s{nom}-dream|\s{part.pl}\Sl parent|\s{inf}-include-\s{inf.pres.ant}|\s{3sg}-seem

    ḍẹ-rýc̣ér|b’hehráy’ê|c’hes|ḍ-ub’hrá|dy-rá|á-ríb’hy’ér|d-át’hád
    \s{2sg}-ask|by.what.means|\s{q}|\s{2sg}-can|\s{part}-voice|\s{gen}-river|\s{inf}-hear

    ŷn-a|c’hes|se|a|aúr-flijé-ss\Sl á|i\Sl źi|as-t’hẹ
    \s{nom.indef}-what|\s{q}|be.\s{3n.sg}|\s{rel}|\s{3n-fut.ant₁}-cause.\s{fut}-\s{cond}-\s{circ₁}|this.\s{acc}|\s{dat}-you

    ḍ-ís|váłé|dẹ-rẹ́b’h
    \s{2sg}-can.\s{subj}|despite.that|\s{inf}-dream

    sy-v́ár|c’hes|dŷn-á|lá-ríb’hy’ér|s-át’hád|dẹ-ré
    \s{3n.pass}-must|\s{q}|\s{part.indef}-what|\s{nom}-river|\s{3n}-hear|\s{inf}-be.true

    as-a|c’hes|sy-dír
    \s{dat}-who|\s{q}|\s{3n.pass}-say

    Au|sẹh|dwá-vvâ|b’hŷnnúb’hâ|’b’h|dy-ríb’hy’ér|sy-át’hás|á-áb’há|sý-vê
    and|this|\s{iness}-moment|anew|in.order.to|\s{part}-river|\s{3s.pass}-hear.\s{subj}|\s{gen}-child|\s{acc.pl}-hand

    dẹ-dec̣ír|ḍír-y’-úrẹ́-ssa
    \s{inf}-tear.apart|\s{2sg.fut.ii}-\s{opt}-can.\s{fut}-\s{cond}

    sý’ẹ|ḅá-lá-nórávíc’h|aúr-dírẹ́-ss\Sl á
    that|\s{circ}₁-\s{nom}-druid₁|\s{3n.fut.ant}₁-say.\s{fut}-\s{cond}-\s{circ}₁

    lý-ahúrs|b’he|lýr|sý\Sl haúáł|b’heý’o-vâ|’z|i\Sl hulvâ
    \s{3pl.pass}-ensure.\s{subj}|in.order.to|their|\s{acc.pl}-prophecies|to.be.quick-\s{adv}|its|\s{acc}\Sl flow

    \s{nom}-river|\s{3n}-deliver
    lá-ríb’hy’ér|s-fúr

    b’hehráy’ê|sý’ẹ|lá-nóráv|s-át’há\Sl d’há|dẹ-ḍ-aḅrâ-á
    how|that|\s{nom}-druid|\s{3sg.n}-hear\Sl \s{pres.ant.3sg.n}|\s{inf₁}-\s{2sg.pass}-learn-\s{pres.ant₁}

    la-sḅé-jú|’sý’ýâ|lé-nóráv|sý\Sl haúáł
    \s{3pl}-want-\s{gnomic}|not|\s{nom.pl}-druid|\s{acc.pl.indef}-prophecy

    sy-ábhecs|b’he|i\Sl d’hec̣ír|abh-vê|sẹh|i-ríb’hy’ér
    \s{3n.pass}-stop.\s{subj}|so.that|\s{acc}-tear.apart|\s{gen.pl}-hand|this|\s{acc}-river

    jaú-válv́e-ś\Sl ẹ́|j-sehul
    \s{1sg.fut.ii₁}-torture-\s{subj}\Sl \s{circ₁}|\s{1sg}-flow

    laú-b’hérẹ́-ś\Sl e|vé|dý\Sl haúáł|dýnóráv|
    \s{3pl.pass.fut.ii₁}-see.\s{fut}-\s{subj}\Sl \s{circ₁}|however|\s{part.pl.indef}-prophecy|\s{part.pl.indef}-druid

    lý-bźé-jú
    \s{3pl.pass}-need-\s{gnomic}

    lý-ábhecs|daú|’b’h|au|sý\Sl haúáł|au|i\Sl d’huý’ýr
    \s{3pl.pass}-prevent.\s{subj}|thus|so.that|and|\s{acc.pl}\Sl prophecy|and|\s{acc}\Sl suffer

    ḅá-|válv́áy’-â|á-nórávíc’h|l-av́árḷýs|b’he|’z|lé\Sl haúáł
    \s{circ₁}-|be.blind-\s{part}|\s{gen}-druid₁|\s{3pl.n}-happen.\s{subj}|so.that|his|\s{nom.pl}-prophecy

    b’há-dec̣irrẹ-ss\Sl á|daú|sý-vê
    \s{2sg.fut.ant₁}-tear.asunder.\s{fut}-\s{cond}\Sl \s{circ₁}|your|\s{acc.pl}-hand.

    c’h-át’hád
    \s{2sg.imper}-hear

    i-ḷý|dẹ-c’haúbhýríf-á|ḍẹ-v́ár|
    \s{acc}-place|\s{inf}-purify-\s{pres.ant}|\s{2sg.pass}-must|

    ḍẹ-râd’h\Sl á|thus|s\Sl bhelbec|daú|b’he-vê|ḍẹ|au|ḍ-át’hér-á
    \s{2sg}-grab\Sl \s{pres.ant}|\s{acc.indef}-shovel|your|\s{instr.pl}-hand|all|and|\s{2sg}-bury-\s{pres.ant}

    s-raûd’hárb|dáhŷn-vúslihé|vú
    \s{acc}-log|\s{iness.indef}-moss|much
}

\section{Unglossed Translations}
\subsection{The Misanthrope – \s{Molière}}
\begin{verse}\it
C’hasbesy’ál y’aúhý, jaír ivaûd de :
Sýhèl bas labíres laválfès,
Sýaud bas abhvaúb’hâ lac’haúblés,
Au sá lac’hlýrs dyn vérjet’hic’hâ
A dyn sdónés láváý’eb’his anhbrí b’hérdy’ŷâ.
Aúráró ijys rá seh áhaúblér
Árívnél lérâ asa jlit’hijy’.
\end{verse}

\Paragraph{Original Text}\smallskip
\begin{verse}
Non, elle est générale, et je hais tous les hommes :
Les uns, parce qu’ils sont méchants et malfaisants,
Et les autres, pour être aux méchants complaisant,
Et n’avoir pas pour eux ces haines vigoureuses
Que doit donner le vice aux âmes vertueuses.
De cette complaisance on voit l’injuste excès
Pour le franc scélérat avec qui j’ai procès.
\end{verse}

\Paragraph{English Translation}\smallskip
\begin{verse}
There’s no exception, and I hate all men:
A part, because they’re wicked and do evil;
The rest, because they fawn upon the wicked,
And fail to feel for them that healthy hatred
Which vice should always rouse in virtuous hearts.
You see the rank injustice of this fawning,
Shown toward the bare-faced scoundrel I’m at law with.
\end{verse}

%% Dictionary.
\chapter{Dictionary}
What follows is the UF dictionary: a complete list of UF words, their etymology and definitions, complete with
select examples from UF literature and simple phrases to illustrate variations. The following conventions apply
in the dictionary:

\Paragraph{Case of Verb Complements}
If the definition or a sense of a transitive verb starts with ‘+\s{case}’, then the direct object of
that verb has case \s{case}. Similarly, if the definition of a ditransitive verb starts with ‘+\s{case1} and
\s{case2}’, then the direct object has \s{case1}, the indirect object \s{case2}.

\Paragraph{Examples}
Examples (from literature) are introduced by a $\diamond$ and usually contain UF text illustrating the word’s
usage, as well as a translation, and possibly explanation. Note that a lot of these examples may use archaic
spellings, e.g. \w{tèl} for \w{ḍèl}.

\Paragraph{Comments}
Roman in the primary definition or a sense of a word indicates that word’s meaning, while italic text is used
to add a comment explaining the word’s usage or etymology.

\Paragraph{Grammatical Forms}
A lot of grammatical forms, including verb and noun affixes, irregular forms of pronouns, most larger numbers,
as well as the entire paradigm of \w{eḍ} ‘to be’, are \textit{not} included in the dictionary since they can
already be found in the grammar. The reader is expected to either be familiar with them already or to search
the rest of the grammar for them.

\def\leftmark{\textbf{\firstmark}\ | \textbf{\botmark}}
\let\rightmark\leftmark
\twocolumn
\ExplSyntaxOn

\def \ex {
    \par
    \setbox0\hbox{\enskip $\diamond$\space}
    \leavevmode \hangindent = \dim_eval:n { \wd0 + 3pt }
    \box0
}

%% Allow breaking after slashes.
\catcode `/ = 13
\def/{\char`/\penalty0}


%% Counter for senses.
\newcounter {sense}
\def \\ { %% 'Magic' macro that is recognised by the ULTRAFRENCHER. DO NOT RENAME.
    \par
    \normalfont \textbf {\arabic{sense}.} \space
    \stepcounter {sense}
    \ignorespaces
}

\parindent = 3pt
\everypar {
    \hangindent = 6pt
    \hangafter = 1 \relax
}

\cs_new:Npn \start_entry: {
    \ifvmode\else\unskip\par\fi
    \vskip 1pt
    \setcounter {sense} {1}
    \noindent
}

\def \this {{ \bf \char`~ }}
\def \w #1 {{\bf\ignorespaces #1}}

%% Word, part of speech, etymology, definition, (forms)
\long \def \entry #1 #2 #3 #4 #5 {
    \start_entry:

    %% Typeset word and part of speech.
    \mark { #1 }
    \textbf { \fontsize{13}{13}\selectfont \ignorespaces #1 } \space
    \textit { \ignorespaces #2 }

    %% Typeset etymology.
    \tl_set:Nn \l_tmpa_tl {#3}
    \tl_if_empty:NTF \l_tmpa_tl { } {
        \space [
            \ignorespaces \tl_use:N \l_tmpa_tl
        ]
    }

    %% Typeset forms, if any.
    \tl_set:Nn \l_tmpa_tl {#5}
    \tl_if_empty:NTF \l_tmpa_tl { } {
        %\space {\nf\scshape{forms}}:
        \space
        \textit { \ignorespaces \tl_use:N \l_tmpa_tl }
        .
    }

    %% Typeset definition.
    \space \ignorespaces #4
    \par
}

%% Reference to another entry.
\long \def \refentry #1 #2 {
    \start_entry:
    \mark { #1 }

    \textbf { \fontsize{13}{13}\selectfont \ignorespaces #1 } \space
    \(\to\) \space
    \textbf { \ignorespaces #2 }
    .
    \par
}

%% Print errors here.
\msg_new:nnn  { ULTRAFRENCHER } { message } { foo }
\def \ULTRAFRENCHERERROR #1 {
    \msg_set:nnn  { ULTRAFRENCHER } { message } { #1 }
    \msg_error:nn { ULTRAFRENCHER } { message }
}

\ExplSyntaxOff

%% Reference to the current word.
\def\this{\textbf{\char`~}}
\def\senseref#1{sense~{\bf#1}}

%%%%%%%%%%%%%%%%%%%%%%%%%%%%%%%%%%%%%%%%%%%%%%%%%%%%%%%%%%%%%%%%%%%%%%%%
%%            This file was generated from DICTIONARY.txt             %%
%%                                                                    %%
%%                         DO NOT EDIT                                %%
%%%%%%%%%%%%%%%%%%%%%%%%%%%%%%%%%%%%%%%%%%%%%%%%%%%%%%%%%%%%%%%%%%%%%%%%

\entry{a}{pron.}{\pf{quoi}}{\textit{Interrogative and relative}.\\\s{indef} What?\\\s{def} Who? Whom?\\\s{indef} \textit{or} \s{def} Which, who, that \textit{(see grammar)}.}{}
\entry{á}{n.}{\pf{âme}}{Spirit.}{}
\entry{aḅ}{v.}{\pf{appeler}}{To call (+\s{acc} sbd./sth.) (+\s{abs} sbd./sth.). \textit{In PF, this verb used to take a double accusative, but this usage disappeared early on in UF, with the second accusative naturally being replaced by the absolutive, likely to avoid ambiguity that was starting to manifest as a result of UF’s increasingly free word order.} \ex \s{Snet’h v.2} \w{jdap rác’hsaý’adâ} ‘I call you a liar’; even in the writings of \s{Snet’h}, the double accusative is no longer attested.}{}
\entry{ábhec}{v.}{\pf{empêcher}}{+\s{acc} To prevent, stop (sth. from happening).}{\s{fut} ábhece, \s{subj} ábhecs}
\entry{abhérś}{v.}{\pf{apercevoir}}{To behold, descry (+\s{part}).}{}
\entry{aḅrâ}{v.}{\pf{apprendre}}{To learn.}{\s{fut} aḅrâdé, \s{subj} aḅrâs}
\entry{aḅraúc̣}{v.}{\pf{approcher}}{To approach, come near, walk up to (+\s{all} sbd./sth.).}{\s{fut} aḅraúc̣é, \s{subj} aḅraúc̣s}
\entry{aḅrdvê}{adv.}{\pf{après-demain}}{The day after tomorrow. \textit{The prefix \w{aḅr} can be prepended as often as necessary, e.g. \w{aḅraḅraḅrdvê} would be ‘in four days’}.}{}
\entry{ab’há}{conj.}{\pf{avant que}}{+\s{opt} Before.}{}
\entry{áb’há}{n.}{\pf{enfant}}{Child.}{}
\entry{ab’haḍ}{v.}{\pf{abattre}}{\\To cut down, fell, knock down, shoot down.\\To butcher, cut apart violently.}{\s{fut} ab’haḍrẹ́, \s{subj} ab’has}
\entry{ab’hásy’ô}{n.}{\pf{aviation}}{Aviation.}{}
\entry{ab’hèc’h}{v.}{\pf{affecter}}{+\s{acc} To affect, influence.}{\s{fut} ab’hèc’hre, \s{subj} ab’hè\-c’hes}
\entry{áb’hẹḍ}{v.}{\pf{embêtter}}{\\+\s{acc} To disturb, inconvenience sbd.\\+\s{part} To harass, bother sbd.}{}
\entry{ab’héy’}{n.}{\pf{abeille}}{Bee.}{}
\entry{ab’hínéb’heḅaý’évrâ}{v.}{\pf{habit ne fait pas le moi\-ne}}{To judge based on appearances.}{\s{fut} ab’hínéb’heḅaý’év́ẹ́, \s{subj} ab’hínéb’heḅaý’\-év́\-ás}
\entry{áb’hóhẹ}{v.}{\pf{enfoncer}}{To push, press, shove, drive (+\s{ill} into).}{}
\entry{ac}{n.}{\pf{hache}}{Axe, hatchet.}{}
\refentry{ach’es}{\w{a} + \w{c’hes}}
\entry{act’he}{v. tr.}{from \w{ac}}{\\To cut or cleave with an axe.\\+\s{acc} To bring an end to.\\+\s{acc def} \textit{of \w{árb} intr. (other than literal)} To get to the point, cut to the chase.\\+\s{acc def} \textit{of \w{árb} and \s{acc}} To bring to light, reveal. \textit{Originally, this idiom did not take a double \s{acc}, but was instead formed with the \s{acc} of ‘tree’ and the \s{ill} of the object, meaning something along the lines of ‘to bring down the tree(s) on sth’—the image here being that of cutting down trees in a wood until only a clearing remains or is ‘brought to light’}.}{\s{fut} acḍe, \s{subj} act’hes}
\entry{ac̣t’he}{v. tr.}{\pf{acheter}}{To buy.}{\s{fut} ac̣ḍrẹ́, \s{subj} ac̣t’hes}
\entry{aḍrá}{v.}{\pf{attraper}}{\\+\s{acc} \textit{or} \s{part} To take.\\\w{aḍrá faúr} \textit{intr.} To take shape, take form.}{}
\entry{ádróid}{n.}{\pf{androïde}}{Android.}{}
\entry{ady’ŷ}{v. or interj.}{\pf{adieu}}{\\Goodbye, farewell.\\+\s{gen} To say goodbye to sbd., bid sbd. farewell.}{}
\entry{ad’he}{v.}{\pf{vader}}{To go.}{\s{fut} í, \s{subj} al}
\entry{ad’hór}{v. tr.}{\pf{adore}}{\\To love, adore.\\+\s{part} To be in love with, have a crush on.\\+\s{gen} To desire, yearn for sbd./sth.\ex \s{Snet’h iv.17} \w{jad’hóré ávvaúríhe} ‘I yearned to remember’ \textit{(compare \w{jad’hóré devvaúríhe} ‘I loved to remember’)}.}{\s{fut} ad’hórérẹ́, \s{subj} ad’hórs}
\entry{ad’hyl}{v.}{\pf{adulte}}{To be adult, grown-up.}{\s{fut} ad’hyle, \s{subj} ad’hyls}
\entry{ád’hýr}{v.}{\pf{endure}}{To resist, endure, withstand (+\s{acc} sth.).}{}
\entry{áẹ}{n.}{\pf{en-haut}}{Sky. \textit{Often plural, especially in a religious sense.}.}{}
\entry{ah}{n.}{\pf{assez}}{\textit{sufficient comparative prefix; see §~\ref{subsubsec:comparison}}.}{}
\entry{áhaúr}{conj.}{\pf{encore}}{+\s{subj} Even though.}{}
\entry{áhaúr}{adv.}{\pf{encore}}{\\Still, yet \textit{(positive context)}.\\Again \textit{(negative context)}.}{}
\entry{áhâłát’hẹ}{v.}{\pf{ensanglanté}}{To be (very) bloody, bloodstained.}{}
\entry{ahúr}{v.}{\pf{assurer}}{To ensure.}{\s{fut} ahúré, \s{subj} ahúrs}
\entry{aír}{v.}{\pf{hair}}{To hate, abhor, detest, loathe, despise (+\s{acc} sbd./sth.).}{}
\entry{ânb’hé}{adv.}{\pf{en effet}, via metathesis from *\w{âné\-b’he}}{Verily, indeed, in fact.}{}
\entry{ánvé}{v. tr.}{\pf{animer}}{To bring to life, animate.}{}
\entry{árb}{n.}{\pf{arbre}}{Tree.}{}
\entry{árḍihyl}{n.}{\pf{particule}}{Particle.}{}
\entry{áríb’h}{v.}{\pf{arriver}}{To arrive.}{}
\entry{ársl}{v.}{\pf{harceler}}{To attack, assail, beset, bully (+\s{acc} sbd.).}{}
\entry{ârýý’}{v.}{\pf{enrouler}}{To wrap (+\s{acc} around sth.).}{}
\entry{ásy’ê}{v.}{\pf{ancien}}{To be ancient.}{\s{fut} ásy’êr, \s{subj} ásy’ês}
\entry{asý’ýâ}{particle}{\pf{pas absolument}}{Not, no. \textit{Commonly \w{’sý’ýâ} after vowels and verbs. This particle is used only in the indicative; see also \w{sá}, \w{t’hé}}.}{}
\entry{át’hád}{v.}{\pf{entendre}}{To hear, perceive (+\s{part} sbd./sth.).}{\s{fut} át’hádé, \s{subj} át’hás}
\entry{át’has}{v.}{\pf{entasser}}{\\\s{+acc} To heap, accumulate.\\\textit{refl.} To pile up, heap.}{}
\entry{át’hér}{v.}{\pf{enterrer}}{+\s{acc} To bury, inter.}{}
\entry{au}{conj.}{\pf{aussi}}{\\And, also, as well, too.\\\w{au} \ldots{} \w{au} \ldots{} ‘both \ldots{} and \ldots’}{}
\entry{aú}{n.}{\pf{homme}}{Man, human.}{}
\entry{aû}{particle}{\pf{non}}{Not-. \textit{Used to negate nouns, adjectives, and adverbs; see §~\ref{subsubsec:noun-negation}}.}{}
\refentry{aubhaus}{ní}
\entry{aublit’hér}{v.}{\pf{oblitérer}}{\\To defeat, vanquish, obliterate (+\s{acc} sbd./sth.).\\To be better than, ‘beat’ (+\s{inf} sbd/sth.).}{}
\entry{aub’heír}{v. (in)tr.}{\pf{obéir}}{To obey.}{}
\entry{auḍ}{adj.}{\pf{autre}}{Other, another.}{}
\entry{auḍé}{v.}{\pf{obtenir}}{\\To obtain, get, acquire.\\+\s{abl} To gain purchase on or hei\-ght or distance from.}{\s{fut} auḍy’édrẹ́}
\entry{auha}{conj.}{\pf{au cas où}}{+\s{opt} In case.}{}
\entry{aujúrdy’í}{adv.}{\pf{aujourd’hui}}{Today. \textit{Archaic, see also \w{júrdy’í}}.}{}
\entry{aúráj}{n.}{\pf{orage}}{\\\textit{(usually pl.)} Storm, tempest, thunderstorm.\ex \s{Snet’h ii.7} \w{phárýaúráj téríbâ} ‘like a terrible storm’.\\\textit{fig.} Upheaval, turmoil, crisis.}{}
\entry{ausc’hýr}{v.}{\pf{obscur}}{To be dark.}{}
\entry{av́ár}{v. irreg.}{\pf{avoir}}{+\s{acc} To have \textit{(usually inalienably)}.}{\s{pres ant} and \s{pret} y, \textit{obsolete} \s{pret} ab’hẹ, \s{fut} aúrẹ́, \s{subj} ès}
\entry{av́árḷý}{v.}{\pf{avoir lieu}}{To take place, happen.}{\s{fut} lav́árḷýé, \s{subj} lav́árḷýs}
\entry{áv́áy’é}{v.}{\pf{envoyer}}{To send.}{\s{fut} áv́áy’érẹ́, \s{subj} áv́áy’és}
\entry{áví}{n.}{\pf{ami}}{Friend.}{}
\entry{ávrê}{conj.}{\pf{à moins que}}{+\s{opt} Unless.}{}
\entry{aý’aúr}{conj.}{\pf{alors}}{While, as (temporal), because.}{}
\entry{Aý’èc’hsád}{n.}{\pf{Alexandre}}{\textit{Male given name}.}{}
\entry{áł}{v.}{from earlier *\w{ḅał} < \pf{parler}}{To speak.}{}
\entry{áȷ́éd}{v.}{\pf{enjoindre}}{To order, enjoin, command.}{}
\entry{ba}{v.}{\pf{baser}}{To base on, found on.}{\s{fut} bare, \s{subj} bas}
\entry{ḅá nórávíc’h}{n. archaic}{\pf{Panoramix}}{Druid. \textit{Only the \w{nórávíc’h} is inflected; infixing of adj. is attested.} \ex \s{Snet’h}, \s{iii.2}: \w{derúb’h phá ráinórávíc’h} ‘to find the great druid’, with infixed \w{rá}.}{}
\entry{baḍ}{v.}{\pf{battre}}{To beat, strike, hit (+\s{acc} sbd./sth.).}{}
\entry{ḅáhẹ}{n.}{\pf{pensée}}{Thought, reflection, meditation, faculty of thinking.}{}
\entry{ḅaj}{n.}{\pf{page}}{Page.}{}
\entry{ḅará}{n.}{\pf{parent}}{Parent.}{}
\entry{ḅarḍ}{v.}{\pf{partir}}{To leave, go away, depart.}{\s{fut} ḅarẹ́, \s{subj} ḅars}
\entry{ḅárḍáḍ}{v.}{\pf{partante}}{(+ \s{aci}) To be interested in, willing to, ready to, prepared for.}{}
\entry{ḅárḍẹ}{n.}{\pf{partie}}{Part, portion, piece, faction of a whole.}{}
\entry{ḅárḍihibhá}{n.}{\pf{participant}}{Participant.}{}
\entry{ḅáréḍ}{v.}{\pf{parraitre}; future stem from \pf{sembler}}{To seem, appear.}{\s{fut} sáb}
\entry{ḅas}{conj.}{\pf{parce que}}{+\s{subj} Because. \textit{Often used to explain motivation rather than cause, as in e.g. ‘We did that because\ldots’}.}{}
\entry{baú}{v. irreg.}{\pf{bon}}{\\To be good, well, healthy.\\To be right, correct, appropriate.\\\textit{usually intr.} To satisfy, fullfill, gratify.}{\s{fut} baúré, \s{subj} véy’ýrs; \s{comp} lẹvéy’ýr, y’ŷvéy’ýr, rêvéy’ýr; \s{sup} révéy’ýr, râdvâv\-éy’\-ýr}
\entry{ḅaú}{n.}{\pf{pont}}{Bridge.}{}
\entry{ḅauheŷnlabhé}{v.}{\pf{poser un lapin}}{To forsake, abandon.}{\s{fut} ḅauheŷnlabhére, \s{subj} ḅauheŷnlabhés}
\entry{ḅauhib}{v.}{\pf{impossible}}{To be impossible, unfeasible.}{\s{fut} ḅauhibre, \s{subj} ḅauh\-ibes}
\entry{Baúré}{n.}{\pf{Borée}}{Boreas, the North Wind.}{}
\entry{ḅáł}{v.}{\pf{parler}}{To speak, talk.}{\s{fut} báłérẹ́}
\entry{ḅáłýr}{n.}{\pf{parleur}}{Speaker, interlocutor.}{}
\entry{ḅelbec}{n.}{\pf{pelle} + \pf{bêche}}{Shovel.}{}
\entry{Bèrḍrá}{n.}{\pf{Bertrand}}{\textit{Male given name}.}{}
\entry{ḅẹt’hẹ}{v. irreg.}{\pf{petit}}{To be small, little.}{\s{fut} rêdẹ́, \s{subj} ḅẹt’hes; \s{comp} lẹrêd, y’ŷrêd, rêrêd; \s{sup} rérêd, râdvârêd}
\entry{ḅét’hýr}{v.}{\pf{peinture}}{To paint.}{}
\entry{ḅéy’í}{n.}{\pf{pays}}{Country, land, region, nation.}{}
\entry{ḅínár}{n.}{\pf{pinard}}{Wine.}{}
\refentry{bír}{vaúb’hẹ}
\entry{biwaú}{n.}{\pf{billion}}{\textit{(obsolete)} Billion (long scale, i.e. $10^{12}$). \textit{Replaced with modern \w{dýwaú})}.}{}
\entry{ḅré}{conj.}{\pf{après que}}{+\s{opt} After.}{}
\entry{ḅusy’ér}{n.}{\pf{poussière}}{Dust.}{}
\entry{bźé}{v.}{\pf{besoin}}{+\s{acc} \textit{or} \s{part} To need, require.}{}
\entry{b’há}{n.}{\pf{vent}}{Wind, breeze.}{}
\entry{b’hár}{n.}{\pf{vague}}{\\Wave.\\\textit{pl.} Ripples, undulations.}{}
\entry{b’hát’hiý’at’hýr}{v.}{\pf{ventilateur}}{To blow.}{}
\entry{b’hauḍ}{v.}{\pf{vôtre}}{To be yours (\s{pl}).}{\s{fut} b’hauḍre, \s{subj} b’haus}
\entry{b’haul}{v.}{\pf{voler}}{To hover, float.}{}
\entry{b’hây’ér}{adv.}{\pf{avant-hier}}{The day before yesterday. \textit{The prefix \w{b’hâ} can be prepended as often as necessary, e.g. \w{b’hâb’hâb’hây’ér} would be ‘four days ago’}.}{}
\entry{b’he}{conj.}{\pf{envers}}{+\s{subj} So that, so as to, to, in order to. \textit{Commonly enclitic \w{’b’h} after vowels}.}{}
\entry{b’hé}{n.}{\pf{vin}}{Grape.}{}
\entry{b’hénvâ}{n.}{\pf{évènement}}{Event, occurrence.}{}
\entry{b’hérḍy’ŷ}{v.}{\pf{vertueux}}{To be virtuous.}{}
\entry{b’hért’he}{n.}{from \pf{vérité}; the /i/ disappeared during Early Middle UF}{Truth.}{}
\entry{b’héy’}{v.}{\pf{veiller}}{\\\textit{intr.} To keep watch, keep guard.\\+\s{spress} To watch over, guard, keep an eye on.}{}
\entry{b’heý’au}{n.}{from archaic \w{b’heý’auhic’h}}{Bicycle.}{}
\entry{b’heý’auhic’h}{n. archaic}{\pf{vélo-cycle}}{Bicycle.}{}
\entry{b’heý’o}{v.}{back-formation from *\w{b’heý’os}, reanalysed as a subjunctive; from \pf{véloce}}{To be quick, fast.}{\s{fut} b’heý’o\-se, \s{subj} b’heý’os}
\entry{b’hí}{n.}{\pf{vigne}}{Vine.}{}
\entry{b’hic’hḍrár}{n.}{\pf{victoire}}{Victory.}{}
\entry{b’hid}{v.}{\pf{vide}}{To be empty.}{}
\entry{b’hizy’ô}{n.}{\pf{vision}}{Vision.}{}
\entry{b’hóy’ẹ}{v.}{\pf{voler}}{To fly. Flight.}{}
\entry{b’huḍ}{n.}{\pf{voûte}}{Vault, arched ceiling.}{}
\entry{b’hýlnẹ́r}{v.}{\pf{invulnérable}}{+\s{instr} To be incapable of being af\-fec\-ted by, invulnerable to.}{\s{fut} b’hýlnẹ́rẹ́, \s{subj} b’hýlnẹ́rs}
\entry{b’hŷnnúb’hâ}{adv.}{old \s{all} of \w{núb’hâ}}{Anew.}{}
\entry{caḍráy’ẹ́}{v.}{\pf{chatoyer}}{To shimmer, iridesce.}{}
\entry{cah}{v.}{\pf{chasser}}{To hunt.}{\s{fut} cahe, \s{subj} cas}
\entry{cahý}{pron. pl. indef.}{\pf{chacun}}{Each other, one another.}{}
\entry{Cár}{n.}{}{\textit{Male given name, equivalent to English ‘Kyle’ or ‘Charles’. Often declined like a regular noun, e.g. \s{nom} \w{Lác̣ár}}.}{}
\entry{cẹ}{v.}{\pf{chaud}}{To be hot.}{}
\entry{će}{v.}{\pf{échouer}}{\\+\s{part} To stumble, do a bad job at.\\+\s{acc} \textit{or} \s{aci} To fail, flunk, not pass.}{\s{fut} ćere, \s{subj} ćes}
\entry{cèc}{n.}{phonetic respelling of \w{cèc’h}}{\textit{(chess)} Check.}{}
\entry{cèc’h}{n.}{\pf{échec}}{Failure, defeat.}{}
\entry{cér}{v.}{\pf{cher}}{\\To be dear, important (+\s{dat} to sbd.). \textit{Possession of a noun qualified with this adjective verb is generally construed with the dative rather than the genitive, e.g. \w{asvẹ áví cérâ} or \w{áví cérâvé} ‘my dear friend’, rather than *\w{vaú áví cérâ}}.\\\textit{with \w{áví} ‘friend’} To be friends with.\ex \w{áví lẹcérvé} ‘he is a (dear) friend of mine’.}{\s{fut} céré, \s{subj} cés}
\entry{cévê}{n.}{\pf{chemin}}{Street.}{}
\entry{c’habhahit’hẹ}{n.}{\pf{capacité}}{Skill, capacity, ability.}{}
\entry{c’hánár}{n.}{\pf{canard}}{\\Ship, boat.\\\s{instr indef} By boat.}{}
\entry{c’hánaú}{n.}{\pf{canot}}{Duck (bir).}{}
\entry{c’háraúciḍ}{v.}{\pf{les carrotes sont cuites}}{To end for good, put to a permanent end.}{\s{fut} c’hár\-aúc\-re, \s{subj} c’háraúc}
\entry{c’hasḅesy’ál}{n.}{cas spécial}{Exception.}{}
\entry{c’haú}{adj.}{see sense 2}{\\Holy.\\\w{c’haú-}\L{} \textit{‘religious prefix’, prepended in derivation to nouns that have a religous connotation; this is historically a back-formation from \w{c’haúfrér} and \w{c’haúhýr} which happen to both start with this ‘prefix’}.}{}
\entry{c’haúáł}{n.}{\w{c’haú} + \w{áł}}{Prophecy.}{}
\entry{c’haúbhárrás}{n.}{\w{c’haú} + \pf{paroisse}}{Parish.}{}
\entry{c’haúbhausy’ô}{n.}{\pf{composition}}{Composition, arrangement, structure.}{}
\entry{c’haúbhèłínáj}{n.}{\w{c’haú} + \pf{pèlerinage}}{Pilgrimage.}{}
\entry{c’haúbhýríf}{n.}{\w{c’haú} + \pf{purifier}}{To purify (+\s{acc} sbd./sth.).}{}
\entry{c’haúḅlér}{.v}{\pf{complaire}}{To be complacent; to be accepting in the presence of +\s{gen} sbd./sth. perceived as negative.}{\s{fut} c’haúḅlére, \s{subj} c’haúḅlés}
\entry{c’haúḅrâd}{v.}{\pf{comprendre}}{+\s{part} To comprehend, understand, gr\-asp.}{\s{fut} c’haúḅrâdrẹ́, \s{subj} c’haúḅrâs}
\entry{c’haúb’héc’h}{v.}{\pf{convaincre}}{To persuade.}{}
\refentry{c’haúḍé}{ní}
\entry{c’haúḍrêd’hẹ}{n.}{\pf{compte-rendu}}{Account, rec\-ord.}{}
\entry{c’haúfí}{v.}{\pf{confiner}}{To contain.}{}
\entry{c’haúfrér}{n.}{\pf{confrère}}{Brother (religious). \textit{Masc. or pl. only, see also \w{c’haúhýr}}.}{}
\entry{c’haúhaúvnaút’hẹ}{n.}{\w{c’haú} + \pf{com\-mu\-nau\-té}}{Mo\-nastery.}{}
\entry{c’haúhýr}{n.}{\pf{consœur}}{Sister (religious). \textit{Fem. only, see also \w{c’haúfrér}}.}{}
\entry{c’haúnéhás}{.n}{\pf{connaissance}}{Knowledge.}{}
\entry{c’haúr}{conj.}{\pf{car} + \pf{comme}}{+\s{subj} As, because, since.}{}
\entry{c’haúv́ájẹ}{n.}{\w{c’haú} + \pf{magie}}{Magic.}{}
\entry{c’haúvnaút’hẹ}{n.}{\pf{communauté}}{Community.}{}
\entry{c’hd’hal}{adv.}{\pf{que dalle}}{Naught, absolutely no\-thing.}{}
\entry{C’hebèc’h}{n.}{\pf{Québec}}{The Promised Land.}{}
\entry{c’hèl}{det. postpos.}{\pf{quelques}}{Some, a few, a couple of.}{}
\entry{c’hèlc’hý}{pron.}{\pf{quelqu’un}}{Someone, somebody, anyone, anybody.}{}
\entry{c’hes}{quest. part.}{\pf{qu’est-ce que}}{\textit{see grammar; often \w{c’h’s} in older texts}.}{}
\entry{c’hesse}{}{contraction of \w{c’hes} + \w{se}}{Is it? \textit{Also substituted for other forms of ‘to be’ in questions, particularly for the plural neuter; stressed on the first syllable}.}{}
\entry{c’hlýr}{v.}{\pf{inclure}}{\\+\s{part} To include.\\+\s{acc} To possess, have \textit{(alienably)}.\\+\s{gen} \textit{usually} \s{indef} To sell, offer, have in stock.}{\s{fut} c’hlýré, \s{subj} c’hlýrs}
\entry{c’hóbhár}{v.}{\pf{comparer}}{To compare.}{\s{fut} c’hóbhárre, \s{subj} c’hó\-bhárs}
\entry{c’hóhid’hẹ́}{v.}{\pf{considérer}}{\\+\s{part} To consider, think ab\-out, ponder.\\+\s{acc} To think through.}{\s{fut} c’hóhid’hẹ́rẹ́, \s{subj} c’hóhid’hés}
\entry{c’hóný}{adj.}{\pf{connu}}{Known, familiar, well-kn\-own.}{}
\entry{c’hór}{n.}{\pf{corps}}{Body.}{}
\entry{c’hóvâ}{v.}{\pf{commencer}}{\\(\s{+ part}) To start, commence, begin.\\\s{+ gen} To start out as.\\\w{âc’hóvâ} \textit{def.} Beginning, start. \textit{Lit. ‘that which is being begun’}.}{\s{fut} c’hóvârẹ́, \s{subj} c’hóv\-ás}
\entry{c’hóvníc’h}{v.}{\pf{communiquer}}{\\To communicate (+\s{instr} with sbd.).\\Communication.}{\s{fut} c’hóvníc’hre, \s{subj} c’hóvníc’hes}
\entry{c’hrír}{v.}{\pf{écrire}}{To write.}{\s{fut} c’hrírẹ́, \s{subj} c’hrís}
\entry{c’hulvâ}{n.}{\pf{écoulement}}{Flow.}{}
\entry{c’húr}{v.}{\pf{court}}{To shrink, reduce in size, narrow.}{}
\entry{c’húr}{v.}{\pf{courrir}}{To run.}{}
\entry{c’hýr}{n.}{\pf{cœur}}{Heart.}{}
\refentry{c’h’s}{c’hes}
\entry{dá}{n.}{\pf{dent}}{Tooth.}{}
\entry{ḍá}{conj.}{\pf{tandis}}{Whereas.}{}
\entry{ḍád}{n.}{\pf{stand}}{Stand, stall, booth.}{}
\entry{dahaúr}{particle}{\pf{d’accord}}{Sure, ok, agreed, fine.}{}
\entry{ḍalẹ}{n.}{\pf{tableau}}{Table.}{}
\entry{ḍalisvâ}{n.}{\pf{établissement}}{Establishment, institution.}{}
\entry{dár}{v.}{\pf{darder}}{To throw, cast, yeet (+\s{acc} sth.).}{}
\entry{ḍaú}{n.}{\pf{tonne}}{Weight.}{}
\entry{daú(c’h)}{particle}{\pf{donc}}{Therefore, then, thus.}{}
\entry{ḍaúb’h}{v. intr.}{\pf{tomber}}{To fall, drop.}{}
\entry{daúb’hedwébhó}{v.}{\pf{tomber dans les pommes}}{To faint.}{\s{fut} daúb’hedwébhóre, \s{subj} daúb’hedwébhós}
\entry{ḍauḍ}{def. pron.}{from earlier \w{ḍẹ auḍ}}{Everything else, any other (one).}{}
\entry{Daúvníc’h}{n.}{}{\textit{male or female given name, equivalent to English ‘Dominic’}.}{}
\refentry{daú’b’h}{daú(c’h) \textnf{+} b’he}
\entry{db’hid’h}{n.}{\pf{individu}}{Person, individual.}{}
\entry{de}{conj.}{\pf{dès que}}{+\s{subj} Once, when once, as soon as.}{}
\entry{dẹ́}{particle}{from \w{Provençal} \textit{den}}{Then (sequential), next.}{}
\entry{ḍẹ}{adj.}{\pf{tout}}{\\All, every, whole, entire.\\\w{ḍẹ auḍ} Obsolete form of \w{ḍauḍ}.}{}
\entry{deḅlér}{v.}{\pf{déplaire}}{To displease (+\s{acc} sbd.), be displeasing.}{}
\entry{dẹb’hní}{v.}{\pf{devenir}}{To become, turn into (+\s{transl} sth./sbd.). \textit{The subject is in the \s{abs} case}.}{}
\entry{dec̣ír}{v.}{\pf{déchirer}}{\\+\s{part} To tear, rip, rend.\\+\s{acc} To rend asunder, tear to pieces.}{\s{fut} dec̣irrẹ, \s{subj} dec̣írs}
\entry{dèc’h}{adj.}{\pf{dextre}}{Right (side), right-handed.}{}
\entry{ḍèc’hníc’hvâ}{adv.}{\pf{techniquement}}{Technically.}{}
\entry{ḍédv́ér}{interj.}{\pf{putain de merde}}{Fuck. \textit{Generic expletive}.}{}
\entry{dẹh}{v.}{\pf{dessous}}{To be below, beneath.}{}
\entry{dehab’híy’}{v. tr.}{\pf{déshabiller}}{To undress +\s{acc} sbd.}{}
\entry{dẹhẹ}{n.}{\pf{dessus}}{\\Top, upper side.\\Surface of a body of water.}{}
\entry{dehid}{v.}{\pf{décider}}{To decide (+\s{inf} to do sth.).}{}
\entry{dej}{particle}{from \w{dejẹ}}{\textit{Emphatic particle; only used in the preterite}.\\\s{pret} + \w{dej} \textit{roughly} To have ever done sth.}{}
\entry{dejẹ}{adv.}{\pf{déjà}}{Already.}{}
\entry{ḍèl}{particle}{\pf{tel}}{\textit{Emphatic particle, used as an intensifier, often postpositive after the verb, but not so much intensifying the verb directly as it does the entire clause.} \ex \s{Snet’h}, \s{ii.34}: \w{lá-árb srýlé dèl} ‘so it was that the tree was burning’ or ‘the tree was burning fiercely’, or ‘indeed, the tree was burning’.}{}
\entry{ḍénéb}{n. pl.}{\pf{ténèbres}}{\textit{exclusively plural \w{lḍénéb}} Darkness.}{}
\entry{ḍèr}{v.}{\pf{taire}}{To silence, shut up.}{\s{fut} ḍérẹ́}
\entry{ḍéraúj}{v.}{\pf{interroger}}{To demand.}{}
\entry{ḍérésḍ}{v.}{\pf{terrestre}}{To be terrestrial, earth-based.}{\s{fut} ḍérésḍrẹ́, \s{subj} ḍérésḍs}
\entry{ḍẹ́ríb}{v.}{\pf{terrible}}{To be terrible (all senses).}{\s{fut} ḍẹ́ríre, \s{subj} ḍẹ́rís}
\entry{dérny’é}{adj.}{\pf{dernier}}{Last, final, ultimate.}{}
\entry{dérny’ẹ́huf}{n.}{\pf{dernier} + \pf{souffle}}{Death.}{}
\entry{ḍérsèd}{v.}{\pf{intercéder}}{To intercede.}{}
\entry{ḍèrvíc’h}{n.}{\pf{thermique}}{Heat, warmth.}{}
\entry{deslẹ}{v.}{\pf{déceler}}{To detect, discover, uncover, reveal.}{\s{fut} deslẹre, \s{subj} deslẹs}
\entry{dévýr}{v.}{\pf{demeurer}}{\\To remain, stay.\\To live, dwell (+\s{iness} somewhere).}{}
\entry{ḍeý’ebhat’hẹ}{n.}{\pf{télépathie}}{Telepathy.}{}
\entry{ḍeý’ebhat’hic’h}{v.}{\pf{télépathique}}{To be telepathic.}{\s{fut} ḍeý’ebh\-at’hic’hre, \s{subj} ḍeý’ebhat’hic’hes}
\entry{dír}{v. tr.}{\pf{dire}}{+\s{acc} To say, tell (+\s{dat} someone).}{\s{fut} dírẹ́, \s{subj} díss}
\entry{díríj}{v.}{\pf{diriger}}{+\s{acc} To direct, run, oversee, operate (a business or establishment).}{\s{fut} díríje, \s{subj} díríjs}
\entry{dónẹ́}{v.}{\pf{donner}}{\s{+ dat \& acc/part} To endow, bestow, give. \textit{The \s{acc} is used when talking about concrete, measurable, and finite objects or sums; the partitive to talk about abstract concepts or parts of objects; the \s{dat} is the person being endowed with}.}{\s{fut} dónrẹ́, \s{subj} dónés}
\entry{ḍúr}{adv.}{\pf{toujours}}{\\\textit{(positive context)} Always.\\\textit{(negative context)} Still.}{}
\entry{ḍúr}{n.}{\pf{tour}}{Tower.}{}
\entry{duý’ýr}{v.}{\pf{douleur}}{To suffer, be in pain.}{}
\entry{dývrê}{particle}{\pf{du moins}}{At least. \textit{As in e.g. ‘At least, I think that \ldots’}.}{}
\entry{dy’ê}{v.}{\pf{tien}}{To be yours (\s{sg}).}{\s{fut} dy’êrẹ́, \s{subj} dy’ês}
\entry{e}{n.}{\pf{eau}}{Water.}{}
\entry{ebhẹ}{v.}{\pf{épais}}{To be thick.}{\s{fut} ebhrẹ, \s{subj} ebhes}
\entry{ec̣}{n.}{\pf{péché}}{Sin, transgression, wrongdoing.}{}
\entry{ec’hlér}{v.}{\pf{éclairer}}{To shine.}{}
\entry{ed}{particle}{\pf{et}}{\textit{Used in numbers, see §~\ref{subsec:numerals}}.}{}
\entry{eḍ}{v. irreg.}{\pf{être}}{To be.}{\s{forms} \textit{active only, see §~\ref{subsec:ed-paradigm}}}
\entry{eḍrrá}{v.}{\pf{étroit}}{Pointy.}{}
\entry{Eḍy’ê}{n.}{}{\textit{male given name, equivalent to English ‘Ste\-phen’}.}{}
\entry{ehyó}{n.}{\pf{écusson}}{Shield.}{}
\entry{el}{n.}{\pf{ailles}}{Wing, blade, fin.}{}
\entry{ez-}{pron.}{\pf{ses}}{Its, her, his, their.}{}
\entry{F}{adj.}{from \pf{fẹ}}{\textit{Logic.} False, $\bot$. \textit{Always capitalised}.}{}
\entry{fahaú}{conj.}{\pf{de façon que}}{+\s{opt} In such a way that, such that, so much so.}{}
\entry{faú}{adv.}{\pf{fort}}{Very, right, really. \textit{Postpositive intensifier placed after adjectives, particularly in the comparative or superlative degrees.}.}{}
\entry{faúr}{adj.}{\pf{fort}}{\textit{obsolete, except in proverbs} Strong, mighty.}{}
\entry{faúr₁}{n.}{\pf{force}}{\\Force, strength, power.\\\w{Faúr} \s{def} \textit{(science fiction, Star Wars)} The Force.}{}
\entry{faúr₂}{n.}{\pf{forme}}{Shape, form. \textit{Sometimes also spelt \w{fór}}.}{}
\entry{fé}{n.}{\pf{fin}}{End.}{}
\entry{fẹ}{v.}{\pf{faux}}{To be false, incorrect, wrong.}{\s{fut} faure, \s{subj} faus}
\entry{fẹhab}{v.}{\pf{faisable}}{To be possible, feasible.}{\s{fut} fẹhabre, \s{subj} fẹhas}
\entry{fèhẹ}{n.}{\pf{faisceau}}{\\Bundle, bunch, cluster.\\Beam, ray.}{}
\entry{fér}{v.}{\pf{faire}}{\\To do, make, build, construct, erect.\\\textit{Expletive; see §~\ref{subsubsec:personal-pronouns}}.}{\s{fut} fẹ́, \s{subj} fés}
\entry{férḍufraú}{v.}{\pf{en faire tout un fromage}}{To make a big fuss a\-bout something.}{\s{fut} fér\-ḍu\-fraúrẹ́, \s{subj} férḍufraús}
\entry{férr-rásvát’h}{n.}{\pf{faire la grasse mat’}}{A long, deep sleep.}{}
\entry{fic’h}{v.}{back-formation from *\w{fic’hs}, reinterpreted as a subjunctive stem; from \pf{fixer}}{To fix, set, establish.}{\s{fut} fic’hre, \s{subj} fic’hs}
\entry{fihas}{v.}{\pf{efficace}}{To be efficient.}{}
\refentry{fór}{faúr₂}
\entry{fórvẹ́}{v.}{\pf{informer}}{To inform (+\s{acc} sbd.) (+\s{aci} of sth.).}{\s{fut} fórv́ẹ́, \s{subj} fórvẹ́s}
\entry{fúr}{v.}{\pf{fournir}}{To deliver, provide (+\s{dat} sbd.) (+ \s{acc} with sth.).}{}
\entry{fý}{n.}{\pf{feu}}{Fire.}{}
\entry{hab’híy’}{v. tr.}{\pf{habiller}}{To dress +\s{acc} sbd.}{}
\entry{í}{n.}{\pf{hymne}}{Legend, myth.}{}
\refentry{ís}{ub’hrá}
\entry{isḍrár}{n.}{\pf{histoire}}{Story, tale.}{}
\entry{íváj}{n. archaic}{\pf{image}}{Image, picture.}{}
\entry{iý’ývî}{v.}{\pf{illuminer}}{To light up, illuminate.}{}
\entry{Já}{n.}{}{\textit{male or female given name, equivalent to English ‘John’ or ‘Joan’}.}{}
\entry{Jac’h}{n.}{\pf{Jacques}}{\textit{Male given name}.}{}
\entry{jávé}{adv.}{\pf{jamais}}{\textit{neg. only} Never, at no time.}{}
\entry{jaý’aú}{n.}{\pf{jalon}}{\\Nail.\\\textit{obsolete} Stake, pole.}{}
\entry{Jed’háy’}{n.}{\pf{Jedi}}{Jedi (Star Wars).}{}
\entry{júrdy’í}{adv.}{from archaic \pf{aujúrdy’í}}{Today.}{}
\entry{jys}{conj.}{\pf{jusqu’à ce que}}{+\s{opt} Until.}{}
\entry{jys}{adv.}{\pf{juste}}{Just, only, merely.}{}
\entry{jys}{v.}{\pf{injuste}}{To be unjust, unfair.}{\s{fut} jysre, \s{subj} jyss}
\entry{lá}{v.}{\pf{planer}}{To fly.}{}
\entry{lab’h}{v.}{\pf{laver}}{\\To wash, clean (+\s{acc} sth.).\\\textit{refl} To wash oneself, take a bath, have a shower.}{}
\entry{Lác}{n.}{}{\textit{female given name, equivalent to English ‘Bi\-anca’}.}{}
\entry{láḍ}{n.}{\pf{plante}}{\\Blade of grass.\\\textit{pl.} Grass.}{}
\entry{lánẹ́}{v.}{\pf{flâner}}{To meander.}{}
\entry{lár}{v.}{\pf{large}}{Wide, broad.}{}
\entry{lârdávrá}{n.}{\pf{langue de bois}}{Evasive, unclear, or overly formal speech.}{}
\entry{las}{v.}{\pf{placer}}{To place, put, set (+\s{acc} sth.).}{}
\entry{laú}{v.}{\pf{long}}{Long. \textit{Often in compounds \w{laú-} ‘long-’}.}{}
\entry{laúrs}{conj.}{\pf{lorsque}}{When (temporal only).}{}
\entry{laúrvé}{conj.}{from \w{laúrs} + \w{vé}}{\textit{(contraction)} But, when. \textit{Stressed on the first syllable}.}{}
\entry{laut’h}{v.}{\pf{flotter}}{Fl\-oat, hover, levitate.}{\s{fut} laut’hre, \s{subj} laut’hes}
\entry{le}{v.}{\pf{laisser} > *\w{lehe}}{\textit{Chiefly in questions or imperative.} To let, allow, permit.}{\s{fut} lere, \s{subj} les}
\entry{lé}{n.}{\pf{plaine}}{Plain, plains.}{}
\entry{lẹ-}{prefix}{\pf{plus}}{\textit{Affirming comparative prefix. See grammar}.}{}
\entry{lec’hḍraúvnẹ́t’hic’h}{v.}{\pf{électromagnétique}}{To be electromagnetic.}{\s{fut} lec’hḍraúvnẹ́t’hic’hre, \s{subj} lec’hḍraúvnẹ́t’hic’hes}
\entry{leḍ}{n.}{\pf{lettre}}{\\Letter (of the alphabet).\\\w{lý’aúleḍ} By the book.}{}
\entry{lèheb’h}{v.}{\pf{laisser-faire}}{+\s{aci} To let happen.}{}
\entry{lẹhuvud}{n.}{\pf{coup de foudre}}{Love at first sight.}{}
\entry{lér}{v.}{\pf{clair}}{To be evident, obvious, frank, clear.}{}
\entry{lér}{v.}{\pf{plaire}}{To please (+\s{acc} sbd.), be pleasing.}{}
\entry{lí}{v.}{\pf{lire}}{\\+\s{part} To read from.\\+\s{acc} To peruse, read entirely.}{\s{fut} lírẹ́, \s{subj} lís}
\entry{lit’hijy’}{v.}{\pf{litigier}}{To litigate, be at law with (+\s{dat} sbd.).}{}
\entry{lívnád}{n.}{\pf{limonade}}{Lemonade.}{}
\entry{liv́uhé}{n.}{\pf{livre} + \pf{bouquin}}{Book.}{}
\entry{lúr}{v.}{\pf{lourd}}{To be bulky, oversized, heavy.}{}
\entry{ly}{particle}{\pf{plus}}{\textit{obsolete variant of \w{lẹ}, sometimes leniting}.}{}
\entry{lý}{n.}{\pf{plume}}{Pen, quill.}{}
\entry{ḷý}{n.}{\pf{lieu}}{\textit{Base of the spatial correlatives. In senses 2–5, case affixes are attached before \this, e.g. sense 2 \s{all} \w{sẹb’héḷý} ‘hither’}.\\Place, location.\\\w{sẹ}\ldots\this \s{def} [from \w{sẹh}] Here, hither, hence, \&c. \textit{Proximal demonstrative (all cases)}.\\\w{sý’\L}\ldots\this \s{def} [from \w{sý’ẹ}] There, thither, thence, \&c. \textit{Distal demonstrative (all cases)}.\\\this\w{hes} \s{indef} [from \w{c’hes}] Where, whither, when\-ce, \&c. \textit{Interrogative (locative cases only)}.\\\w{s’}/\w{sá\L}\ldots\this \s{indef} [from \w{sá}] No\-where, from no\-where, \&c. \textit{Negative (locative cases only)}.}{}
\entry{lybhárdyt’há}{adv.}{pluspart du temps}{Often.}{}
\entry{lýr}{pron.}{\pf{leur}}{Their.}{}
\entry{lýrḍ}{v.}{from \pf{leur}; the ‘ḍ’ was added in analogy with \w{naúḍ} and \w{b’hauḍ}}{To be theirs.}{\s{fut} lýrḍre, \s{subj} lýrs}
\entry{lys}{adv.}{\pf{plus} /plys/}{\textit{neg. only} No longer, not any more. \textit{The meaning of this and \w{lẹ} swapped at some point for unknown reasons}.}{}
\entry{lýv́á}{v. \s{3rd} person only}{\pf{pleuvoir}}{To rain.}{\s{fut} lýv́áre, \s{subj} lýv́ás}
\entry{lývy’ér}{n.}{\pf{lumière}}{Light.}{}
\entry{lyzy’ýr}{adj.}{\pf{plusieurs}}{Several.}{}
\entry{n}{n.}{\pf{haine}}{Hate, hatred, loathing.}{}
\entry{nájẹ}{v.}{\pf{nager}}{To swim.}{\s{fut} náȷ́ẹ, \s{subj} nájes}
\entry{nárrahóḍ}{v.}{\pf{raconter} + \pf{narrer}; subj. from \pf{filer}}{To narrate, recount \s{+part sth.}, tell (\s{+dat} sbd.) a story.}{\s{fut} nárrahóḍe, \s{subj} fils}
\entry{nát’hýr}{n.}{\pf{nature}}{\\\textit{(chiefly)} \s{indef} Nature, the natural world.\\\s{def} The way something is.}{}
\entry{naúḍ}{v.}{\pf{nôtre}}{To be ours.}{\s{fut} naúḍre, \s{subj} naús}
\entry{néḍ}{v. dep.}{\pf{naître}}{To be born. \s{This is a deponent verb whose subject takes the \s{acc} and which only takes passive affixes.}.}{\s{sub} néhs}
\entry{nérjẹ}{n.}{\pf{énergie}}{Energy.}{}
\entry{nés}{adj.}{from earlier \w{nésḍ}}{Left (side), left-handed.}{}
\entry{nésḍ}{adj. archaic}{\pf{senestre}}{Left (side), left-handed.}{}
\entry{ní}{v.}{\pf{nier}, \s{fut} from \pf{contrer}, \s{subj} from \pf{oposer}}{To deny, ref\-use, reject, rebut (+\s{acc} sbd./sth.).}{\s{fut} c’haúḍé, \s{subj} aubhaus}
\entry{ní}{conj.}{\pf{ni}}{Neither, nor.}{}
\entry{níb’hẹ}{n.}{\pf{niveau}}{\\Level, degree.\\\s{def iness + gen} On the level of.}{}
\entry{nór}{v.}{back-formation from *\w{nórâ} from \pf{ignorant}}{To be ignorant, unaware, oblivious.}{}
\entry{nóráv}{n.}{from archaic \w{ḅá nórávíc’h}}{Druid.}{}
\entry{núb’h}{v.}{\pf{nouveau}}{To be new.}{\s{fut} núb’he, \s{subj} núb’hs}
\refentry{p-}{ḅ-}
\refentry{ph-}{ḅ-}
\refentry{p’h-}{bh-}
\entry{R}{adj.}{from \pf{ré}}{\textit{Logic.} True, $\top$. \textit{Always capitalised}.}{}
\entry{r}{n.}{\pf{air}}{Air. \textit{Frequently plural}.}{}
\entry{ra}{conj.}{\pf{swa} > *\w{rá}}{\\Or. \textit{exclusive, see also~\w{u}}.\\\w{u}/\w{ra} \ldots\ \w{ra} \ldots\ ‘either \ldots\ or \ldots’ \textit{(exclusive)}.}{}
\entry{râ}{v.}{\pf{gagner}}{To win, gain, earn (+\s{acc} sth.).}{}
\entry{rá₁}{n.}{\pf{loi}}{Law, rule, regulation.}{}
\entry{rá₂}{adj.}{\pf{grand}}{Big, large, great.}{}
\entry{rá₃}{n.}{\pf{mois}}{Month.}{}
\entry{rá₄}{n.}{\pf{voix}}{Voice.}{}
\entry{rá₅}{n.}{\pf{bras}}{Arm.}{}
\entry{Ráb’h}{n.}{unknown; presumably the name of some celebrity or local deity}{\\\textit{indecl.} \s{def sg} \textit{always} \s{nom} \textit{or} \s{voc} Ráb’h. \textit{Main god of the ULTRAFRENCH pantheon; usually male. Old-fashioned also often all-caps \w{RÁB’H}.}.\ex \s{Snet’h}, \s{i.17}: \w{au lebálá daú RÁB’H} ‘and thus spake Ráb’h’.\\\w{Ráb’h sénýr} \s{def sg} Lord Ráb’h. \textit{Used for sense~{\bf 1} in all other cases; as with all names, only \w{sénýr} is inflected. Old-fashioned often \w{RÁB’H Sénýr}}.\ex \s{Snet’h}, \s{8.1}: \w{au labraúc RÁB’H naút B’héhénýr} ‘and they came to our Lord Ráb’h’.\\\textit{(rarely)} The main god of another culture. \textit{Only attested figuratively. Not capitalised in this sense, and declined like a regular word.}.\ex \s{Snet’h}, \s{ii.3}: \w{ledéraújá’z derévôt’he láráb’h} ‘their god demanded they return’.}{}
\refentry{ráb’h}{v́ár}
\entry{ráb’háy’}{v.}{\pf{travailler}, \s{fut} and \s{subj} from \pf{bos\-ser}}{To work.}{\s{fut} bohér, \s{subj} bos}
\entry{rác’hánár}{n.}{from \w{ráhe} + \w{c’hánár}}{Airship, dirigible.}{}
\entry{rác’hsaý’ad}{v.}{\pf{raconter des salades}}{To lie, tell tall tales, overexaggerate.}{\s{fut} rác’h\-sa\-ý’e, \s{subj} rác’hsaýs}
\entry{rád}{v. tr.}{\pf{rendre}}{To surrender +\s{acc} sth. (\s{dat} to sbd.).}{}
\entry{râd}{v.}{\pf{prendre}}{+\s{acc} \textit{or} \s{part} To grab. \textit{The \s{part} usually implies that only a part or some of a larger whole is grabbed, e.g. a handful of sand)}.}{}
\entry{râdrásôn}{v.}{\pf{prendre ses jambe à son cou}}{To run.}{\s{fut} râdrásônre, \s{subj} râdrásôns}
\entry{rádrénẹ́}{v. + \s{aci}}{\pf{les doigts dans le nez}}{To put no effort into.}{\s{fut} rádrénrẹ́, \s{subj} rádrénẹ́s}
\entry{râdvâ-}{prefix}{\pf{grandement}}{\textit{Superlative prefix. See grammar}.}{}
\entry{rád’hérn}{n.}{from \w{rá} + \w{dérny’é}}{\textit{(always definite)} Last month.}{}
\entry{rád’hsy’ô}{n.}{\pf{traditon}}{Tradition, custom.}{}
\entry{rád’hyc’hsy’ô}{n.}{\pf{traduction}}{Translation.}{}
\entry{râhaúḍ}{v.}{\pf{recontrer}}{To meet, encounter, come face to face (+\s{all} with sbd.).}{\s{fut} râhaúḍre, \s{subj} râhaús}
\entry{ráhe}{n.}{\pf{oiseau}}{Bird.}{}
\entry{ráhé}{n.}{from \w{ráhe} + \w{ráhó}}{Flying fish.}{}
\entry{ráhé}{n.}{\pf{voisin}}{Neighbour.}{}
\entry{ráhẹ}{conj.}{\pf{quoique}}{+\s{subj} Although, though.}{}
\entry{râhẹ}{n.}{\pf{français}}{Human, person.}{}
\entry{ráhis}{v.}{\pf{raciste}}{To be racist.}{\s{fut} ráhise, \s{subj} ráhiss}
\entry{ráhó}{n.}{\pf{poisson}}{Fish.}{}
\entry{ráhó}{n.}{\pf{gazon}}{Grassland, grassy field, meadow.}{}
\entry{ráhut’h}{n.}{\pf{grand} + \pf{couteau}}{Sword, blade \ex \w{áráhut’h’t ilý ly b’haúr} ‘the pen is mightier than the sword’ \textit{(originally a fossilised, obsolete ACI: \w{á\-hut’h\-rá éḍ ilý lẹb’haúr})}.}{}
\entry{rál}{n.}{\pf{toile}}{Canvas.}{}
\entry{rár}{v.}{\pf{voir}}{To see (+\s{part} sbd./sth.).}{\s{fut} b’hérẹ́, \s{subj} rárs}
\entry{rárd}{v.}{\pf{regarder}}{\\+\s{acc} To watch.\\+\s{part} To look at.}{\s{fut} rárdre, \s{subj} rárds}
\entry{râsír}{v.}{\pf{transpirer}}{+\s{aci}.\\To come to light, become known, transpire.\\\s{pres ant} For it to be clear, apparent, evident that \ldots \textit{Lit. ‘it has come to light that \ldots’}.}{\s{fut} râsírẹ́, \s{subj} râsírs}
\entry{rát’hẹ}{particle}{\pf{vois-tu}}{You see, you know.}{}
\entry{raû}{n. archaic}{\pf{tronc}}{Log (of a tree).}{}
\entry{raû}{interj.}{\pf{gône}}{Kid. \textit{This is grammatically a vocative—not that one could tell since it looks identical to the absolutive}.}{}
\entry{raú(b’hc’h)-}{prefix}{from \w{rób’hoc’h}}{\textit{Causative prefix, see §~\ref{subsec:diachrony-and-derivation}}.}{}
\entry{raúb’hẹ}{n.}{\pf{robot}}{Robot.}{}
\entry{raûc}{n.}{\pf{tronche}}{Head.}{}
\entry{raûd’hárb}{n.}{\pf{tronc d’arbre}}{Log (of a tree).}{}
\entry{raúhérẹ́}{v.}{from \w{raú-} + \w{sérẹ́}}{To tighten, make tighter (+\s{acc}).}{\s{fut} raúhérrẹ́, \s{subj} raúhérẹ́s}
\entry{raúhy’b’h}{v.}{from \w{raú-} + \w{sy’b’h}}{To raise, lift up (+\s{acc} sth.) (+\s{ela} from sth.).}{}
\entry{raúl}{n.}{\pf{parole}}{\\Language, speech, word.\\\w{Raúl} \textit{(definite only)} Short for \w{T’hebhaú Raúl}. \textit{\s{nom sg} irreg. \w{Raúl}; all other forms are regular}.}{}
\entry{raúvá}{n.}{\pf{fromage}}{Moon.}{}
\entry{ráv́â}{adv.}{\pf{rarement}}{\textit{neg. only} Seldom, rarely (ever).}{}
\entry{rávér}{n.}{\pf{grammaire}}{\\Grammar, the grammatical rules of a language.\\A textbook describing the grammar of a language.}{}
\entry{ráy’á}{v.}{\pf{voyage}}{\\To travel, go on a journey.\\\textit{n.} Travel, voyage, journey.}{}
\entry{ráy’é}{v.}{\pf{noyer}}{To drown.}{}
\entry{ráy’ê}{n.}{\pf{moyen}}{\\Way, means, method.\\\w{ráy’ê y’aúhý} + \s{aci} There is no way, that \ldots{}.\\\s{instr pl} \w{b’hehráy’ê} How, by what means, in this way.}{}
\entry{ráý’ẹ}{v.}{\pf{râler}}{To complain, grumble.}{}
\entry{ré}{v.}{\pf{vrai}}{To be true, correct, right.}{\s{fut} rẹ́, \s{subj} rés}
\entry{ré}{n.}{\pf{rai}}{Ray, beam.}{}
\entry{ré}{v.}{\pf{créer}, \s{subj} from \pf{fabriquer}}{To create, make (\s{+acc} sth.).}{\s{fut} rẹ́éré, \s{subj} faríc’hs}
\entry{ré}{adj.}{\pf{près}}{Near, close, nearby.}{}
\entry{ré}{v. intr.}{\pf{errer}}{To wander, roam (+\s{perl} across sth.).}{}
\entry{ré}{adv.}{en vain}{In vain, for nothing.\textit{Usually preceded directly by the verb it applies to}.}{}
\entry{ré}{n.}{\pf{souhait}}{Wish.}{}
\entry{rê}{conj.}{\pf{bien que}}{+\s{subj} Although, though.}{}
\entry{rê}{v.}{\pf{trine}}{To be composed of three parts or people; triune.}{\s{fut} rêrẹ́, \s{subj} rês}
\entry{rê}{n.}{\pf{airain}}{Copper.}{}
\entry{rê}{n.}{\pf{point}}{Point (in a score).}{}
\entry{ré-}{prefix}{\pf{très}}{\textit{Superlative prefix. See grammar}.}{}
\entry{rê-}{prefix}{\pf{moins}}{\textit{Neutral comparative prefix. See grammar}.}{}
\entry{réaû}{n.}{from \w{ré}}{Creation, making.}{}
\entry{rébh}{v.}{\pf{préparer}}{To anticipate (+\s{acc} sth.).}{}
\entry{rébhós}{n.}{\pf{réponse}}{Answer, response, reply.}{}
\entry{rẹ́b’h}{v. or n.}{\pf{rêver}}{\\To dream (+\s{gen} of sth.).\\Dream, a dreaming.}{\s{fut} rẹ́v́e, \s{subj} rẹ́b’hs}
\entry{réb’hní}{v.}{\pf{prévenir}}{\\To prevent, stop (+\s{acc} sth. from happening).\\To forewarn (+\s{part} of sth.).}{\s{fut} réb’hníre, \s{subj} réb’hnís}
\entry{réḍ}{v.}{\pf{souhaiter}}{To wish (+\s{acc/aci} for sth.).}{}
\entry{rêd}{v.}{\pf{craindre}}{+s{opt} To fear, lest \ldots \textit{Construed with the negated optative}.}{\s{fut} rêdrẹ́, \s{subj} rês}
\refentry{rêd}{ḅẹt’hẹ}
\entry{rêdrsýrśẹ}{v.}{\pf{prendre sur soi}}{\\+\s{aci} To take upon onself to do sth.\\+\s{pci} To take upon oneself to start doing sth.}{}
\entry{rẹ́dy’í}{v.}{\pf{réduire}}{To reduce (+\s{acc} \textit{or pass.} sbd./sth.) (+\s{all} to sth.).}{\s{fut} rẹ́dy’ré, \s{subj} rẹ́dy’ís}
\entry{rêd’hes}{particle}{\pf{bien sûr}}{Of course, certainly, surely.}{}
\entry{rẹ́flec̣}{v.}{\pf{réfléchir}}{To think (+\s{part} sth.).}{}
\entry{réhẹv́}{v.}{\pf{recevoir}}{To receive.}{\s{fut} réhẹv́é, \s{subj} rẹsy}
\entry{rêr}{n.}{\pf{fringues}}{\\An article of clothing, garment, piece of clothing.\\\textit{pl.} Clothes, garments.}{}
\entry{rés}{n.}{\pf{reste}}{Rest, remainder.}{}
\entry{rét’hád}{v.}{\pf{prétendre}}{To claim, allege.}{\s{fut} rét’hádrẹ́, \s{subj} rét’h\-ádes}
\entry{rét’hẹ}{v.}{\pf{traiter}}{To handle, take care of, deal with.}{\s{fut} rét’hẹre, \s{subj} rét’hes}
\entry{rét’hír}{v.}{\pf{retirer}}{\\(+\s{acc}) To pull, draw, withdraw.\\+\s{part} To pull on sth. without actually moving it; to try to pull sth.}{\s{fut} rét’hírẹ́, \s{subj} rét’hírs}
\entry{révôt’hẹ}{v.}{\pf{remonter}}{To return, come back.}{}
\entry{ríb’hy’ér}{n.}{\pf{rivière}}{River.}{}
\entry{rívnél}{n.}{\pf{criminel}}{Scoundrel, someone without virtue.}{}
\entry{ríy’ŷrệ}{n.}{\pf{prieuré}}{Priory.}{}
\entry{rjẹ}{n.}{\pf{Hergé}}{Comic book.}{}
\entry{rób’hoc’h}{v.}{\pf{provoquer}; future from \pf{infliger}}{\s{+acc} To cause, make happen.}{\s{fut} flijé, \s{subj} rób’hoc’hs}
\entry{rrá}{v.}{\pf{croire}}{Believe (something or someone).}{\s{fut} rrẹ́, \s{subj} rrás}
\entry{rráḍraúc}{n.}{\pf{droit} + \pf{gauche}}{Side.}{}
\entry{rrád’hahánár}{n.}{\pf{froid de canard}}{Extreme cold, coldness.}{}
\entry{rúb’h}{v.}{\pf{trouver}}{To find, discover.}{}
\entry{rvá}{interj.}{of unknown origin}{Alas, woe, oh. \textit{Exclamation of distress, surprise, sadness, or regret}.}{\textit{after words that end with ‘r’, this is spelt \w{-vá} instead}}
\entry{rýc̣ér}{v.}{\pf{requerir}}{To ask, question.}{}
\entry{rýd}{v.}{\pf{rude}}{To be uneven, rough, rugged.}{}
\entry{rýl}{v.}{\pf{brûler}}{\\\s{+acc} To burn.\\\s{+part} To scorch, singe.}{}
\entry{rýl}{n.}{\pf{gueule}}{Face.}{}
\entry{rýrŷ}{v.}{\pf{rugueux}}{To be rough, rugged.}{}
\entry{rýsḍ}{v.}{\pf{frustrer}}{To frustrate, vex, annoy.}{}
\entry{rývýr}{.n}{\pf{rumeur}}{History.}{}
\entry{rýý’ẹ́}{v.}{\pf{céruléen}}{To be cerulean, sky-blue.}{}
\entry{rzaúsḍ}{v.}{\pf{exhaustif}}{\\To be exhaustive, comprehensive, complete.\\To be finished, completed.}{\s{fut} rzaúsḍre, \s{subj} rzaúsḍs}
\entry{s}{conj.}{\pf{si}}{If, when, whenever.}{}
\entry{sá}{particle}{\pf{sans}}{Not, no. \textit{Always enclitic \w{s’} before vowels. This particle is used only in the subjunctive; see also \w{asý’ýâ}, \w{t’hé}}.}{}
\entry{sá}{conj.}{\pf{sans que}}{+\s{subj} Without (doing sth.).}{}
\entry{sáḍy’ér}{n.}{\pf{sanctuaire}}{Sanctuary, shrine.}{}
\entry{sáhẹ}{v.}{\pf{insensé}}{To be preposterous, absurd, nonsensical.}{\s{fut} sáhere, \s{subj} sáhes}
\entry{saj}{v.}{\pf{sage}}{To be wise, prudent.}{}
\entry{sajès}{n.}{\pf{sagesse}}{Wisdom.}{}
\entry{Sásc’hríḍ}{n. never lenited}{\pf{sanskrit}}{The Sanskrit language.}{}
\entry{sásy’él}{v.}{\pf{essentiel}}{To be essential.}{\s{fut} sásy’élẹ́, \s{subj} sásy’éls}
\entry{sauc’h}{conj.}{\pf{sauf que}}{+\s{subj} Except that.}{}
\entry{saul}{n.}{\pf{sol}}{Sun.}{}
\entry{saúr}{n.}{\pf{sorte}}{\\Kind, sort, type, form.\\\s{def + gen} (some) kind(s) of.}{}
\entry{saut’h}{v. intr. or tr.}{\pf{sauter}}{To teleport, translocate, warp (+\s{acc} sth.).}{}
\entry{sauz}{n.}{\pf{chose}}{Thing, object.}{}
\entry{sauz-aud}{adj.}{\pf{autre chose}}{Something else, another thing.}{}
\refentry{sauzaud}{sauz-aud}
\entry{sav́á}{v.}{\pf{savoir}}{To know (+\s{part/acc} sth. \textit{case depends on the depth of the speaker’s understanding}).}{\s{fut} saúr, \s{subj} sac}
\entry{Sávýy’él}{n.}{\pf{Samuel}}{\textit{Male given name}.}{}
\entry{sḅé}{v.}{\pf{espérer}}{\\To want (+\s{acc/inf} sth.).\\+\s{opt} To wish, want, desire.}{\s{fut} sḅérẹ́, \s{subj} sḅés}
\entry{sḅrí}{n.}{\pf{espirit}}{Soul.}{}
\entry{sb’hé}{v.}{\pf{se baigner}}{To bathe.}{}
\entry{sẹ}{particle}{\pf{ainsi}}{So, thus, as a result.}{}
\entry{séḅ}{v.}{\pf{simple}}{To be plain, simple.}{\s{fut} séḅrẹ́, \s{subj} séḅs}
\entry{seb’haúd}{v. intr.}{\pf{s’effondrer}}{To cave in, collapse.}{}
\entry{sèd’h}{part.}{from \pf{c’est du}}{It is due to (+\s{gen} sth. / +\s{aci} the fact that...).}{}
\entry{sẹh}{det.}{\pf{ceci}}{+\s{def} \textit{noun} This, these. \textit{Precedes and is attached to nouns}.}{}
\entry{sẹhérél}{v.}{\pf{se quereller}}{To quarrel, argue, fight about (+\s{part}).}{}
\entry{sehul}{v.}{\pf{s’écouler}}{To flow.}{}
\entry{sẹhúr}{v.}{\pf{secourir}}{To help, succour, give aid (+\s{dat} to sb.) (+\s{aci}/\s{acc} with sth.).}{\s{fut} sẹhúrre, \s{subj} sẹhús}
\entry{sénýr}{n.}{\pf{seigneur}}{\\Lord.\\\textit{Short for \w{Ráb’h sénýr}}.}{}
\entry{sẹrád}{v. intr.}{\pf{se rendre}}{To surrender.}{}
\entry{sérḍé}{det.}{\pf{certain}}{Certain, particular but not specified.}{}
\entry{sérẹ́}{v.}{\pf{serré}}{\\To be tight, close-fitting, snug.\\\s{indef} \textit{usually} \s{instr} \w{c’hýr sérệ} A heavy heart.}{\s{fut} sérrẹ́, \s{subj} sérẹ́s}
\entry{sèt’h}{v.}{\pf{sentir}}{To feel.}{\s{fut} sèt’he, \s{subj} sès}
\entry{séy’ẹ́}{v.}{\pf{essayer}}{+\s{part} \textit{or} \s{inf} To try, attempt.}{\s{fut} séy’ẹ́rẹ́, \s{subj} séy’ẹ́s}
\entry{siḍ}{n.}{\pf{site}}{Facility, site.}{}
\entry{sisḍé}{n.}{\pf{système}}{System.}{}
\entry{Sit’h}{n.}{\pf{Sith}}{Sith (Star Wars).}{}
\entry{sit’há}{conj.}{\pf{si tant est que}}{+\s{opt} Supposing that; if, assuming that.}{}
\entry{sívý’ér}{v.}{\pf{similaire}}{To be similar, alike (+\s{gen} to sth.).}{}
\entry{Snet’h}{n.}{}{\textit{Family name, equivalent to English ‘Smyth’}.}{}
\entry{sol}{n.}{\pf{sol}}{Ground, floor, earth, soil.\textit{The plural may be used to indicate a large quantity of soil}.}{}
\entry{suḍ}{v.}{\pf{soutenir}}{\\+\s{acc} To support, hold up.\\+\s{part} To help support, hold up part of.}{}
\entry{sud’hénvâ}{adv.}{\pf{soudainement}}{Suddenly.}{}
\entry{suf}{n.}{\pf{souffre}}{Pain.}{}
\entry{sufb’h}{n.}{\pf{souffle} + \pf{vie}}{Life.}{}
\entry{susy’é}{v.}{\pf{soucier}}{+\s{part, pci} To care about, worry about.}{\s{fut} susy’ére, \s{subj} susy’és}
\entry{swi}{det.}{\pf{celui}}{The one, that one, this one.}{}
\entry{sybhẹ́rýr}{v.}{\pf{supérieur}}{\textit{intr. or} +\s{gen} To be superior to, better than, higher than.}{\s{fut} sybhẹ́rýrẹ́, \s{subj} sybhẹ́rýrs}
\entry{syḅlẹ}{v.}{\pf{suppléer}}{\\To supplement (\s{acc} sth.) (+\s{instr} with sth.). \textit{If no \s{instr} is present, the subject is implied to be the supplement}.\\\w{syḅlâ} \textit{adj.} Additional, extra.}{}
\entry{syhyý’á}{v.}{\pf{succulent}}{To be succulent, delicious.}{\s{fut} syhyý’áré, \s{subj} syhyý’ás}
\entry{syl}{v.}{\pf{seul}}{\\To be the only one.\\To be lone, alone.}{\s{fut} syle, \s{subj} syls}
\entry{sy’b’h}{v. intr.}{\pf{se lever}}{To rise (+\s{ela} from sth.).}{}
\entry{sy’ê}{v.}{\pf{sien}}{To be his, hers, its.}{\s{fut} sy’êrẹ́, \s{subj} sy’ês}
\entry{sý’ẹ}{det.}{\pf{cela}}{+\s{def} \textit{noun} That, those. \textit{Precedes and is attached to nouns; often \w{sý’} before vowels, with one apostrophe, not two}.}{}
\refentry{s’}{sá}
\refentry{t-}{ḍ-}
\entry{t’hé}{conj.}{\pf{de peur que} > *\w{dbhýrc’h} > *\w{dýrc’h} > *\w{dc’hý} > \this}{Not, no. \textit{Always \w{t’h’\N} before vowels, but does not nasalise if the ‘é’ is still present. This particle is used only in the optative; see also \w{asý’ýâ}, \w{sá}}.}{}
\entry{T’hebhaú}{n. or adj.}{from \w{t’hebhaúz}}{(ULTRA-) France, (ULTRA-)French.}{}
\entry{T’hebhaú Raúl}{n. def. sg.}{from \w{t’hebhaúz} + \w{raúl}}{The ULTRAFR\-ENCH language. \textit{Only \w{T’hebhaú} is declined as though the entire phrase were one word. In informal speech and writing, this is typically shortened to \w{Raúl}}.}{\s{nom sg} \textit{irreg.} \w{T’hebhaú Raúl}}
\entry{t’hebhaúz}{v.}{\pf{jeter l’éponge}}{To be (ULTRA-)French.}{\s{fut} t’hebhaúźe, \s{subj} t’hebhaúś}
\entry{t’hiy’e}{v.}{from \w{yt’hiy’ihẹ}; \s{subj} via ba\-ck-formation from the \s{fut}}{+\s{part} To use, make use of.}{\s{fut} t’hiźe, \s{subj} \s{t’hizes}}
\entry{u}{conj.}{\pf{ou}}{\\Or. \textit{Inclusive, see also \w{ra}}.\\\w{u} \ldots\ \w{u} \ldots\ ‘\ldots\ or \ldots’ \textit{(inclusive)}.}{}
\entry{ub’h}{v.}{\pf{ouvrir}}{To open.}{\s{fut} uv́, \s{subj} ub’hs}
\entry{ub’hrá}{v.}{\pf{pouvoir}}{\\+\s{inf/aci} To be able to, can. \textit{Never construed with an \s{inf} if it in and of itself is the infinitive of an \s{aci} or \s{pci}, in which case the variant with the \s{part} (\senseref{2}) is used instead}.\\+\s{part} To be capable of \ldots\\\s{opt cond i + aci} To be possible; may. \textit{Dynamic or epistemic, never deontic; this and sense 4 are essentially a more emphatic optative}.\\\s{opt cond ii + aci} Might. \textit{Dynamic or epistemic, never deontic}.}{\s{fut} úrẹ́, \s{subj} ís}
\entry{ulíy’ẹ́}{v.}{\pf{oublier}}{To forget.}{\s{fut} ulíy’ẹ́rẹ́, \s{subj} ulíy’ẹ́s}
\entry{úrbh}{conj.}{\pf{pour peu que}}{+\s{opt} Provided that, so long as.}{}
\entry{urdálbhaúrḍ}{n.}{\pf{avoir un oursin dans le portefeuille}}{A very rich person; billionaire.}{}
\refentry{úrẹ́}{ub’hrá}
\entry{uy’ed’háb’hrí}{v.}{\pf{rouler dans la farine}}{To scam, cheat, swindle.}{\s{fut} uy’e\-d’háv́e, \s{subj} uy’ed’háb’hrís}
\entry{vá}{n.}{\pf{mât}}{Mast.}{}
\refentry{vá}{rvá}
\entry{vádłabhaud’hávúrsab’hád’háváb’hrárḍuẹ}{v. literary}{\pf{vendre la peau de ours avant de avoir tué}}{To depend on predictions of the future. \textit{Of disputed origin; first attested in the works of the Early UF comedian \s{J. A. B. Snet’h}}.}{\s{fut} vád\-ła\-bhau\-d’há\-vúr\-sa\-b’há\-d’há\-vá\-b’hrár\-ḍu\-re, \s{s} vád\-ła\-bhau\-d’há\-vúr\-sa\-b’há\-d’há\-vá\-b’hrár\-ḍus}
\entry{vâhẹ}{v.}{\pf{manquer}}{\\+\s{gen} To lack, want.\\+\s{part} \textit{or} \s{pass} To miss. \textit{The object and subject of this verb are swapped compared to English ‘to miss’, e.g. \w{b’hývvâhé} (\s{2pl.act} + \s{1sg.pass}) ‘I miss you (\s{pl})’, lit. roughly ‘you (\s{pl}) are wanting to me’)}.\\+\s{acc} To miss out on.}{\s{fut} vâhérẹ́, \s{subj} vâhés}
\entry{váj}{n.}{from \w{íváj}}{Image, picture.}{}
\entry{válḍrét’hás}{n.}{\pf{maltraitance}}{Torture.}{}
\entry{válfèz}{v.}{\pf{malfaisant}}{To be malfeasant, evil, malevolent.}{\s{fut} válfèź, \s{subj} válfès}
\entry{válv́áy’}{v.}{\pf{malvoyant}}{To be blind.}{}
\entry{válvê}{v.}{\pf{malmener}}{To mistreat, torture.}{\s{fut} válv́e, \s{subj} válvês}
\entry{v́ár}{v. irreg.}{\pf{devoir}}{\\\s{pass} +\s{aci} Must, have to, be obliged to. \textit{The subject is always in the passive in this sense only}.\\+\s{dat} To owe sbd. (+\s{acc} sth.).\\\s{cond i + aci} Even if; \textit{e.g.} \w{aúrdyssa dẹće} \textit{‘even if he should fail’}.}{\s{cond i, ii} dy, \s{fut} dv́e, \s{subj} ráb’h}
\entry{vás}{n. \s{pl def}}{\pf{masses}}{The masses, the people.}{}
\entry{vaúb’hẹ}{v. irreg.}{\pf{mauvais}}{\\To be bad.\\To be wrong, incorrect, inappropriate.}{\s{fut} bíré, \s{subj} bíres; \s{comp} lẹbír, y’ŷbír, rêbír; \s{sup} réb’hír, râdvâbír}
\entry{vaûd}{n.}{\pf{monde}}{World.}{}
\entry{vaûḍ}{v.}{\pf{montrer}}{To show, display (+\s{acc} sth.).}{}
\entry{vaúd’hér}{v.}{\pf{modérer}}{To be moderate.}{}
\entry{vaût’há}{n.}{\pf{montagne}}{Mountain.}{}
\entry{váý’eb’his}{n.}{\pf{maléfice}}{Vice.}{}
\entry{váý’ýr}{n.}{\pf{malheur}}{Tragedy, misfortune.}{}
\entry{váłé}{conj.}{\pf{malgré que}}{+\s{subj} Despite that, in spite of.}{}
\entry{vé}{conj.}{\pf{mais}}{But, however, although.}{}
\entry{vê₁}{adv.}{\pf{demain}}{Tomorrow.}{}
\entry{vê₂}{n.}{\pf{main}}{Hand.}{}
\entry{véc}{n.}{\pf{mèche}}{\\A strand of hair.\\\s{pl.} Hair.}{}
\entry{véḍ}{v.}{\pf{mettre}}{To lay, put, place (+\s{acc} sth.).}{}
\entry{véhýr}{conj.}{\pf{dans la mesure où}}{Insofar as.}{}
\entry{véhýr}{v/n.}{\pf{mesure}}{\\To measure.\\Measurement.}{\s{fut} véhýrẹ́, \s{subj} véhýrs}
\entry{vér}{n.}{\pf{mère}}{\textit{(informal)} Mum, mom.}{}
\entry{vérjet’hic’h}{v.}{\pf{énergétique}}{To be vigorous, energetic.}{}
\entry{vérr}{n.}{\pf{mer}}{Sea, ocean.}{}
\entry{vérs}{interj.}{\pf{merci}}{\\Thank you. (+\s{gen} for sth.).\\\w{dyvérs fér} To thank (+\s{dat} sbd.) (+\s{gen} for sth.).}{}
\refentry{vérvá}{vér + vá}
\entry{vêt’hnâ}{adv.}{from \pf{maintenant}, lenited for unknown reasons}{Now.}{}
\refentry{véy’ýr}{baú}
\entry{víd’hẹ}{n.}{\pf{midi}}{Noon, midday.}{}
\entry{Víd’hic’hlaúry’ê}{n.}{\pf{Midichlorien}}{Midichlorian (Star Wars).}{}
\entry{vísy’ô}{n.}{\pf{émission}}{\\Emission.\\Programme, broadcast, show.}{}
\entry{vnásḍér}{n.}{\pf{monastère}}{Castle.}{}
\entry{vú}{adj.}{\pf{moult}}{Many, much, a lot of.}{}
\entry{vúb’hvâ}{n.}{\pf{movement}}{Movement, motion.}{}
\entry{vúslihé}{n.}{\pf{mousse} + \pf{lichen}}{Moss.}{}
\entry{vvâ}{n.}{\pf{maman}}{Mother.}{}
\entry{vvâ}{n.}{\pf{moment}}{Moment, instant.}{}
\entry{vvaúríhe}{v. (in)tr.}{\pf{mémoriser}}{To remember.}{\s{fut} vvaúríźe, \s{subj} vvaúríhes}
\entry{vŷ}{v.}{\pf{mener}}{To lead.}{\s{fut} menre, \s{subj} mens}
\entry{w}{v.}{\pf{enlever}}{To remove (+\s{acc} sth.).}{}
\entry{ýr}{v.}{\pf{heurter}}{To hit, strike.}{\s{fut} ýrḍ, \s{subj} ýrs}
\entry{yt’hiy’ihẹ}{v.}{\pf{utiliser}}{+\s{part} \textit{Archaic}. To use, make use of.}{\s{fut} yt’hiy’iźe, \s{subj} yt’hiy’\-ihẹs}
\entry{Yý’is}{n.}{\pf{Ulysse}}{\textit{Male given name}.}{}
\entry{y’ác’hraúníc’h}{v.}{\pf{diachronique}}{To be diachronic.}{\s{fut} y’ác’hraú\-níc’hre, \s{subj} y’ác’hraúníc’hes}
\entry{y’aúhý}{part.}{\pf{il n’y a aucun}}{There is no, there are no, there is none.}{}
\entry{ý’aúhý}{part.}{\pf{il y a aucun}}{There is, there are.}{}
\entry{y’aúý’}{v.}{back-formation from \w{y’aúý’vâ}, displacing earlier \w{y’aúý’á}}{To be violent, vehement.}{}
\entry{y’aúý’á}{v. archaic}{back-formation from \w{y’aúý’ávâ}}{To be violent, vehement.}{}
\entry{y’aúý’ávâ}{adv. archaic}{\pf{violament}}{Violently, vehemently.}{}
\entry{y’aúý’vâ}{adv.}{back-formation from \w{y’aúý’á}, displacing earlier \w{y’aúý’ávâ}}{Violently, vehemently.}{}
\entry{y’é}{pron.}{\pf{rien}}{Nothing. \textit{Like most negative polarity items, this induces negation of the verb}.}{}
\entry{y’ẹ́}{v.}{\pf{nier}}{To forbid, deny.}{\s{fut} y’ẹ́rẹ́, \s{subj} y’ẹ́s}
\entry{y’ê}{v.}{\pf{mien}}{To be mine.}{\s{fut} y’êrẹ́, \s{subj} y’ês}
\entry{y’éjúré}{n.}{\pf{siège} + \pf{tabouret}}{Chair, seat.}{}
\entry{y’ér}{adv.}{\pf{hier}}{Yesterday.}{}
\entry{y’í}{n.}{\pf{nuit}}{Night.}{}
\entry{y’í}{n.}{\pf{puits}}{Well (water source).}{}
\entry{y’íhá}{v.}{\pf{puissant}}{To be powerful, mighty, puissant.}{}
\entry{y’ír}{v. (in)tr.}{\pf{ouïr}}{To understand, listen, \textit{(rarely)} hear.}{\s{fut} aúré, \s{subj} rás}
\entry{y’ís}{conj.}{\pf{puisque}}{Considering that, since, because. \textit{Unlike \w{c’haúr}, this does not take the subjunctive; it is used to indicate the (potential) cause of something}.}{}
\entry{y’úr}{n.}{\pf{jour}}{\\Day.\\\w{órdy’úr ád’y’úr} Day after day. \textit{Contracted \s{ela} and \s{ill}}.}{}
\entry{y’ŷ}{n.}{from \w{y’ŷvéłáfrí}}{Eye.}{}
\entry{y’ŷ-}{prefix}{\pf{mieux}}{\textit{Denying comparative prefix. See grammar}.}{}
\entry{y’ŷvéłáfrí}{n. pl. archaic}{\pf{yeux de merlan frit}}{Eyes.}{}
\entry{Zauḅ}{n.}{\pf{Ésope}}{Aesop.}{}
\refentry{’sý’ýâ}{asý’ýâ}


\end{document}
