\documentclass[a4paper, 12pt, oneside, final]{article}
\usepackage[margin=2cm]{geometry}
\usepackage{fontspec}
\usepackage{unicode-math}
\usepackage[english]{babel}
\usepackage{csquotes}
\usepackage{array, tabularx, multirow}
\usepackage{longtable}

\setmainfont{Minion 3}
\setmathfont{latinmodern-math.otf}
\setmathfont[range=\mathit]{Minion 3 Italic}
\frenchspacing

\AtBeginDocument{
    \def\today{
        \number\day\space
        \ifcase\month\or
        January\or February\or March\or April\or May\or June\or
        July\or August\or September\or October\or November\or December\fi\space
        \number\year
    }
}

\makeatletter
\def\footnoterule{%
    \kern-3\p@
    \hrule\@width.4\columnwidth
    \kern2.6\p@
}

\def\@makefntext#1{%
    \setlength\parindent{1em}%
    \noindent
    {%
    \mbox{\llap{{}\textsuperscript{\@thefnmark}\kern.5pt}}{#1}%1
    }%
}
\makeatother

\title{ULTRAFRENCH}
\author{Agma Schwa \& Ætérnal}
\date{\today}

\ExplSyntaxOn
\cs_new:Npn \__two_cols:nnnnn #1 #2 #3 #4 #5 {
    \ifvmode\else\unskip\par\fi
    \noindent\leavevmode
    \hbox to \hsize {
        \hbox to #3 { \vtop {#1} }
        \hskip   #4
        \hbox to #5 { \vtop {#2} }
    }\par
}

\NewDocumentCommand \TwoCols {
    D[]{.475\hsize}
    D[]{.475\hsize}
    D[]{0pt plus 1fill}
    +m
    +m
} {
    \__two_cols:nnnnn{#4}{#5}{#1}{#3}{#2}
}

\def \UF { \bfseries \itshape }
\let \nf \normalfont

\def \d {ḍ}
\def \D {Ḍ}
\def \b {ḅ}
\def \B {Ḅ}
\def \l {ḷ}
\def \L {Ḷ}
\def \e {ẹ}
\def \E {Ẹ}

\char_set_catcode_active:N \*
\cs_set_protected:Npn \__md_star:w #1*{ \textit{#1} }
\def\* { \detokenize{*} }
\cs_new:Npn * { \__md_star:w }

\cs_new:Npn \items { \itemize\itemsep6pt }
\cs_new:Npn \enditems { \enditemize }

\ExplSyntaxOff

\newlength{\EnumItemSep} \EnumItemSep-3pt

\newenvironment{enum}[1][0]{%
    \vspace{-.5em}%
    \settowidth{\leftmargini}{99.\hskip\labelsep}%
    \begin{enumerate}\setcounter{enumi}{#1}\itemsep\EnumItemSep
}{%
    \end{enumerate}%
    \vspace{-.5em}%
}

\def\parheading#1{\noindent\textbf{#1}}

\let\Sub\textsubscript

\begin{document}
\maketitle
\thispagestyle{empty}
\clearpage
\setcounter{page}{1}

\section{Phonology and Evolution from Modern Pseudo-French}\label{sec:phonology}{\def\arraystretch{1.25}\setlength{\tabcolsep}{.4em}
\noindent\begin{tabular}{@{}|l|l|l|l|l|l|l@{\quad}|l|l|l|}                                                   \cline{1-6} \cline{8-10}
               & Labial & Coronal  & Palatal  & Velar & Glottal &&           & Front        & Back        \\ \cline{1-6} \cline{8-10}
    Stop       & b, bʱ  & d        &          & k     &         && Close     & i ĩ ĩ̃ i̥, y ỹ ỹ̃ ẙ & u ũ ũ̃ u̥ \\ \cline{1-6} \cline{8-10}
    Nasal      &        & n        &          &       &         && Close-mid & e ẽ ẽ̃ e̥      & o o̥         \\ \cline{1-6} \cline{8-10}
    Fricative  & ɸ β, ʋ̃ & s z, θ ð & ɕ ʑ      & x     & h       && Mid       & \multicolumn{2}{c|}{ə ⟨\e{}⟩ ə̥}   \\ \cline{1-6} \cline{8-10}
    Approx.    &        &          & ɥ ɥ̃, j̊   & ɰ ɰ̃   &         && Open-mid  & ɛ ɛ̃ ɛ̃̃ ɛ̥      & ɔ̃ ɔ̃̃         \\ \cline{1-6} \cline{8-10}
    Lat. Fric. &        & ɮ̃        & ʎ̝̃        &       &         && Open      & a ḁ          & ɑ̃ ɑ̃̃         \\ \cline{1-6} \cline{8-10}
\end{tabular}}\bigskip

\parheading{Legend}\par\noindent
Ṽ = nasalised vowel, Ṽ̃ = nasal vowel, V = any vowel (or, in conjunction with Ṽ/Ṽ̃, oral vowel)\\
N = nasal consonant, C̃ = nasalised consonant (e.g. /ɰ̃/, but not true nasals), C = any consonant.\medskip

\TwoCols[.45\hsize][.45\hsize][0pt]{
\parheading{Preliminary Changes}
\begin{enum}
    \item g, ʁ, w > ɰ ⟨r⟩
    \item œ, œ̃, ø > y, ỹ, ỹ
    \item ɔ > o
    \item y > j / \_(\#)V
    \item V\kern-2pt\raisebox{2pt}{\Sub α} > $\emptyset$ / \_\#V\kern-2pt\raisebox{2pt}{\Sub α}
    \item lj, lɥ > ʎ
    \item j > ɥ ⟨y’⟩
    \item ɰ > ɥ / \_i
    \item C > $\emptyset$ / \#\_C
    \item C > $\emptyset$ / C\_\#
    \item k > x\footnotemark ⟨c’h⟩
    \item ʃ, ʒ > ɕ ⟨ç⟩, ʑ ⟨j⟩
    \item nt > nθ
    \item t > \d{} [d] (‘hard /d/’)
    \item p > \b{} [b] (‘hard /b/’)
    \item f, v > ɸ ⟨f⟩, β ⟨b’h⟩
\end{enum}
}{
\parheading{Great Nasal Shift}
\begin{enum}[15]
    \item Ṽl > ɰ̃ ⟨w⟩
    \item V, Ṽ > Ṽ, Ṽ̃ / \_[NC̃ɥɰ], [NC̃ɥɰ]\_
    \item ə̃, ə̃̃, ã, ã̃, õ, õ̃ > ɛ̃, ɛ̃̃, ɑ̃, ɑ̃̃, ɔ̃, ɔ̃̃
    \item N, C̃ > $\emptyset$ / V\_\#
    \item ɲ, ŋ > n
    \item V, Ṽ > $\emptyset$ / N \_ N
    \item m, l, ʎ > ʋ̃ ⟨v⟩, ɮ̃ ⟨l⟩, ʎ̝̃ ⟨\l⟩
\end{enum}

\parheading{Intervocalic Lenition (/ V\_V is implied)}
\begin{enum}[21]
    \item x, s, z > h
    \item ɕ, ɮ̃, ʎ̝̃ > j̊ ⟨ç⟩, ɥ̃ ⟨l⟩, ɰ̃ ⟨\l⟩
    \item nθ > n
    \item d, \d{}, b, \b{} > ð ⟨d’h⟩, θ ⟨t’h⟩, β, bʱ ⟨bh⟩
\end{enum}

\parheading{Late Changes}
\begin{enum}[25]
    \item ə > $\emptyset$ / C\_C
    \item V[-nasalised, -nasal] > V̥ / \_\#
\end{enum}
}\medskip

\footnotetext{[χ] around back vowels, [ɕ] elsewhere.}

\parheading{Example}\\
Our example sentence is ‘Charles, you bought a fish on the mountain?’ or \textbf{\textit{Charles, tu as acheté un pois\-son dans la montagne?}}
The resulting UF sentence is \textit{\textbf{Çár, y’áçt’he ý ráhó dwávôt’há}}.
In this simple case, the grammar is more or less the same in PF and UF, so no grammatical changes apply here.\par\medskip

\noindent\begin{tabular}{@{}llll}
PF &ʃaʁl ty a aʃəte œ̃ pwasɔ̃ dɑ̃ la mɔ̃taɲ       &(17)&ɕaɰ ɥaɕə\d{}e ỹ ɰasɔ̃ dɰ̃a mɔ̃\d{}aɲ \\
(1)&ʃaɰl ty a aʃəte œ̃ pɰasɔ̃ dɑ̃ la mɔ̃taɲ       &(18)&ɕãɰ ɥãɕə\d{}e ỹ ɰãsɔ̃ dɰ̃ã mɔ̃̃\d{}ãɲ \\
(2)&ʃaɰl ty a aʃəte ỹ pɰasɔ̃ dɑ̃ la mɔ̃taɲ       &(19)&ɕɑ̃ɰ ɥɑ̃ɕə\d{}e ỹ ɰɑ̃sɔ̃ dɰ̃ɑ̃ mɔ̃̃\d{}ɑ̃ɲ \\
(4)&ʃaɰl tja aʃəte ỹ pɰasɔ̃ dɑ̃ la mɔ̃taɲ        &(20)&ɕɑ̃ɰ ɥɑ̃ɕə\d{}e ỹ ɰɑ̃sɔ̃ dɰ̃ɑ̃ mɔ̃̃\d{}ɑ̃ \\
(5)&ʃaɰl tjaʃəte ỹ pɰasɔ̃ dɑ̃ la mɔ̃taɲ          &(23)&ɕɑ̃ɰ ɥɑ̃ɕə\d{}e ỹ ɰɑ̃sɔ̃ dɰ̃ɑ̃ ʋ̃ɔ̃̃\d{}ɑ̃ \\
(7)&ʃaɰl tɥaʃəte ỹ pɰasɔ̃ dɑ̃ la mɔ̃taɲ          &(24)&ɕɑ̃ɰ ɥɑ̃ɕə\d{}e ỹ ɰɑ̃hɔ̃ dɰ̃ɑ̃ ʋ̃ɔ̃̃\d{}ɑ̃ \\
(9)&ʃaɰl ɥaʃəte ỹ ɰasɔ̃ dɑ̃ la mɔ̃taɲ            &(25)&ɕɑ̃ɰ ɥɑ̃j̊ə\d{}e ỹ ɰɑ̃hɔ̃ dɰ̃ɑ̃ ʋ̃ɔ̃̃\d{}ɑ̃ \\
(10)&ʃaɰ ɥaʃəte ỹ ɰasɔ̃ dɑ̃ la mɔ̃taɲ             &(27)&ɕɑ̃ɰ ɥɑ̃j̊əθe ỹ ɰɑ̃hɔ̃ dɰ̃ɑ̃ ʋ̃ɔ̃̃θɑ̃ \\
(12)&ɕaɰ ɥaɕəte ỹ ɰasɔ̃ dɑ̃ la mɔ̃taɲ            &(28)&ɕɑ̃ɰ ɥɑ̃j̊θe ỹ ɰɑ̃hɔ̃ dɰ̃ɑ̃ ʋ̃ɔ̃̃θɑ̃ \\
(14)&ɕaɰ ɥaɕə\d{}e ỹ ɰasɔ̃ dɑ̃ la mɔ̃\d{}aɲ  &(29)&ɕɑ̃ɰ ɥɑ̃j̊θe̥ ỹ ɰɑ̃hɔ̃ dɰ̃ɑ̃ ʋ̃ɔ̃̃θɑ̃ \\
\end{tabular}

\section{Accidence}\label{sec:accidence}
\subsection{Verbal Morphology}\label{subsec:verbal-morphology}
Verbs in UF are inflected for person, number, tense, aspect, mood, and voice. Verbal inflexion is mainly done
by means of concatenating a vast set of prefixes onto a verb, with the occasional suffix and circumfix making
its appearance. This chapter details these affixes, their meanings, uses, forms, and restrictions.


\subsubsection{Active/Passive Affixes}\label{subsubsec:active-passive-affixes}
UF has a set of active/subject as well as passive/object prefixes which can be used on their own or in combination
with one another, though at most one active and one passive prefix may be combined with a verb.\footnote{Irrespective
of whether they are personal or infinitive prefixes. For instance, it would also be illegal to combine e.g. the active
infinitive prefix with the first person active singular prefix.} Table~\ref{tab:active-passive-prefixes}
below lists those prefixes, two of which are actually circumfixes.

\begin{table}[h]
\centering
\noindent\begin{tabular}{@{}|>{}l|>{\it}l|>{\it}l|>{}l|>{}l|>{\it}l|>{\it}l|}\cline{1-3}\cline{5-7}
 Active&\nf Sg&\nf Pl& & Passive&\nf Sg&\nf Pl\\\cline{1-3}\cline{5-7}
1st&j-&ó-/r-/w- -ó&&1st&v-&ó-/r-/w-\\\cline{1-3}\cline{5-7}
2nd&\d{}(\e)-&b’h(y)- -é&&2nd&\d{}(\e)-&b’h(y)-\\\cline{1-3}\cline{5-7}
3rd m&l(\e)-&l(\e)-&&3rd m&y’-&lý-\\\cline{1-3}\cline{5-7}
3rd f&ll(a)-&ll(\e)-&&3rd f&y’- &lý-\\\cline{1-3}\cline{5-7}
3rd n&ý’- &l(a)-&&3rd n&ý’-&lý-\\\cline{1-3}\cline{5-7}
Infinitive&\multicolumn{2}{c|}{\it d(\e)-}&&Infinitive&\multicolumn{2}{c|}{\it à-/h-}\\\cline{1-3}\cline{5-7}
\end{tabular}
\caption{Active (left) and passive (right) verbal affixes.}\label{tab:active-passive-prefixes}
\end{table}

\noindent A great degree of syncretism can be observed in the third-person forms. The gender distinction in the
third person singular that diachronically resulted from gendered personal pronouns is almost non-existent in the
plural; the reason for this development is that those forms are derived from the old dative form, which lacked
this distinction altogether.
Furthermore, the active first and second person plural are only distinguished from their passive counterparts by
the presence of additional suffixes in the former.

The first person plural prefix varies if there is a vowel following it: if it is
any vowel that is not a variant of ‘o’, the prefix is realised as *r-* instead, e.g. *ad’hór* ‘love’ to
*rad’hóró* ‘we love’. If the vowel a variant of ‘o’, the prefix is realised as *w-* instead, e.g. *ob’heír* ‘obey’
to *wob’heíró* ‘we obey’.

The passive infinitive prefix *à-* coalesces with any vowel following it: it becomes *á* if it
is followed by a non-nasal variant of ‘a’, e.g. *ad’hór* to *ád’hór* ‘to be loved’; *â* if it is
followed by a nasal variant of ‘a’, e.g. *ánvé* ‘give life to’ to *ânvé* ‘to be animated’; and *h-* if it is
followed by any other vowel, e.g. *ob’heír* to *hob’heír* ‘to be obeyed’.

The parenthesised vowels are used if the prefix is followed by a consonant, e.g. *dír* ‘say’ to *ll\e{}dír*
‘they (f) say’ and *b’hydíré* ‘you (pl) say’, but *ad’hór* to *llad’hór* ‘they (f) love’ and *b’had’hóré* ‘you
(pl) love’. The prefixes *ó-* and *à-* retain their main forms if followed by a consonant,
e.g. *dír* ‘say’ to *ódíró* ‘We say’ and *àdír* ‘to be said’. The exception to this is that second person plural *b’h(y)-*
drops the *y* if followed by a glide, e.g. *y’ír* ‘to hear’ to *b’hy’íré* ‘you (pl) hear’ (not \**b’hyy’íré*).

When multiple prefixes are used together, active prefixes precede passive prefixes, except that infinitive prefixes
always come first, e.g. *ad’hór* ‘love’ to *jvad’hór* ‘I love myself’ (not \**vjad’hór*) and *b’hy’ad’hóré* ‘you (pl) love him/her’,
but *d\e{}vad’hór* ‘to love me’ and *àb’had’hóré* ‘to be loved by you (pl)’. Recall that at most one infinitive prefix
may be used.

By way of illustration, consider the paradigm of the verb *ad’hór* as shown in Table~\ref{tab:adhor-paradigm} below.
Since this word starts with a vowel, the parenthesised vowels in Table~\ref{tab:active-passive-prefixes} above
are not used. Furthermore, since it starts with a non-nasal ‘a’-like vowel, the *ó-* prefix is realised as *r-*
and the *à-* prefix coalesces with the initial ‘a’ of the stem to form *á*.

% TEMPLATE:
%\noindent\begin{tabular}{@{}|>{}l|>{\it}l|>{\it}l|>{}l|>{}l|>{\it}l|>{\it}l|}\cline{1-3}\cline{5-7}
%\nf Active&\nf Sg&\nf Pl&\nf &\nf Passive&\nf Sg&\nf Pl \\\cline{1-3}\cline{5-7}
%1st       &   &  &&1st   &   &   \\\cline{1-3}\cline{5-7}
%2nd       &   &  &&2nd   &   &   \\\cline{1-3}\cline{5-7}
%3rd m     &   &  &&3rd m &   &   \\\cline{1-3}\cline{5-7}
%3rd f     &   &  &&3rd f &   &   \\\cline{1-3}\cline{5-7}
%3rd n     &   &  &&3rd n &   &   \\\cline{1-3}\cline{5-7}
%Infinitive& \multicolumn{2}{c|}{\it }  && Infinitive & \multicolumn{2}{c|}{\it } \\\cline{1-3}\cline{5-7}
%\end{tabular}

\begin{table}[h]
\centering
\noindent\begin{tabular}{@{}|>{}l|>{\it}l|>{\it}l|>{}l|>{}l|>{\it}l|>{\it}l|}\cline{1-3}\cline{5-7}
\nf Active&\nf Sg&\nf Pl&\nf &\nf Passive&\nf Sg&\nf Pl\\\cline{1-3}\cline{5-7}
1st&jad’hór&rad’hóró&&1st&vad’hór&rad’hór\\\cline{1-3}\cline{5-7}
2nd&\d{}ad’hór&b’had’hóré&&2nd&\d{}ad’hór&b’had’hór\\\cline{1-3}\cline{5-7}
3rd m&lad’hór&lad’hór&&3rd m&y’ad’hór&lýad’hór\\\cline{1-3}\cline{5-7}
3rd f&llad’hór&llad’hór&&3rd f&y’ad’hór &lýad’hór\\\cline{1-3}\cline{5-7}
3rd n&ý’ad’hór&lad’hór&&3rd n&ý’ad’hór&lýad’hór\\\cline{1-3}\cline{5-7}
Infinitive&\multicolumn{2}{c|}{\it dad’hór}&&Infinitive&\multicolumn{2}{c|}{\it ád’hór}\\\cline{1-3}\cline{5-7}
\end{tabular}
\caption{Paradigm of the verb \emph{ad’hór}.}\label{tab:adhor-paradigm}
\end{table}

\noindent For comparison, the paradigm of the verb *vvóríhe* ‘remember’ is shown in Table~\ref{tab:vvorihe-paradigm} below.
Since it starts with a consonant, the parenthesised vowels in Table~\ref{tab:active-passive-prefixes} are used, and any
prefixes that end with a vowel remain unchanged.

\begin{table}[h]
\centering
\noindent\begin{tabular}{@{}|>{}l|>{\it}l|>{\it}l|>{}l|>{}l|>{\it}l|>{\it}l|}\cline{1-3}\cline{5-7}
\nf Active&\nf Sg&\nf Pl&\nf &\nf Passive&\nf Sg&\nf Pl\\\cline{1-3}\cline{5-7}
1st&jvvóríhe&óvvóríhy’ó&&1st&vvvóríhe&óvvóríhe\\\cline{1-3}\cline{5-7}
2nd&ḍẹvvóríhe&b’hyvvóríhê&&2nd&ḍẹvvóríhe&b’hyvvóríhe\\\cline{1-3}\cline{5-7}
3rd m&lẹvvóríhe&lẹvvóríhe&&3rd m&y’vvóríhe&lývvóríhe\\\cline{1-3}\cline{5-7}
3rd f&llavvóríhe&llẹvvóríhe&&3rd f&y’vvóríhe&lývvóríhe\\\cline{1-3}\cline{5-7}
3rd n&ý’vvóríhe&lavvóríhe&&3rd n&ý’vvóríhe&lývvóríhe\\\cline{1-3}\cline{5-7}
Infinitive&\multicolumn{2}{c|}{\it dẹvvóríhe}&&Infinitive&\multicolumn{2}{c|}{\it àvvóríhe}\\\cline{1-3}\cline{5-7}
\end{tabular}
\caption{Paradigm of the verb \emph{vvóríhe}.}\label{tab:vvorihe-paradigm}
\end{table}

\subsubsection{The Conjugation of \textit{eḍ} ‘to be’}

\begin{table}[h]
\centering
\noindent\begin{tabular}{@{}|>{}l|>{\it}l|>{\it}l|}\hline
&\nf Sg&\nf Pl\\\hline
1st       & vy’í & ósó \\\hline
2nd       & ḍe   & b’heḍ \\\hline
3rd m     & le   & lẹsó \\\hline
3rd f     & lle  & llẹsó \\\hline
3rd n     & ý’e  & lasó \\\hline
Infinitive& \multicolumn{2}{c|}{\it éḍ} \\\hline
\end{tabular}
\caption{Paradigm of the verb \emph{eḍ}.}\label{tab:ed-paradigm}
\end{table}







\twocolumn
\clearpage
\section{Dictionary}
\ExplSyntaxOn

\cs_new:Npn \start_entry: {
    \hangindent = 6pt
    \hangafter = 1
    \noindent
}

\def \pfabbr { \textsc { pf \space } }

%% word, part of speech, etymology, definition, (forms)
\def \entry #1 #2 #3 #4 #5 {
    \start_entry:

    %% Typeset word and part of speech.
    \textbf { \ignorespaces #1 } \space
    \textit { \ignorespaces #2 }

    %% Typeset etymology.
    \tl_set:Nn \l_tmpa_tl {#3}
    \tl_if_empty:NTF \l_tmpa_tl { } {
        \space [
            \ignorespaces \textit { \tl_use:N \l_tmpa_tl }
        ]
    }

    %% Typeset forms, if any.
    \tl_set:Nn \l_tmpa_tl {#5}
    \tl_if_empty:NTF \l_tmpa_tl { } {
        \space \textsc{forms}: \space
        \ignorespaces \tl_use:N \l_tmpa_tl
        .
    }

    %% Typeset definition.
    \space \ignorespaces #4 .
    \par
}

%% Reference to another entry.
\def \refentry #1 #2 {
    \start_entry:

    \textbf { \ignorespaces #1 } \space
    \(\to\) \space
    \textit { \ignorespaces #2 }
    .
    \par
}

\ExplSyntaxOff

%% This is a generated file
%% Do not edit this file manually
%%
%% To update this file, edit DICTIONARY.txt and rerun
%% GENERATE-DICTIONARY.sh

\entry{aç̇t’he}{v. tr.}{\pfabbr acheter}{To buy}{}
\entry{ad’hór}{v. tr.}{\pfabbr adore}{To love, adore}{}
\entry{ánvé}{v. tr.}{\pfabbr animer}{To bring to life, animate}{}
\entry{ḅárḍáḍ}{v.}{\pfabbr partante}{ (+ \s{aci}) To be interested in, willing to, ready to, prepared for}{}
\entry{ḅẹt’hẹ}{adj.}{\pfabbr petit}{Small, little}{}
\refentry{b’heḍ}{eḍ}
\refentry{b’heḍy’é}{eḍ}
\refentry{b’hu}{eḍ}
\entry{Çár}{n.}{}{*male given name, equivalent to English ‘Kyle’ or ‘Charles’*}{}
\entry{c’hes}{part.}{\pfabbr qu'est-ce que}{*interrogative particle*}{}
\entry{c’húr}{v.}{\pfabbr court}{To shrink, reduce in size, narrow}{}
\entry{ḍalẹ}{n.}{\pfabbr tableau}{Table}{}
\refentry{ḍe}{eḍ}
\refentry{ḍet’h}{eḍ}
\entry{dír}{v. tr.}{\pfabbr dire}{To say, tell}{}
\entry{Dóvníc’h}{n.}{}{*male or female given name, equivalent to English ‘Dominic’*}{}
\refentry{ḍyf}{eḍ}
\entry{ebhẹ}{adj.}{\pfabbr épais}{Thick}{}
\refentry{éḍ}{eḍ}
\entry{eḍrrá}{adj.}{\pfabbr étroit}{Pointy}{}
\entry{eḍ}{v. irreg. }{\pfabbr être}{To be}{*active only*. \s{pres: sg} *vy’í*, *ḍe*, *le*, *lle*, *s*; \s{pl} *ósó*, *b’heḍ*, *lẹsó*, *llẹsó*, *lasó*; \s{inf} *éḍ*. \s{pres ant: sg} *vẹ*, *ḍyf*, *leb’h*, *lleb’h*, *seb’h*; \s{pl} *ófý*, *b’hu*, *lẹfýr*, *llẹfýr*, *lafýr*; \s{inf} *éfyḍ* \s{pret: sg} *vet’h*, *ḍet’h*, *let’h*, *llet’h*, *set’h*; \s{pl} *weḍy’ó*, *b’heḍy’é*, *let’he*, *llet’he*, *laet’h*; \s{inf} *ét’hẹd*}
\entry{Eḍy’ê}{n.}{}{*male given name, equivalent to English ‘Stephen’*}{}
\refentry{éfyḍ vet’h}{eḍ}
\entry{ehyó}{n.}{\pfabbr écusson}{Shield}{}
\refentry{ét’hẹd }{eḍ}
\entry{Já}{n.}{}{*male or female given name, equivalent to English ‘John’ or ‘Joan’*}{}
\refentry{laet’h}{eḍ}
\refentry{lafýr}{eḍ}
\entry{lár}{adj.}{\pfabbr large}{Wide, broad}{}
\refentry{lasó}{eḍ}
\refentry{leb’h}{eḍ}
\refentry{le}{eḍ}
\refentry{lẹfýr}{eḍ}
\refentry{lẹsó}{eḍ}
\refentry{let’h}{eḍ}
\refentry{let’he}{eḍ}
\refentry{lleb’h}{eḍ}
\refentry{lle}{eḍ}
\refentry{llẹfýr}{eḍ}
\refentry{llẹsó}{eḍ}
\refentry{llet’h}{eḍ}
\refentry{llet’he}{eḍ}
\entry{ló}{adj.}{\pfabbr long}{Long}{}
\entry{lúr}{adj.}{\pfabbr lourd}{Bulky, oversized, heavy}{}
\entry{ob’heír}{v. (in)tr.}{\pfabbr obéir}{To obey}{}
\refentry{ófý}{eḍ}
\refentry{ósó}{eḍ}
\entry{rá}{adj.}{\pfabbr grand}{Big, large, great}{}
\entry{ráhó}{n.}{\pfabbr poisson}{Fish}{}
\entry{rvá}{interj.}{of unknown origin}{Alas, woe, oh. *Exclamation of distress, surprise, sadness, or regret*}{ *after words that end with ‘r’, this is spelt ‘-vá’ instead*}
\refentry{seb’h}{eḍ}
\refentry{s}{eḍ}
\refentry{set’h}{eḍ}
\refentry{vá }{rvá}
\refentry{vẹ}{eḍ}
\entry{vôt’há}{n.}{\pfabbr montagne}{Mountain}{}
\entry{vvóríhe}{v. (in)tr.}{\pfabbr mémoriser}{To remember}{}
\refentry{vy’í}{eḍ}
\refentry{weḍy’ó}{eḍ}
\entry{y’ír}{v. (in)tr.}{\pfabbr ouïr}{To hear, understand, listen}{}






\end{document}

