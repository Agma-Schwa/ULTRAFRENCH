\documentclass[a4paper, 12pt, oneside, final]{article}
\usepackage[margin=2cm]{geometry}
\usepackage{fontspec}
\usepackage{unicode-math}
\usepackage[english]{babel}
\usepackage{csquotes}
\usepackage{array, tabularx, multirow}
\usepackage{longtable}
\usepackage{float}

\setmainfont[Numbers=OldStyle]{Minion 3}
\setmathfont{latinmodern-math.otf}
\setmathfont[range=\mathit]{Minion 3 Italic}
\frenchspacing

\AtBeginDocument{
    \def\today{
        \number\day\space
        \ifcase\month\or
        January\or February\or March\or April\or May\or June\or
        July\or August\or September\or October\or November\or December\fi\space
        \number\year
    }
}

\makeatletter
\def\footnoterule{%
    \kern-3\p@
    \hrule\@width.4\columnwidth
    \kern2.6\p@
}

\def\@makefntext#1{%
    \setlength\parindent{1em}%
    \noindent
    {%
    \mbox{\llap{{}\textsuperscript{\@thefnmark}\kern.5pt}}{#1}%1
    }%
}
\makeatother

\title{A Comprehensive Diachronic Grammar of Modern ULTRAFRENCH}
\author{Agma Schwa \& Ætérnal}
\date{\today}

\ExplSyntaxOn
\cs_new:Npn \__two_cols:nnnnn #1 #2 #3 #4 #5 {
    \ifvmode\else\unskip\par\fi
    \noindent\leavevmode
    \hbox to \hsize {
        \hbox to #3 { \vtop {#1} }
        \hskip   #4
        \hbox to #5 { \vtop {#2} }
    }\par
}

\NewDocumentCommand \TwoCols {
    D[]{.475\hsize}
    D[]{.475\hsize}
    D[]{0pt plus 1fill}
    +m
    +m
} {
    \__two_cols:nnnnn{#4}{#5}{#1}{#3}{#2}
}

\def \UF { \bfseries \itshape }
\let \nf \normalfont

\def \d {ḍ}
\def \D {Ḍ}
\def \b {ḅ}
\def \B {Ḅ}
\let \s \textsc

\def \L {\textsuperscript{L}}
\def \N {\textsuperscript{N}}

\char_set_catcode_active:N \*
\cs_set_protected:Npn \__md_star:w #1*{ \textit{#1} }
\def\* { \detokenize{*} }
\cs_new:Npn * { \__md_star:w }

\cs_new:Npn \items { \itemize\itemsep6pt }
\cs_new:Npn \enditems { \enditemize }

\ExplSyntaxOff

\newlength{\EnumItemSep} \EnumItemSep-3pt

\newenvironment{enum}[1][0]{%
    \vspace{-.5em}%
    \settowidth{\leftmargini}{99.\hskip\labelsep}%
    \begin{enumerate}\setcounter{enumi}{#1}\itemsep\EnumItemSep
}{%
    \end{enumerate}%
    \vspace{-.5em}%
}

\def\parheading#1{\noindent\textbf{#1}}

\let\Sub\textsubscript

\begin{document}
\maketitle
\thispagestyle{empty}
\clearpage
\setcounter{page}{1}

\tableofcontents
\clearpage

\section{Phonology and Evolution from Modern Pseudo-French}\label{sec:phonology}{\def\arraystretch{1.25}\setlength{\tabcolsep}{.4em}
\noindent\begin{tabular}{@{}|l|l|l|l|l|l|l@{\quad}|l|l|l|}                                                   \cline{1-6} \cline{8-10}
               & Labial & Coronal  & Palatal  & Velar & Glottal &&           & Front        & Back        \\ \cline{1-6} \cline{8-10}
    Stop       & b, bʱ  & d        &          &       &         && Close     & i ĩ ĩ̃ i̥, y ỹ ỹ̃ ẙ & u ũ ũ̃ u̥ \\ \cline{1-6} \cline{8-10}
    Nasal      &        & n        &          &       &         && Close-mid & e ẽ ẽ̃ e̥      & o o̥         \\ \cline{1-6} \cline{8-10}
    Fricative  & ɸ β, ʋ̃ & s z, θ ð & ç ɕ ʑ    & x χ   & h       && Mid       & \multicolumn{2}{c|}{ə ⟨ẹ⟩ ə̥}   \\ \cline{1-6} \cline{8-10}
    Approx.    &        &          & ɥ ɥ̃, j̊   & ɰ ɰ̃   &         && Open-mid  & ɛ ɛ̃ ɛ̃̃ ɛ̥      & ɔ̃ ɔ̃̃         \\ \cline{1-6} \cline{8-10}
    Lat. Fric. &        & ɮ̃        & ʎ̝̃        &       &         && Open      & a ḁ          & ɑ̃ ɑ̃̃         \\ \cline{1-6} \cline{8-10}
\end{tabular}}\bigskip

\parheading{Legend}\par\noindent
Ṽ = nasalised vowel, Ṽ̃ = nasal vowel, V = any vowel (or, in conjunction with Ṽ/Ṽ̃, oral vowel)\\
N = nasal consonant, C̃ = nasalised consonant (e.g. /ɰ̃/, but not true nasals), C = any consonant.\medskip
\def\scalpha{\kern-2pt\raisebox{2pt}{\Sub α}}

%% NOTE: In case the changes below and the ones listed
%% in the Lexurgy file differ, the latter are authoritative,
%% as I may forget to update these here sometimes.

\TwoCols[.45\hsize][.45\hsize][0pt]{
\parheading{Preliminary Changes}
\begin{enum}
    \item g, ʁ, w > ɰ ⟨r⟩
    \item œ, œ̃, ø > y, ỹ, ỹ
    \item ɔ > o
    \item y > j / \_(\#)V
    \item V\scalpha > $\emptyset$ / \_\#V\scalpha
    \item lj, lɥ > ʎ
    \item j > ɥ ⟨y’⟩
    \item ɰ > ɥ / \_i
    \item C > $\emptyset$ / \#\_C
    \item C > $\emptyset$ / C\_\#
    \item k > x\footnotemark ⟨c’h⟩
    \item ʃ, ʒ > ɕ ⟨ç⟩, ʑ ⟨j⟩
    \item nt > nθ
    \item t > \d{} [d] (‘hard /d/’)
    \item p > \b{} [b] (‘hard /b/’)
    \item f, v > ɸ ⟨f⟩, β ⟨b’h⟩
\end{enum}
}{
\parheading{Great Nasal Shift}
\begin{enum}[15]
    \item Ṽl > ɰ̃ ⟨w⟩
    \item V > Ṽ̃ / [NC̃ɥɰ]\_N\#
    \item V, Ṽ > Ṽ, Ṽ̃ / \_[NC̃ɥɰ], [NC̃ɥɰ]\_
    \item ə̃, ə̃̃, ã, ã̃, õ, õ̃ > ɛ̃, ɛ̃̃, ɑ̃, ɑ̃̃, ɔ̃, ɔ̃̃
    \item N, C̃ > $\emptyset$ / V\_\#
    \item ɲ, ŋ > n
    \item V, Ṽ > $\emptyset$ / N \_ N
    \item m, l, ʎ > ʋ̃ ⟨v⟩, ɮ̃ ⟨l⟩, ʎ̝̃ ⟨ḷ⟩
\end{enum}

\parheading{Intervocalic Lenition (/ V\_V is implied)}
\begin{enum}[21]
    \item x, s, z > h\footnotemark
    \item ɕ, ɮ̃, ʎ̝̃ > j̊ ⟨ç̇⟩, ɥ̃, ɰ̃
    \item nθ > n
    \item d, \d{}, b, \b{} > ð ⟨d’h⟩, θ ⟨t’h⟩, β, bʱ ⟨bh⟩
    \item ɸ > β / V\_V
\end{enum}

\parheading{Late Changes}
\begin{enum}[25]
    \item C[+stop, -alveolar]C\scalpha > C\scalpha
    \item h > $\emptyset$ / hV\_
    \item ə > $\emptyset$ / C\_C
    \item V[-nasalised, -nasal] > ə̥ / \_\#
\end{enum}
}\medskip

\footnotetext{[χ] around back vowels, [ɕ] elsewhere. For the purpose of sound changes, both are treated as [x].}
\footnotetext[2]{[ç] before variants of /i/ and /y/, [h] elsewhere.}

\subsection{Orthography}
The spelling of most UF sounds is indicated above; the less exotic consonants are spelt as
one might expect. That is, /b, d, n, ɸ, s, z, h/ are spelt ⟨b, d, n, f, s, z, h⟩, respectively.

The ‘hard’ voiced *ḅ*, *ḍ* which are pronounced exactly like their regular counterparts, are normally also spelt ⟨b⟩ and
⟨d⟩. However, the dot below is commonly used in dictionaries and grammatical material to distinguish between the two
as they differ from one another in how they are lenited.

%% TODO: ORTHOGRAPHY. Dot below is schwa if it’s an e, nasalised e instead of ɛ if it has an accent,
%% hard d, b, as well as palatal l. dot above ç indicates lenition.

\section{Accidence}\label{sec:accidence}
\subsection{Verbal Morphology}\label{subsec:verbal-morphology}
Verbs in UF are inflected for person, number, tense, aspect, mood, and voice. Verbal inflexion is mainly done
by means of concatenating a vast set of prefixes onto a verb, with the occasional suffix and circumfix making
its appearance. This chapter details these affixes, their meanings, uses, forms, and restrictions.


\subsubsection{Active/Passive Affixes}\label{subsubsec:active-passive-affixes}
UF has a set of active/subject as well as passive/object prefixes which can be used on their own or in combination
with one another, though at most one active and one passive prefix may be combined with a verb.\footnote{Irrespective
of whether they are personal or infinitive prefixes. For instance, it would also be illegal to combine e.g. the active
infinitive prefix with the first person active singular prefix.} Table~\ref{tab:active-passive-prefixes}
below lists those prefixes, two of which are actually circumfixes.

\begin{table}[H]
\centering
\noindent\begin{tabular}{@{}|>{}l|>{\it}l|>{\it}l|>{}l|>{}l|>{\it}l|>{\it}l|}\cline{1-3}\cline{5-7}
 Active&\nf Sg&\nf Pl& & Passive&\nf Sg&\nf Pl\\\cline{1-3}\cline{5-7}
1st&j-&ó-/r-/w- -(y’)ó&&1st&v-&ó-/r-/w-\\\cline{1-3}\cline{5-7}
2nd&\d{}(ẹ)-&b’h(y)- -(y’)é&&2nd&\d{}(ẹ)-&b’h(y)-\\\cline{1-3}\cline{5-7}
3rd m&l(ẹ)-&l(ẹ)-&&3rd m&y’-&lý-\\\cline{1-3}\cline{5-7}
3rd f&ll(a)-&ll(ẹ)-&&3rd f&y’- &lý-\\\cline{1-3}\cline{5-7}
3rd n&s- &l(a)-&&3rd n&sy-&lý-\\\cline{1-3}\cline{5-7}
Infinitive&\multicolumn{2}{c|}{\it d(ẹ)-}&&Infinitive&\multicolumn{2}{c|}{\it à-/h-}\\\cline{1-3}\cline{5-7}
\end{tabular}
\caption{Active (left) and passive (right) verbal affixes.}\label{tab:active-passive-prefixes}
\end{table}

\noindent A great degree of syncretism can be observed in the third-person forms. The gender distinction in the
\s{3sg} that diachronically resulted from gendered personal pronouns is almost non-existent in the
plural; the reason for this development is that those forms are derived from the old dative form, which lacked
this distinction altogether.

The \s{act 1pl, 2pl} forms are only distinguished from their passive counterparts by
the presence of additional suffixes in the former. The \s{3sg n} in the active and passive is derived from the PF
demonstrative \**ce* and its variants; the \s{3pl n} is derived from the other \s{3pl} forms.

The \s{1pl} prefix varies if there is a vowel following it: if it is
any vowel that is not a variant of ‘o’, the prefix is realised as *r-* instead, e.g. *ad’hór* ‘love’ to
*rad’hóró* ‘we love’. If the vowel a variant of ‘o’, the prefix is realised as *w-* instead, e.g. *ob’heír* ‘obey’
to *wob’heíró* ‘we obey’.\footnote{Diachronically, the base form of this prefix is \**o-*, whence e.g.
\**oad’hóró* > *rad’hóró* and \**oob’heíró* > *wob’heíró*.}

The \s{inf pass} prefix *à-* coalesces with any vowel following it: it becomes *á* if it
is followed by a non-nasal variant of ‘a’, e.g. *ad’hór* to *ád’hór* ‘to be loved’; *â* if it is
followed by a nasal variant of ‘a’, e.g. *ánvé* ‘give life to’ to *ânvé* ‘to be animated’; and *h-* if it is
followed by any other vowel, e.g. *ob’heír* to *hob’heír* ‘to be obeyed’.

The parenthesised vowels are used if the prefix is followed by a consonant, e.g. *dír* ‘say’ to *llẹ{}dír*
‘they (\s{f}) say’ and *b’hydíré* ‘you (\s{pl}) say’, but *ad’hór* to *llad’hór* ‘they (\s{f}) love’ and *b’had’hóré* ‘you
(\s{pl}) love’. The prefixes *ó-* and *à-* retain their main forms if followed by a consonant,
e.g. *dír* ‘say’ to *ódíró* ‘We say’ and *àdír* ‘to be said’. The exception to this is that \s{2pl} *b’h(y)-*
drops the *y* if followed by a glide, e.g. *y’ír* ‘to hear’ to *b’hy’íré* ‘you (\s{pl}) hear’ (not \**b’hyy’íré*).

The *y’* in the suffix parts of the \s{1pl, 2pl act} are dropped if the verb ends with a consonant, e.g. *ad’hór*
to *b’hád’hóré*, or if it ends with a vowel that is a variant of ‘o’ in the case of the \s{1pl} and ‘e’ in the case
of the \s{2pl}, in which cases the vowels are contracted and a level of nasalisation is added, e.g. *vvóríhe*
‘to remember’ to *b’hyvvóríhé* ‘you (\s{pl}) remember’ (not \**b’hyvvóríhy’é*). In all other cases, the *y’* is retained,
e.g. *óvvóríhey’ó* ‘we remember’.

When multiple prefixes are used together, active prefixes precede passive prefixes, except that infinitive prefixes
always come first, e.g. *ad’hór* ‘love’ to *jvad’hór* ‘I love myself’ (not \**vjad’hór*) and *b’hy’ad’hóré* ‘you (\s{pl}) love him/her’,
but *dẹvad’hór* ‘to love me’ and *àb’had’hóré* ‘to be loved by you (\s{pl})’. Recall that at most one infinitive prefix
may be used.

By way of illustration, consider the paradigm of the verb *ad’hór* as shown in Table~\ref{tab:adhor-paradigm} below.
Since this word starts with a vowel, the parenthesised vowels in Table~\ref{tab:active-passive-prefixes} above
are not used. Furthermore, since it starts with a non-nasal ‘a’-like vowel, the *ó-* prefix is realised as *r-*
and the *à-* prefix coalesces with the initial ‘a’ of the stem to form *á*.

% TEMPLATE:
%\noindent\begin{tabular}{@{}|>{}l|>{\it}l|>{\it}l|>{}l|>{}l|>{\it}l|>{\it}l|}\cline{1-3}\cline{5-7}
%\nf Active&\nf Sg&\nf Pl&\nf &\nf Passive&\nf Sg&\nf Pl \\\cline{1-3}\cline{5-7}
%1st       &   &  &&1st   &   &   \\\cline{1-3}\cline{5-7}
%2nd       &   &  &&2nd   &   &   \\\cline{1-3}\cline{5-7}
%3rd m     &   &  &&3rd m &   &   \\\cline{1-3}\cline{5-7}
%3rd f     &   &  &&3rd f &   &   \\\cline{1-3}\cline{5-7}
%3rd n     &   &  &&3rd n &   &   \\\cline{1-3}\cline{5-7}
%Infinitive& \multicolumn{2}{c|}{\it }  && Infinitive & \multicolumn{2}{c|}{\it } \\\cline{1-3}\cline{5-7}
%\end{tabular}

\begin{table}[H]
\centering
\noindent\begin{tabular}{@{}|>{}l|>{\it}l|>{\it}l|>{}l|>{}l|>{\it}l|>{\it}l|}\cline{1-3}\cline{5-7}
\nf Active&\nf Sg&\nf Pl&\nf &\nf Passive&\nf Sg&\nf Pl\\\cline{1-3}\cline{5-7}
1st&jad’hór&rad’hóró&&1st&vad’hór&rad’hór\\\cline{1-3}\cline{5-7}
2nd&\d{}ad’hór&b’had’hóré&&2nd&\d{}ad’hór&b’had’hór\\\cline{1-3}\cline{5-7}
3rd m&lad’hór&lad’hór&&3rd m&y’ad’hór&lýad’hór\\\cline{1-3}\cline{5-7}
3rd f&llad’hór&llad’hór&&3rd f&y’ad’hór &lýad’hór\\\cline{1-3}\cline{5-7}
3rd n&ý’ad’hór&lad’hór&&3rd n&ý’ad’hór&lýad’hór\\\cline{1-3}\cline{5-7}
Infinitive&\multicolumn{2}{c|}{\it dad’hór}&&Infinitive&\multicolumn{2}{c|}{\it ád’hór}\\\cline{1-3}\cline{5-7}
\end{tabular}
\caption{Paradigm of the Verb \emph{ad’hór}.}\label{tab:adhor-paradigm}
\end{table}

\noindent For comparison, the paradigm of the verb *vvóríhe* ‘remember’ is shown in Table~\ref{tab:vvorihe-paradigm} below.
Since it starts with a consonant, the parenthesised vowels in Table~\ref{tab:active-passive-prefixes} are used, and any
prefixes that end with a vowel remain unchanged.

\begin{table}[H]
\centering
\noindent\begin{tabular}{@{}|>{}l|>{\it}l|>{\it}l|>{}l|>{}l|>{\it}l|>{\it}l|}\cline{1-3}\cline{5-7}
\nf Active&\nf Sg&\nf Pl&\nf &\nf Passive&\nf Sg&\nf Pl\\\cline{1-3}\cline{5-7}
1st&jvvóríhe&óvvóríhey’ó&&1st&vvvóríhe&óvvóríhe\\\cline{1-3}\cline{5-7}
2nd&ḍẹvvóríhe&b’hyvvóríhé&&2nd&ḍẹvvóríhe&b’hyvvóríhe\\\cline{1-3}\cline{5-7}
3rd m&lẹvvóríhe&lẹvvóríhe&&3rd m&y’vvóríhe&lývvóríhe\\\cline{1-3}\cline{5-7}
3rd f&llavvóríhe&llẹvvóríhe&&3rd f&y’vvóríhe&lývvóríhe\\\cline{1-3}\cline{5-7}
3rd n&ý’vvóríhe&lavvóríhe&&3rd n&ý’vvóríhe&lývvóríhe\\\cline{1-3}\cline{5-7}
Infinitive&\multicolumn{2}{c|}{\it dẹvvóríhe}&&Infinitive&\multicolumn{2}{c|}{\it àvvóríhe}\\\cline{1-3}\cline{5-7}
\end{tabular}
\caption{Paradigm of the Verb \emph{vvóríhe}.}\label{tab:vvorihe-paradigm}
\end{table}

\subsection{Tense and Aspect Marking}\label{subsec:tense-and-aspect-marking}
Tense in PF is marked by additional sets of affixes that are appended to the verb in addition to the active/passive affixes.
There are two broad groups of such affixes: suffixes, which are appended to the end of the verb and replace the \s{act 1pl, 2pl} suffixes
in those persons, as well as circumfixes and prefixes, which are inserted before the active/passive markers and replace the
replace the \s{act 1pl, 2pl} suffixes in some cases.


\subsubsection{Suffixed Tenses}
The present anterior and preterite are formed by appending a set of suffixes to the verb. Table~\ref{tab:present-anterior-and-preterite-suffixes}
below lists the suffixes for those tenses. The present anterior has a perfective aspect, while the preterite has an imperfective aspect. The
former is commonly used to describe events that are completed—particularly events that occurred recently, hence the name—while the latter
is used to describe events that are ongoing or habitual.


\begin{table}[H]
\centering
\noindent\begin{tabular}{@{}|>{}l|>{\it}l|>{\it}l|>{}l|>{}l|>{\it}l|>{\it}l|}\cline{1-3}\cline{5-7}
\nf Present Anterior&\nf Sg&\nf Pl&\nf &\nf Preterite&\nf Sg&\nf Pl \\\cline{1-3}\cline{5-7}
1st       & -\L é & -\L â &&1st    & -\L á  & -y’ô  \\\cline{1-3}\cline{5-7}
2nd       & -\L á & -\L áḍ &&2nd   & -\L é  & -y’ẹ́  \\\cline{1-3}\cline{5-7}
3rd       & -\L á & -\L ér &&3rd m & -\L é  & -\L é   \\\cline{1-3}\cline{5-7}
Infinitive& \multicolumn{2}{c|}{\it -á }  && Infinitive & \multicolumn{2}{c|}{\it -é } \\\cline{1-3}\cline{5-7}
\end{tabular}
\caption{Present Anterior and Preterite Affixes.}\label{tab:present-anterior-and-preterite-suffixes}
\end{table}

\noindent Neither tense distinguishes gender in the third person. All suffixes, except for the infinitive and \s{1pl, 2pl pret},
lenite any consonant *before* them, e.g. *ḅárḍáḍ* ‘to be willing’ to *jḅárḍát’hé* ‘I was willing’ but *dẹḅárḍáḍá*
‘to have been willing’.

Diachronically, the \s{1sg pret} is an interesting case; in EUF, it was originally \**-é*, but it later changed to *-á*
to distinguish it from the \s{2sg, 3sg pres ant}. The remaining forms—save the infinitives, which are derived from the
tenses’ definite endings by analogy—originated from the PF simple past tenses.

The table below lists the example paradigm of the verb *ad’hór* in the present anterior and preterite tenses.
Observe that there is no difference between the \s{1pl, 2pl} active and passive.

\begin{table}[H]
\centering
\noindent\begin{tabular}{@{}|>{}l|>{\it}l|>{\it}l|>{}l|>{}l|>{\it}l|>{\it}l|}\cline{1-3}\cline{5-7}
\nf Active & \nf Sg   & \nf Pl     & \nf & \nf Passive & \nf Sg   & \nf Pl    \\\cline{1-3}\cline{5-7}
1st        & jad’hóré  & rad’hórâ     &     & 1st         & vad’hóré  & rad’hórâ   \\\cline{1-3}\cline{5-7}
2nd        & ḍad’hórá  & b’had’hóráḍ  &     & 2nd         & ḍad’hórá  & b’had’hóráḍ \\\cline{1-3}\cline{5-7}
3rd m      & lad’hórá  & lad’hórér    &     & 3rd m       & y’ad’hórá & lýad’hórér  \\\cline{1-3}\cline{5-7}
3rd f      & llad’hórá & llad’hórér   &     & 3rd f       & y’ad’hórá & lýad’hórér  \\\cline{1-3}\cline{5-7}
3rd n      & ý’ad’hórá & lad’hórér    &     & 3rd n       & ý’ad’hórá & lýad’hórér  \\\cline{1-3}\cline{5-7}
Infinitive & \multicolumn{2}{c|}{\it dad’hórá} & & Infinitive & \multicolumn{2}{c|}{\it ád’hórá} \\\cline{1-3}\cline{5-7}
\end{tabular}
\caption{Present Anterior Paradigm of the Verb \emph{ad’hór}.}\label{tab:adhor-paradigm-pres-ant}
\end{table}

\begin{table}[H]
\centering
\noindent\begin{tabular}{@{}|>{}l|>{\it}l|>{\it}l|>{}l|>{}l|>{\it}l|>{\it}l|}\cline{1-3}\cline{5-7}
\nf Active & \nf Sg   & \nf Pl     & \nf & \nf Passive & \nf Sg   & \nf Pl    \\\cline{1-3}\cline{5-7}
1st        & jad’hórá  & rad’hóry’ô     &     & 1st     & vad’hórá  & rad’hóry’ô   \\\cline{1-3}\cline{5-7}
2nd        & ḍad’hóré  & b’had’hóry’ẹ́  &     & 2nd      & ḍad’hóré  & b’had’hóry’ẹ́ \\\cline{1-3}\cline{5-7}
3rd m      & lad’hóré  & lad’hóré    &     & 3rd m      & y’ad’hóré & lýad’hóré  \\\cline{1-3}\cline{5-7}
3rd f      & llad’hóré & llad’hóré   &     & 3rd f      & y’ad’hóré & lýad’hóré  \\\cline{1-3}\cline{5-7}
3rd n      & ý’ad’hóré & lad’hóré   &     & 3rd n       & ý’ad’hóré & lýad’hóré  \\\cline{1-3}\cline{5-7}
Infinitive & \multicolumn{2}{c|}{\it dad’hóré} & & Infinitive & \multicolumn{2}{c|}{\it ád’hóré} \\\cline{1-3}\cline{5-7}
\end{tabular}
\caption{Preterite Paradigm of the Verb \emph{ad’hór}.}\label{tab:adhor-paradigm-pret}
\end{table}


\subsection{Irregular Verbs}\label{subsec:irregular-verbs}
\subsubsection{The Conjugation of \textit{eḍ} ‘to be’}


\begin{table}[H]
\centering
\noindent\begin{tabular}{@{}|>{}l|>{\it}l|>{\it}l|l|l|>{\it}l|>{\it}l|l|l|>{\it}l|>{\it}l|}\cline{1-3}\cline{5-7}\cline{9-11}
Present&\nf Sg&\nf Pl    && Pres. Ant.&\nf Sg&\nf Pl    && Preterite&\nf Sg&\nf Pl      \\\cline{1-3}\cline{5-7}\cline{9-11}
1st       & vy’í & ósó   && 1st       & vẹ     & ófý    && 1st       & vet’h & weḍy’ó   \\\cline{1-3}\cline{5-7}\cline{9-11}
2nd       & ḍe   & b’heḍ && 2nd       & ḍyf    & b’hu   && 2nd       & ḍet’h & b’heḍy’é \\\cline{1-3}\cline{5-7}\cline{9-11}
3rd m     & le   & lẹsó  && 3rd m     & leb’h  & lẹfýr  && 3rd m     & let’h & let’he   \\\cline{1-3}\cline{5-7}\cline{9-11}
3rd f     & lle  & llẹsó && 3rd f     & lle’bh & llẹfýr && 3rd f     & llet’h & llet’he \\\cline{1-3}\cline{5-7}\cline{9-11}
3rd n     & s    & lasó  && 3rd n     & seb’h  & lafýr  && 3rd n     & set’h & laet’h   \\\cline{1-3}\cline{5-7}\cline{9-11}
Infinitive& \multicolumn{2}{c|}{\it éḍ} && Infinitive& \multicolumn{2}{c|}{\it éfyḍ} && Infinitive& \multicolumn{2}{c|}{\it ét’hẹd} \\\cline{1-3}\cline{5-7}\cline{9-11}
\end{tabular}
\caption{Paradigm of the verb \emph{eḍ}.}\label{tab:ed-paradigm}
\end{table}

\noindent The etymology of these forms is mostly from a gradual simplification of coalesced forms of the personal
pronouns with the PF copula. To compensate for the fact that PF lacks certain forms that are present in UF, some
of the forms were coined by analogy. For instance, the \s{pres ant inf} *éfyḍ* is derived from the \s{pres ant}
stem \**fy* and the \s{pres inf} *éḍ*, and the same is true for the \s{pret inf} *ét’hẹd*.

For obvious reasons, the copula lacks passive forms. At the same time, the first person forms are manifestly
derived from the first person passive pronoun, for unknown reasons.

\subsection{Noun Morphology}\label{subsec:noun-morphology}
UF has 4 declensions. A definite and indefinite vocalic declension, and a definite and indefinite consonantal declension.
As their names might suggest, the former two are used for nouns that start with a vowel, and the latter two for nouns
that start with a consonant. UF has no morphologically separate articles; rather, the old PF articles have been incorporated
into the declensions. Furthermore, UF no longer has a gender distinction in nouns.

\subsubsection{Declension}
The table below shows the affixes of the definite and indefinite declensions. The declensions are mostly identical,
except that, as with the conjugation of verbs, the consonantal prefixes often end in a vowel (marked below with
parentheses), which are not present in the vocalic declension.

\begin{table}[H]
\centering
\noindent\begin{tabular}{@{}|l|>{\it}l|>{\it}l|l|l|>{\it}l|>{\it}l|}\cline{1-3}\cline{5-7}
Definite    &\nf Sg&\nf Pl && Indefinite       &\nf Sg&\nf Pl\\\cline{1-3}\cline{5-7}

Nominative  & lá-\L & lé-\L                    &&Nominative &ŷn-\N & ý-\L         \\\cline{1-3}\cline{5-7}
Vocative    & $\emptyset$-\L & $\emptyset$-\L  &&Vocative   & / & /                     \\\cline{1-3}\cline{5-7}
Partitive   & dy-\L  & dẹ-\L                   &&Partitive  &dŷn-\N & dý-\L                      \\\cline{1-3}\cline{5-7}
Accusative  & y’i-\L  & sý-\L                  &&Accusative & s-\L & s-                       \\\cline{1-3}\cline{5-7}
...         &  &                               &&           & &              \\\cline{1-3}\cline{5-7}
Inessive    & dwá- & dwé-                      &&Inessive   & dáhŷn- & dáhŷ-                    \\\cline{1-3}\cline{5-7}
\end{tabular}
\caption{UF Declension.}\label{tab:table-uf-declension}
\end{table}

\noindent Most of these forms cause lenition in the initial consonant of the noun, e.g. *ḍalẹ* ‘table’ to
\s{def acc sg} *s’thalẹ*; this lenition is blocked in the \s{indef acc pl} due to the presence of a hypercorrected ‘s’
in PF \**ces*, e.g. *s’ḍalẹ* ‘the tables (\s{acc})’ (not *s’thalẹ*, which is the singular), as well as in
less commonly used forms such as the \s{def} inessive *dwáḍalẹ* ‘on the table’.

The \s{indef nom sg} *ŷn-* prefix and some other forms nasalise nouns; as a reminder, this means that in
nouns starting with *ḍ*, the *ḍ* is deleted, e.g. *ŷnalẹ* ‘a table’;
it causes nasalisation in words that start with a vowel e.g. *ehyó* ‘shield’ to *ŷnéhyó* ‘a shield’. The indefinite vocative
does not exist, as that would make little sense. As lenition, nasalisation too is blocked in rarer forms, e.g. \s{indef} inessive
*dáhŷnḍalẹ* ‘on a table’.

The diachrony of these forms is mostly from the PF definite and indefinite pronouns, though some forms, such as the
accusative, are borrowed from demonstratives instead (\s{def} from PF \**celui* and \s{indef} from PF \**ce*); the definite
partitive forms are from the PF partitive article, and
the indefinite forms are formed with an additional *d-* by analogy to the definite forms. The locative cases are combinations
of the articles and PF prepositions.

\begin{table}[H]
\centering
\noindent\begin{tabular}{@{}|l|>{\it}l|>{\it}l|l|l|>{\it}l|>{\it}l|}\cline{1-3}\cline{5-7}
Definite    &\nf Sg&\nf Pl && Indefinite&\nf Sg&\nf Pl\\\cline{1-3}\cline{5-7}

Nominative  & lát’halẹ  & lét’halẹ   &&Nominative & ŷnalẹ & ýt’halẹ         \\\cline{1-3}\cline{5-7}
Vocative    & t’halẹ    & t’halẹ    &&Vocative   & / & /                      \\\cline{1-3}\cline{5-7}
Partitive   & dyt’halẹ  & dẹt’halẹ &&Partitive   & dŷnalẹ & dýt’halẹ                     \\\cline{1-3}\cline{5-7}
Accusative  & y’it’halẹ & sýt’halẹ  &&Accusative & st’halẹ & sḍalẹ                       \\\cline{1-3}\cline{5-7}
...         &  &  &&           & &              \\\cline{1-3}\cline{5-7}
Inessive    & dwáḍalẹ & dwéḍalẹ &&Inessive   & dáhŷnḍalẹ & dáhýḍalẹ                    \\\cline{1-3}\cline{5-7}
\end{tabular}
\caption{Consonantal declension of *ḍalẹ*.}\label{tab:vocalic-declension}
\end{table}



\section{Examples}
\noindent\begin{tabular}{@{}lllll}
\multicolumn{5}{@{}l}{\it Çár-vá, sráhó dwávôt’há daçt’heá?}\\
Ç̇ár &-vá &s-ráhó &dwá-vôt’há &ḍ-aç̇t’he-á\\
ˈj̊ɑ̃ɰ&ɰʋ̃ɑ̃&ˌsɰɑ̃ˈhɔ̃&dɰɑ̃ˌʋ̃ɔ̃̃ˈθɑ̃&daj̊ˈθe.ɑ̃\\
Charles.\s{voc}&\s{particle}&\s{indef.acc}-fish&\s{def.iness}-mountain&\s{2sg.act}-buy-\s{pres.ant.2sg}\\
\multicolumn{5}{@{}l}{‘Charles, you bought a fish on the mountain?’}\\
\end{tabular}




\twocolumn
\clearpage
\section{Dictionary}
\ExplSyntaxOn

\cs_new:Npn \start_entry: {
    \hangindent = 6pt
    \hangafter = 1
    \noindent
}

\def \pfabbr { \textsc { pf \space } }

%% word, part of speech, etymology, definition, (forms)
\def \entry #1 #2 #3 #4 #5 {
    \start_entry:

    %% Typeset word and part of speech.
    \textbf { \ignorespaces #1 } \space
    \textit { \ignorespaces #2 }

    %% Typeset etymology.
    \tl_set:Nn \l_tmpa_tl {#3}
    \tl_if_empty:NTF \l_tmpa_tl { } {
        \space [
            \ignorespaces \textit { \tl_use:N \l_tmpa_tl }
        ]
    }

    %% Typeset forms, if any.
    \tl_set:Nn \l_tmpa_tl {#5}
    \tl_if_empty:NTF \l_tmpa_tl { } {
        \space {\nf\scshape{forms}}: \space
        \ignorespaces \tl_use:N \l_tmpa_tl
        .
    }

    %% Typeset definition.
    \space \ignorespaces #4 .
    \par
}

%% Reference to another entry.
\def \refentry #1 #2 {
    \start_entry:

    \textbf { \ignorespaces #1 } \space
    \(\to\) \space
    \textit { \ignorespaces #2 }
    .
    \par
}

\ExplSyntaxOff

%% This is a generated file
%% Do not edit this file manually
%%
%% To update this file, edit DICTIONARY.txt and rerun
%% GENERATE-DICTIONARY.sh

\entry{aç̇t’he}{v. tr.}{\pfabbr acheter}{To buy}{}
\entry{ad’hór}{v. tr.}{\pfabbr adore}{To love, adore}{}
\entry{ánvé}{v. tr.}{\pfabbr animer}{To bring to life, animate}{}
\entry{ḅárḍáḍ}{v.}{\pfabbr partante}{ (+ \s{aci}) To be interested in, willing to, ready to, prepared for}{}
\entry{ḅẹt’hẹ}{adj.}{\pfabbr petit}{Small, little}{}
\refentry{b’heḍ}{eḍ}
\refentry{b’heḍy’é}{eḍ}
\refentry{b’hu}{eḍ}
\entry{Çár}{n.}{}{*male given name, equivalent to English ‘Kyle’ or ‘Charles’*}{}
\entry{c’hes}{part.}{\pfabbr qu'est-ce que}{*interrogative particle*}{}
\entry{c’húr}{v.}{\pfabbr court}{To shrink, reduce in size, narrow}{}
\entry{ḍalẹ}{n.}{\pfabbr tableau}{Table}{}
\refentry{ḍe}{eḍ}
\refentry{ḍet’h}{eḍ}
\entry{dír}{v. tr.}{\pfabbr dire}{To say, tell}{}
\entry{Dóvníc’h}{n.}{}{*male or female given name, equivalent to English ‘Dominic’*}{}
\refentry{ḍyf}{eḍ}
\entry{ebhẹ}{adj.}{\pfabbr épais}{Thick}{}
\refentry{éḍ}{eḍ}
\entry{eḍrrá}{adj.}{\pfabbr étroit}{Pointy}{}
\entry{eḍ}{v. irreg. }{\pfabbr être}{To be}{*active only*. \s{pres: sg} *vy’í*, *ḍe*, *le*, *lle*, *s*; \s{pl} *ósó*, *b’heḍ*, *lẹsó*, *llẹsó*, *lasó*; \s{inf} *éḍ*. \s{pres ant: sg} *vẹ*, *ḍyf*, *leb’h*, *lleb’h*, *seb’h*; \s{pl} *ófý*, *b’hu*, *lẹfýr*, *llẹfýr*, *lafýr*; \s{inf} *éfyḍ* \s{pret: sg} *vet’h*, *ḍet’h*, *let’h*, *llet’h*, *set’h*; \s{pl} *weḍy’ó*, *b’heḍy’é*, *let’he*, *llet’he*, *laet’h*; \s{inf} *ét’hẹd*}
\entry{Eḍy’ê}{n.}{}{*male given name, equivalent to English ‘Stephen’*}{}
\refentry{éfyḍ vet’h}{eḍ}
\entry{ehyó}{n.}{\pfabbr écusson}{Shield}{}
\refentry{ét’hẹd }{eḍ}
\entry{Já}{n.}{}{*male or female given name, equivalent to English ‘John’ or ‘Joan’*}{}
\refentry{laet’h}{eḍ}
\refentry{lafýr}{eḍ}
\entry{lár}{adj.}{\pfabbr large}{Wide, broad}{}
\refentry{lasó}{eḍ}
\refentry{leb’h}{eḍ}
\refentry{le}{eḍ}
\refentry{lẹfýr}{eḍ}
\refentry{lẹsó}{eḍ}
\refentry{let’h}{eḍ}
\refentry{let’he}{eḍ}
\refentry{lleb’h}{eḍ}
\refentry{lle}{eḍ}
\refentry{llẹfýr}{eḍ}
\refentry{llẹsó}{eḍ}
\refentry{llet’h}{eḍ}
\refentry{llet’he}{eḍ}
\entry{ló}{adj.}{\pfabbr long}{Long}{}
\entry{lúr}{adj.}{\pfabbr lourd}{Bulky, oversized, heavy}{}
\entry{ob’heír}{v. (in)tr.}{\pfabbr obéir}{To obey}{}
\refentry{ófý}{eḍ}
\refentry{ósó}{eḍ}
\entry{rá}{adj.}{\pfabbr grand}{Big, large, great}{}
\entry{ráhó}{n.}{\pfabbr poisson}{Fish}{}
\entry{rvá}{interj.}{of unknown origin}{Alas, woe, oh. *Exclamation of distress, surprise, sadness, or regret*}{ *after words that end with ‘r’, this is spelt ‘-vá’ instead*}
\refentry{seb’h}{eḍ}
\refentry{s}{eḍ}
\refentry{set’h}{eḍ}
\refentry{vá }{rvá}
\refentry{vẹ}{eḍ}
\entry{vôt’há}{n.}{\pfabbr montagne}{Mountain}{}
\entry{vvóríhe}{v. (in)tr.}{\pfabbr mémoriser}{To remember}{}
\refentry{vy’í}{eḍ}
\refentry{weḍy’ó}{eḍ}
\entry{y’ír}{v. (in)tr.}{\pfabbr ouïr}{To hear, understand, listen}{}






\end{document}

