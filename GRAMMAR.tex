\documentclass[a4paper, 12pt, oneside, final]{article}
\usepackage[margin=2cm]{geometry}
\usepackage{fontspec}
\usepackage{unicode-math}
\usepackage[english]{babel}
\usepackage{csquotes}
\usepackage{array, tabularx, multirow}
\usepackage{longtable}
\usepackage{float}
\usepackage{tabularray}
\usepackage{graphicx}

\setmainfont[Numbers=OldStyle]{Minion 3}
\setmathfont{latinmodern-math.otf}
\setmathfont[range=\mathit]{Minion 3 Italic}

%% %%%%%%%%%%%%%%%%%%%%%%%%%%%%%%%%%%%%%%%%%%%%%%%%%%%%%%%%%%%%%%%%%%%%%%%%%%%%%
%%  Environment and Layout
%% %%%%%%%%%%%%%%%%%%%%%%%%%%%%%%%%%%%%%%%%%%%%%%%%%%%%%%%%%%%%%%%%%%%%%%%%%%%%%
\ExplSyntaxOn
\makeatletter

\def \dlabelstyle #1 {
    \def\descriptionlabel ##1 {\hspace\labelsep \normalfont #1 ##1}
}

\cs_new:Npn \__two_cols:nnnnn #1 #2 #3 #4 #5 {
    \ifvmode\else\unskip\par\fi
    \noindent\leavevmode
    \hbox to \hsize {
        \hbox to #3 { \vtop {#1} }
        \hskip   #4
        \hbox to #5 { \vtop {#2} }
    } \par
}

\NewDocumentCommand \TwoCols {
    D[]{.475\hsize}
    D[]{.475\hsize}
    D[]{0pt plus 1fill}
    +m
    +m
} {
    \__two_cols:nnnnn{#4}{#5}{#1}{#3}{#2}
}

\cs_new:Npn \__gloss_insert_table_header: {
    %% Generating columns doesn’t seem to work, so this hack will do. If you
    %% need a gloss with more columns than this, I suggest you pause and take
    %% a moment to reflect on your life choices.
    \begin {tabular} { @{} *{100}l }
}

\cs_new:Npn \__gloss_table_start: {
    \ifvmode\else\unskip\par\fi
    \addvspace { 8pt }
    %\vtop{\iffalse}\fi
    \noindent
}

\cs_new:Npn \__gloss_table_end: {
    \end {tabular}
    \ifvmode\else\par\fi
    \addvspace { 8pt }
    %\iffalse{\fi}
    \everypar { \setbox\z@\lastbox \everypar{} }
}

%% Rescan a token list.
\cs_new:Npn \__gloss_rescan:n #1 {
    { \tl_rescan:nn {} { #1 } }
}

%% Format a single line of a gloss.
%%
%% The line number must be stored in \g_tempb_int.
%%
%% This takes the line #1 and formats it, inserting the contents of #2
%% at the start of each line, and #3 at the beginning of each column.
%% #3 can be a macro that takes one argument, in which case it will be
%% passed the contents of the column.
\cs_new:Npn \__gloss_format_line:nnn #1 #2 #3 {
    #2

    \seq_set_split:Nnn \l_tmpa_seq { | } { #1 }
    \bool_set_true:N \l_tmpa_bool

    \seq_map_inline:Nn \l_tmpa_seq {
        \bool_if:NF \l_tmpa_bool { & }
        \bool_set_false:N \l_tmpa_bool
        #3 { \__gloss_rescan:n {##1} }
    }
}

%% Format a gloss.
%%
%% This iterates over all lines in #1 and calls #2 on it.
\cs_new:Npn \__gloss_format:nn #1 #2 {
    \int_gset:Nn \g_tmpb_int { 1 }
    \tl_map_inline:nn { #1 } {
        \tl_if_blank:nF { ##1 } {
            %% Insert \cr or table header if this is the first line. We
            %% need to emit this here, as otherwise, TeX will encounter
            %% unexpandable tokens before the formatter (#2) is executed,
            %% which will start the cell before any actual content has
            %% been inserted; this means we can no longer tell TeX to
            %% \omit the cell header, which causes \multicolumn to break
            %% horribly.
            %%
            %% By emitting the table header here, we ensure that instead,
            %% the formatter is the first thing TeX gets to see after the
            %% initial table header and after every \cr.
            \int_compare:nNnTF { \g_tmpb_int } > { 1 } { \\ } { \__gloss_insert_table_header: }
            #2 { ##1 }
            \int_gincr:N \g_tmpb_int
        }
    }
}

%% Two lines means object language + gloss.
\cs_new:Npn \__gloss_two_lines:n #1 {
    \cs_gset:Npn \__formatter:n ##1 {
        \__gloss_format_line:nnn { ##1 } {} {
            \int_compare:nNnT { \g_tmpb_int } = { 1 } { \itshape }
        }
    }

    \__gloss_format:nn { #1 } { \__formatter:n }
}

%% Full gloss (text, object language, pronunciation, gloss, and translation).
\cs_new:Npn \__gloss_five_lines:n #1 {
    \cs_gset:Npn \__formatter_i:n ##1 { \multicolumn {100} {@{}l} {\itshape\bfseries\__gloss_rescan:n {##1}} }
    \cs_gset:Npn \__formatter_ii:n ##1 { \__gloss_format_line:nnn {##1} {} {\itshape} }
    \cs_gset:Npn \__formatter_iii:n ##1 { \__gloss_format_line:nnn {##1} {} {} }
    \cs_gset:Npn \__formatter_iv:n ##1 { \__gloss_format_line:nnn {##1} {} {} }
    \cs_gset:Npn \__formatter_v:n ##1 { \multicolumn {100} {@{}l} {\__gloss_rescan:n {##1}} }

    \__gloss_format:nn { #1 } {
        \cs:w __formatter_ \int_to_roman:n \g_tmpb_int :n \cs_end:
    }
}

%% Full gloss w/o IPA (text, object language, gloss, and translation).
\cs_new:Npn \__gloss_four_lines:n #1 {
    \cs_gset:Npn \__formatter_i:n ##1 { \multicolumn {100} {@{}l} {\itshape\bfseries \__gloss_rescan:n {##1} } }
    \cs_gset:Npn \__formatter_ii:n ##1 { \__gloss_format_line:nnn {##1} {} {\itshape} }
    \cs_gset:Npn \__formatter_iii:n ##1 { \__gloss_format_line:nnn {##1} {} {} }
    \cs_gset:Npn \__formatter_iv:n ##1 { \multicolumn {100} {@{}l} {\__gloss_rescan:n {##1}} }

    \__gloss_format:nn { #1 } {
        \cs:w __formatter_ \int_to_roman:n \g_tmpb_int :n \cs_end:
    }
}

\NewDocumentCommand \gloss {
    > { \exp_args:Nx \SplitList { \iow_char:N \^^M } } +v
} {
    \__gloss_table_start:

    %% Count lines.
    \int_gset:Nn \g_tmpa_int { 0 }
    \tl_map_inline:nn { #1 } {
        \tl_if_blank:nF { ##1 } {
            \int_gincr:N \g_tmpa_int
        }
    }

    %% Dispatch the appropriate number of lines.
    \int_case:nnF { \g_tmpa_int } {
        2 { \__gloss_two_lines:n { #1 } }
        4 { \__gloss_four_lines:n { #1 } }
        5 { \__gloss_five_lines:n { #1 } }
    }

    %% Any other line count is an error.
    {
        \msg_new:nnn { gloss } { too-many-lines } {
            Too~many~lines~in~gloss:~expected~2,~got~\int_use:N \g_tmpa_int
        }

        \msg_error:nn { gloss } { too-many-lines }
    }

    %% Close the table.
    \__gloss_table_end:
}

\def \footnoterule {
    \kern -3\p@
    \hrule \@width .4\columnwidth
    \kern 2.6\p@
}

\def \@makefntext #1 {
    \setlength \parindent { 1em }
    \noindent {
        \mbox {
            \llap { {}\textsuperscript{\@thefnmark} \kern.5pt }
        } { #1 }
    }
}

\newlength{\EnumItemSep} \EnumItemSep-3pt

\newenvironment { enum } [1] [0] {
    \vspace { -.5em }
    \settowidth \leftmargini { 99.\hskip\labelsep }
    \begin { enumerate }
    \setcounter { enumi } { #1 }
    \itemsep \EnumItemSep
} {
    \end { enumerate }
    \vspace { -.5em }
}

\newenvironment { dlist } [1] [{}] {
    \vspace { -.5em }
    \begingroup
    \def\descriptionlabel ##1 {\hspace\labelsep \normalfont #1 ##1}
    \settowidth \leftmargini { 99.\hskip\labelsep }
    \begin { description }
    \itemsep \EnumItemSep
} {
    \end { description }
    \endgroup
    \vspace { -.5em }
}

%% Make it so *...* works just like in Markdown.
\char_set_catcode_active:N \*
\def\* { \detokenize{*} }
\cs_new:Npn * { \__md_star:w }

\cs_set_protected:Npn \__md_star:w {
    \peek_charcode_remove:NTF * { \__md_starstar:w } { \__md_singlestar:w }
}

\cs_set_protected:Npn \__md_starstar:w {
    \peek_charcode_remove:NTF * { \__md_triplestar:w } { \__md_doublestar:w }
}

\cs_set_protected:Npn \__md_singlestar:w #1*   { \textit{#1} }
\cs_set_protected:Npn \__md_doublestar:w #1**  { \textbf{#1} }
\cs_set_protected:Npn \__md_triplestar:w #1*** { \textbf{\textit{#1}} }

\cs_new:Npn \items {
    \begingroup
    \itemize\kern-\topsep
    \itemsep0pt
}

\cs_new:Npn \enditems {
    \enditemize\kern-\topsep
    \endgroup
}


%% %%%%%%%%%%%%%%%%%%%%%%%%%%%%%%%%%%%%%%%%%%%%%%%%%%%%%%%%%%%%%%%%%%%%%%%%%%%%%
%%  Settings and Utility
%% %%%%%%%%%%%%%%%%%%%%%%%%%%%%%%%%%%%%%%%%%%%%%%%%%%%%%%%%%%%%%%%%%%%%%%%%%%%%%
\def \UF { \bfseries \itshape }
\let \nf \normalfont

\def \d {ḍ}
\def \D {Ḍ}
\def \b {ḅ}
\def \B {Ḅ}
\def \L {\textsuperscript{L}}
\def \N {\textsuperscript{N}}
\long \def \s #1 {{\normalfont\scshape #1 }}

\let \Sl \textbackslash
\let \Sub \textsubscript
\def \parheading #1 { \noindent \textbf{#1} }

\def \Item #1 {
    \item [ \textsc{\textbf{#1}} ]
}

\def \Paragraph #1 {
    \ifvmode\else\unskip\par\fi
    \addvspace \bigskipamount
    \noindent \leavevmode \ignorespaces \textbf{#1} \par
    \everypar { \setbox 0 \lastbox \everypar {} }
    \nobreak
}

\frenchspacing

\AtBeginDocument {
    \def \today {
        \int_value:w \day \space
        \int_case:nn { \month } {
             1 { January }
             2 { February }
             3 { March }
             4 { April }
             5 { May }
             6 { June }
             7 { July }
             8 { August }
             9 { September }
            10 { October }
            11 { November }
            12 { December }
        } \space
        \int_value:w \year
    }
}

\makeatother
\ExplSyntaxOff

%% %%%%%%%%%%%%%%%%%%%%%%%%%%%%%%%%%%%%%%%%%%%%%%%%%%%%%%%%%%%%%%%%%%%%%%%%%%%%%
%%  Document
%% %%%%%%%%%%%%%%%%%%%%%%%%%%%%%%%%%%%%%%%%%%%%%%%%%%%%%%%%%%%%%%%%%%%%%%%%%%%%%
\title{A Comprehensive Diachronic Grammar of Modern ULTRAFRENCH}
\author{Agma Schwa \& Ætérnal}
\date{\today}

\begin{document}
\maketitle
\thispagestyle{empty}
\clearpage
\setcounter{page}{1}

\tableofcontents
\clearpage

\section{Phonology and Evolution from Modern Pseudo-French}\label{sec:phonology}{\def\arraystretch{1.25}\setlength{\tabcolsep}{.4em}
\noindent\begin{tabular}{@{}|l|l|l|l|l|l|l@{\quad}|l|l|l|}                                                   \cline{1-6} \cline{8-10}
                       & Labial & Coronal  & Palatal  & Velar & Glottal &&            & Front        & Back        \\ \cline{1-6} \cline{8-10}
    Stop               & b, bʱ  & d        &          &       &         && Close      & i ĩ ĩ̃ i̥      & u ũ ũ̃ u̥ \\ \cline{1-6} \cline{8-10}
    Nasal              &        & n        &          &       &         && Near-close & ʏ ʏ̃ ʏ̃̃ ʏ̊      &             \\ \cline{1-6} \cline{8-10}
    Fricative          & ɸ β, ʋ̃ & s z, θ ð & ç ɕ ʑ    & x χ   & h       && Close-mid  & e ẽ ẽ̃ e̥      & o o̥         \\ \cline{1-6} \cline{8-10}
    Fric. (ʁ-coloured) & βʶ     & sʶ zʶ ɮ̃ʶ & ɕʶ ʑʶ    &       &         && Mid        & ə ə̣          &             \\\cline{1-6} \cline{8-10}
    Trill              &        &          &          & ʀ     &         && Open-mid   & ɛ ɛ̃ ɛ̃̃ ɛ̥      & ɔ̃ ɔ̃̃         \\ \cline{1-6} \cline{8-10}
    Approximant        &        &          & ɥ ɥ̃, j̊   & ɰ ɰ̃   &         && Near-open  & ɐ ɐ̥          &             \\ \cline{1-6} \cline{8-10}
    Lateral Fricative  &        & ɮ̃        & ʎ̝̃        &       &         && Open       &              & ɑ̃ ɑ̃̃         \\ \cline{1-6} \cline{8-10}
\end{tabular}}\bigskip

\parheading{Legend}\par\noindent
Ṽ = nasalised vowel, Ṽ̃ = nasal vowel, V = any vowel (or, in conjunction with Ṽ/Ṽ̃, oral vowel)\\
N = nasal consonant, C̃ = nasalised consonant (e.g. /ɰ̃/, but not true nasals), C = any consonant.\medskip
\def\scalpha{\kern-2pt\raisebox{2pt}{\Sub α}}

%% NOTE: In case the changes below and the ones listed
%% in the Lexurgy file differ, the latter are authoritative,
%% as I may forget to update these here sometimes.

\TwoCols[.45\hsize][.45\hsize][0pt]{
\parheading{Preliminary Changes}
\begin{enum}
    \item g, w > ɰ ⟨r⟩
    \item œ, œ̃, ø > y, ỹ, ỹ
    \item ɔ > o
    \item u > v / \_o
    \item y > j / \_(\#)V
    \item V\scalpha > $\emptyset$ / \_\#V\scalpha
    \item lj, lɥ > ʎ
    \item j > ɥ ⟨y’⟩
    \item ɰ > ɥ / \_i
    \item ʁʁ > ʀ
    \item sʁ, ʃʁ, zʁ, ʒʁ > sʶ, ʃʶ, zʶ, ʒʶ
    \item vʁ > vʶ
    \item ʁ > ɰ
    \item C > $\emptyset$ / \#\_C
    \item C > $\emptyset$ / C\_\#
    \item k > x ⟨c’h⟩
    \item ʃ, ʃʶ, ʒ, ʒʶ > ɕ, ɕʶ, ʑ, ʑʶ
    \item nt > nθ
    \item t > \d{} [d] (‘hard /d/’)
    \item p > \b{} [b] (‘hard /b/’)
    \item f, v, vʶ > ɸ ⟨f⟩, β ⟨b’h⟩, βʶ ⟨v́⟩
\end{enum}
}{
\parheading{Great Nasal Shift}
\begin{enum}[15]
    \item Ṽl > ɰ̃ ⟨w⟩
    \item V > Ṽ̃ / [NC̃ɥɰ]\_N\#
    \item V, Ṽ > Ṽ, Ṽ̃ / \_[NC̃ɥɰ], [NC̃ɥɰ]\_
    \item ə̃, ə̃̃, ã, ã̃, õ, õ̃ > ɛ̃, ɛ̃̃, ɑ̃, ɑ̃̃, ɔ̃, ɔ̃̃
    \item N, C̃ > $\emptyset$ / V\_\#
    \item ɲ, ŋ > n
    \item V, Ṽ > $\emptyset$ / N \_ N
    \item m, l, ʎ > ʋ̃ ⟨v⟩, ɮ̃ ⟨l⟩, ʎ̝̃ ⟨ḷ⟩
    \item ɮ̃ɰ, ɰɮ̃ > ɮ̃ʶ ⟨ł⟩
\end{enum}

\parheading{Intervocalic Lenition (/ V\_V is implied)}
\begin{enum}[21]
    \item x, s, z > h
    \item ɕ, ɮ̃, ʎ̝̃ > j̊ ⟨ç̇⟩, ɥ̃, ɰ̃
    \item nθ > n
    \item d, \d{}, b, \b{} > ð ⟨d’h⟩, θ ⟨t’h⟩, β, bʱ ⟨bh⟩
    \item ɸ > β / V\_V
\end{enum}

\parheading{Late Changes}
\begin{enum}[25]
    \item C[+stop, -alveolar]C\scalpha > C\scalpha
    \item C[+stop]C\scalpha[+stop] > C\scalpha
    \item h > $\emptyset$ / hV\_
    \item ə > $\emptyset$ / C\_C
    \item V[-nasalised, -nasal] > ə̥ / \_\#
\end{enum}
}\medskip

\subsection{Pronunciation, Allophony, and Stress}\label{subsec:pronunciation-allophony-and-stress}
There is not a lot of allophony in UF, save that /x/ is realised as [χ] around back vowels and [ɕ] elsewhere, e.g.
*c’húr* /xũɰ/ ‘to shrink’ is pronounced [χũˑˠ]. Furthermore, /h/ is [ç] before variants of /i/ and /y/, and [h] elsewhere.

The vast majority PF words are stressed on the last syllable of the root, e.g. *ad’hór* ‘to love’ /aˈðɔ̃ɰ/, but *b’had’hóré*
‘you (\s{pl}) love’ /βaˈðɔ̃.ɰɛ̃/. The stress is not indicated in writing, neither in actual texts, nor in this
grammar or in dictionaries. The main exception to this are names, which are generally stressed on the first syllable,
and receive secondary stress on the last syllable,\footnote{That is, unless the name ends in an obvious suffix, in which case the last
syllable before any such suffixes receives secondary stress; however, this is generally quite rare.} e.g. *Daúvníc’h* /ˈdɔ̃ʋ̃ˌnĩx/.

The only exception to this rule are certain particles and irregular verbs, some of which have irregular stress; for instance,
the forms of *eḍ* ‘to be’ are all stressed on the first syllable. Any such words that deviate from the norm will be pointed
out in this grammar and in dictionaries.

Oral vowels before the stressed syllable are often somewhat muted or reduced, albeit still audible, and stressed vowels are lengthened if they
are nasalised, e.g. the pronunciation of *ad’hór*, which we just transcribed as /aˈðɔ̃ɰ/, is actually closer to [ɐ̯ˈðɔ̃ˠˑ].
Word-final voiceless `e` is always /ə̥/.

Oral vowels have a nasalised and nasal counterpart. /i/, /y/—which is actually [ʏ]—and /u/ do not vary in quality when nasalised.
/a/ is normally [ɐ],
but becomes [ɑ] when nasalised or nasal. Similarly, /e/ becomes [ɛ], and /o/ becomes [ɔ]. Note that nasalised [ẽ] exists, but it’s
rare. The quality never changes when going from nasalised to nasal. The schwa has no nasal(lised) counterpart. Lastly, oral vowel
also have voiceless counterparts, whose quality is the same as that of the base vowel.

The difference between nasalised vowels and nasal vowels is that the former are merely coarticulated with nasalisation, whereas
the latter are completely and utterly *in the nose*—no air escapes through the mouth when a nasal vowel is articulated, and all
the air flows just through the nose. Middle UF and some modern dialects also distinguish between sinistral and dextral nasal
vowels,\footnote{Sinistral nasal vowels are articulated with the left nostril, and dextral nasal vowels with the right nostril.}
but this distinction is no longer present in the modern standard language.

Furthermore, as indicated in that same example, word-final /ɰ/ is often realised as velarisation of the preceding vowel;
the same, however, is not the case for /ɰ̃/. Initial /ɰ/ is sometimes elided after words that end with /ɰ/, particularly
in particles (e.g. *rvá* ‘alas’).

Lenition causes the changes marked above as ‘Intervocalic Lenition’ to be applied to a consonant; furthermore,
ʁ-coloured consonants are replaced with their regular counterparts, and *h* disappears completely.

\subsection{Orthography}
The spelling of most UF sounds is indicated above; the less exotic consonants are spelt as
one might expect. That is, /b, d, n, ɸ, s, z, h/ are spelt ⟨b, d, n, f, s, z, h⟩, respectively.

Several fricatives are spelt with an apostrophe followed by a ‘h’, viz. /x/ ⟨c’h⟩, /θ/ ⟨t’h⟩, /ð/ ⟨d’h⟩,
and /β/ ⟨b’h⟩. Conventional letters are used for rather unconventional sounds, mostly for diachronic reasons:
/l/ does not exist in UF, so ⟨l⟩ is either /ɮ̃/ or /ʎ̝̃/, ⟨v⟩ is /ʋ̃/, ⟨j⟩ is /ʑ/, ⟨r⟩ is /ɰ/, ⟨w⟩ is /ɰ̃/. The vowel
/y/ is spelt ⟨y⟩, and its consonantal equivalent /ɥ/ as well as nasalised /ɥ̃/ are spelt with an apostrophe, that is
⟨y’⟩ and ⟨ý’⟩. The ʁ-fricated fricatives /βʶ, ɮ̃ʶ, sʶ, ɕʶ, zʶ, ʑʶ/
are spelt ⟨v́, ł, ś, ḉ, ȷ́, ź⟩, respectively.

Double consonant letters indicate a lengthened consonant; these are rare, but they can occur in any position. The only
exception to this is ⟨rr⟩, which is not /ɰː/, but rather /ʀ/. UF does not have phonemic vowel length (though recall
that phonetic lengthening occurs in some situations), so a double vowel letter is always pronounced as two separate vowels.

The vowels are mostly spelt as one might expect; nasalised vowels are indicated by an acute, and nasal vowels by a circumflex.
The variants of /i, y, u, a, e/ are spelt with ⟨i, y, u, a, e⟩ as their base letters. Nasal /ẽ/ and /ẽ̃/ as well as Schwa are
indicated by adding a dot below the ⟨e⟩; the vowel /o/ is spelt ⟨au⟩ or ⟨o⟩ for diachronic reasons;\footnote{As is always the
case in cases like this, hypercorrection is frequent, and ⟨au⟩ is often preferred word-initially, even if the
PF root was spelt with ⟨o⟩. In general, UF speakers seem to prefer ⟨au⟩ over ⟨o⟩, except word-finally and after ⟨w⟩, except
that in verb affixes, *au* is quite common word-finally. The sequence ⟨wau⟩ does not exist in UF.} in the case of
⟨au⟩, the acute and circumflex are added to the ⟨u⟩. The diphthong /au/ is spelt ⟨äu⟩, ⟨aü⟩, or with accents on both vowels. Oral
/ɛ/ is rare and is spelt ⟨è⟩. Word-initially and word-finally, a grave instead indicates that the vowel is voiceless. Word-final
voiceless /e/ is always /ə̥/, but confusingly, it is also just spelt ⟨e⟩, since ⟨è⟩ is already /ɛ/.\footnote{Thus, a word-final ⟨e⟩
can be /e/, such as in *vvaúríhe* /ʋ̃ːɔ̃ɰĩˈhe/ ‘to remember’, or /ə̥/, such as in *dale* /daɮ̃ə̥/ ‘table’. As a rule of thumb, it is
usually /e/ at the end of verb stems—but not verb forms in general—and /ə̥/ elsewhere. Fortunately they are differentiated by a
dot below in dictionaries and in this grammar: *vvaúríhe* vs *ḍalẹ*.}

The ‘hard’ voiced *ḅ*, *ḍ* which are pronounced exactly like their regular counterparts, are normally also spelt ⟨b⟩ and
⟨d⟩. However, the dot is commonly used in dictionaries and grammatical material to distinguish between the two
as they differ from one another in how they are lenited. Furthemore, a dot below or above a letter is commonly to indicate
a variety of different things, depending on the letter:
\begin{items}\itemsep .5ex plus .1ex minus .1ex\relax
\item a dot below in *ḅ*, *ḍ* indicates that they are the ‘hard’ variants of the letter, which are pronounced
      the same, but lenited differently;
\item a dot below in *ḷ* indicates that it is palatal /ʎ̝̃/ instead of alveolar /ɮ̃/;
\item a dot below in *ẹ* indicates that it is a schwa;
\item a dot below nasalised *ẹ́*, *ệ* indicates that they are /ẽ/, /ẽ̃/ instead of /ɛ̃/, /ɛ̃̃/;
\item a dot above in *ċ* indicates that it is lenited /j̊/.
\end{items}

\noindent Thus, in non-grammatical writing, the following are indistinguishable:
\begin{items}\itemsep .5ex plus .1ex minus .1ex\relax
\item *l* can be palatal /ʎ̝̃/ or alveolar /ɮ̃/;
\item *e* can be a schwa, or /e/;
\item *é*, *ê* can be /ɛ̃/, /ɛ̃̃/ or /ẽ/, /ẽ̃/;
\item *c* can be /ɕ/ or /j̊/.
\end{items}

\noindent Elided initial /ɰ/ is indicated by omitting the *r* in writing and attaching the word to the previous one with a hyphen,
e.g. *-vá* ‘alas’.

UF seldom uses hyphens to separate or join words and instead prefers to spell them as one word instead; an exception
to this is that affixes that end with a vowel are typically separated from the word they are attached to with a hyphen
if that word starts with (a variant of) the same vowel. For example, the \s{def nom sg} of *el* ‘wing’
is *láel*, but the plural is *lé-el*.

\subsubsection{Lenition and Nasalisation}
Certain morphological elements subject surrounding context to lenition or nasalisation. Nasalisation affects vowels,
which become more nasal (that is, (voiceless) oral vowels become nasalised, and nasalised vowels become nasal; nasal
vowels are unaffected), as well as *ḍ*, which becomes *n*.

Lenition is more complicated; it affects only consonants and causes a softening similar to what happened diachronically
between vowels. All ʁ-fricated consonants simply lose their ʁ-frication. Furthermore, the
following consonants are also affected by lenition:

\begin{table}[H]
\centering
\itshape
\begin{tabular}{l|lll|l|l|l|ll|l|l|l|}
\bf Consonant & x & s & z       & c & l  & ḷ & b & f          & ḅ   & d   & ḍ    \\\hline
\bf Lenited & \multicolumn{3}{c|}{h} & ċ & ý’ & w & \multicolumn{2}{c|}{b’h} & bh  & d’h & t’h  \\
\end{tabular}
\nf
\caption{Consonants Affected by Lenition}\label{tab:lenition}
\end{table}

\noindent Note that double consonants are typically unaffected by morphological lenition, e.g. *dír* ‘to say’,
whose subjunctive stem is *díss*, forms *aúdíssâ* (rougly ‘we should have said’), not \**aúdíhhâ*.

\subsubsection{Glossing}
To simplify glosses, cases are assumed to be definite and singular unless otherwise stated, and verb forms are
assumed to be indicative, present tense, and active, unless otherwise stated.

\section{Accidence}\label{sec:accidence}
\subsection{Verbal Morphology}\label{subsec:verbal-morphology}
Verbs in UF are inflected for person, number, tense, aspect, mood, and voice. Verbal inflexion is mainly done
by means of concatenating a vast set of prefixes onto a verb, with the occasional suffix and circumfix making
its appearance. This chapter details these affixes, their meanings, uses, forms, and restrictions.


\subsubsection{Active/Passive Affixes}\label{subsubsec:active-passive-affixes}
UF has a set of active/subject as well as passive/object prefixes which can be used on their own or in combination
with one another, though at most one active and one passive prefix may be combined with a verb.\footnote{Irrespective
of whether they are personal or infinitive prefixes. For instance, it would also be illegal to combine e.g. the active
infinitive prefix with the first person active singular prefix.} Table~\ref{tab:active-passive-prefixes}
below lists those prefixes, two of which are actually circumfixes.

\begin{table}[H]
\centering
\noindent\begin{tabular}{@{}|>{}l|>{\it}l|>{\it}l|>{}l|>{}l|>{\it}l|>{\it}l|}\cline{1-3}\cline{5-7}
 Active&\nf Sg&\nf Pl& & Passive&\nf Sg&\nf Pl\\\cline{1-3}\cline{5-7}
1st&j-&aú-/r-/w- -(y’)ó&&1st&v-&aú-/r-/w-\\\cline{1-3}\cline{5-7}
2nd&\d{}(ẹ)-&b’h(y)- -(y’)é&&2nd&\d{}(ẹ)-&b’h(y)-\\\cline{1-3}\cline{5-7}
3rd m&l(ẹ)-&l(ẹ)-&&3rd m&y’-&lý-\\\cline{1-3}\cline{5-7}
3rd f&ll(a)-&ll(ẹ)-&&3rd f&y’- &lý-\\\cline{1-3}\cline{5-7}
3rd n&s- &l(a)-&&3rd n&sy-&lý-\\\cline{1-3}\cline{5-7}
Infinitive&\multicolumn{2}{c|}{\it d(ẹ)-}&&Infinitive&\multicolumn{2}{c|}{\it à-/h-}\\\cline{1-3}\cline{5-7}
Participle&\multicolumn{2}{c|}{\it -â}&&Participle&\multicolumn{2}{c|}{\it â-}\\\cline{1-3}\cline{5-7}
\end{tabular}
\caption{Active (left) and passive (right) verbal affixes.}\label{tab:active-passive-prefixes}
\end{table}

\noindent A great degree of syncretism can be observed in the third-person forms. The gender distinction in the
\s{3sg} that diachronically resulted from gendered personal pronouns is almost non-existent in the
plural; the reason for this development is that those forms are derived from the old dative form, which lacked
this distinction altogether.

The \s{act 1pl, 2pl} forms are only distinguished from their passive counterparts by
the presence of additional suffixes in the former. The \s{3sg n} in the active and passive is derived from the PF
demonstrative \**ce* and its variants; the \s{3pl n} is derived from the other \s{3pl} forms.

\Paragraph{Usage Notes}
\begin{items}
\Item{1pl}
The \s{1pl} prefix varies if there is a vowel following it: if it is
any vowel that is not a variant of ‘o’, the prefix is realised as *r-* instead, e.g. *ad’hór* ‘love’ to
*rad’hóró* ‘we love’. If the vowel a variant of ‘o’, the prefix is realised as *w-* instead, e.g. *aub’heír* ‘obey’
to *wob’heíró* ‘we obey’.\footnote{Diachronically, the base form of this prefix is \**o-*, whence e.g.
\**oad’hóró* > *rad’hóró* and \**oob’heíró* > *wob’heíró*.} Note that this also leads to a change in spelling: stem-initial
⟨au⟩ is changed to ⟨o⟩.

\Item{1,2 pl}
The *y’* in the suffix parts of the \s{1pl, 2pl act} are dropped if the verb ends with a consonant, e.g. *ad’hór*
to *b’hád’hóré*, or if it ends with a vowel that is a variant of ‘o’ in the case of the \s{1pl} and ‘e’ in the case
of the \s{2pl}, in which cases the vowels are contracted and a level of nasalisation is added, e.g. *vvaúríhe*
‘to remember’ to *b’hyvvaúríhé* ‘you (\s{pl}) remember’ (not \**b’hyvvaúríhy’é*). In all other cases, the *y’* is retained,
e.g. *aúvvaúríhey’ó* ‘we remember’.

\Item{inf}
The \s{inf pass} prefix *à-* coalesces with any vowel following it: it becomes *á* if it
is followed by a non-nasal variant of ‘a’, e.g. *ad’hór* to *ád’hór* ‘to be loved’; *â* if it is
followed by a nasal variant of ‘a’, e.g. *ánvé* ‘give life to’ to *ânvé* ‘to be animated’; and *h-* if it is
followed by any other vowel, e.g. *aub’heír* to *haub’heír* ‘to be obeyed’.

\Item{part}
The participle affixes are commonly used to form adjectives since the vast majority of adjectives in UF are actually ‘adjective verbs’
with a meaning of ‘to be X’. The participle can be used to convert such a verb back into a regular adjective, e.g. *lár* ‘to be wide’
to *lárâ* ‘wide’. Like the passive infinitive affix, the participle affixes coalesce with vowels and always form a maximally
nasal vowel, e.g. *vvaúríhe* ‘to remember’ forms *vvaúríhê* ‘remembering’, and *ad’hór* forms *âd’hór* ‘being loved’. As with other
coalescence rules, the *-â* instead *replaces* a word-final *ẹ*, and *ẹ* only: e.g. *ḅẹt’hẹ* ‘to be small’ becomes *ḅẹt’hâ* ‘being small’.

\Item{{\upshape\textit{-ẹ-}}}
The parenthesised vowels are used if the prefix is followed by a consonant, e.g. *dír* ‘say’ to *llẹ{}dír*
‘they (\s{f}) say’ and *b’hydíré* ‘you (\s{pl}) say’, but *ad’hór* to *llad’hór* ‘they (\s{f}) love’ and *b’had’hóré* ‘you
(\s{pl}) love’. The prefixes *aú-* and *à-* retain their main forms if followed by a consonant,
e.g. *dír* ‘say’ to *aúdíró* ‘We say’ and *àdír* ‘to be said’.

\Item{{\upshape\textit{-y-}}}
The exception to this is that \s{2pl} *b’h(y)-*
drops the *y* if followed by a glide, e.g. *y’ír* ‘to hear’ to *b’hy’íré* ‘you (\s{pl}) hear’ (not \**b’hyy’íré*).
\end{items}

\Paragraph{Combining Prefixes}
When multiple prefixes are used together, active prefixes precede passive prefixes, except that infinitive and participle prefixes
always come first, e.g. *ad’hór* ‘love’ to *jvad’hór* ‘I love myself’ (not \**vjad’hór*) and *b’hy’ad’hóré* ‘you (\s{pl}) love him/her’,
but *dẹvad’hór* ‘to love me’ and *àb’had’hóré* ‘to be loved by you (\s{pl})’. Recall that at most one infinitive prefix
and at most one participle affix may be used.

\Paragraph{Example Paradigms}
By way of illustration, consider the paradigm of the verb *ad’hór* as shown in Table~\ref{tab:adhor-paradigm} below.
Since this word starts with a vowel, the parenthesised vowels in Table~\ref{tab:active-passive-prefixes} above
are not used. Furthermore, since it starts with a non-nasal ‘a’-like vowel, the *aú-* prefix is realised as *r-*
and the *à-* prefix coalesces with the initial ‘a’ of the stem to form *á*.

% TEMPLATE:
%\noindent\begin{tabular}{@{}|>{}l|>{\it}l|>{\it}l|>{}l|>{}l|>{\it}l|>{\it}l|}\cline{1-3}\cline{5-7}
%\nf Active&\nf Sg&\nf Pl&\nf &\nf Passive&\nf Sg&\nf Pl \\\cline{1-3}\cline{5-7}
%1st       &   &  &&1st   &   &   \\\cline{1-3}\cline{5-7}
%2nd       &   &  &&2nd   &   &   \\\cline{1-3}\cline{5-7}
%3rd m     &   &  &&3rd m &   &   \\\cline{1-3}\cline{5-7}
%3rd f     &   &  &&3rd f &   &   \\\cline{1-3}\cline{5-7}
%3rd n     &   &  &&3rd n &   &   \\\cline{1-3}\cline{5-7}
%Infinitive& \multicolumn{2}{c|}{\it }  && Infinitive & \multicolumn{2}{c|}{\it } \\\cline{1-3}\cline{5-7}
%\end{tabular}

\begin{table}[H]
\centering
\noindent\begin{tabular}{@{}|>{}l|>{\it}l|>{\it}l|>{}l|>{}l|>{\it}l|>{\it}l|}\cline{1-3}\cline{5-7}
\nf Active&\nf Sg&\nf Pl&\nf &\nf Passive&\nf Sg&\nf Pl\\\cline{1-3}\cline{5-7}
1st&jad’hór&rad’hóró&&1st&vad’hór&rad’hór\\\cline{1-3}\cline{5-7}
2nd&\d{}ad’hór&b’had’hóré&&2nd&\d{}ad’hór&b’had’hór\\\cline{1-3}\cline{5-7}
3rd m&lad’hór&lad’hór&&3rd m&y’ad’hór&lýad’hór\\\cline{1-3}\cline{5-7}
3rd f&llad’hór&llad’hór&&3rd f&y’ad’hór &lýad’hór\\\cline{1-3}\cline{5-7}
3rd n&sad’hór&lad’hór&&3rd n&ý’ad’hór&lýad’hór\\\cline{1-3}\cline{5-7}
Infinitive&\multicolumn{2}{c|}{\it dad’hór}&&Infinitive&\multicolumn{2}{c|}{\it ád’hór}\\\cline{1-3}\cline{5-7}
Participle&\multicolumn{2}{c|}{\it ad’hórâ}&&Participle&\multicolumn{2}{c|}{\it âd’hór}\\\cline{1-3}\cline{5-7}
\end{tabular}
\caption{Paradigm of the Verb \emph{ad’hór}.}\label{tab:adhor-paradigm}
\end{table}

\noindent For comparison, the paradigm of the verb *vvaúríhe* ‘remember’ is shown in Table~\ref{tab:vvorihe-paradigm} below.
Since it starts with a consonant, the parenthesised vowels in Table~\ref{tab:active-passive-prefixes} are used, and any
prefixes that end with a vowel remain unchanged.

\begin{table}[H]
\centering
\noindent\begin{tabular}{@{}|>{}l|>{\it}l|>{\it}l|>{}l|>{}l|>{\it}l|>{\it}l|}\cline{1-3}\cline{5-7}
\nf Active&\nf Sg&\nf Pl&\nf &\nf Passive&\nf Sg&\nf Pl\\\cline{1-3}\cline{5-7}
1st&jvvaúríhe&aúvvaúríhey’ó&&1st&vvvaúríhe&aúvvaúríhe\\\cline{1-3}\cline{5-7}
2nd&ḍẹvvaúríhe&b’hyvvóríhé&&2nd&ḍẹvvaúríhe&b’hyvvaúríhe\\\cline{1-3}\cline{5-7}
3rd m&lẹvvaúríhe&lẹvvaúríhe&&3rd m&y’vvaúríhe&lývvaúríhe\\\cline{1-3}\cline{5-7}
3rd f&llavvaúríhe&llẹvvaúríhe&&3rd f&y’vvaúríhe&lývvaúríhe\\\cline{1-3}\cline{5-7}
3rd n&ý’vvaúríhe&lavvaúríhe&&3rd n&ý’vvaúríhe&lývvaúríhe\\\cline{1-3}\cline{5-7}
Infinitive&\multicolumn{2}{c|}{\it dẹvvaúríhe}&&Infinitive&\multicolumn{2}{c|}{\it àvvaúríhe}\\\cline{1-3}\cline{5-7}
Participle&\multicolumn{2}{c|}{\it vvaúríhê}&&Participle&\multicolumn{2}{c|}{\it âvvaúríhe}\\\cline{1-3}\cline{5-7}
\end{tabular}
\caption{Paradigm of the Verb \emph{vvaúríhe}.}\label{tab:vvorihe-paradigm}
\end{table}

\subsection{Tense and Aspect Marking}\label{subsec:tense-and-aspect-marking}
Tense in PF is marked by additional sets of affixes that are appended to the verb in addition to the active/passive affixes.
There are two broad groups of such affixes: suffixes, which are appended to the end of the verb and replace the \s{act 1pl, 2pl} suffixes
in those persons, as well as circumfixes and prefixes, which are inserted before the active/passive markers and replace the
replace the \s{act 1pl, 2pl} suffixes in some cases.

\subsubsection{Suffixed Tenses}
The present anterior and preterite are formed by appending a set of suffixes to the verb. Table~\ref{tab:present-anterior-and-preterite-suffixes}
below lists the suffixes for those tenses. The present anterior has a perfect or perfective aspect, while the preterite has an imperfective aspect. The
former is commonly used to describe events that are completed or extend to the present—particularly events that occurred recently, hence the name—while the latter
is used to describe events that are ongoing or habitual.


\begin{table}[H]
\centering
\noindent\begin{tabular}{@{}|>{}l|>{\it}l|>{\it}l|>{}l|>{}l|>{\it}l|>{\it}l|}\cline{1-3}\cline{5-7}
\nf Present Anterior&\nf Sg&\nf Pl&\nf &\nf Preterite&\nf Sg&\nf Pl \\\cline{1-3}\cline{5-7}
1st       & -\L é & -\L â &&1st    & -\L á  & -y’aû  \\\cline{1-3}\cline{5-7}
2nd       & -\L á & -\L áḍ &&2nd   & -\L é  & -y’ẹ́  \\\cline{1-3}\cline{5-7}
3rd       & -\L á & -\L ér &&3rd m & -\L é  & -\L é   \\\cline{1-3}\cline{5-7}
Infinitive& \multicolumn{2}{c|}{\it -á }  && Infinitive & \multicolumn{2}{c|}{\it -é } \\\cline{1-3}\cline{5-7}
Participle& \multicolumn{2}{c|}{\it -ér }  && Participle & \multicolumn{2}{c|}{\it -ár } \\\cline{1-3}\cline{5-7}
\end{tabular}
\caption{Present Anterior and Preterite Affixes.}\label{tab:present-anterior-and-preterite-suffixes}
\end{table}

\noindent Neither tense distinguishes gender in the third person. All suffixes, except for the infinitive and \s{1pl, 2pl pret},
lenite any consonant *before* them, e.g. *ḅárḍáḍ* ‘to be willing’ to *jḅárḍát’hé* ‘I was willing’ but *dẹḅárḍáḍá*
‘to have been willing’.

Diachronically, the \s{1sg pret} is an interesting case; in EUF, it was originally \**-é*, but it later changed to *-á*
to distinguish it from the \s{2sg, 3sg pres ant}. The remaining forms—save the infinitives, which are derived from the
tenses’ definite endings by analogy—originated from the PF simple past tenses.

The table below lists the example paradigm of the verb *ad’hór* in the present anterior and preterite tenses.
Observe that there is no difference between the \s{1pl, 2pl} active and passive.

The participle suffixes coalesce with present participle affixes to form *êr* in the present anterior and *âr* in the preterite,
where applicable, e.g. present *ad’hórâ* ‘loving’ becomes *ad’hórêr* ‘having loved’.

In both tenses, the suffixes coalesce with vowels before them, replacing them and nasalising them if
they are already nasal, e.g. *jvvaúríé* ‘I remembered’.

If a verb takes both and active and a passive person affix, the suffix aligns with the active affix; thus
‘she loved me’ is *llavad’hórá* and not \**llavád’hóré*.

\begin{table}[H]
\centering
\noindent\begin{tabular}{@{}|>{}l|>{\it}l|>{\it}l|>{}l|>{}l|>{\it}l|>{\it}l|}\cline{1-3}\cline{5-7}
\nf Active & \nf Sg   & \nf Pl     & \nf & \nf Passive & \nf Sg   & \nf Pl    \\\cline{1-3}\cline{5-7}
1st        & jad’hóré  & rad’hórâ     &     & 1st         & vad’hóré  & rad’hórâ   \\\cline{1-3}\cline{5-7}
2nd        & ḍad’hórá  & b’had’hóráḍ  &     & 2nd         & ḍad’hórá  & b’had’hóráḍ \\\cline{1-3}\cline{5-7}
3rd m      & lad’hórá  & lad’hórér    &     & 3rd m       & y’ad’hórá & lýad’hórér  \\\cline{1-3}\cline{5-7}
3rd f      & llad’hórá & llad’hórér   &     & 3rd f       & y’ad’hórá & lýad’hórér  \\\cline{1-3}\cline{5-7}
3rd n      & ý’ad’hórá & lad’hórér    &     & 3rd n       & ý’ad’hórá & lýad’hórér  \\\cline{1-3}\cline{5-7}
Infinitive & \multicolumn{2}{c|}{\it dad’hórá} & & Infinitive & \multicolumn{2}{c|}{\it ád’hórá} \\\cline{1-3}\cline{5-7}
Participle&\multicolumn{2}{c|}{\it ad’hórêr}&&Participle&\multicolumn{2}{c|}{\it âd’hórér}\\\cline{1-3}\cline{5-7}
\end{tabular}
\caption{Present Anterior Paradigm of the Verb \emph{ad’hór}.}\label{tab:adhor-paradigm-pres-ant}
\end{table}

\begin{table}[H]
\centering
\noindent\begin{tabular}{@{}|>{}l|>{\it}l|>{\it}l|>{}l|>{}l|>{\it}l|>{\it}l|}\cline{1-3}\cline{5-7}
\nf Active & \nf Sg   & \nf Pl     & \nf & \nf Passive & \nf Sg   & \nf Pl    \\\cline{1-3}\cline{5-7}
1st        & jad’hórá  & rad’hóry’aû     &     & 1st     & vad’hórá  & rad’hóry’aû   \\\cline{1-3}\cline{5-7}
2nd        & ḍad’hóré  & b’had’hóry’ẹ́  &     & 2nd      & ḍad’hóré  & b’had’hóry’ẹ́ \\\cline{1-3}\cline{5-7}
3rd m      & lad’hóré  & lad’hóré    &     & 3rd m      & y’ad’hóré & lýad’hóré  \\\cline{1-3}\cline{5-7}
3rd f      & llad’hóré & llad’hóré   &     & 3rd f      & y’ad’hóré & lýad’hóré  \\\cline{1-3}\cline{5-7}
3rd n      & ý’ad’hóré & lad’hóré   &     & 3rd n       & ý’ad’hóré & lýad’hóré  \\\cline{1-3}\cline{5-7}
Infinitive & \multicolumn{2}{c|}{\it dad’hóré} & & Infinitive & \multicolumn{2}{c|}{\it ád’hóré} \\\cline{1-3}\cline{5-7}
Participle&\multicolumn{2}{c|}{\it ad’hórâr}&&Participle&\multicolumn{2}{c|}{\it âd’hórár}\\\cline{1-3}\cline{5-7}
\end{tabular}
\caption{Preterite Paradigm of the Verb \emph{ad’hór}.}\label{tab:adhor-paradigm-pret}
\end{table}

\subsubsection{Future I}
The future tenses, that is, the Future, Future Anterior (a tense similar to the future perfect), as well
as the Conditional, are formed by adding prefixes to the present forms. The prefix is the same in all persons and numbers,
except that there is a separate prefix for the infinitive.

In the Future, much to the UF learner’s dismay, this prefix can go in two separate positions: either before the person marker(s) or
inbetween the person marker(s) and the stem. The former case is more common in speech, while the later is more literary
and strongly preferred in writing and poetry as well as in formal speech. But even in informal speech, the Future I alone
will still not be enough to get by, as the Conditional, a *very* common tense, is formed using the Future II.

First, let us examine the former, simpler case, commonly called the Future I. The prefix is *aú-* if the verb form
after it starts with a consonant (except glides), *aúr-* in all other cases; e.g. *aújad’hór* ‘I shall love’, but
*aúrý’ad’hór* ‘it will love’. In the infinitive passive, it
contracts with the initial *à-* or *á-* to *áu* or *aû*, e.g. *aûd’hór* ‘to be about to be loved’.\footnote{This form
has no direct equivalent in English and is fairly hard to translate on its own.} No contraction happens
if the infinitive starts with *â*, e.g. *aúrânvé* ‘to be about to be animated’. Since
there is little point in writing a table for just the prefixes, Table~\ref{tab:adhor-paradigm-future-1} instead shows the Future I paradigm
of the verb \emph{ad’hór}.

\begin{table}[H]
\centering
\noindent\begin{tabular}{@{}|>{}l|>{\it}l|>{\it}l|>{}l|>{}l|>{\it}l|>{\it}l|}\cline{1-3}\cline{5-7}
\nf Active&\nf Sg&\nf Pl&\nf &\nf Passive&\nf Sg&\nf Pl\\\cline{1-3}\cline{5-7}
1st&aújad’hór&aúrad’hóró&&1st&aúvad’hór&aúrad’hór\\\cline{1-3}\cline{5-7}
2nd&aú\d{}ad’hór&aúb’had’hóré&&2nd&aú\d{}ad’hór&aúb’had’hór\\\cline{1-3}\cline{5-7}
3rd m&aúlad’hór&aúlad’hór&&3rd m&aúry’ad’hór&aúlýad’hór\\\cline{1-3}\cline{5-7}
3rd f&aúllad’hór&aúllad’hór&&3rd f&aúry’ad’hór &aúlýad’hór\\\cline{1-3}\cline{5-7}
3rd n&aúrý’ad’hór&aúlad’hór&&3rd n&aúrý’ad’hór&aúlýad’hór\\\cline{1-3}\cline{5-7}
Infinitive&\multicolumn{2}{c|}{\it aúdad’hór}&&Infinitive&\multicolumn{2}{c|}{\it aûd’hór}\\\cline{1-3}\cline{5-7}
Participle&\multicolumn{2}{c|}{\it aúrad’hórâ}&&Participle&\multicolumn{2}{c|}{\it aúrâd’hór}\\\cline{1-3}\cline{5-7}
\end{tabular}
\caption{Future I Paradigm of the Verb \emph{ad’hór}.}\label{tab:adhor-paradigm-future-1}
\end{table}

\subsubsection{Future II}
The Future I paradigm is fairly straight-forward; unfortunately, the Future II is a lot worse: not only do the affixes
vary a lot more, but they are different depending on whether verb form following them starts with a vowel or a consonant.\footnote{This is
not a problem in the Future I, since the prefix is never adjacent to the stem.}
The vocalic and consonantal Future II affixes are shown in Tables~\ref{tab:future-2-vocalic}~and~\ref{tab:future-2-consonantal} below, respectively.

The diachrony of these forms is somewhat unclear—especially that of the participles. It would appear, however, that they result from a coalescence
of the personal pronouns with forms of some auxiliary (likely PF *avoir* and *aller*) as well as the PF future. It appears that
the \s{2sg} is derived from the formal PF \s{2pl} pronoun, which is in line with the fact that the Future II is generally
considered more formal than the almost colloquial Future I. The *v́* in the \s{2pl act} seems to be the result of metathesis.

\begin{table}[H]
\centering
\noindent\begin{tabular}{@{}|>{}l|>{\it}l|>{\it}l|>{}l|>{}l|>{\it}l|>{\it}l|}\cline{1-3}\cline{5-7}
Active&\nf Sg&\nf Pl& & Passive&\nf Sg&\nf Pl\\\cline{1-3}\cline{5-7}
1st   &b’h- -(ẹ)  &náý’- -aú      &&1st    &v- -é    &náý’-       \\\cline{1-3}\cline{5-7}
2nd   &ḍír- -(ẹ)  &b’haý’- -(r)ẹ́  &&2nd    &ḍír-     &b’haý’-     \\\cline{1-3}\cline{5-7}
3rd m &ł-  -(ẹ)   &lb’h- -aú      &&3rd m  &l-       &lb’h- -(r)e \\\cline{1-3}\cline{5-7}
3rd f &èł-  -(ẹ)  &lb’h- -aú      &&3rd f  &l-       &lb’h- -(r)e \\\cline{1-3}\cline{5-7}
3rd n &aúł-  -(ẹ) &lb’h- -aú      &&3rd n  &s-       &lb’h- -(r)e \\\cline{1-3}\cline{5-7}
Infinitive&\multicolumn{2}{c|}{\it d- -è}&&Infinitive&\multicolumn{2}{c|}{\it h-}\\\cline{1-3}\cline{5-7}
Participle&\multicolumn{2}{c|}{\it -ŷr}&&Participle&\multicolumn{2}{c|}{\it á- -ýr}\\\cline{1-3}\cline{5-7}
\end{tabular}
\caption{Vocalic Future II Affixes.}\label{tab:future-2-vocalic}
\end{table}

\begin{table}[H]
\centering
\noindent\begin{tabular}{@{}|>{}l|>{\it}l|>{\it}l|>{}l|>{}l|>{\it}l|>{\it}l|}\cline{1-3}\cline{5-7}
Active&\nf Sg&\nf Pl& & Passive&\nf Sg&\nf Pl\\\cline{1-3}\cline{5-7}
1st   &jaú- -ẹ́  &aúnraû- -aú &&1st   &vaú- -é  &naú-       \\\cline{1-3}\cline{5-7}
2nd   &b’há- -(ẹ) &v́aú- -e     &&2nd   &\d{}á-  &b’haú-     \\\cline{1-3}\cline{5-7}
3rd m &aúr-  -(ẹ) &laú- -aú    &&3rd m &y’aúr-  &laú- -(r)e \\\cline{1-3}\cline{5-7}
3rd f &aúr-  -(ẹ) &laú- -aú    &&3rd f &y’aúr-  &laú- -(r)e \\\cline{1-3}\cline{5-7}
3rd n &aúr-  -(ẹ) &laú- -aú    &&3rd n &saúr-   &laú- -(r)e \\\cline{1-3}\cline{5-7}
Infinitive&\multicolumn{2}{c|}{\it dẹ- -è}&&Infinitive&\multicolumn{2}{c|}{\it haú-}\\\cline{1-3}\cline{5-7}
Participle&\multicolumn{2}{c|}{\it -(r)ŷ}&&Participle&\multicolumn{2}{c|}{\it á- -(r)ý}\\\cline{1-3}\cline{5-7}
\end{tabular}
\caption{Consonantal Future II Affixes.}\label{tab:future-2-consonantal}
\end{table}

\Paragraph{Future Stem}
Many verbs have a different future stem that is used in all future tenses (except the Future I); for example, the future
stem of *vvaúríhe* ‘to remember’, is *vvaúríźe*; thus, we have
*jvvaúríhe* ‘to remember’ but *jaúvvaúríźẹ́* ‘I shall remember’. Note also that these forms already include the
active/passive affixes, which is why it’s *jaúvvaúríźẹ́* and not \**jaújvvaúríźẹ́* or \**jjaúvvaúríźẹ́*.
As in the present, the dictionary form of the future
stem is a verbal noun; thus, *vvaúríźe* roughly means ‘the act of being about to remember’.\footnote{As noted before, infinitive
and gerund forms of future tenses are difficult to translate into English.}

\Paragraph{Stem-final vowel elision and *-(ẹ)*}
The future stem usually ends with a vowel, which is dropped if any future suffix or a suffix that starts with a vowel is added, e.g.
*laúvvaúríźaú* ‘they will remember’, not \**laúvvaúríźeaú*. Note that in the case of future suffixes, even those that start
with a consonant cause the vowel to be dropped. The only exception to this is the suffix *-(ẹ)*, which is found in a number of
Future II forms; that suffix is dropped instead, e.g. *aúrvvaúríźe* ‘she will remember’, not \**aúrvvaúríźẹ*.

\Paragraph{Nasal Stems}
Some future stems are nasalising, which is the case if the final vowel is a nasal vowel; in such cases, that vowel
is still dropped if a suffix is added, but if that suffix starts with a vowel, nasalisation is applied to it, e.g.
in the case of *dír*, whose future stem is *dírẹ́*, we have *aúnraûdíraû* ‘we shall say’: the *-aú* suffix merges
with the nasalisation of the final vowel to become *aû*. Unlike with regular stems, the Future II *-(ẹ)* *does*
replace the final vowel and becomes *-ẹ́* for such verbs, e.g. *aúrdírẹ́* ‘he will say’, and \s{1sg fut pass}
vocalic *-é* becomes *-ê*.

\Paragraph{*r-* Dropping}
Initial *r* in Future II suffixes is dropped if the
last consonant before the final vowel of the future stem is *w*, or an ʁ-coloured consonant such as *ź*, e.g.
*laúvvaúríźe* ‘they will be remembered’, not \**laúvvaúríźre*. If the last consonant of the future stem is *r*, since
any following vowel (whether nasalised or not) is deleted when a Future II suffix is added, the final *r* of the stem and
the initial *-r* of the Future II suffixes that have one coalesce to *rr*, e.g. *b’haý’ad’hórérre* ‘you (\s{pl}) will
love’.

\Paragraph{Affix Stacking}
Note that when more than one affix is used, at most one can be a future affix, e.g. *jaúsyvvaúríźẹ́* ‘I shall remember it’
and not \**jaúsaúrvvaúríźẹ́*. Generally, the active prefix will be the future affix, but it is possible to use the
passive future affixes instead for emphasis e.g. *jy’aúrvvaúríźe* roughly ‘him, I shall remember’; often, this is
also used to aid in establishing a contrast to some other part of the sentence that does not have this inversion.

Since some of the passive future affixes also have suffix parts—unlike the present affixes, where the passive forms are
all prefixes—we can end up with multiple suffixes in addition to multiple prefixes, in which case active prefixes, instead
of simply preceding the passive ones, can be thought of as effectively ‘wrapping’ them, e.g. *aúlaúvvaúríźey’ó* ‘we shall
remember them’, which contains *laúvvaúríźe* ‘they will be remembered’.

Finally, as always, infinitive prefixes come first. If combined with other affixes, it will generally be the future affix,
e.g. *haúlývvaúríźe* roughly ‘to be about to remember them’ but, as with passive affixes, variations are possible for emphasis
or contrastive power, e.g. *dẹlaúvvaúríźe*, which puts more emphasis on ‘them’.

\Paragraph{Examples}
Table~\ref{tab:future-2-adhor} below shows the complete (vocalic) Future II paradigm of the verb *ad’hór* ‘to love’, and
Table~\ref{tab:future-2-vvaurihe} the complete (consonantal) Future II paradigm of II *vvaúríhe* ‘to remember’; recall
that the future stems of these verbs are *ad’hórérẹ́* and *vvaúríźe*.

\begin{table}[H]
\centering
\noindent\begin{tabular}{@{}|>{}l|>{\it}l|>{\it}l|>{}l|>{}l|>{\it}l|>{\it}l|}\cline{1-3}\cline{5-7}
Active&\nf Sg&\nf Pl& & Passive&\nf Sg&\nf Pl\\\cline{1-3}\cline{5-7}
1st   &b’had’hórérẹ́  &náý’ad’hóréraû      &&1st    &vad’hórérệ    &náý’ad’hórérẹ́   \\\cline{1-3}\cline{5-7}
2nd   &ḍírad’hórérẹ́  &b’haý’ad’hórérrẹ́    &&2nd    &ḍírad’hórérẹ́  &b’haý’ad’hórérẹ́ \\\cline{1-3}\cline{5-7}
3rd m &ład’hórérẹ́    &lb’had’hóréraû      &&3rd m  &lad’hórérẹ́    &lb’had’hórérre  \\\cline{1-3}\cline{5-7}
3rd f &èład’hórérẹ́   &lb’had’hóréraû      &&3rd f  &lad’hórérẹ́    &lb’had’hórérre  \\\cline{1-3}\cline{5-7}
3rd n &aúład’hórérẹ́  &lb’had’hóréraû      &&3rd n  &sad’hórérẹ́    &lb’had’hórérre  \\\cline{1-3}\cline{5-7}
Infinitive&\multicolumn{2}{c|}{\it dad’hóréré}&&Infinitive&\multicolumn{2}{c|}{\it had’hórérẹ́}\\\cline{1-3}\cline{5-7}
Participle&\multicolumn{2}{c|}{\it ad’hórérŷr}&&Participle&\multicolumn{2}{c|}{\it ád’hórérýr}\\\cline{1-3}\cline{5-7}
\end{tabular}
\caption{Vocalic Future II Paradigm of *ad’hór*.}\label{tab:future-2-adhor}
\end{table}

\begin{table}[H]
\centering
\noindent\begin{tabular}{@{}|>{}l|>{\it}l|>{\it}l|>{}l|>{}l|>{\it}l|>{\it}l|}\cline{1-3}\cline{5-7}
Active&\nf Sg&\nf Pl& & Passive&\nf Sg&\nf Pl\\\cline{1-3}\cline{5-7}
1st   &jaúvvaúríźẹ́  &aúnraûvvaúríźaú &&1st   &vaúvvaúríźé    &naúvvaúríźe       \\\cline{1-3}\cline{5-7}
2nd   &b’hávvaúríźẹ &v́aúvvaúríźe     &&2nd   &\d{}ávvaúríźe  &b’haúvvaúríźe     \\\cline{1-3}\cline{5-7}
3rd m &aúrvvaúríźẹ  &laúvvaúríźaú    &&3rd m &y’aúrvvaúríźe  &laúvvaúríźe \\\cline{1-3}\cline{5-7}
3rd f &aúrvvaúríźẹ  &laúvvaúríźaú    &&3rd f &y’aúrvvaúríźe  &laúvvaúríźe \\\cline{1-3}\cline{5-7}
3rd n &aúrvvaúríźẹ  &laúvvaúríźaú    &&3rd n &saúrvvaúríźe   &laúvvaúríźe \\\cline{1-3}\cline{5-7}
Infinitive&\multicolumn{2}{c|}{\it dẹvvaúríźè}&&Infinitive&\multicolumn{2}{c|}{\it haúvvaúríźe}\\\cline{1-3}\cline{5-7}
Infinitive&\multicolumn{2}{c|}{\it vvaúríźŷ}&&Infinitive&\multicolumn{2}{c|}{\it ávvaúríźý}\\\cline{1-3}\cline{5-7}
\end{tabular}
\caption{Consonantal Future II Paradigm of *vvaúríhe*.}\label{tab:future-2-vvaurihe}
\end{table}

\subsubsection{Future Anterior}
The Future Anterior tense is formed by combining the Future II and the Present Anterior affixes. The \s{pres ant} suffixes
are applied after the \s{fut ii} affixes. The vocalic and consonantal affixes are shown in
Tables~\ref{tab:future-anterior-vocalic}~and~\ref{tab:future-anterior-consonantal}.

\begin{table}[H]
\centering
\noindent\begin{tabular}{@{}|>{}l|>{\it}l|>{\it}l|>{}l|>{}l|>{\it}l|>{\it}l|}\cline{1-3}\cline{5-7}
Active&\nf Sg&\nf Pl& & Passive&\nf Sg&\nf Pl\\\cline{1-3}\cline{5-7}
1st   &b’h- -\L é  &náý’- -aúrâ      &&1st    &v- -\L ê    &náý’- -\L â      \\\cline{1-3}\cline{5-7}
2nd   &ḍír- -\L á  &b’haý’- -(r)ệḍ   &&2nd    &ḍír- -\L á  &b’haý’- -\L áḍ    \\\cline{1-3}\cline{5-7}
3rd m &ł-   -\L á  &lb’h- -aûr       &&3rd m  &l- -\L á    &lb’h- -(r)ér \\\cline{1-3}\cline{5-7}
3rd f &èł-  -\L á  &lb’h- -aûr       &&3rd f  &l- -\L á    &lb’h- -(r)ér \\\cline{1-3}\cline{5-7}
3rd n &aúł- -\L á  &lb’h- -aûr       &&3rd n  &s- -\L á    &lb’h- -(r)ér \\\cline{1-3}\cline{5-7}
Infinitive&\multicolumn{2}{c|}{\it d- -á}&&Infinitive&\multicolumn{2}{c|}{\it h- -á}\\\cline{1-3}\cline{5-7}
Participle&\multicolumn{2}{c|}{\it -ŷrér}&&Participle&\multicolumn{2}{c|}{\it á- -ýrér}\\\cline{1-3}\cline{5-7}
\end{tabular}
\caption{Vocalic Future Anterior Affixes.}\label{tab:future-anterior-vocalic}
\end{table}

\begin{table}[H]
\centering
\noindent\begin{tabular}{@{}|>{}l|>{\it}l|>{\it}l|>{}l|>{}l|>{\it}l|>{\it}l|}\cline{1-3}\cline{5-7}
Active&\nf Sg&\nf Pl& & Passive&\nf Sg&\nf Pl\\\cline{1-3}\cline{5-7}
1st   &jaú-  -\L ệ  &aúnraû- -aúrâ  &&1st   &vaú- -\L ê    &naú- -\L â      \\\cline{1-3}\cline{5-7}
2nd   &b’há- -\L á  &v́aú- -éḍ       &&2nd   &\d{}á- -\L á  &b’haú- -\L áḍ    \\\cline{1-3}\cline{5-7}
3rd m &aúr-  -\L á  &laú- -aûr      &&3rd m &y’aúr- -\L á  &laú- -(r)ér \\\cline{1-3}\cline{5-7}
3rd f &aúr-  -\L á  &laú- -aûr      &&3rd f &y’aúr- -\L á  &laú- -(r)ér \\\cline{1-3}\cline{5-7}
3rd n &aúr-  -\L á  &laú- -aûr      &&3rd n &saúr-  -\L á  &laú- -(r)ér \\\cline{1-3}\cline{5-7}
Infinitive&\multicolumn{2}{c|}{\it dẹ- -á}&&Infinitive&\multicolumn{2}{c|}{\it haú- -á}\\\cline{1-3}\cline{5-7}
Participle&\multicolumn{2}{c|}{\it -(r)ŷr}&&Participle&\multicolumn{2}{c|}{\it á- -(r)ýr}\\\cline{1-3}\cline{5-7}
\end{tabular}
\caption{Consonantal Future Anterior Affixes.}\label{tab:future-anterior-consonantal}
\end{table}

\noindent
Note that again, nasalised stems add another level of nasalisation, and vowel-dropping still applies, but
this time, there is no *-ẹ* dropping, since none of the affixes end with *ẹ* anymore.

\Paragraph{Coalescence}
All vowel suffixes coalesce with the final vowel of the stem; if the suffix vowel is nasal, a level of nasalisation is
added, e.g. *aúrvvaúrízá* ‘he will have remembered’ from the future stem *vvaúríźe*. Note also that the *ź* is lenited
to *z*; the quality of the suffix vowel overrides that of the stem vowel. *r* contraction still happens as in the
Future II.

Tables~\ref{tab:future-ant-adhor}~and~\ref{tab:future-ant-vvaurihe} below show
the paradigm of the verbs *ad’hór* ‘to love’ and *vvaúríhe* ‘to remember’ in the Future Anterior tense. Note that
both the rules for the Future Anterior tense as well as the Present Anterior tense apply here.

\begin{table}[H]
\centering
\noindent\begin{tabular}{@{}|>{}l|>{\it}l|>{\it}l|>{}l|>{}l|>{\it}l|>{\it}l|}\cline{1-3}\cline{5-7}
Active&\nf Sg&\nf Pl& & Passive&\nf Sg&\nf Pl\\\cline{1-3}\cline{5-7}
1st   &b’had’hórérệ  &náý’ad’hóréraûrâ     &&1st    &vad’hórérệ    &náý’ad’hórérậ   \\\cline{1-3}\cline{5-7}
2nd   &ḍírad’hórérậ  &b’haý’ad’hórérrệḍ    &&2nd    &ḍírad’hórérậ  &b’haý’ad’hórérậḍ \\\cline{1-3}\cline{5-7}
3rd m &ład’hórérậ    &lb’had’hóréraûr      &&3rd m  &lad’hórérậ    &lb’had’hórérrér  \\\cline{1-3}\cline{5-7}
3rd f &èład’hórérậ   &lb’had’hóréraûr      &&3rd f  &lad’hórérậ    &lb’had’hórérrér  \\\cline{1-3}\cline{5-7}
3rd n &aúład’hórérậ  &lb’had’hóréraûr      &&3rd n  &sad’hórérậ    &lb’had’hórérrér  \\\cline{1-3}\cline{5-7}
Infinitive&\multicolumn{2}{c|}{\it dad’hórérâ}&&Infinitive&\multicolumn{2}{c|}{\it had’hórérậ}\\\cline{1-3}\cline{5-7}
Participle&\multicolumn{2}{c|}{\it ad’hórérŷrér}&&Participle&\multicolumn{2}{c|}{\it ád’hórérýrér}\\\cline{1-3}\cline{5-7}
\end{tabular}
\caption{Vocalic Future Anterior Paradigm of *ad’hór*.}\label{tab:future-ant-adhor}
\end{table}

\begin{table}[H]
\centering
\noindent\begin{tabular}{@{}|>{}l|>{\it}l|>{\it}l|>{}l|>{}l|>{\it}l|>{\it}l|}\cline{1-3}\cline{5-7}
Active&\nf Sg&\nf Pl& & Passive&\nf Sg&\nf Pl\\\cline{1-3}\cline{5-7}
1st   &jaúvvaúrízệ  &aúnraûvvaúríźaúrâ &&1st   &vaúvvaúrízê    &naúvvaúrízâ       \\\cline{1-3}\cline{5-7}
2nd   &b’hávvaúrízá &v́aúvvaúríźéḍ     &&2nd   &\d{}ávvaúrízá  &b’haúvvaúrízáḍ     \\\cline{1-3}\cline{5-7}
3rd m &aúrvvaúrízá  &laúvvaúríźaûr    &&3rd m &y’aúrvvaúrízá  &laúvvaúríźér \\\cline{1-3}\cline{5-7}
3rd f &aúrvvaúrízá  &laúvvaúríźaûr    &&3rd f &y’aúrvvaúrízá  &laúvvaúríźér \\\cline{1-3}\cline{5-7}
3rd n &aúrvvaúrízá  &laúvvaúríźaûr    &&3rd n &saúrvvaúrízá   &laúvvaúríźér \\\cline{1-3}\cline{5-7}
Infinitive&\multicolumn{2}{c|}{\it dẹvvaúríźá}&&Infinitive&\multicolumn{2}{c|}{\it haúvvaúríźe}\\\cline{1-3}\cline{5-7}
Infinitive&\multicolumn{2}{c|}{\it vvaúríźŷr}&&Infinitive&\multicolumn{2}{c|}{\it ávvaúríźý}\\\cline{1-3}\cline{5-7}
\end{tabular}
\caption{Consonantal Future Anterior Paradigm of *vvaúríhe*.}\label{tab:future-ant-vvaurihe}
\end{table}

\subsubsection{Conditional I and II}
The Conditional tenses are fairly simple—so long as you know the Future II and Future Anterior, that is. Both Conditionals
are formed by adding the *-ss(a)-* infix between the Future II stem and any suffixes. As always, the vowel is omitted if
the suffix after the infix starts with a vowel. For instance, Table~\ref{tab:cond-ii-vvaurihe}
below shows the consonantal Conditional II paradigm vor *vvaúríhe* ‘to be able to’. Note that the *ss* in this form
are *never* lenited:

\begin{table}[H]
\tabcolsep4pt
\centering
\noindent\begin{tabular}{@{}|>{}l|>{\it}l|>{\it}l|>{}l|>{}l|>{\it}l|>{\it}l|}\cline{1-3}\cline{5-7}
Active&\nf Sg&\nf Pl& & Passive&\nf Sg&\nf Pl\\\cline{1-3}\cline{5-7}
1st   &jaúvvaúríźessệ  &aúnraûvvaúríźessaúrâ &&1st   &vaúvvaúríźessê    &naúvvaúríźessâ       \\\cline{1-3}\cline{5-7}
2nd   &b’hávvaúríźessá &v́aúvvaúríźesséḍ      &&2nd   &\d{}ávvaúríźessá  &b’haúvvaúríźessáḍ     \\\cline{1-3}\cline{5-7}
3rd m &aúrvvaúríźessá  &laúvvaúríźessaûr     &&3rd m &y’aúrvvaúríźessá  &laúvvaúríźessrér \\\cline{1-3}\cline{5-7}
3rd f &aúrvvaúríźessá  &laúvvaúríźessaûr     &&3rd f &y’aúrvvaúríźessá  &laúvvaúríźessrér \\\cline{1-3}\cline{5-7}
3rd n &aúrvvaúríźessá  &laúvvaúríźessaûr     &&3rd n &saúrvvaúríźessá   &laúvvaúríźessrér \\\cline{1-3}\cline{5-7}
Inf&\multicolumn{2}{c|}{\it dẹvvaúríźessá}&&Inf&\multicolumn{2}{c|}{\it haúvvaúríźesse}\\\cline{1-3}\cline{5-7}
Inf&\multicolumn{2}{c|}{\it vvaúríźessŷr}&&Inf&\multicolumn{2}{c|}{\it ávvaúríźessý}\\\cline{1-3}\cline{5-7}
\end{tabular}
\caption{Consonantal Conditional II Paradigm of *vvaúríhe*.}\label{tab:cond-ii-vvaurihe}
\end{table}

\subsection{Miscellaneous Tenses}
\subsubsection{The Gnomic}
The gnomic tense is marked by the infix *-j(ú)-* after the stem: *ad’hór* ‘to love’ to *rad’hórjô* ‘We love (for ever)’.
The *ú* is omitted if the infix is followed by the vowel, in which case it causes nasalisation.

\subsection{Imperative}
The imperative mood exists only in the present tense, and only in the second and third person. It is formed by
affixing the following suffixes to the stem.
\begin{table}[H]
\centering
\noindent\begin{tabular}{@{}|>{}l|>{\it}l|>{\it}l|>{}l|>{}l|>{\it}l|>{\it}l|}\cline{1-3}\cline{5-7}
 Active&\nf Sg&\nf Pl& & Passive&\nf Sg&\nf Pl\\\cline{1-3}\cline{5-7}
2nd &c’h(e)-     &c’heb’h(y)- &&2nd& -rá   &-nú\\\cline{1-3}\cline{5-7}
3rd &\multicolumn{2}{c|}{\it c’hel(ẹ)-} &&3rd& -ḷẹ   &-b’hẹ\\\cline{1-3}\cline{5-7}
\end{tabular}
\caption{Imperative affixes.}\label{tab:imperative-affixes}
\end{table}

\noindent The diachrony of these forms is likely from subjunctive constructions w/ \s{pf} \**que* in the active
and from suffixed pronouns in the passive. Note that imperative affixes are added *in place* of
present active/passive affixes, e.g. *c’hedír!* ‘speak!’, not \**c’heḍẹdír*. As usual, the parenthesised
vowels are omitted if the verb form starts with a vowel, e.g. *c’had’hór!* ‘love!’.

Imperative affixes can be combined with active/passive affixes, though, as usual, an active imperative prefix
can only be paired with a passive present affix, and vice versa. Active imperative prefixes are always placed
first, e.g. *c’hevad’hór!* ‘love me!’, and passive affixes are placed last, e.g. *b’had’hórérá* ‘be loved by
us!’. The negation of the imperative uses the subjunctive and is explained in §~\ref{subsubsec:negated-subjunctive}.

\subsection{Subjunctive}\label{subsec:subjunctive}
The UF subjunctive forms are fortunately fairly simple: they use the same affixes as the present, past, and future
forms, except that each verb has a different subjunctive stem as well as a future subjunctive stem; the subjunctive
stem is typically formed by adding an *-s* to the end of the corresponding indicative stem, e.g. *ad’hór*
‘to love’ to *ad’hórs*; thus we have, e.g. *jad’hórs* ‘I would love’, and *rád’hórsó* ‘We would love’.

The future subjunctive stem is formed by adding the desinence *-śe* to the end of the future stem. For example,
the future stem of *ad’hór* is *ad’hórérẹ́*, so the future subjunctive stem is *ad’hórérẹ́śe*; similarly, the future
stem of *vvaúríhe* is *vvaúríźe*, so the future subjunctive stem is *vvaúríźeśe*.
There are several main uses of the UF subjunctive, each of which we shall examine in more detail below:
\begin{enum}
\item in reported speech, e.g. *lladírá vad’hórhé* ‘she said she loved me’;
\item with certain subordinating conjunctions, such as *b’he* ‘so that’;
\item to express deontic modality, e.g. *ḍẹḅars* ‘you may leave’;
\item as a jussive, e.g. *rad’hesó* ‘let’s go’;
\item as a negative imperative, e.g. *sá ḍẹḅars* ‘don’t leave’;
\item irrealis conditionals (see §~\ref{subsec:conditionals}).
\end{enum}

\subsubsection{Reported Speech}
UF does not use backshifting in reported speech, but rather, the corresponding subjunctive form is used. For instance,
*jḍad’hór* ‘I love you’ becomes *jdíré jḍad’hórs* ‘I said I love you’. Note that the tense stays the same in this
example: present indicative becomes present subjunctive. Accordingly, *jḍad’hóré* ‘I loved you’ becomes *jdíré
jḍad’hórsé* ‘I said I loved you’.

Consequently, the tense of the verb in reported speech is independent of the tense of the matrix clause, e.g.
*b’had’hrệ* ‘I shall go’ becomes *jdíré b’had’hrẹ́sé* ‘I said I would go’,\footnote{Note the lenition here because
of the present anterior suffix: *b’had’hrẹ́sé*, not \**b’had’hrẹ́śé*.} with *b’had’hrẹ́sé* being the Future II
subjunctive form of *b’had’hrẹ́*.

\subsubsection{Dependent Clauses}
The following subordinating conjunctions take the subjunctive:
\TwoCols[.45\hsize][.45\hsize] {
\begin{dlist}[\bfseries\itshape]
    \item[áhaúr] ‘even though’
    \item[ḅas] ‘because’
    \item[b’he] ‘so that’
    \item[c’haúr] ‘as’ (viz. ‘because’)
    \item[daúc’h] ‘therefore’
    \item[de] ‘once’
\end{dlist}
} {
\begin{dlist}[\bfseries\itshape]
    \item[ráhẹ] ‘though’
    \item[rê] ‘although’
    \item[s] ‘if’ (see §~\ref{subsec:conditionals})
    \item[sá] ‘without’
    \item[sauc’h] ‘except that’
    \item[váłé] ‘despite that’
\end{dlist}
}

\noindent Note that not all subordinating conjunctions take the subjunctive. For instance, the conjunction *y’is*
‘because’ takes the indicative: *jḍad’hórs c’haúr* ‘as I love you’, but *jḍad’hór y’ís* ‘because I love you’.

\subsubsection{Deontic Modality}
The subjunctive can also be used on its own, in which case it assumes a deontic or jussive meaning;
in the first person, it is generally a jussive, e.g. *rad’hesó* ‘let’s go’, but the jussive sense is not restricted
to the first person, e.g. *lẹsyrét’hes* ‘he take care of it’ (in the sense of ‘let him take care of it’).

The deontic sense is also apparent from that last example: *lẹsyrét’hes* can also be interpreted to mean ‘he
may take care of it’, which can either be a statement of permission or a condescending order. Note that even
though UF also has a word for ‘let’ (namely *le*), it is mostly used in questions or commands, while the
deontic subjunctive is used to grant permission.

\subsubsection{Negation}\label{subsubsec:negated-subjunctive}
The subjunctive is negated with the particle *sá*, rather than with *asý’ýâ*. The particle *sá* is placed
immediately before the verb form it negates, e.g. *sá jḍad’hórs c’haúr* ‘as I don’t love you’. It is reduced
to *s’* before vowels, but interestingly, it does not cause nasalisation in that case, e.g. *s’aúsydíssâ c’haúr*
‘as we didn’t say it’.

On its own, the negated subjunctive is used to express a negative imperative in the second and third person,
e.g. *sá ḍẹḅars* ‘don’t leave’, and a negative jussive in the first person e.g. *sá rad’hesó*, ‘let’s not go’.

\subsection{Optative}\label{subsec:optative}
The UF optative is used to express wishes, hopes, as well as in certain conditional constructions. It is formed
by affixing *y’(ẹ)\L* to the verb stem, e.g. *dẹvy’ẹvvaúríhe* ‘may you remember me’. Some prefixes in the future end with
*ý’*; this is dropped in the optative: e.g. *náý’ad’hóraú* ‘we shall love’ becomes *náy’ad’hóraú* ‘may we love’. Note
that the bare optative is difficult to translate into English; a more precise explanation of what these forms actually
mean will be given below. Uses of the optative include:
\begin{enum}
\item wishes, hopes, dreams, and aspirations;
\item with certain subordinating conjunctions, such as *auha* ‘in case’;
\item talking about fears;
\item counterfactual conditionals (see §~\ref{subsec:conditionals}).
\end{enum}

\subsubsection{Wishes and Hopes}
The most traditional use of the optative is to express wishes and hopes, e.g. *dẹvy’ẹvvaúríhe* ‘may you remember me’. In
the present or future tense, this use indicates a wish for something to happen; in the present tense, its meaning is
that of a wish for a condition to be true in the present in the face of uncertainty or lack of knowledge; thus, the
actual meaning of *dẹvy’ẹvvaúríhe* is roughly ‘I hope that you remember me’.\footnote{The context of this utterance could be
meeting someone again after a long time apart and hoping that they still remember you.} In the future tense, it indicates a wish
that a situation will be true in the future, e.g. *b’hávy’ẹvvaúríźe* ‘may you remember me’.

In the past tenses, the optative indicates dismay, regret, or disappointment that something did not happen, e.g.
\s{pres ant} *dẹvy’ẹvvaúríhá* ‘if only you had remembered me’.

\subsubsection{Dependent Clauses}
The following subordinating conjunctions take the optative:
\TwoCols[.45\hsize][.45\hsize] {
\begin{dlist}[\bfseries\itshape]
    \item[auha] ‘in case’
    \item[ab’há] ‘before’
    \item[ávrê] ‘unless’
    \item[ḅré] ‘after’
\end{dlist}
}{
\begin{dlist}[\bfseries\itshape]
    \item[fahaú] ‘in such a way that’
    \item[jys] ‘until’
    \item[sit’há] ‘supposing that’
    \item[úrbh] ‘provided that’
\end{dlist}
}

\subsubsection{Negation and Verbs of Fearing}
As with the negated subjunctive, the negated optative also has a separate negation particle, namely *t’hé*
(spelt *t’h’\N* before vowels). Note that a negated optative indicates that the speaker wishes that something does
or had not happened, e.g. *t’hé dẹvy’ẹvvaúríhá* ‘if only you had not remembered me’. The negation thus negates
the wish, and not the act of wishing; for the latter, the indicative or subjunctive together with a verb such
as *sḅé* ‘to wish’ are used instead.

Verbs of fearing are typically construed with a dependent clause in the negated optative, e.g. *jréd’hé
t’hé b’háy’ẹbharẹ́* ‘I was afraid lest you might leave’.

\subsection{Irregular Verbs}\label{subsec:irregular-verbs}
\subsubsection{The Conjugation of \textit{eḍ} ‘to be’}


\begin{table}[H]
\centering
\noindent\begin{tabular}{@{}|>{}l|>{\it}l|>{\it}l|l|l|>{\it}l|>{\it}l|l|l|>{\it}l|>{\it}l|}\cline{1-3}\cline{5-7}\cline{9-11}
Present&\nf Sg&\nf Pl    && Pres. Ant.&\nf Sg&\nf Pl    && Preterite&\nf Sg&\nf Pl      \\\cline{1-3}\cline{5-7}\cline{9-11}
1st       & vy’í & aúsó   && 1st      & vẹ     & aúfý   && 1st       & vet’h & weḍy’ó   \\\cline{1-3}\cline{5-7}\cline{9-11}
2nd       & ḍe   & b’heḍ && 2nd       & ḍyf    & b’hu   && 2nd       & ḍet’h & b’heḍy’é \\\cline{1-3}\cline{5-7}\cline{9-11}
3rd m     & le   & lẹsó  && 3rd m     & leb’h  & lẹfýr  && 3rd m     & let’h & let’he   \\\cline{1-3}\cline{5-7}\cline{9-11}
3rd f     & lle  & llẹsó && 3rd f     & lle’bh & llẹfýr && 3rd f     & llet’h & llet’he \\\cline{1-3}\cline{5-7}\cline{9-11}
3rd n     & s    & lasó  && 3rd n     & seb’h  & lafýr  && 3rd n     & set’h & laet’h   \\\cline{1-3}\cline{5-7}\cline{9-11}
Infinitive& \multicolumn{2}{c|}{\it éḍ} && Infinitive& \multicolumn{2}{c|}{\it éfyḍ} && Infinitive& \multicolumn{2}{c|}{\it ét’hẹd} \\\cline{1-3}\cline{5-7}\cline{9-11}
\end{tabular}
\caption{Paradigm of the verb \emph{eḍ}.}\label{tab:ed-paradigm}
\end{table}

\noindent The etymology of these forms is mostly from a gradual simplification of coalesced forms of the personal
pronouns with the PF copula. To compensate for the fact that PF lacks certain forms that are present in UF, some
of the forms were coined by analogy. For instance, the \s{pres ant inf} *éfyḍ* is derived from the \s{pres ant}
stem \**fy* and the \s{pres inf} *éḍ*, and the same is true for the \s{pret inf} *ét’hẹd*.

For obvious reasons, the copula lacks passive forms. At the same time, the first person forms are manifestly
derived from the first person passive pronoun, for unknown reasons.

Unlike nearly every other word in the language, all forms of the copula are summarily stressed on the first
syllable.

\subsection{Noun Morphology}\label{subsec:noun-morphology}
UF has 4 declensions. A definite and indefinite vocalic declension, and a definite and indefinite consonantal declension.
As their names might suggest, the former two are used for nouns that start with a vowel, and the latter two for nouns
that start with a consonant. UF has no morphologically separate articles; rather, the old PF articles have been incorporated
into the declensions. Furthermore, UF no longer has a gender distinction in nouns.

\subsubsection{Declension}
The table below shows the affixes of the definite and indefinite declensions. The declensions are mostly identical,
except that, as with the conjugation of verbs, the consonantal prefixes often end in a vowel (marked below with
parentheses), which are not present in the vocalic declension.

\begin{table}[H]
\centering
\noindent\begin{tabular}{@{}|l|>{\it}l|>{\it}l|l|l|>{\it}l|>{\it}l|}\cline{1-3}\cline{5-7}
Definite    &\nf Sg&\nf Pl && Indefinite       &\nf Sg&\nf Pl\\\cline{1-3}\cline{5-7}

Absolutive    & $\emptyset$     & l-      &&Absolutive    & $\emptyset$-\N & $\emptyset$-\L   \\\cline{1-3}\cline{5-7}
Nominative    & lá-\L   & lé-\L   &&Nominative    & ŷn-\N & ý-\L      \\\cline{1-3}\cline{5-7}
Vocative      & $\emptyset$-\L  & $\emptyset$-\L  &&Vocative      & / & /             \\\cline{1-3}\cline{5-7}
Partitive     & dy-\L   & dẹ-\L   &&Partitive     & dŷn-\N & dý-\L    \\\cline{1-3}\cline{5-7}
Accusative    & i-\L    & sý-\L   &&Accusative    & s-\L & s-         \\\cline{1-3}\cline{5-7}
Genitive      & á-\L    & abh-\L  &&Genitive      & sý-\N & sý-\L     \\\cline{1-3}\cline{5-7}
Dative        & as-\L   & a-\L    &&Dative        & an-\N & an-\L     \\\cline{1-3}\cline{5-7}
Inessive      & dwá-    & dwé-    &&Inessive      & dáhŷn- & dáhŷ-    \\\cline{1-3}\cline{5-7}
Ablative      & rê(d)-  & rês-    &&Ablative      & rêdýn- & rêdý-    \\\cline{1-3}\cline{5-7}
Allative      & b’hé-\L & b’hér-  &&Allative      & b’hŷn-\N & b’hý-\L  \\\cline{1-3}\cline{5-7}
Considerative & słá-    & słé-    &&Considerative & sý’óýn- & sý’óý-  \\\cline{1-3}\cline{5-7}
Instrumental  & b’hel-  & b’he-   &&Instrumental  & b’hehý(n)- & b’heh-  \\\cline{1-3}\cline{5-7}
...           &         &         &&              & &                 \\\cline{1-3}\cline{5-7}
\end{tabular}
\caption{UF Declension.}\label{tab:table-uf-declension}
\end{table}

\noindent Most of these forms cause lenition in the initial consonant of the noun, e.g. *ḍalẹ* ‘table’ to
\s{def acc sg} *s’thalẹ*; this lenition is blocked in the \s{indef acc pl} due to the presence of a hypercorrected ‘s’
in PF \**ces*, e.g. *s’ḍalẹ* ‘the tables (\s{acc})’ (not *s’thalẹ*, which is the singular), as well as in
less commonly used forms such as the \s{def} inessive *dwáḍalẹ* ‘on the table’.

The \s{indef nom sg} *ŷn-* prefix and some other forms nasalise nouns; as a reminder, this means that in
nouns starting with *ḍ*, the *ḍ* is deleted, e.g. *ŷnalẹ* ‘a table’;
it causes nasalisation in words that start with a vowel e.g. *ehyó* ‘shield’ to *ŷnéhyó* ‘a shield’. The indefinite vocative
does not exist, as that would make little sense. As lenition, nasalisation too is blocked in rarer forms, e.g. \s{indef} inessive
*dáhŷnḍalẹ* ‘on a table’.

The absolutive case is used for the predicate noun of predicative sentences, e.g. *Aúsó ḍe ráhó* ‘We are all fish’.

The considerative case can be translated as ‘according to’, or ‘in the opinion of’, and is used to express the opinion of
the speaker or point out something as an opinion, belief, or hypothesis of someone or something.

The *d* in the \s{def abl sg} is omitted if the noun starts with a consonant, e.g. *rêḍalẹ* ‘from the table’; be careful
especially with words that start with *s*, whose \s{abl sg} is often mistaken for a plural, e.g. *rêsol* ‘from the floor’,
but *rêssol* ‘from the floors’.

The diachrony of these forms is mostly from the PF definite and indefinite pronouns, though some forms, such as the
accusative, are borrowed from demonstratives instead (\s{def} from PF \**celui* and \s{indef} from PF \**ce*); the definite
partitive forms are from the PF partitive article, and
the indefinite forms are formed with an additional *d-* by analogy to the definite forms. The locative cases are combinations
of the articles and PF prepositions. The ablative is from PF \**loin de* ‘away from’. The diachrony of the genitive singular
is unclear.

\begin{table}[H]
\centering
\noindent\begin{tabular}{@{}|l|>{\it}l|>{\it}l|l|l|>{\it}l|>{\it}l|}\cline{1-3}\cline{5-7}
Definite    &\nf Sg&\nf Pl && Indefinite&\nf Sg&\nf Pl\\\cline{1-3}\cline{5-7}

Nominative  & lát’halẹ  & lét’halẹ   &&Nominative & ŷnalẹ & ýt’halẹ         \\\cline{1-3}\cline{5-7}
Vocative    & t’halẹ    & t’halẹ    &&Vocative   & / & /                      \\\cline{1-3}\cline{5-7}
Partitive   & dyt’halẹ  & dẹt’halẹ &&Partitive   & dŷnalẹ & dýt’halẹ                     \\\cline{1-3}\cline{5-7}
Accusative  & it’halẹ & sýt’halẹ  &&Accusative & st’halẹ & sḍalẹ                       \\\cline{1-3}\cline{5-7}
...         &  &  &&           & &              \\\cline{1-3}\cline{5-7}
Inessive    & dwáḍalẹ & dwéḍalẹ &&Inessive   & dáhŷnḍalẹ & dáhýḍalẹ                    \\\cline{1-3}\cline{5-7}
\end{tabular}
\caption{Consonantal declension of *ḍalẹ*.}\label{tab:vocalic-declension}
\end{table}

\subsection{Adjectives}
UF does not have many actual adjectives. Most words in UF are either nouns or verbs, and most ‘adjectives’ are just
participles, which can always be used like adjectives. Indeed, there are a lot of verbs whose meaning is something
along the lines of ‘to be X’, whose present participle behaves like the adjective ‘X’, e.\,g. *ḅẹt’hẹ* ‘to be small’
to *ḅẹt’hâ* ‘small’ (literally ‘being small’).

Adjectives generally follow the noun they modify and are never inflected, e.g. *át’halẹ ḅẹt’hâ* ‘of a small table’.
There is no established order of adjectives.

%% ‘lẹ-’: from PF ‘plus’. Comparative prefix.
\subsubsection{Comparison}
Unlike in many other languages, there are 3 comparatives in UF: The affirming comparative, so called
because it affirms the positive (‘better, and also good’); the denying comparative, which denies the positive
(‘better, but not good’), and the neutral comparative, which does not make any statement about the positive
(‘better’).

To illustrate the difference between the three: We might say that an ant is ‘bigger’ than a grain of sand, but
an ant is still not big, all things considered. By contrast, an elephant may be ‘smaller’ than a mountain,
but that doesn’t mean that an elephant is small.

In UF, the comparatives are expressed by three infixes, which are prefixed directly to the stem. The affirming
comparative prefix is *lẹ*, the denying comparative prefix is *y’ŷ*, and the neutral comparative prefix is *rê*.
Thus, we have *ḅẹt’hâ* ‘small’, *lẹḅẹt’hâ* ‘smaller, and also small’, *y’ŷḅẹt’hâ* ‘smaller, but not small’, and
*rêḅẹt’hâ* ‘smaller’.

The comparative prefixes can also be applied to verbs, though they usually only make sense for the aforementioned
‘adjective verbs’, e.g. *jy’ŷḅẹt’hẹ* ‘I am smaller, but still big’. Note that these prefixes
might cause a verb’s forms to change from vocalic to consonantal, e.g. *ebhẹ* ‘to be thick’ (future stem *ebhrẹ*)
is vocalic *náy’ebhraú* ‘we shall be thick’ in the positive, but consonantal *aúnraûy’ŷebhraû* ‘we shall be
thicker, but not thick’ in the negative comparative.

The affirming comparative can also be used absolutely, with the meaning of ‘to a large degree’. Thus,
we have *ḅẹt’hâ* ‘small’, and *lẹḅẹt’hâ* ‘tiny’; sometimes, this also leads to a slight change in meaning
or perception, e.g. *ebhâ* ‘thick’, but *lẹ-ebhâ* ‘thicc’.

The affirming and denying comparative can also mean ‘too X’ and ‘not X enough’, respectively; thus, *lẹḅẹt’hâ*
can also mean ‘too small’, and *y’ŷḅẹt’hâ* can also mean ‘not small enough’, though this meaning is somewhat
uncommon in isolation and most commonly found in constructions (see below).

The superlative is formed with one of two prefixes: *ré\L* and *râdvâ*. Be careful not to confuse the former
with the neutral comparative *rê*! The two prefixes are largely interchangeable, however, the former is more
literary and also older. The latter is a more recent development to reduce potential ambiguity with the
neutral comparative. Note that *ré* lenites, whereas *râdvâ* does not. Thus, we have *rébhẹt’hâ* or
*râdvâḅẹt’hâ* ‘smallest’.

\subsubsection{Constructions}
The comparative can be used together with an infinitive, ACI, or PCI. The affirming comparative here has the meaning
of ‘too X to \ldots’, and the denying comparative has the meaning of ‘not X enough to \ldots’. A good illustrative
example of this is the following UF proverb:

\gloss {
    Láráhó slẹlúrá b’héd’hẹhẹ dẹnájẹ.
    Lá-ráhó|s-lẹ-lúr-á|b’hé\Sl d’hẹhẹ|dẹ-nájẹ
    \s{nom}-fish|\s{3n}-\s{aff.comp}-bulky-\s{3sg.pres.ant}|\s{all}\Sl surface|\s{inf}-swim
    ‘The fish was too bulky to swim to the surface’\footnotemark
}

\footnotetext{This is a very common proverb (often also just \textit{láráhó slẹlúr} ‘The fish is too bulky’)
and roughly means that something has gone too far or gone on for too long (‘Now you’ve done it’ or ‘Now
it’s too late’). Variations of it exists; in the optative, for instance, this proverb means ‘Let’s not overdo this’.}

\section{Syntax}\label{sec:syntax}
UF syntax is unfortunately complicated in what morphological constructs are used in what situations, and
the rules are not always clear. The following is a list of the most common constructions.

\subsection{Independent Clauses}
The UF independent clause typically consists of a finite verb together with a subject perhaps several
objects. The verb is conjugated to agree with the subject in person, number, and gender in some cases.

\gloss {
    Rab’hadó iárb.
    r-ab’haḍ-ó| i-árb
    \s{1pl}-fell-\s{1pl}| \s{acc}-tree
    ‘We are felling the tree.’
}

The unmarked tense in UF is the present tense, which can generally be translated as either a present or
present continuous tense in English. For general truths and facts, the gnomic tense is generally used
instead.

\gloss {
    Rab’hadjô sárb.
    r-ab’haḍ-jô| s-árb
    \s{1pl}-fell-\s{gnomic}\Sl\s{1pl}| \s{acc.indef}-tree
    ‘We fell trees.’
}

The object is incorporated into the verb if it is a personal pronoun, in which case there are rules for
the order in which these affixes occur (see Section~\ref{subsec:verbal-morphology}).

\gloss {
    Lerab’hat’há.
    lẹ-r-ab’ha\Sl t’há.
    \s{3sgm}-\s{1pl.pass}-fell\Sl\s{3sg.pres.ant}
    ‘He felled us.’
}

Word order is rather lax due to the presence of case marking, and any constituent can be fronted for
emphasis, but the default word order is SVO or SOV.

\gloss {
    B’hehýnác aúlýab’hat’hâ.
    b’hehýn-ác| aú-lý-ab’ha\Sl t’hâ.
    \s{instr.indef}-axe|\s{1pl}-\s{3pl.pass}-fell\Sl\s{1pl.pres.ant}
    ‘With an axe, we have felled them.’
}

Note that words belonging to the same phrase are typically juxtaposed as adjectives are not inflected. However,
this rule may sometimes be broken, particularly in poetry. Consider, for example, the following passage in alexandrine
metre, written by the renowned poet \s{J.\,Y.\,B.\,Smyth}, where we can find the verb positioned between a possessive
pronoun and its associated noun:

\gloss {
    Au lýr náý’acḍaúrâ sýeċ asvaúr sýárb.
    Au |lýr |náý’-acḍ-aúrâ|sý-eċ|as-vaúr|sý-árb
    And|their|\s{1pl.fut.ant}-cleave-\s{circ}|\s{acc.pl}-sin|\s{dat}-world|\s{acc}-tree
    ‘And we shall indeed have revealed their sins to the world’\footnotemark
}

\footnotetext{See the dictionary entry for *act’he*, sense 4, for more information about the use of this word here,
which normally means ‘cleave’. The literal meaning of this sentence is roughly: ‘And we shall have brought
down the trees upon their sins, to (= for the benefit of) the world’.}


\subsection{Negated Clauses}
Negation in the indicative is expressed using the particle *asý’ýâ* ‘not’, which is often appended to verbs
as *’sý’ýâ*. For a discussion of negation in the subjunctive and optative, see
Sections~\ref{subsec:subjunctive}~and~\ref{subsec:optative}.
By default, the particle is placed right after the verb:
\gloss {
    Aúlýab’hat’hâ’sý’ýâ b’hehýnác.
    aú-lý-ab’ha\Sl t’hâ|’sý’ýâ|b’hehýn-ác.
    \s{1pl}-\s{3pl.pass}-fell\Sl\s{1pl.pres.ant}|not|\s{instr.indef}-axe
    ‘We have not felled them with an axe.’
}

In case of a fronted constituent, the particle is placed after that constituent:
\gloss {
    B’hehýnác asý’ýâ aúlýab’hat’hâ.
    b’hehýn-ác|asý’ýâ|aú-lý-ab’ha\Sl t’hâ.
    \s{instr.indef}-axe|not|\s{1pl}-\s{3pl.pass}-fell\Sl\s{1pl.pres.ant}
    ‘It is not with an axe that we have felled them.’
}

Note that it is not valid to both front a constituent and not move the negation. For example,
the following sentence sounds very awkward and no UF speaker would ever say or write this,
save perhaps to sound extremely ironic.
\gloss {
    \#B’hehýnác aúlýab’hat’hâ’sý’ýâ.
    b’hehýn-ác|aú-lý-ab’ha\Sl t’hâ|’sý’ýâ.
    \s{instr.indef}-axe|\s{1pl}-\s{3pl.pass}-fell\Sl\s{1pl.pres.ant}|not
    *Roughly:* ‘With an axe, we have not-felled them.’
}

UF makes frequent use of double negation in conjunction with words that create a negative context
such as *jávé* ‘never’, *y’ê* ‘nothing’, or *ráv́â* ‘seldom’. Typically, such words are frontend,
and consequently, the negation particle then appears appended to them, e.g.:
\gloss {
    Ráv́â’sý’ýâ st’halẹ jaċt'hé.
    Ráv́â|’sý’ýâ |s\Sl t’halẹ |j-aċt'h\Sl é
    seldom|not |\s{acc.indef}\Sl table|\s{1sg}-buy\Sl\s{3sg.pres.ant}
    ‘Rarely have I ever bought a table.’
}

Note that double negation is required for the sentence to make sense; UF learners often forget
about that, which can lead to rather awkward constructs such as:
\gloss {
    \#Ráv́â st’halẹ jaċt'hé.
    Ráv́â|s\Sl t’halẹ |j-aċt'h\Sl é
    seldom|\s{acc.indef}\Sl table|\s{1sg}-buy\Sl\s{3sg.pres.ant}
    *Roughly:* ‘I rarely-bought a table.’
}

Still, if a fronted constituent is present, the negation particle is placed after that constituent:
\gloss {
    St’halẹ’sý’ýâ ráv́â  jaċt'hé.
    s\Sl t’halẹ|’sý’ýâ | ráv́â |j-aċt'h\Sl é
    \s{acc.indef}\Sl table|not| seldom |\s{1sg}-buy\Sl\s{3sg.pres.ant}
    ‘A table I have bought rarely.’
}

Foreigners often make the mistake of assuming that the negation particle is part of a word,
e.g. that *ráv́â’sý’ýâ* means ‘seldom’. As such, UF speakers, when imitating a foreigner, may
sometimes use more than one negation particle in a single sentence. Note that this is very
much not proper language; such constructions are summarily comedic and best compared to phrases
such as ‘it do be like that’ in English:
\gloss {
    \*Ráv́â’sý’ýâ st’halẹ jaċt'hé’sý’ýâ
    Ráv́â|’sý’ýâ |s\Sl t’halẹ |j-aċt'h\Sl é|’sý’ýâ
    seldom|not |\s{acc.indef}\Sl table|\s{1sg}-buy\Sl\s{3sg.pres.ant}|not
    *Roughly:* ‘Rarely-not I bought a table.’
}

\subsection{Interrogative Clauses}
In UF, questions are generally marked by one or more particles. Unlike in many other languages, the verb generally
does not move, except perhaps for emphasis. The most fundamental kind of question is a yes-no question, which is
marked by the interrogative particle *c’hes*. The particle typically occurs in second position in the sentence:
\gloss {
    St’halẹ c’hes jaċt'hé?
    s\Sl t’halẹ |c’hes |j-aċt'h\Sl é
    \s{acc.indef}\Sl table|Q|\s{1sg}-buy\Sl\s{3sg.pres.ant}
    ‘Did I buy a table?’
}

Negation is placed in the usual position. A negated question is marked by the negation particle *sý’ýâ*,
and the expected answer is ‘yes’:
\gloss {
    St’halẹ c’hes jaċt'hé’sý’ýâ?
    s\Sl t’halẹ |c’hes |j-aċt'h\Sl é|’sý’ýâ
    \s{acc.indef}\Sl table|Q|\s{1sg}-buy\Sl\s{3sg.pres.ant}|not
    ‘Did I not buy a table?’
}

Alternatively, the particle *(r)vá* can be used to indicate that the speaker expects the answer to be ‘no’
or to indicate disbelief, surprise, or amazement. Note that this particle *replaces* the question particle.
Attempting to use both particles in the same sentence is ungrammatical and will likely be interpreted as
stuttering.
\gloss {
    St’halẹvá jaċt'hé?
    s\Sl t’halẹ |vá |j-aċt'h\Sl é
    \s{acc.indef}\Sl table|Q|\s{1sg}-buy\Sl\s{3sg.pres.ant}
    ‘I bought a table?’
}

Of course, these questions can also be negated:
\gloss {
    St’halẹvá jaċt'hé’sý’ýâ?
    s\Sl t’halẹ |vá |j-aċt'h\Sl é|’sý’ýâ
    \s{acc.indef}\Sl table|Q|\s{1sg}-buy\Sl\s{3sg.pres.ant}|not
    ‘I didn’t buy a table?’
}

The precise meaning of these questions is as follows: In *St’halẹ c’hes jaċt'hé?* (‘Did I buy a table?’),
the speaker is asking whether they themselves bought a table; a plausible situation would be that they
simply forgot whether they did. Its negation, *St’halẹ c’hes jaċt'hé’sý’ýâ?* (‘Did I not buy a table?’),
could be used if the speaker is sure they bought a table sometime ago, but they can’t seem to find it and
are starting to doubt themselves (‘Did I not buy a table? I’m sure I did.’).

By contrast, the question *St’halẹvá jaċt'hé?*) would be an assertion of disbelief; maybe the speaker
found a table in their loft, and they can’t seem to remember buying it, but the price tag is still there.
Finally, its negation *St’halẹvá jaċt'hé’sý’ýâ?* would most likely be the speaker expressing their frustration
over the fact that they can’t seem to find their table and asserting that, in fact, they know for sure that
they did indeed buy a table (‘Did I not buy a table? I know I did!’).

Fronting of the verb in the last two cases generally indicates confusion rather than amazement or anger and
is most commonly used in response to someone else’s statement so as to ask for clarification (‘What do you mean
“I bought a table”; what are you talking about?’).
\gloss {
    Jaċt'hévá st’halẹ?
    j-aċt'h\Sl é |vá|s\Sl t’halẹ
    \s{acc.indef}\Sl table|Q|\s{1sg}-buy\Sl\s{3sg.pres.ant}
    ‘I \textit{bought} a \textit{table}?!’
}

The same applies to the negated version of such a question:
\gloss {
    Jaċt'hé’sý’ýâvá st’halẹ?
    j-aċt'h\Sl é |’sý’ýâ|vá|s\Sl t’halẹ
    \s{acc.indef}\Sl table|not|Q|\s{1sg}-buy\Sl\s{3sg.pres.ant}
    ‘I \textit{didn’t} buy a \textit{table}?!’
}

Note the order of particles: negation precedes the question particle. Placing them the other way around
makes it sound like you’re trying to correct yourself from *Jaċt'hévá* to *Jaċt'hé’sý’ýâ*.

\subsection{Conditionals}\label{subsec:conditionals}
Notes:
- Simple indicative conditional use the indicative ‘If X, then Y’, ‘If it rains, the ground is wet’;
  ‘If A is 2 and B is 3, then A + B is 5’.

- Irrealis conditionals (conditionals that describe a situation that is not true, and could never be true)
  use the subjunctive ‘If it were raining right now, we would be wet’.

- Potential conditionals, which describe a situation that could happen, and which the
  speaker considers plausible use the optative ‘If we were to go left now, we’d fall off a cliff’. These
  conditionals are only possible in the present and future.

- Counterfactual conditionals, which describe a situation that could be true, but isn’t. These conditionals
  exist only in the present and past and also use the optative ‘If we had gone left, we would have fallen off
  a cliff’.

\section{Examples}\label{sec:examples}
\subsubsection{Simple Glossing Example}

\gloss {
    Cárvá, sráhó dwávaût’há dact’heá?
    Ċár |vá |s-ráhó |dwá-vaût’há |ḍ-aċt’he-á
    ˈj̊ɑ̃ːˠ|ʋ̃ɑ̃|ˌsɰɑ̃ˈhɔ̃|dɰɑ̃ˌʋ̃ɔ̃̃ˈθɑ̃|da̯j̊ˈθe.ɑ̃
    Charles.\s{voc}|\s{particle}|\s{indef.acc}-fish|\s{def.iness}-mountain|\s{2sg}-buy-\s{pres.ant.2sg}
    ‘Charles, you bought a fish on the mountain?’
}


\subsubsection{I Don’t Think This Warrants Explaining}
{\itshape\bfseries
Słérá de c’hóný áb’hásy’ô, ráy’ê y’aúhý dís dyb’hóy’e sab’héy’. Ez lé-el lalebet’he z’ihór bet’hê rêsol daudé.

}

\gloss {
    słé-rá| ḍẹ| c’hóný| áb’hásy’ô| ráy’ê| y’aúhý| ḍ-ís
    \s{cons.pl}-law|all|well.known|\s{gen}\Sl aviation|way|there.is.no|\s{inf-subj}\Sl can
}

\gloss {
    dy-b’hóy’ẹ | s-ab’héy’ |ez |lé-el | la-lẹ-bet’hẹ|z’
    \s{part}-to.fly|\s{acc.indef}-bee|its|\s{nom.pl}-wing|\s{3pl-aff.comp}-be.small|its
}

\gloss {
    ihór | bet’hê | rê-sol | ḍ-auḍé
    \s{acc}\Sl body| be.small\Sl\s{part}|\s{abl}-soil|\s{inf}-obtain
}

%% NOTES:
%% ‘ez’: its, her, his. Before a word that starts w/ a vowel, this becomes *z* (e.g. **ez ihór* > *z’ihór*).
%%       No, the apostrophe doesn’t make sense in that position.
%%       Otherwise, after a word ending w/ a vowel, it becomes *’z*
%%
%%
%% Tous les 36 du mois -> Include the ‘36’ in the word!
%%
%% Comparative + ACI: ‘too X to...’. With negation ‘Not X enough to ...’.
%%
%% The PF infinitive endings -ir, -er, etc. are often dropped to form the base form (e.g. auḍé < auḅḍénír).
%%
%% Adjectives are not declined and follow the noun they modify.
%%
%% Word order for ACIs puts the infinitive at the very beginning and the accusative at the very end, or vice versa,
%%   and everything that belongs to that phrase inbetween.
%%
%% ‘-t’he’, FUT ‘-ḍe’, SUBJ ‘t’hes’ is a productive word formation suffix that can be used to
%% turn a noun into a verb that roughly means ‘to use that thing’. E.g. ac ‘axe’ -> act’he  ‘roughly: to use
%% an axe’.

\medskip\noindent
‘According to all known laws of aviation, there is no way a bee should be able to fly. Its wings are too
small to get its fat little body off the ground. The bee, of course, flies anyway because bees don't care
what humans think is impossible.’

\medskip\noindent
Literal translation: ‘According to all known laws of aviation, there is no way that a bee should be capable of flight.
Its wings are too small for its little body to obtain [distance] from the ground.




%% Dictionary.
\clearpage
\def\leftmark{\firstmark\ | \botmark}
\let\rightmark\leftmark

\twocolumn[\section{Dictionary}]

\ExplSyntaxOn

\cs_new:Npn \start_entry: {
    \hangindent = 6pt
    \hangafter = 1
    \noindent
}

\def \pfabbr {{\normalfont\scshape  pf \space }}

%% Word, part of speech, etymology, definition, (forms)
\long \def \entry #1 #2 #3 #4 #5 {
    \start_entry:

    %% Typeset word and part of speech.
    \mark { #1 }
    \textbf { \ignorespaces #1 } \space
    \textit { \ignorespaces #2 }

    %% Typeset etymology.
    \tl_set:Nn \l_tmpa_tl {#3}
    \tl_if_empty:NTF \l_tmpa_tl { } {
        \space [
            \ignorespaces \textit { \tl_use:N \l_tmpa_tl }
        ]
    }

    %% Typeset forms, if any.
    \tl_set:Nn \l_tmpa_tl {#5}
    \tl_if_empty:NTF \l_tmpa_tl { } {
        %\space {\nf\scshape{forms}}:
        \space
        \ignorespaces \tl_use:N \l_tmpa_tl
        .
    }

    %% Typeset definition.
    \space \ignorespaces #4 .
    \par
}

%% Reference to another entry.
\long \def \refentry #1 #2 {
    \start_entry:

    \textbf { \ignorespaces #1 } \space
    \(\to\) \space
    \textit { \ignorespaces #2 }
    .
    \par
}

\ExplSyntaxOff

%% This is a generated file
%% Do not edit this file manually
%%
%% To update this file, edit DICTIONARY.txt and rerun
%% GENERATE-DICTIONARY.sh

\entry{aç̇t’he}{v. tr.}{\pfabbr acheter}{To buy}{}
\entry{ad’hór}{v. tr.}{\pfabbr adore}{To love, adore}{}
\entry{ánvé}{v. tr.}{\pfabbr animer}{To bring to life, animate}{}
\entry{ḅárḍáḍ}{v.}{\pfabbr partante}{ (+ \s{aci}) To be interested in, willing to, ready to, prepared for}{}
\entry{ḅẹt’hẹ}{adj.}{\pfabbr petit}{Small, little}{}
\refentry{b’heḍ}{eḍ}
\refentry{b’heḍy’é}{eḍ}
\refentry{b’hu}{eḍ}
\entry{Çár}{n.}{}{*male given name, equivalent to English ‘Kyle’ or ‘Charles’*}{}
\entry{c’hes}{part.}{\pfabbr qu'est-ce que}{*interrogative particle*}{}
\entry{c’húr}{v.}{\pfabbr court}{To shrink, reduce in size, narrow}{}
\entry{ḍalẹ}{n.}{\pfabbr tableau}{Table}{}
\refentry{ḍe}{eḍ}
\refentry{ḍet’h}{eḍ}
\entry{dír}{v. tr.}{\pfabbr dire}{To say, tell}{}
\entry{Dóvníc’h}{n.}{}{*male or female given name, equivalent to English ‘Dominic’*}{}
\refentry{ḍyf}{eḍ}
\entry{ebhẹ}{adj.}{\pfabbr épais}{Thick}{}
\refentry{éḍ}{eḍ}
\entry{eḍrrá}{adj.}{\pfabbr étroit}{Pointy}{}
\entry{eḍ}{v. irreg. }{\pfabbr être}{To be}{*active only*. \s{pres: sg} *vy’í*, *ḍe*, *le*, *lle*, *s*; \s{pl} *ósó*, *b’heḍ*, *lẹsó*, *llẹsó*, *lasó*; \s{inf} *éḍ*. \s{pres ant: sg} *vẹ*, *ḍyf*, *leb’h*, *lleb’h*, *seb’h*; \s{pl} *ófý*, *b’hu*, *lẹfýr*, *llẹfýr*, *lafýr*; \s{inf} *éfyḍ* \s{pret: sg} *vet’h*, *ḍet’h*, *let’h*, *llet’h*, *set’h*; \s{pl} *weḍy’ó*, *b’heḍy’é*, *let’he*, *llet’he*, *laet’h*; \s{inf} *ét’hẹd*}
\entry{Eḍy’ê}{n.}{}{*male given name, equivalent to English ‘Stephen’*}{}
\refentry{éfyḍ vet’h}{eḍ}
\entry{ehyó}{n.}{\pfabbr écusson}{Shield}{}
\refentry{ét’hẹd }{eḍ}
\entry{Já}{n.}{}{*male or female given name, equivalent to English ‘John’ or ‘Joan’*}{}
\refentry{laet’h}{eḍ}
\refentry{lafýr}{eḍ}
\entry{lár}{adj.}{\pfabbr large}{Wide, broad}{}
\refentry{lasó}{eḍ}
\refentry{leb’h}{eḍ}
\refentry{le}{eḍ}
\refentry{lẹfýr}{eḍ}
\refentry{lẹsó}{eḍ}
\refentry{let’h}{eḍ}
\refentry{let’he}{eḍ}
\refentry{lleb’h}{eḍ}
\refentry{lle}{eḍ}
\refentry{llẹfýr}{eḍ}
\refentry{llẹsó}{eḍ}
\refentry{llet’h}{eḍ}
\refentry{llet’he}{eḍ}
\entry{ló}{adj.}{\pfabbr long}{Long}{}
\entry{lúr}{adj.}{\pfabbr lourd}{Bulky, oversized, heavy}{}
\entry{ob’heír}{v. (in)tr.}{\pfabbr obéir}{To obey}{}
\refentry{ófý}{eḍ}
\refentry{ósó}{eḍ}
\entry{rá}{adj.}{\pfabbr grand}{Big, large, great}{}
\entry{ráhó}{n.}{\pfabbr poisson}{Fish}{}
\entry{rvá}{interj.}{of unknown origin}{Alas, woe, oh. *Exclamation of distress, surprise, sadness, or regret*}{ *after words that end with ‘r’, this is spelt ‘-vá’ instead*}
\refentry{seb’h}{eḍ}
\refentry{s}{eḍ}
\refentry{set’h}{eḍ}
\refentry{vá }{rvá}
\refentry{vẹ}{eḍ}
\entry{vôt’há}{n.}{\pfabbr montagne}{Mountain}{}
\entry{vvóríhe}{v. (in)tr.}{\pfabbr mémoriser}{To remember}{}
\refentry{vy’í}{eḍ}
\refentry{weḍy’ó}{eḍ}
\entry{y’ír}{v. (in)tr.}{\pfabbr ouïr}{To hear, understand, listen}{}






\end{document}
