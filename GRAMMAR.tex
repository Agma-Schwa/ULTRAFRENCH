\documentclass[a4paper, 12pt, oneside, final]{article}
\usepackage[margin=2cm]{geometry}
\usepackage{fontspec}
\usepackage{unicode-math}
\usepackage[english]{babel}
\usepackage{csquotes}
\usepackage{array, tabularx, multirow}
\usepackage{longtable}
\usepackage{float}

\setmainfont[Numbers=OldStyle]{Minion 3}
\setmathfont{latinmodern-math.otf}
\setmathfont[range=\mathit]{Minion 3 Italic}
\frenchspacing

\AtBeginDocument{
    \def\today{
        \number\day\space
        \ifcase\month\or
        January\or February\or March\or April\or May\or June\or
        July\or August\or September\or October\or November\or December\fi\space
        \number\year
    }
}

\makeatletter
\def\footnoterule{%
    \kern-3\p@
    \hrule\@width.4\columnwidth
    \kern2.6\p@
}

\def\@makefntext#1{%
    \setlength\parindent{1em}%
    \noindent
    {%
    \mbox{\llap{{}\textsuperscript{\@thefnmark}\kern.5pt}}{#1}%1
    }%
}
\makeatother

\title{A Comprehensive Diachronic Grammar of Modern ULTRAFRENCH}
\author{Agma Schwa \& Ætérnal}
\date{\today}

\ExplSyntaxOn
\cs_new:Npn \__two_cols:nnnnn #1 #2 #3 #4 #5 {
    \ifvmode\else\unskip\par\fi
    \noindent\leavevmode
    \hbox to \hsize {
        \hbox to #3 { \vtop {#1} }
        \hskip   #4
        \hbox to #5 { \vtop {#2} }
    }\par
}

\NewDocumentCommand \TwoCols {
    D[]{.475\hsize}
    D[]{.475\hsize}
    D[]{0pt plus 1fill}
    +m
    +m
} {
    \__two_cols:nnnnn{#4}{#5}{#1}{#3}{#2}
}

\def \UF { \bfseries \itshape }
\let \nf \normalfont

\def \d {ḍ}
\def \D {Ḍ}
\def \b {ḅ}
\def \B {Ḅ}
\let \s \textsc

\def \L {\textsuperscript{L}}
\def \N {\textsuperscript{N}}

\char_set_catcode_active:N \*
\cs_set_protected:Npn \__md_star:w #1*{ \textit{#1} }
\def\* { \detokenize{*} }
\cs_new:Npn * { \__md_star:w }

\cs_new:Npn \items { \itemize\itemsep6pt }
\cs_new:Npn \enditems { \enditemize }

\ExplSyntaxOff

\newlength{\EnumItemSep} \EnumItemSep-3pt

\newenvironment{enum}[1][0]{%
    \vspace{-.5em}%
    \settowidth{\leftmargini}{99.\hskip\labelsep}%
    \begin{enumerate}\setcounter{enumi}{#1}\itemsep\EnumItemSep
}{%
    \end{enumerate}%
    \vspace{-.5em}%
}

\def\parheading#1{\noindent\textbf{#1}}

\let\Sub\textsubscript

\begin{document}
\maketitle
\thispagestyle{empty}
\clearpage
\setcounter{page}{1}

\tableofcontents
\clearpage

\section{Phonology and Evolution from Modern Pseudo-French}\label{sec:phonology}{\def\arraystretch{1.25}\setlength{\tabcolsep}{.4em}
\noindent\begin{tabular}{@{}|l|l|l|l|l|l|l@{\quad}|l|l|l|}                                                   \cline{1-6} \cline{8-10}
               & Labial & Coronal  & Palatal  & Velar & Glottal &&           & Front        & Back        \\ \cline{1-6} \cline{8-10}
    Stop       & b, bʱ  & d        &          &       &         && Close     & i ĩ ĩ̃ i̥, y ỹ ỹ̃ ẙ & u ũ ũ̃ u̥ \\ \cline{1-6} \cline{8-10}
    Nasal      &        & n        &          &       &         && Close-mid & e ẽ ẽ̃ e̥      & o o̥         \\ \cline{1-6} \cline{8-10}
    Fricative  & ɸ β, ʋ̃ & s z, θ ð & ç ɕ ʑ    & x χ   & h       && Mid       & \multicolumn{2}{c|}{ə ⟨ẹ⟩ ə̥}   \\ \cline{1-6} \cline{8-10}
    Approx.    &        &          & ɥ ɥ̃, j̊   & ɰ ɰ̃   &         && Open-mid  & ɛ ɛ̃ ɛ̃̃ ɛ̥      & ɔ̃ ɔ̃̃         \\ \cline{1-6} \cline{8-10}
    Lat. Fric. &        & ɮ̃        & ʎ̝̃        &       &         && Open      & a ḁ          & ɑ̃ ɑ̃̃         \\ \cline{1-6} \cline{8-10}
\end{tabular}}\bigskip

\parheading{Legend}\par\noindent
Ṽ = nasalised vowel, Ṽ̃ = nasal vowel, V = any vowel (or, in conjunction with Ṽ/Ṽ̃, oral vowel)\\
N = nasal consonant, C̃ = nasalised consonant (e.g. /ɰ̃/, but not true nasals), C = any consonant.\medskip
\def\scalpha{\kern-2pt\raisebox{2pt}{\Sub α}}

%% NOTE: In case the changes below and the ones listed
%% in the Lexurgy file differ, the latter are authoritative,
%% as I may forget to update these here sometimes.

\TwoCols[.45\hsize][.45\hsize][0pt]{
\parheading{Preliminary Changes}
\begin{enum}
    \item g, ʁ, w > ɰ ⟨r⟩
    \item œ, œ̃, ø > y, ỹ, ỹ
    \item ɔ > o
    \item y > j / \_(\#)V
    \item V\scalpha > $\emptyset$ / \_\#V\scalpha
    \item lj, lɥ > ʎ
    \item j > ɥ ⟨y’⟩
    \item ɰ > ɥ / \_i
    \item C > $\emptyset$ / \#\_C
    \item C > $\emptyset$ / C\_\#
    \item k > x\footnotemark ⟨c’h⟩
    \item ʃ, ʒ > ɕ ⟨ç⟩, ʑ ⟨j⟩
    \item nt > nθ
    \item t > \d{} [d] (‘hard /d/’)
    \item p > \b{} [b] (‘hard /b/’)
    \item f, v > ɸ ⟨f⟩, β ⟨b’h⟩
\end{enum}
}{
\parheading{Great Nasal Shift}
\begin{enum}[15]
    \item Ṽl > ɰ̃ ⟨w⟩
    \item V > Ṽ̃ / [NC̃ɥɰ]\_N\#
    \item V, Ṽ > Ṽ, Ṽ̃ / \_[NC̃ɥɰ], [NC̃ɥɰ]\_
    \item ə̃, ə̃̃, ã, ã̃, õ, õ̃ > ɛ̃, ɛ̃̃, ɑ̃, ɑ̃̃, ɔ̃, ɔ̃̃
    \item N, C̃ > $\emptyset$ / V\_\#
    \item ɲ, ŋ > n
    \item V, Ṽ > $\emptyset$ / N \_ N
    \item m, l, ʎ > ʋ̃ ⟨v⟩, ɮ̃ ⟨l⟩, ʎ̝̃ ⟨ḷ⟩
\end{enum}

\parheading{Intervocalic Lenition (/ V\_V is implied)}
\begin{enum}[21]
    \item x, s, z > h\footnotemark
    \item ɕ, ɮ̃, ʎ̝̃ > j̊ ⟨ç̇⟩, ɥ̃, ɰ̃
    \item nθ > n
    \item d, \d{}, b, \b{} > ð ⟨d’h⟩, θ ⟨t’h⟩, β, bʱ ⟨bh⟩
\end{enum}

\parheading{Late Changes}
\begin{enum}[25]
    \item C[+stop, -alveolar]C\scalpha > C\scalpha
    \item h > $\emptyset$ / hV\_
    \item ə > $\emptyset$ / C\_C
    \item V[-nasalised, -nasal] > ə̥ / \_\#
\end{enum}
}\medskip

\footnotetext{[χ] around back vowels, [ɕ] elsewhere. For the purpose of sound changes, both are treated as [x].}
\footnotetext[2]{[ç] before variants of /i/ and /y/, [h] elsewhere.}

\subsection{Orthography}
The spelling of most UF sounds is indicated above; the less exotic consonants are spelt as
one might expect. That is, /b, d, n, ɸ, s, z, h/ are spelt ⟨b, d, n, f, s, z, h⟩, respectively.

The ‘hard’ voiced *ḅ*, *ḍ* which are pronounced exactly like their regular counterparts, are normally also spelt ⟨b⟩ and
⟨d⟩. However, the dot below is commonly used in dictionaries and grammatical material to distinguish between the two
as they differ from one another in how they are lenited.

%% TODO: ORTHOGRAPHY. Dot below is schwa if it’s an e, nasalised e instead of ɛ if it has an accent,
%% hard d, b, as well as palatal l. dot above ç indicates lenition.

\section{Accidence}\label{sec:accidence}
\subsection{Verbal Morphology}\label{subsec:verbal-morphology}
Verbs in UF are inflected for person, number, tense, aspect, mood, and voice. Verbal inflexion is mainly done
by means of concatenating a vast set of prefixes onto a verb, with the occasional suffix and circumfix making
its appearance. This chapter details these affixes, their meanings, uses, forms, and restrictions.


\subsubsection{Active/Passive Affixes}\label{subsubsec:active-passive-affixes}
UF has a set of active/subject as well as passive/object prefixes which can be used on their own or in combination
with one another, though at most one active and one passive prefix may be combined with a verb.\footnote{Irrespective
of whether they are personal or infinitive prefixes. For instance, it would also be illegal to combine e.g. the active
infinitive prefix with the first person active singular prefix.} Table~\ref{tab:active-passive-prefixes}
below lists those prefixes, two of which are actually circumfixes.

\begin{table}[H]
\centering
\noindent\begin{tabular}{@{}|>{}l|>{\it}l|>{\it}l|>{}l|>{}l|>{\it}l|>{\it}l|}\cline{1-3}\cline{5-7}
 Active&\nf Sg&\nf Pl& & Passive&\nf Sg&\nf Pl\\\cline{1-3}\cline{5-7}
1st&j-&ó-/r-/w- -(y’)ó&&1st&v-&ó-/r-/w-\\\cline{1-3}\cline{5-7}
2nd&\d{}(ẹ)-&b’h(y)- -(y’)é&&2nd&\d{}(ẹ)-&b’h(y)-\\\cline{1-3}\cline{5-7}
3rd m&l(ẹ)-&l(ẹ)-&&3rd m&y’-&lý-\\\cline{1-3}\cline{5-7}
3rd f&ll(a)-&ll(ẹ)-&&3rd f&y’- &lý-\\\cline{1-3}\cline{5-7}
3rd n&s- &l(a)-&&3rd n&sy-&lý-\\\cline{1-3}\cline{5-7}
Infinitive&\multicolumn{2}{c|}{\it d(ẹ)-}&&Infinitive&\multicolumn{2}{c|}{\it à-/h-}\\\cline{1-3}\cline{5-7}
\end{tabular}
\caption{Active (left) and passive (right) verbal affixes.}\label{tab:active-passive-prefixes}
\end{table}

\noindent A great degree of syncretism can be observed in the third-person forms. The gender distinction in the
\s{3sg} that diachronically resulted from gendered personal pronouns is almost non-existent in the
plural; the reason for this development is that those forms are derived from the old dative form, which lacked
this distinction altogether.

The \s{act 1pl, 2pl} forms are only distinguished from their passive counterparts by
the presence of additional suffixes in the former. The \s{3sg n} in the active and passive is derived from the PF
demonstrative \**ce* and its variants; the \s{3pl n} is derived from the other \s{3pl} forms.

The \s{1pl} prefix varies if there is a vowel following it: if it is
any vowel that is not a variant of ‘o’, the prefix is realised as *r-* instead, e.g. *ad’hór* ‘love’ to
*rad’hóró* ‘we love’. If the vowel a variant of ‘o’, the prefix is realised as *w-* instead, e.g. *ob’heír* ‘obey’
to *wob’heíró* ‘we obey’.\footnote{Diachronically, the base form of this prefix is \**o-*, whence e.g.
\**oad’hóró* > *rad’hóró* and \**oob’heíró* > *wob’heíró*.}

The \s{inf pass} prefix *à-* coalesces with any vowel following it: it becomes *á* if it
is followed by a non-nasal variant of ‘a’, e.g. *ad’hór* to *ád’hór* ‘to be loved’; *â* if it is
followed by a nasal variant of ‘a’, e.g. *ánvé* ‘give life to’ to *ânvé* ‘to be animated’; and *h-* if it is
followed by any other vowel, e.g. *ob’heír* to *hob’heír* ‘to be obeyed’.

The parenthesised vowels are used if the prefix is followed by a consonant, e.g. *dír* ‘say’ to *llẹ{}dír*
‘they (\s{f}) say’ and *b’hydíré* ‘you (\s{pl}) say’, but *ad’hór* to *llad’hór* ‘they (\s{f}) love’ and *b’had’hóré* ‘you
(\s{pl}) love’. The prefixes *ó-* and *à-* retain their main forms if followed by a consonant,
e.g. *dír* ‘say’ to *ódíró* ‘We say’ and *àdír* ‘to be said’. The exception to this is that \s{2pl} *b’h(y)-*
drops the *y* if followed by a glide, e.g. *y’ír* ‘to hear’ to *b’hy’íré* ‘you (\s{pl}) hear’ (not \**b’hyy’íré*).

The *y’* in the suffix parts of the \s{1pl, 2pl act} are dropped if the verb ends with a consonant, e.g. *ad’hór*
to *b’hád’hóré*, or if it ends with a vowel that is a variant of ‘o’ in the case of the \s{1pl} and ‘e’ in the case
of the \s{2pl}, in which cases the vowels are contracted and a level of nasalisation is added, e.g. *vvóríhe*
‘to remember’ to *b’hyvvóríhé* ‘you (\s{pl}) remember’ (not \**b’hyvvóríhy’é*). In all other cases, the *y’* is retained,
e.g. *óvvóríhey’ó* ‘we remember’.

When multiple prefixes are used together, active prefixes precede passive prefixes, except that infinitive prefixes
always come first, e.g. *ad’hór* ‘love’ to *jvad’hór* ‘I love myself’ (not \**vjad’hór*) and *b’hy’ad’hóré* ‘you (\s{pl}) love him/her’,
but *dẹvad’hór* ‘to love me’ and *àb’had’hóré* ‘to be loved by you (\s{pl})’. Recall that at most one infinitive prefix
may be used.

By way of illustration, consider the paradigm of the verb *ad’hór* as shown in Table~\ref{tab:adhor-paradigm} below.
Since this word starts with a vowel, the parenthesised vowels in Table~\ref{tab:active-passive-prefixes} above
are not used. Furthermore, since it starts with a non-nasal ‘a’-like vowel, the *ó-* prefix is realised as *r-*
and the *à-* prefix coalesces with the initial ‘a’ of the stem to form *á*.

% TEMPLATE:
%\noindent\begin{tabular}{@{}|>{}l|>{\it}l|>{\it}l|>{}l|>{}l|>{\it}l|>{\it}l|}\cline{1-3}\cline{5-7}
%\nf Active&\nf Sg&\nf Pl&\nf &\nf Passive&\nf Sg&\nf Pl \\\cline{1-3}\cline{5-7}
%1st       &   &  &&1st   &   &   \\\cline{1-3}\cline{5-7}
%2nd       &   &  &&2nd   &   &   \\\cline{1-3}\cline{5-7}
%3rd m     &   &  &&3rd m &   &   \\\cline{1-3}\cline{5-7}
%3rd f     &   &  &&3rd f &   &   \\\cline{1-3}\cline{5-7}
%3rd n     &   &  &&3rd n &   &   \\\cline{1-3}\cline{5-7}
%Infinitive& \multicolumn{2}{c|}{\it }  && Infinitive & \multicolumn{2}{c|}{\it } \\\cline{1-3}\cline{5-7}
%\end{tabular}

\begin{table}[H]
\centering
\noindent\begin{tabular}{@{}|>{}l|>{\it}l|>{\it}l|>{}l|>{}l|>{\it}l|>{\it}l|}\cline{1-3}\cline{5-7}
\nf Active&\nf Sg&\nf Pl&\nf &\nf Passive&\nf Sg&\nf Pl\\\cline{1-3}\cline{5-7}
1st&jad’hór&rad’hóró&&1st&vad’hór&rad’hór\\\cline{1-3}\cline{5-7}
2nd&\d{}ad’hór&b’had’hóré&&2nd&\d{}ad’hór&b’had’hór\\\cline{1-3}\cline{5-7}
3rd m&lad’hór&lad’hór&&3rd m&y’ad’hór&lýad’hór\\\cline{1-3}\cline{5-7}
3rd f&llad’hór&llad’hór&&3rd f&y’ad’hór &lýad’hór\\\cline{1-3}\cline{5-7}
3rd n&ý’ad’hór&lad’hór&&3rd n&ý’ad’hór&lýad’hór\\\cline{1-3}\cline{5-7}
Infinitive&\multicolumn{2}{c|}{\it dad’hór}&&Infinitive&\multicolumn{2}{c|}{\it ád’hór}\\\cline{1-3}\cline{5-7}
\end{tabular}
\caption{Paradigm of the Verb \emph{ad’hór}.}\label{tab:adhor-paradigm}
\end{table}

\noindent For comparison, the paradigm of the verb *vvóríhe* ‘remember’ is shown in Table~\ref{tab:vvorihe-paradigm} below.
Since it starts with a consonant, the parenthesised vowels in Table~\ref{tab:active-passive-prefixes} are used, and any
prefixes that end with a vowel remain unchanged.

\begin{table}[H]
\centering
\noindent\begin{tabular}{@{}|>{}l|>{\it}l|>{\it}l|>{}l|>{}l|>{\it}l|>{\it}l|}\cline{1-3}\cline{5-7}
\nf Active&\nf Sg&\nf Pl&\nf &\nf Passive&\nf Sg&\nf Pl\\\cline{1-3}\cline{5-7}
1st&jvvóríhe&óvvóríhey’ó&&1st&vvvóríhe&óvvóríhe\\\cline{1-3}\cline{5-7}
2nd&ḍẹvvóríhe&b’hyvvóríhé&&2nd&ḍẹvvóríhe&b’hyvvóríhe\\\cline{1-3}\cline{5-7}
3rd m&lẹvvóríhe&lẹvvóríhe&&3rd m&y’vvóríhe&lývvóríhe\\\cline{1-3}\cline{5-7}
3rd f&llavvóríhe&llẹvvóríhe&&3rd f&y’vvóríhe&lývvóríhe\\\cline{1-3}\cline{5-7}
3rd n&ý’vvóríhe&lavvóríhe&&3rd n&ý’vvóríhe&lývvóríhe\\\cline{1-3}\cline{5-7}
Infinitive&\multicolumn{2}{c|}{\it dẹvvóríhe}&&Infinitive&\multicolumn{2}{c|}{\it àvvóríhe}\\\cline{1-3}\cline{5-7}
\end{tabular}
\caption{Paradigm of the Verb \emph{vvóríhe}.}\label{tab:vvorihe-paradigm}
\end{table}

\subsection{Tense and Aspect Marking}\label{subsec:tense-and-aspect-marking}
Tense in PF is marked by additional sets of affixes that are appended to the verb in addition to the active/passive affixes.
There are two broad groups of such affixes: suffixes, which are appended to the end of the verb and replace the \s{act 1pl, 2pl} suffixes
in those persons, as well as circumfixes and prefixes, which are inserted before the active/passive markers and replace the
replace the \s{act 1pl, 2pl} suffixes in some cases.


\subsubsection{Suffixed Tenses}
The present anterior and preterite are formed by appending a set of suffixes to the verb. Table~\ref{tab:present-anterior-and-preterite-suffixes}
below lists the suffixes for those tenses. The present anterior has a perfective aspect, while the preterite has an imperfective aspect. The
former is commonly used to describe events that are completed—particularly events that occurred recently, hence the name—while the latter
is used to describe events that are ongoing or habitual.


\begin{table}[H]
\centering
\noindent\begin{tabular}{@{}|>{}l|>{\it}l|>{\it}l|>{}l|>{}l|>{\it}l|>{\it}l|}\cline{1-3}\cline{5-7}
\nf Present Anterior&\nf Sg&\nf Pl&\nf &\nf Preterite&\nf Sg&\nf Pl \\\cline{1-3}\cline{5-7}
1st       & -\L é & -\L â &&1st    & -\L á  & -y’ô  \\\cline{1-3}\cline{5-7}
2nd       & -\L á & -\L áḍ &&2nd   & -\L é  & -y’ẹ́  \\\cline{1-3}\cline{5-7}
3rd       & -\L á & -\L ér &&3rd m & -\L é  & -\L é   \\\cline{1-3}\cline{5-7}
Infinitive& \multicolumn{2}{c|}{\it -á }  && Infinitive & \multicolumn{2}{c|}{\it -é } \\\cline{1-3}\cline{5-7}
\end{tabular}
\caption{Present Anterior and Preterite Affixes.}\label{tab:present-anterior-and-preterite-suffixes}
\end{table}

\noindent Neither tense distinguishes gender in the third person. All suffixes, except for the infinitive and \s{1pl, 2pl pret},
lenite any consonant *before* them, e.g. *ḅárḍáḍ* ‘to be willing’ to *jḅárḍát’hé* ‘I was willing’ but *dẹḅárḍáḍá*
‘to have been willing’.

Diachronically, the \s{1sg pret} is an interesting case; in EUF, it was originally \**-é*, but it later changed to *-á*
to distinguish it from the \s{2sg, 3sg pres ant}. The remaining forms—save the infinitives, which are derived from the
tenses’ definite endings by analogy—originated from the PF simple past tenses.

The table below lists the example paradigm of the verb *ad’hór* in the present anterior and preterite tenses.
Observe that there is no difference between the \s{1pl, 2pl} active and passive.

\begin{table}[H]
\centering
\noindent\begin{tabular}{@{}|>{}l|>{\it}l|>{\it}l|>{}l|>{}l|>{\it}l|>{\it}l|}\cline{1-3}\cline{5-7}
\nf Active & \nf Sg   & \nf Pl     & \nf & \nf Passive & \nf Sg   & \nf Pl    \\\cline{1-3}\cline{5-7}
1st        & jad’hóré  & rad’hórâ     &     & 1st         & vad’hóré  & rad’hórâ   \\\cline{1-3}\cline{5-7}
2nd        & ḍad’hórá  & b’had’hóráḍ  &     & 2nd         & ḍad’hórá  & b’had’hóráḍ \\\cline{1-3}\cline{5-7}
3rd m      & lad’hórá  & lad’hórér    &     & 3rd m       & y’ad’hórá & lýad’hórér  \\\cline{1-3}\cline{5-7}
3rd f      & llad’hórá & llad’hórér   &     & 3rd f       & y’ad’hórá & lýad’hórér  \\\cline{1-3}\cline{5-7}
3rd n      & ý’ad’hórá & lad’hórér    &     & 3rd n       & ý’ad’hórá & lýad’hórér  \\\cline{1-3}\cline{5-7}
Infinitive & \multicolumn{2}{c|}{\it dad’hórá} & & Infinitive & \multicolumn{2}{c|}{\it ád’hórá} \\\cline{1-3}\cline{5-7}
\end{tabular}
\caption{Present Anterior Paradigm of the Verb \emph{ad’hór}.}\label{tab:adhor-paradigm-pres-ant}
\end{table}

\begin{table}[H]
\centering
\noindent\begin{tabular}{@{}|>{}l|>{\it}l|>{\it}l|>{}l|>{}l|>{\it}l|>{\it}l|}\cline{1-3}\cline{5-7}
\nf Active & \nf Sg   & \nf Pl     & \nf & \nf Passive & \nf Sg   & \nf Pl    \\\cline{1-3}\cline{5-7}
1st        & jad’hórá  & rad’hóry’ô     &     & 1st     & vad’hórá  & rad’hóry’ô   \\\cline{1-3}\cline{5-7}
2nd        & ḍad’hóré  & b’had’hóry’ẹ́  &     & 2nd      & ḍad’hóré  & b’had’hóry’ẹ́ \\\cline{1-3}\cline{5-7}
3rd m      & lad’hóré  & lad’hóré    &     & 3rd m      & y’ad’hóré & lýad’hóré  \\\cline{1-3}\cline{5-7}
3rd f      & llad’hóré & llad’hóré   &     & 3rd f      & y’ad’hóré & lýad’hóré  \\\cline{1-3}\cline{5-7}
3rd n      & ý’ad’hóré & lad’hóré   &     & 3rd n       & ý’ad’hóré & lýad’hóré  \\\cline{1-3}\cline{5-7}
Infinitive & \multicolumn{2}{c|}{\it dad’hóré} & & Infinitive & \multicolumn{2}{c|}{\it ád’hóré} \\\cline{1-3}\cline{5-7}
\end{tabular}
\caption{Preterite Paradigm of the Verb \emph{ad’hór}.}\label{tab:adhor-paradigm-pret}
\end{table}


\subsection{Irregular Verbs}\label{subsec:irregular-verbs}
\subsubsection{The Conjugation of \textit{eḍ} ‘to be’}

\begin{table}[H]
\centering
\noindent\begin{tabular}{@{}|>{}l|>{\it}l|>{\it}l|}\hline
&\nf Sg&\nf Pl\\\hline
1st       & vy’í & ósó \\\hline
2nd       & ḍe   & b’heḍ \\\hline
3rd m     & le   & lẹsó \\\hline
3rd f     & lle  & llẹsó \\\hline
3rd n     & ý’e  & lasó \\\hline
Infinitive& \multicolumn{2}{c|}{\it éḍ} \\\hline
\end{tabular}
\caption{Paradigm of the verb \emph{eḍ}.}\label{tab:ed-paradigm}
\end{table}

\subsection{Noun Morphology}\label{subsec:noun-morphology}
UF has 4 declensions. A definite and indefinite vocalic declension, and a definite and indefinite consonantal declension.
As their names might suggest, the former two are used for nouns that start with a vowel, and the latter two for nouns
that start with a consonant. UF has no morphologically separate articles; rather, the old PF articles have been incorporated
into the declensions. Furthermore, UF no longer has a gender distinction in nouns.

\subsubsection{Declension}
The table below shows the affixes of the definite and indefinite declensions. The declensions are mostly identical,
except that, as with the conjugation of verbs, the consonantal prefixes often end in a vowel (marked below with
parentheses), which are not present in the vocalic declension.

\begin{table}[H]
\centering
\noindent\begin{tabular}{@{}|l|>{\it}l|>{\it}l|l|l|>{\it}l|>{\it}l|}\cline{1-3}\cline{5-7}
Definite    &\nf Sg&\nf Pl && Indefinite       &\nf Sg&\nf Pl\\\cline{1-3}\cline{5-7}

Nominative  & lá-\L & lé-\L                    &&Nominative &ŷn-\N & ý-\L         \\\cline{1-3}\cline{5-7}
Vocative    & $\emptyset$-\L & $\emptyset$-\L  &&Vocative   & / & /                     \\\cline{1-3}\cline{5-7}
Partitive   & dy-\L  & dẹ-\L                   &&Partitive  &dŷn-\N & dý-\L                      \\\cline{1-3}\cline{5-7}
Accusative  & y’i-\L  & sý-\L                  &&Accusative & s-\L & s-                       \\\cline{1-3}\cline{5-7}
...         &  &                               &&           & &              \\\cline{1-3}\cline{5-7}
Inessive    & dwá- & dwé-                      &&Inessive   & dáhŷn- & dáhŷ-                    \\\cline{1-3}\cline{5-7}
\end{tabular}
\caption{UF Declension.}\label{tab:table-uf-declension}
\end{table}

\noindent Most of these forms cause lenition in the initial consonant of the noun, e.g. *ḍalẹ* ‘table’ to
\s{def acc sg} *s’thalẹ*; this lenition is blocked in the \s{indef acc pl} due to the presence of a hypercorrected ‘s’
in PF \**ces*, e.g. *s’ḍalẹ* ‘the tables (\s{acc})’ (not *s’thalẹ*, which is the singular), as well as in
less commonly used forms such as the \s{def} inessive *dwáḍalẹ* ‘on the table’.

The \s{indef nom sg} *ŷn-* prefix and some other forms nasalise nouns; as a reminder, this means that in
nouns starting with *ḍ*, the *ḍ* is deleted, e.g. *ŷnalẹ* ‘a table’;
it causes nasalisation in words that start with a vowel e.g. *ehyó* ‘shield’ to *ŷnéhyó* ‘a shield’. The indefinite vocative
does not exist, as that would make little sense. As lenition, nasalisation too is blocked in rarer forms, e.g. \s{indef} inessive
*dáhŷnḍalẹ* ‘on a table’.

The diachrony of these forms is mostly from the PF definite and indefinite pronouns, though some forms, such as the
accusative, are borrowed from demonstratives instead (\s{def} from PF \**celui* and \s{indef} from PF \**ce*); the definite
partitive forms are from the PF partitive article, and
the indefinite forms are formed with an additional *d-* by analogy to the definite forms. The locative cases are combinations
of the articles and PF prepositions.

\begin{table}[H]
\centering
\noindent\begin{tabular}{@{}|l|>{\it}l|>{\it}l|l|l|>{\it}l|>{\it}l|}\cline{1-3}\cline{5-7}
Definite    &\nf Sg&\nf Pl && Indefinite&\nf Sg&\nf Pl\\\cline{1-3}\cline{5-7}

Nominative  & lát’halẹ  & lét’halẹ   &&Nominative & ŷnalẹ & ýt’halẹ         \\\cline{1-3}\cline{5-7}
Vocative    & t’halẹ    & t’halẹ    &&Vocative   & / & /                      \\\cline{1-3}\cline{5-7}
Partitive   & dyt’halẹ  & dẹt’halẹ &&Partitive   & dŷnalẹ & dýt’halẹ                     \\\cline{1-3}\cline{5-7}
Accusative  & y’it’halẹ & sýt’halẹ  &&Accusative & st’halẹ & sḍalẹ                       \\\cline{1-3}\cline{5-7}
...         &  &  &&           & &              \\\cline{1-3}\cline{5-7}
Inessive    & dwáḍalẹ & dwéḍalẹ &&Inessive   & dáhŷnḍalẹ & dáhýḍalẹ                    \\\cline{1-3}\cline{5-7}
\end{tabular}
\caption{Consonantal declension of *ḍalẹ*.}\label{tab:vocalic-declension}
\end{table}



\section{Examples}
\noindent\begin{tabular}{@{}lllll}
\multicolumn{5}{@{}l}{\it Çár-vá, sráhó dwávôt’há daçt’heá?}\\
Ç̇ár &-vá &s-ráhó &dwá-vôt’há &ḍ-aç̇t’he-á\\
ˈj̊ɑ̃ɰ&ɰʋ̃ɑ̃&ˌsɰɑ̃ˈhɔ̃&dɰɑ̃ˌʋ̃ɔ̃̃ˈθɑ̃&daj̊ˈθe.ɑ̃\\
Charles.\s{voc}&\s{particle}&\s{indef.acc}-fish&\s{def.iness}-mountain&\s{2sg.act}-buy-\s{pres.ant.2sg}\\
\multicolumn{5}{@{}l}{‘Charles, you bought a fish on the mountain?’}\\
\end{tabular}




\twocolumn
\clearpage
\section{Dictionary}
\ExplSyntaxOn

\cs_new:Npn \start_entry: {
    \hangindent = 6pt
    \hangafter = 1
    \noindent
}

\def \pfabbr { \textsc { pf \space } }

%% word, part of speech, etymology, definition, (forms)
\def \entry #1 #2 #3 #4 #5 {
    \start_entry:

    %% Typeset word and part of speech.
    \textbf { \ignorespaces #1 } \space
    \textit { \ignorespaces #2 }

    %% Typeset etymology.
    \tl_set:Nn \l_tmpa_tl {#3}
    \tl_if_empty:NTF \l_tmpa_tl { } {
        \space [
            \ignorespaces \textit { \tl_use:N \l_tmpa_tl }
        ]
    }

    %% Typeset forms, if any.
    \tl_set:Nn \l_tmpa_tl {#5}
    \tl_if_empty:NTF \l_tmpa_tl { } {
        \space {\nf\scshape{forms}}: \space
        \ignorespaces \tl_use:N \l_tmpa_tl
        .
    }

    %% Typeset definition.
    \space \ignorespaces #4 .
    \par
}

%% Reference to another entry.
\def \refentry #1 #2 {
    \start_entry:

    \textbf { \ignorespaces #1 } \space
    \(\to\) \space
    \textit { \ignorespaces #2 }
    .
    \par
}

\ExplSyntaxOff

%%%%%%%%%%%%%%%%%%%%%%%%%%%%%%%%%%%%%%%%%%%%%%%%%%%%%%%%%%%%%%%%%%%%%%%%
%%            This file was generated from DICTIONARY.txt             %%
%%                                                                    %%
%%                         DO NOT EDIT                                %%
%%%%%%%%%%%%%%%%%%%%%%%%%%%%%%%%%%%%%%%%%%%%%%%%%%%%%%%%%%%%%%%%%%%%%%%%

\entry{a}{pron.}{\pf{quoi}}{\textit{Interrogative and relative}.\\\s{indef} What?\\\s{def} Who? Whom?\\\s{indef} \textit{or} \s{def} Which, who, that \textit{(see grammar)}.}{}
\entry{á}{n.}{\pf{âme}}{Spirit.}{}
\entry{aḅ}{v.}{\pf{appeler}}{To call (+\s{acc} sbd./sth.) (+\s{abs} sbd./sth.). \textit{In PF, this verb used to take a double accusative, but this usage disappeared early on in UF, with the second accusative naturally being replaced by the absolutive, likely to avoid ambiguity that was starting to manifest as a result of UF’s increasingly free word order.} \ex \s{Snet’h v.2} \w{jdap rác’hsaý’adâ} ‘I call you a liar’; even in the writings of \s{Snet’h}, the double accusative is no longer attested.}{}
\entry{ábhec}{v.}{\pf{empêcher}}{+\s{acc} To prevent, stop (sth. from happening).}{\s{fut} ábhece, \s{subj} ábhecs}
\entry{abhérś}{v.}{\pf{apercevoir}}{To behold, descry (+\s{part}).}{}
\entry{aḅrâ}{v.}{\pf{apprendre}}{To learn.}{\s{fut} aḅrâdé, \s{subj} aḅrâs}
\entry{aḅraúc̣}{v.}{\pf{approcher}}{To approach, come near, walk up to (+\s{all} sbd./sth.).}{\s{fut} aḅraúc̣é, \s{subj} aḅraúc̣s}
\entry{aḅrdvê}{adv.}{\pf{après-demain}}{The day after tomorrow. \textit{The prefix \w{aḅr} can be prepended as often as necessary, e.g. \w{aḅraḅraḅrdvê} would be ‘in four days’}.}{}
\entry{ab’há}{conj.}{\pf{avant que}}{+\s{opt} Before.}{}
\entry{áb’há}{n.}{\pf{enfant}}{Child.}{}
\entry{ab’haḍ}{v.}{\pf{abattre}}{\\To cut down, fell, knock down, shoot down.\\To butcher, cut apart violently.}{\s{fut} ab’haḍrẹ́, \s{subj} ab’has}
\entry{ab’hásy’ô}{n.}{\pf{aviation}}{Aviation.}{}
\entry{ab’hèc’h}{v.}{\pf{affecter}}{+\s{acc} To affect, influence.}{\s{fut} ab’hèc’hre, \s{subj} ab’hè\-c’hes}
\entry{áb’hẹḍ}{v.}{\pf{embêtter}}{\\+\s{acc} To disturb, inconvenience sbd.\\+\s{part} To harass, bother sbd.}{}
\entry{ab’héy’}{n.}{\pf{abeille}}{Bee.}{}
\entry{ab’hínéb’heḅaý’évrâ}{v.}{\pf{habit ne fait pas le moi\-ne}}{To judge based on appearances.}{\s{fut} ab’hínéb’heḅaý’év́ẹ́, \s{subj} ab’hínéb’heḅaý’\-év́\-ás}
\entry{áb’hóhẹ}{v.}{\pf{enfoncer}}{To push, press, shove, drive (+\s{ill} into).}{}
\entry{ac}{n.}{\pf{hache}}{Axe, hatchet.}{}
\refentry{ach’es}{\w{a} + \w{c’hes}}
\entry{act’he}{v. tr.}{from \w{ac}}{\\To cut or cleave with an axe.\\+\s{acc} To bring an end to.\\+\s{acc def} \textit{of \w{árb} intr. (other than literal)} To get to the point, cut to the chase.\\+\s{acc def} \textit{of \w{árb} and \s{acc}} To bring to light, reveal. \textit{Originally, this idiom did not take a double \s{acc}, but was instead formed with the \s{acc} of ‘tree’ and the \s{ill} of the object, meaning something along the lines of ‘to bring down the tree(s) on sth’—the image here being that of cutting down trees in a wood until only a clearing remains or is ‘brought to light’}.}{\s{fut} acḍe, \s{subj} act’hes}
\entry{ac̣t’he}{v. tr.}{\pf{acheter}}{To buy.}{\s{fut} ac̣ḍrẹ́, \s{subj} ac̣t’hes}
\entry{aḍrá}{v.}{\pf{attraper}}{\\+\s{acc} \textit{or} \s{part} To take.\\\w{aḍrá faúr} \textit{intr.} To take shape, take form.}{}
\entry{ádróid}{n.}{\pf{androïde}}{Android.}{}
\entry{ady’ŷ}{v. or interj.}{\pf{adieu}}{\\Goodbye, farewell.\\+\s{gen} To say goodbye to sbd., bid sbd. farewell.}{}
\entry{ad’he}{v.}{\pf{vader}}{To go.}{\s{fut} í, \s{subj} al}
\entry{ad’hór}{v. tr.}{\pf{adore}}{\\To love, adore.\\+\s{part} To be in love with, have a crush on.\\+\s{gen} To desire, yearn for sbd./sth.\ex \s{Snet’h iv.17} \w{jad’hóré ávvaúríhe} ‘I yearned to remember’ \textit{(compare \w{jad’hóré devvaúríhe} ‘I loved to remember’)}.}{\s{fut} ad’hórérẹ́, \s{subj} ad’hórs}
\entry{ad’hyl}{v.}{\pf{adulte}}{To be adult, grown-up.}{\s{fut} ad’hyle, \s{subj} ad’hyls}
\entry{ád’hýr}{v.}{\pf{endure}}{To resist, endure, withstand (+\s{acc} sth.).}{}
\entry{áẹ}{n.}{\pf{en-haut}}{Sky. \textit{Often plural, especially in a religious sense.}.}{}
\entry{ah}{n.}{\pf{assez}}{\textit{sufficient comparative prefix; see §~\ref{subsubsec:comparison}}.}{}
\entry{áhaúr}{conj.}{\pf{encore}}{+\s{subj} Even though.}{}
\entry{áhaúr}{adv.}{\pf{encore}}{\\Still, yet \textit{(positive context)}.\\Again \textit{(negative context)}.}{}
\entry{áhâłát’hẹ}{v.}{\pf{ensanglanté}}{To be (very) bloody, bloodstained.}{}
\entry{ahúr}{v.}{\pf{assurer}}{To ensure.}{\s{fut} ahúré, \s{subj} ahúrs}
\entry{aír}{v.}{\pf{hair}}{To hate, abhor, detest, loathe, despise (+\s{acc} sbd./sth.).}{}
\entry{ânb’hé}{adv.}{\pf{en effet}, via metathesis from *\w{âné\-b’he}}{Verily, indeed, in fact.}{}
\entry{ánvé}{v. tr.}{\pf{animer}}{To bring to life, animate.}{}
\entry{árb}{n.}{\pf{arbre}}{Tree.}{}
\entry{árḍihyl}{n.}{\pf{particule}}{Particle.}{}
\entry{áríb’h}{v.}{\pf{arriver}}{To arrive.}{}
\entry{ársl}{v.}{\pf{harceler}}{To attack, assail, beset, bully (+\s{acc} sbd.).}{}
\entry{ârýý’}{v.}{\pf{enrouler}}{To wrap (+\s{acc} around sth.).}{}
\entry{ásy’ê}{v.}{\pf{ancien}}{To be ancient.}{\s{fut} ásy’êr, \s{subj} ásy’ês}
\entry{asý’ýâ}{particle}{\pf{pas absolument}}{Not, no. \textit{Commonly \w{’sý’ýâ} after vowels and verbs. This particle is used only in the indicative; see also \w{sá}, \w{t’hé}}.}{}
\entry{át’hád}{v.}{\pf{entendre}}{To hear, perceive (+\s{part} sbd./sth.).}{\s{fut} át’hádé, \s{subj} át’hás}
\entry{át’has}{v.}{\pf{entasser}}{\\\s{+acc} To heap, accumulate.\\\textit{refl.} To pile up, heap.}{}
\entry{át’hér}{v.}{\pf{enterrer}}{+\s{acc} To bury, inter.}{}
\entry{au}{conj.}{\pf{aussi}}{\\And, also, as well, too.\\\w{au} \ldots{} \w{au} \ldots{} ‘both \ldots{} and \ldots’}{}
\entry{aú}{n.}{\pf{homme}}{Man, human.}{}
\entry{aû}{particle}{\pf{non}}{Not-. \textit{Used to negate nouns, adjectives, and adverbs; see §~\ref{subsubsec:noun-negation}}.}{}
\refentry{aubhaus}{ní}
\entry{aublit’hér}{v.}{\pf{oblitérer}}{\\To defeat, vanquish, obliterate (+\s{acc} sbd./sth.).\\To be better than, ‘beat’ (+\s{inf} sbd/sth.).}{}
\entry{aub’heír}{v. (in)tr.}{\pf{obéir}}{To obey.}{}
\entry{auḍ}{adj.}{\pf{autre}}{Other, another.}{}
\entry{auḍé}{v.}{\pf{obtenir}}{\\To obtain, get, acquire.\\+\s{abl} To gain purchase on or hei\-ght or distance from.}{\s{fut} auḍy’édrẹ́}
\entry{auha}{conj.}{\pf{au cas où}}{+\s{opt} In case.}{}
\entry{aujúrdy’í}{adv.}{\pf{aujourd’hui}}{Today. \textit{Archaic, see also \w{júrdy’í}}.}{}
\entry{aúráj}{n.}{\pf{orage}}{\\\textit{(usually pl.)} Storm, tempest, thunderstorm.\ex \s{Snet’h ii.7} \w{phárýaúráj téríbâ} ‘like a terrible storm’.\\\textit{fig.} Upheaval, turmoil, crisis.}{}
\entry{ausc’hýr}{v.}{\pf{obscur}}{To be dark.}{}
\entry{av́ár}{v. irreg.}{\pf{avoir}}{+\s{acc} To have \textit{(usually inalienably)}.}{\s{pres ant} and \s{pret} y, \textit{obsolete} \s{pret} ab’hẹ, \s{fut} aúrẹ́, \s{subj} ès}
\entry{av́árḷý}{v.}{\pf{avoir lieu}}{To take place, happen.}{\s{fut} lav́árḷýé, \s{subj} lav́árḷýs}
\entry{áv́áy’é}{v.}{\pf{envoyer}}{To send.}{\s{fut} áv́áy’érẹ́, \s{subj} áv́áy’és}
\entry{áví}{n.}{\pf{ami}}{Friend.}{}
\entry{ávrê}{conj.}{\pf{à moins que}}{+\s{opt} Unless.}{}
\entry{aý’aúr}{conj.}{\pf{alors}}{While, as (temporal), because.}{}
\entry{Aý’èc’hsád}{n.}{\pf{Alexandre}}{\textit{Male given name}.}{}
\entry{áł}{v.}{from earlier *\w{ḅał} < \pf{parler}}{To speak.}{}
\entry{áȷ́éd}{v.}{\pf{enjoindre}}{To order, enjoin, command.}{}
\entry{ba}{v.}{\pf{baser}}{To base on, found on.}{\s{fut} bare, \s{subj} bas}
\entry{ḅá nórávíc’h}{n. archaic}{\pf{Panoramix}}{Druid. \textit{Only the \w{nórávíc’h} is inflected; infixing of adj. is attested.} \ex \s{Snet’h}, \s{iii.2}: \w{derúb’h phá ráinórávíc’h} ‘to find the great druid’, with infixed \w{rá}.}{}
\entry{baḍ}{v.}{\pf{battre}}{To beat, strike, hit (+\s{acc} sbd./sth.).}{}
\entry{ḅáhẹ}{n.}{\pf{pensée}}{Thought, reflection, meditation, faculty of thinking.}{}
\entry{ḅaj}{n.}{\pf{page}}{Page.}{}
\entry{ḅará}{n.}{\pf{parent}}{Parent.}{}
\entry{ḅarḍ}{v.}{\pf{partir}}{To leave, go away, depart.}{\s{fut} ḅarẹ́, \s{subj} ḅars}
\entry{ḅárḍáḍ}{v.}{\pf{partante}}{(+ \s{aci}) To be interested in, willing to, ready to, prepared for.}{}
\entry{ḅárḍẹ}{n.}{\pf{partie}}{Part, portion, piece, faction of a whole.}{}
\entry{ḅárḍihibhá}{n.}{\pf{participant}}{Participant.}{}
\entry{ḅáréḍ}{v.}{\pf{parraitre}; future stem from \pf{sembler}}{To seem, appear.}{\s{fut} sáb}
\entry{ḅas}{conj.}{\pf{parce que}}{+\s{subj} Because. \textit{Often used to explain motivation rather than cause, as in e.g. ‘We did that because\ldots’}.}{}
\entry{baú}{v. irreg.}{\pf{bon}}{\\To be good, well, healthy.\\To be right, correct, appropriate.\\\textit{usually intr.} To satisfy, fullfill, gratify.}{\s{fut} baúré, \s{subj} véy’ýrs; \s{comp} lẹvéy’ýr, y’ŷvéy’ýr, rêvéy’ýr; \s{sup} révéy’ýr, râdvâv\-éy’\-ýr}
\entry{ḅaú}{n.}{\pf{pont}}{Bridge.}{}
\entry{ḅauheŷnlabhé}{v.}{\pf{poser un lapin}}{To forsake, abandon.}{\s{fut} ḅauheŷnlabhére, \s{subj} ḅauheŷnlabhés}
\entry{ḅauhib}{v.}{\pf{impossible}}{To be impossible, unfeasible.}{\s{fut} ḅauhibre, \s{subj} ḅauh\-ibes}
\entry{Baúré}{n.}{\pf{Borée}}{Boreas, the North Wind.}{}
\entry{ḅáł}{v.}{\pf{parler}}{To speak, talk.}{\s{fut} báłérẹ́}
\entry{ḅáłýr}{n.}{\pf{parleur}}{Speaker, interlocutor.}{}
\entry{ḅelbec}{n.}{\pf{pelle} + \pf{bêche}}{Shovel.}{}
\entry{Bèrḍrá}{n.}{\pf{Bertrand}}{\textit{Male given name}.}{}
\entry{ḅẹt’hẹ}{v. irreg.}{\pf{petit}}{To be small, little.}{\s{fut} rêdẹ́, \s{subj} ḅẹt’hes; \s{comp} lẹrêd, y’ŷrêd, rêrêd; \s{sup} rérêd, râdvârêd}
\entry{ḅét’hýr}{v.}{\pf{peinture}}{To paint.}{}
\entry{ḅéy’í}{n.}{\pf{pays}}{Country, land, region, nation.}{}
\entry{ḅínár}{n.}{\pf{pinard}}{Wine.}{}
\refentry{bír}{vaúb’hẹ}
\entry{biwaú}{n.}{\pf{billion}}{\textit{(obsolete)} Billion (long scale, i.e. $10^{12}$). \textit{Replaced with modern \w{dýwaú})}.}{}
\entry{ḅré}{conj.}{\pf{après que}}{+\s{opt} After.}{}
\entry{ḅusy’ér}{n.}{\pf{poussière}}{Dust.}{}
\entry{bźé}{v.}{\pf{besoin}}{+\s{acc} \textit{or} \s{part} To need, require.}{}
\entry{b’há}{n.}{\pf{vent}}{Wind, breeze.}{}
\entry{b’hár}{n.}{\pf{vague}}{\\Wave.\\\textit{pl.} Ripples, undulations.}{}
\entry{b’hát’hiý’at’hýr}{v.}{\pf{ventilateur}}{To blow.}{}
\entry{b’hauḍ}{v.}{\pf{vôtre}}{To be yours (\s{pl}).}{\s{fut} b’hauḍre, \s{subj} b’haus}
\entry{b’haul}{v.}{\pf{voler}}{To hover, float.}{}
\entry{b’hây’ér}{adv.}{\pf{avant-hier}}{The day before yesterday. \textit{The prefix \w{b’hâ} can be prepended as often as necessary, e.g. \w{b’hâb’hâb’hây’ér} would be ‘four days ago’}.}{}
\entry{b’he}{conj.}{\pf{envers}}{+\s{subj} So that, so as to, to, in order to. \textit{Commonly enclitic \w{’b’h} after vowels}.}{}
\entry{b’hé}{n.}{\pf{vin}}{Grape.}{}
\entry{b’hénvâ}{n.}{\pf{évènement}}{Event, occurrence.}{}
\entry{b’hérḍy’ŷ}{v.}{\pf{vertueux}}{To be virtuous.}{}
\entry{b’hért’he}{n.}{from \pf{vérité}; the /i/ disappeared during Early Middle UF}{Truth.}{}
\entry{b’héy’}{v.}{\pf{veiller}}{\\\textit{intr.} To keep watch, keep guard.\\+\s{spress} To watch over, guard, keep an eye on.}{}
\entry{b’heý’au}{n.}{from archaic \w{b’heý’auhic’h}}{Bicycle.}{}
\entry{b’heý’auhic’h}{n. archaic}{\pf{vélo-cycle}}{Bicycle.}{}
\entry{b’heý’o}{v.}{back-formation from *\w{b’heý’os}, reanalysed as a subjunctive; from \pf{véloce}}{To be quick, fast.}{\s{fut} b’heý’o\-se, \s{subj} b’heý’os}
\entry{b’hí}{n.}{\pf{vigne}}{Vine.}{}
\entry{b’hic’hḍrár}{n.}{\pf{victoire}}{Victory.}{}
\entry{b’hid}{v.}{\pf{vide}}{To be empty.}{}
\entry{b’hizy’ô}{n.}{\pf{vision}}{Vision.}{}
\entry{b’hóy’ẹ}{v.}{\pf{voler}}{To fly. Flight.}{}
\entry{b’huḍ}{n.}{\pf{voûte}}{Vault, arched ceiling.}{}
\entry{b’hýlnẹ́r}{v.}{\pf{invulnérable}}{+\s{instr} To be incapable of being af\-fec\-ted by, invulnerable to.}{\s{fut} b’hýlnẹ́rẹ́, \s{subj} b’hýlnẹ́rs}
\entry{b’hŷnnúb’hâ}{adv.}{old \s{all} of \w{núb’hâ}}{Anew.}{}
\entry{caḍráy’ẹ́}{v.}{\pf{chatoyer}}{To shimmer, iridesce.}{}
\entry{cah}{v.}{\pf{chasser}}{To hunt.}{\s{fut} cahe, \s{subj} cas}
\entry{cahý}{pron. pl. indef.}{\pf{chacun}}{Each other, one another.}{}
\entry{Cár}{n.}{}{\textit{Male given name, equivalent to English ‘Kyle’ or ‘Charles’. Often declined like a regular noun, e.g. \s{nom} \w{Lác̣ár}}.}{}
\entry{cẹ}{v.}{\pf{chaud}}{To be hot.}{}
\entry{će}{v.}{\pf{échouer}}{\\+\s{part} To stumble, do a bad job at.\\+\s{acc} \textit{or} \s{aci} To fail, flunk, not pass.}{\s{fut} ćere, \s{subj} ćes}
\entry{cèc}{n.}{phonetic respelling of \w{cèc’h}}{\textit{(chess)} Check.}{}
\entry{cèc’h}{n.}{\pf{échec}}{Failure, defeat.}{}
\entry{cér}{v.}{\pf{cher}}{\\To be dear, important (+\s{dat} to sbd.). \textit{Possession of a noun qualified with this adjective verb is generally construed with the dative rather than the genitive, e.g. \w{asvẹ áví cérâ} or \w{áví cérâvé} ‘my dear friend’, rather than *\w{vaú áví cérâ}}.\\\textit{with \w{áví} ‘friend’} To be friends with.\ex \w{áví lẹcérvé} ‘he is a (dear) friend of mine’.}{\s{fut} céré, \s{subj} cés}
\entry{cévê}{n.}{\pf{chemin}}{Street.}{}
\entry{c’habhahit’hẹ}{n.}{\pf{capacité}}{Skill, capacity, ability.}{}
\entry{c’hánár}{n.}{\pf{canard}}{\\Ship, boat.\\\s{instr indef} By boat.}{}
\entry{c’hánaú}{n.}{\pf{canot}}{Duck (bir).}{}
\entry{c’háraúciḍ}{v.}{\pf{les carrotes sont cuites}}{To end for good, put to a permanent end.}{\s{fut} c’hár\-aúc\-re, \s{subj} c’háraúc}
\entry{c’hasḅesy’ál}{n.}{cas spécial}{Exception.}{}
\entry{c’haú}{adj.}{see sense 2}{\\Holy.\\\w{c’haú-}\L{} \textit{‘religious prefix’, prepended in derivation to nouns that have a religous connotation; this is historically a back-formation from \w{c’haúfrér} and \w{c’haúhýr} which happen to both start with this ‘prefix’}.}{}
\entry{c’haúáł}{n.}{\w{c’haú} + \w{áł}}{Prophecy.}{}
\entry{c’haúbhárrás}{n.}{\w{c’haú} + \pf{paroisse}}{Parish.}{}
\entry{c’haúbhausy’ô}{n.}{\pf{composition}}{Composition, arrangement, structure.}{}
\entry{c’haúbhèłínáj}{n.}{\w{c’haú} + \pf{pèlerinage}}{Pilgrimage.}{}
\entry{c’haúbhýríf}{n.}{\w{c’haú} + \pf{purifier}}{To purify (+\s{acc} sbd./sth.).}{}
\entry{c’haúḅlér}{.v}{\pf{complaire}}{To be complacent; to be accepting in the presence of +\s{gen} sbd./sth. perceived as negative.}{\s{fut} c’haúḅlére, \s{subj} c’haúḅlés}
\entry{c’haúḅrâd}{v.}{\pf{comprendre}}{+\s{part} To comprehend, understand, gr\-asp.}{\s{fut} c’haúḅrâdrẹ́, \s{subj} c’haúḅrâs}
\entry{c’haúb’héc’h}{v.}{\pf{convaincre}}{To persuade.}{}
\refentry{c’haúḍé}{ní}
\entry{c’haúḍrêd’hẹ}{n.}{\pf{compte-rendu}}{Account, rec\-ord.}{}
\entry{c’haúfí}{v.}{\pf{confiner}}{To contain.}{}
\entry{c’haúfrér}{n.}{\pf{confrère}}{Brother (religious). \textit{Masc. or pl. only, see also \w{c’haúhýr}}.}{}
\entry{c’haúhaúvnaút’hẹ}{n.}{\w{c’haú} + \pf{com\-mu\-nau\-té}}{Mo\-nastery.}{}
\entry{c’haúhýr}{n.}{\pf{consœur}}{Sister (religious). \textit{Fem. only, see also \w{c’haúfrér}}.}{}
\entry{c’haúnéhás}{.n}{\pf{connaissance}}{Knowledge.}{}
\entry{c’haúr}{conj.}{\pf{car} + \pf{comme}}{+\s{subj} As, because, since.}{}
\entry{c’haúv́ájẹ}{n.}{\w{c’haú} + \pf{magie}}{Magic.}{}
\entry{c’haúvnaút’hẹ}{n.}{\pf{communauté}}{Community.}{}
\entry{c’hd’hal}{adv.}{\pf{que dalle}}{Naught, absolutely no\-thing.}{}
\entry{C’hebèc’h}{n.}{\pf{Québec}}{The Promised Land.}{}
\entry{c’hèl}{det. postpos.}{\pf{quelques}}{Some, a few, a couple of.}{}
\entry{c’hèlc’hý}{pron.}{\pf{quelqu’un}}{Someone, somebody, anyone, anybody.}{}
\entry{c’hes}{quest. part.}{\pf{qu’est-ce que}}{\textit{see grammar; often \w{c’h’s} in older texts}.}{}
\entry{c’hesse}{}{contraction of \w{c’hes} + \w{se}}{Is it? \textit{Also substituted for other forms of ‘to be’ in questions, particularly for the plural neuter; stressed on the first syllable}.}{}
\entry{c’hlýr}{v.}{\pf{inclure}}{\\+\s{part} To include.\\+\s{acc} To possess, have \textit{(alienably)}.\\+\s{gen} \textit{usually} \s{indef} To sell, offer, have in stock.}{\s{fut} c’hlýré, \s{subj} c’hlýrs}
\entry{c’hóbhár}{v.}{\pf{comparer}}{To compare.}{\s{fut} c’hóbhárre, \s{subj} c’hó\-bhárs}
\entry{c’hóhid’hẹ́}{v.}{\pf{considérer}}{\\+\s{part} To consider, think ab\-out, ponder.\\+\s{acc} To think through.}{\s{fut} c’hóhid’hẹ́rẹ́, \s{subj} c’hóhid’hés}
\entry{c’hóný}{adj.}{\pf{connu}}{Known, familiar, well-kn\-own.}{}
\entry{c’hór}{n.}{\pf{corps}}{Body.}{}
\entry{c’hóvâ}{v.}{\pf{commencer}}{\\(\s{+ part}) To start, commence, begin.\\\s{+ gen} To start out as.\\\w{âc’hóvâ} \textit{def.} Beginning, start. \textit{Lit. ‘that which is being begun’}.}{\s{fut} c’hóvârẹ́, \s{subj} c’hóv\-ás}
\entry{c’hóvníc’h}{v.}{\pf{communiquer}}{\\To communicate (+\s{instr} with sbd.).\\Communication.}{\s{fut} c’hóvníc’hre, \s{subj} c’hóvníc’hes}
\entry{c’hrír}{v.}{\pf{écrire}}{To write.}{\s{fut} c’hrírẹ́, \s{subj} c’hrís}
\entry{c’hulvâ}{n.}{\pf{écoulement}}{Flow.}{}
\entry{c’húr}{v.}{\pf{court}}{To shrink, reduce in size, narrow.}{}
\entry{c’húr}{v.}{\pf{courrir}}{To run.}{}
\entry{c’hýr}{n.}{\pf{cœur}}{Heart.}{}
\refentry{c’h’s}{c’hes}
\entry{dá}{n.}{\pf{dent}}{Tooth.}{}
\entry{ḍá}{conj.}{\pf{tandis}}{Whereas.}{}
\entry{ḍád}{n.}{\pf{stand}}{Stand, stall, booth.}{}
\entry{dahaúr}{particle}{\pf{d’accord}}{Sure, ok, agreed, fine.}{}
\entry{ḍalẹ}{n.}{\pf{tableau}}{Table.}{}
\entry{ḍalisvâ}{n.}{\pf{établissement}}{Establishment, institution.}{}
\entry{dár}{v.}{\pf{darder}}{To throw, cast, yeet (+\s{acc} sth.).}{}
\entry{ḍaú}{n.}{\pf{tonne}}{Weight.}{}
\entry{daú(c’h)}{particle}{\pf{donc}}{Therefore, then, thus.}{}
\entry{ḍaúb’h}{v. intr.}{\pf{tomber}}{To fall, drop.}{}
\entry{daúb’hedwébhó}{v.}{\pf{tomber dans les pommes}}{To faint.}{\s{fut} daúb’hedwébhóre, \s{subj} daúb’hedwébhós}
\entry{ḍauḍ}{def. pron.}{from earlier \w{ḍẹ auḍ}}{Everything else, any other (one).}{}
\entry{Daúvníc’h}{n.}{}{\textit{male or female given name, equivalent to English ‘Dominic’}.}{}
\refentry{daú’b’h}{daú(c’h) \textnf{+} b’he}
\entry{db’hid’h}{n.}{\pf{individu}}{Person, individual.}{}
\entry{de}{conj.}{\pf{dès que}}{+\s{subj} Once, when once, as soon as.}{}
\entry{dẹ́}{particle}{from \w{Provençal} \textit{den}}{Then (sequential), next.}{}
\entry{ḍẹ}{adj.}{\pf{tout}}{\\All, every, whole, entire.\\\w{ḍẹ auḍ} Obsolete form of \w{ḍauḍ}.}{}
\entry{deḅlér}{v.}{\pf{déplaire}}{To displease (+\s{acc} sbd.), be displeasing.}{}
\entry{dẹb’hní}{v.}{\pf{devenir}}{To become, turn into (+\s{transl} sth./sbd.). \textit{The subject is in the \s{abs} case}.}{}
\entry{dec̣ír}{v.}{\pf{déchirer}}{\\+\s{part} To tear, rip, rend.\\+\s{acc} To rend asunder, tear to pieces.}{\s{fut} dec̣irrẹ, \s{subj} dec̣írs}
\entry{dèc’h}{adj.}{\pf{dextre}}{Right (side), right-handed.}{}
\entry{ḍèc’hníc’hvâ}{adv.}{\pf{techniquement}}{Technically.}{}
\entry{ḍédv́ér}{interj.}{\pf{putain de merde}}{Fuck. \textit{Generic expletive}.}{}
\entry{dẹh}{v.}{\pf{dessous}}{To be below, beneath.}{}
\entry{dehab’híy’}{v. tr.}{\pf{déshabiller}}{To undress +\s{acc} sbd.}{}
\entry{dẹhẹ}{n.}{\pf{dessus}}{\\Top, upper side.\\Surface of a body of water.}{}
\entry{dehid}{v.}{\pf{décider}}{To decide (+\s{inf} to do sth.).}{}
\entry{dej}{particle}{from \w{dejẹ}}{\textit{Emphatic particle; only used in the preterite}.\\\s{pret} + \w{dej} \textit{roughly} To have ever done sth.}{}
\entry{dejẹ}{adv.}{\pf{déjà}}{Already.}{}
\entry{ḍèl}{particle}{\pf{tel}}{\textit{Emphatic particle, used as an intensifier, often postpositive after the verb, but not so much intensifying the verb directly as it does the entire clause.} \ex \s{Snet’h}, \s{ii.34}: \w{lá-árb srýlé dèl} ‘so it was that the tree was burning’ or ‘the tree was burning fiercely’, or ‘indeed, the tree was burning’.}{}
\entry{ḍénéb}{n. pl.}{\pf{ténèbres}}{\textit{exclusively plural \w{lḍénéb}} Darkness.}{}
\entry{ḍèr}{v.}{\pf{taire}}{To silence, shut up.}{\s{fut} ḍérẹ́}
\entry{ḍéraúj}{v.}{\pf{interroger}}{To demand.}{}
\entry{ḍérésḍ}{v.}{\pf{terrestre}}{To be terrestrial, earth-based.}{\s{fut} ḍérésḍrẹ́, \s{subj} ḍérésḍs}
\entry{ḍẹ́ríb}{v.}{\pf{terrible}}{To be terrible (all senses).}{\s{fut} ḍẹ́ríre, \s{subj} ḍẹ́rís}
\entry{dérny’é}{adj.}{\pf{dernier}}{Last, final, ultimate.}{}
\entry{dérny’ẹ́huf}{n.}{\pf{dernier} + \pf{souffle}}{Death.}{}
\entry{ḍérsèd}{v.}{\pf{intercéder}}{To intercede.}{}
\entry{ḍèrvíc’h}{n.}{\pf{thermique}}{Heat, warmth.}{}
\entry{deslẹ}{v.}{\pf{déceler}}{To detect, discover, uncover, reveal.}{\s{fut} deslẹre, \s{subj} deslẹs}
\entry{dévýr}{v.}{\pf{demeurer}}{\\To remain, stay.\\To live, dwell (+\s{iness} somewhere).}{}
\entry{ḍeý’ebhat’hẹ}{n.}{\pf{télépathie}}{Telepathy.}{}
\entry{ḍeý’ebhat’hic’h}{v.}{\pf{télépathique}}{To be telepathic.}{\s{fut} ḍeý’ebh\-at’hic’hre, \s{subj} ḍeý’ebhat’hic’hes}
\entry{dír}{v. tr.}{\pf{dire}}{+\s{acc} To say, tell (+\s{dat} someone).}{\s{fut} dírẹ́, \s{subj} díss}
\entry{díríj}{v.}{\pf{diriger}}{+\s{acc} To direct, run, oversee, operate (a business or establishment).}{\s{fut} díríje, \s{subj} díríjs}
\entry{dónẹ́}{v.}{\pf{donner}}{\s{+ dat \& acc/part} To endow, bestow, give. \textit{The \s{acc} is used when talking about concrete, measurable, and finite objects or sums; the partitive to talk about abstract concepts or parts of objects; the \s{dat} is the person being endowed with}.}{\s{fut} dónrẹ́, \s{subj} dónés}
\entry{ḍúr}{adv.}{\pf{toujours}}{\\\textit{(positive context)} Always.\\\textit{(negative context)} Still.}{}
\entry{ḍúr}{n.}{\pf{tour}}{Tower.}{}
\entry{duý’ýr}{v.}{\pf{douleur}}{To suffer, be in pain.}{}
\entry{dývrê}{particle}{\pf{du moins}}{At least. \textit{As in e.g. ‘At least, I think that \ldots’}.}{}
\entry{dy’ê}{v.}{\pf{tien}}{To be yours (\s{sg}).}{\s{fut} dy’êrẹ́, \s{subj} dy’ês}
\entry{e}{n.}{\pf{eau}}{Water.}{}
\entry{ebhẹ}{v.}{\pf{épais}}{To be thick.}{\s{fut} ebhrẹ, \s{subj} ebhes}
\entry{ec̣}{n.}{\pf{péché}}{Sin, transgression, wrongdoing.}{}
\entry{ec’hlér}{v.}{\pf{éclairer}}{To shine.}{}
\entry{ed}{particle}{\pf{et}}{\textit{Used in numbers, see §~\ref{subsec:numerals}}.}{}
\entry{eḍ}{v. irreg.}{\pf{être}}{To be.}{\s{forms} \textit{active only, see §~\ref{subsec:ed-paradigm}}}
\entry{eḍrrá}{v.}{\pf{étroit}}{Pointy.}{}
\entry{Eḍy’ê}{n.}{}{\textit{male given name, equivalent to English ‘Ste\-phen’}.}{}
\entry{ehyó}{n.}{\pf{écusson}}{Shield.}{}
\entry{el}{n.}{\pf{ailles}}{Wing, blade, fin.}{}
\entry{ez-}{pron.}{\pf{ses}}{Its, her, his, their.}{}
\entry{F}{adj.}{from \pf{fẹ}}{\textit{Logic.} False, $\bot$. \textit{Always capitalised}.}{}
\entry{fahaú}{conj.}{\pf{de façon que}}{+\s{opt} In such a way that, such that, so much so.}{}
\entry{faú}{adv.}{\pf{fort}}{Very, right, really. \textit{Postpositive intensifier placed after adjectives, particularly in the comparative or superlative degrees.}.}{}
\entry{faúr}{adj.}{\pf{fort}}{\textit{obsolete, except in proverbs} Strong, mighty.}{}
\entry{faúr₁}{n.}{\pf{force}}{\\Force, strength, power.\\\w{Faúr} \s{def} \textit{(science fiction, Star Wars)} The Force.}{}
\entry{faúr₂}{n.}{\pf{forme}}{Shape, form. \textit{Sometimes also spelt \w{fór}}.}{}
\entry{fé}{n.}{\pf{fin}}{End.}{}
\entry{fẹ}{v.}{\pf{faux}}{To be false, incorrect, wrong.}{\s{fut} faure, \s{subj} faus}
\entry{fẹhab}{v.}{\pf{faisable}}{To be possible, feasible.}{\s{fut} fẹhabre, \s{subj} fẹhas}
\entry{fèhẹ}{n.}{\pf{faisceau}}{\\Bundle, bunch, cluster.\\Beam, ray.}{}
\entry{fér}{v.}{\pf{faire}}{\\To do, make, build, construct, erect.\\\textit{Expletive; see §~\ref{subsubsec:personal-pronouns}}.}{\s{fut} fẹ́, \s{subj} fés}
\entry{férḍufraú}{v.}{\pf{en faire tout un fromage}}{To make a big fuss a\-bout something.}{\s{fut} fér\-ḍu\-fraúrẹ́, \s{subj} férḍufraús}
\entry{férr-rásvát’h}{n.}{\pf{faire la grasse mat’}}{A long, deep sleep.}{}
\entry{fic’h}{v.}{back-formation from *\w{fic’hs}, reinterpreted as a subjunctive stem; from \pf{fixer}}{To fix, set, establish.}{\s{fut} fic’hre, \s{subj} fic’hs}
\entry{fihas}{v.}{\pf{efficace}}{To be efficient.}{}
\refentry{fór}{faúr₂}
\entry{fórvẹ́}{v.}{\pf{informer}}{To inform (+\s{acc} sbd.) (+\s{aci} of sth.).}{\s{fut} fórv́ẹ́, \s{subj} fórvẹ́s}
\entry{fúr}{v.}{\pf{fournir}}{To deliver, provide (+\s{dat} sbd.) (+ \s{acc} with sth.).}{}
\entry{fý}{n.}{\pf{feu}}{Fire.}{}
\entry{hab’híy’}{v. tr.}{\pf{habiller}}{To dress +\s{acc} sbd.}{}
\entry{í}{n.}{\pf{hymne}}{Legend, myth.}{}
\refentry{ís}{ub’hrá}
\entry{isḍrár}{n.}{\pf{histoire}}{Story, tale.}{}
\entry{íváj}{n. archaic}{\pf{image}}{Image, picture.}{}
\entry{iý’ývî}{v.}{\pf{illuminer}}{To light up, illuminate.}{}
\entry{Já}{n.}{}{\textit{male or female given name, equivalent to English ‘John’ or ‘Joan’}.}{}
\entry{Jac’h}{n.}{\pf{Jacques}}{\textit{Male given name}.}{}
\entry{jávé}{adv.}{\pf{jamais}}{\textit{neg. only} Never, at no time.}{}
\entry{jaý’aú}{n.}{\pf{jalon}}{\\Nail.\\\textit{obsolete} Stake, pole.}{}
\entry{Jed’háy’}{n.}{\pf{Jedi}}{Jedi (Star Wars).}{}
\entry{júrdy’í}{adv.}{from archaic \pf{aujúrdy’í}}{Today.}{}
\entry{jys}{conj.}{\pf{jusqu’à ce que}}{+\s{opt} Until.}{}
\entry{jys}{adv.}{\pf{juste}}{Just, only, merely.}{}
\entry{jys}{v.}{\pf{injuste}}{To be unjust, unfair.}{\s{fut} jysre, \s{subj} jyss}
\entry{lá}{v.}{\pf{planer}}{To fly.}{}
\entry{lab’h}{v.}{\pf{laver}}{\\To wash, clean (+\s{acc} sth.).\\\textit{refl} To wash oneself, take a bath, have a shower.}{}
\entry{Lác}{n.}{}{\textit{female given name, equivalent to English ‘Bi\-anca’}.}{}
\entry{láḍ}{n.}{\pf{plante}}{\\Blade of grass.\\\textit{pl.} Grass.}{}
\entry{lánẹ́}{v.}{\pf{flâner}}{To meander.}{}
\entry{lár}{v.}{\pf{large}}{Wide, broad.}{}
\entry{lârdávrá}{n.}{\pf{langue de bois}}{Evasive, unclear, or overly formal speech.}{}
\entry{las}{v.}{\pf{placer}}{To place, put, set (+\s{acc} sth.).}{}
\entry{laú}{v.}{\pf{long}}{Long. \textit{Often in compounds \w{laú-} ‘long-’}.}{}
\entry{laúrs}{conj.}{\pf{lorsque}}{When (temporal only).}{}
\entry{laúrvé}{conj.}{from \w{laúrs} + \w{vé}}{\textit{(contraction)} But, when. \textit{Stressed on the first syllable}.}{}
\entry{laut’h}{v.}{\pf{flotter}}{Fl\-oat, hover, levitate.}{\s{fut} laut’hre, \s{subj} laut’hes}
\entry{le}{v.}{\pf{laisser} > *\w{lehe}}{\textit{Chiefly in questions or imperative.} To let, allow, permit.}{\s{fut} lere, \s{subj} les}
\entry{lé}{n.}{\pf{plaine}}{Plain, plains.}{}
\entry{lẹ-}{prefix}{\pf{plus}}{\textit{Affirming comparative prefix. See grammar}.}{}
\entry{lec’hḍraúvnẹ́t’hic’h}{v.}{\pf{électromagnétique}}{To be electromagnetic.}{\s{fut} lec’hḍraúvnẹ́t’hic’hre, \s{subj} lec’hḍraúvnẹ́t’hic’hes}
\entry{leḍ}{n.}{\pf{lettre}}{\\Letter (of the alphabet).\\\w{lý’aúleḍ} By the book.}{}
\entry{lèheb’h}{v.}{\pf{laisser-faire}}{+\s{aci} To let happen.}{}
\entry{lẹhuvud}{n.}{\pf{coup de foudre}}{Love at first sight.}{}
\entry{lér}{v.}{\pf{clair}}{To be evident, obvious, frank, clear.}{}
\entry{lér}{v.}{\pf{plaire}}{To please (+\s{acc} sbd.), be pleasing.}{}
\entry{lí}{v.}{\pf{lire}}{\\+\s{part} To read from.\\+\s{acc} To peruse, read entirely.}{\s{fut} lírẹ́, \s{subj} lís}
\entry{lit’hijy’}{v.}{\pf{litigier}}{To litigate, be at law with (+\s{dat} sbd.).}{}
\entry{lívnád}{n.}{\pf{limonade}}{Lemonade.}{}
\entry{liv́uhé}{n.}{\pf{livre} + \pf{bouquin}}{Book.}{}
\entry{lúr}{v.}{\pf{lourd}}{To be bulky, oversized, heavy.}{}
\entry{ly}{particle}{\pf{plus}}{\textit{obsolete variant of \w{lẹ}, sometimes leniting}.}{}
\entry{lý}{n.}{\pf{plume}}{Pen, quill.}{}
\entry{ḷý}{n.}{\pf{lieu}}{\textit{Base of the spatial correlatives. In senses 2–5, case affixes are attached before \this, e.g. sense 2 \s{all} \w{sẹb’héḷý} ‘hither’}.\\Place, location.\\\w{sẹ}\ldots\this \s{def} [from \w{sẹh}] Here, hither, hence, \&c. \textit{Proximal demonstrative (all cases)}.\\\w{sý’\L}\ldots\this \s{def} [from \w{sý’ẹ}] There, thither, thence, \&c. \textit{Distal demonstrative (all cases)}.\\\this\w{hes} \s{indef} [from \w{c’hes}] Where, whither, when\-ce, \&c. \textit{Interrogative (locative cases only)}.\\\w{s’}/\w{sá\L}\ldots\this \s{indef} [from \w{sá}] No\-where, from no\-where, \&c. \textit{Negative (locative cases only)}.}{}
\entry{lybhárdyt’há}{adv.}{pluspart du temps}{Often.}{}
\entry{lýr}{pron.}{\pf{leur}}{Their.}{}
\entry{lýrḍ}{v.}{from \pf{leur}; the ‘ḍ’ was added in analogy with \w{naúḍ} and \w{b’hauḍ}}{To be theirs.}{\s{fut} lýrḍre, \s{subj} lýrs}
\entry{lys}{adv.}{\pf{plus} /plys/}{\textit{neg. only} No longer, not any more. \textit{The meaning of this and \w{lẹ} swapped at some point for unknown reasons}.}{}
\entry{lýv́á}{v. \s{3rd} person only}{\pf{pleuvoir}}{To rain.}{\s{fut} lýv́áre, \s{subj} lýv́ás}
\entry{lývy’ér}{n.}{\pf{lumière}}{Light.}{}
\entry{lyzy’ýr}{adj.}{\pf{plusieurs}}{Several.}{}
\entry{n}{n.}{\pf{haine}}{Hate, hatred, loathing.}{}
\entry{nájẹ}{v.}{\pf{nager}}{To swim.}{\s{fut} náȷ́ẹ, \s{subj} nájes}
\entry{nárrahóḍ}{v.}{\pf{raconter} + \pf{narrer}; subj. from \pf{filer}}{To narrate, recount \s{+part sth.}, tell (\s{+dat} sbd.) a story.}{\s{fut} nárrahóḍe, \s{subj} fils}
\entry{nát’hýr}{n.}{\pf{nature}}{\\\textit{(chiefly)} \s{indef} Nature, the natural world.\\\s{def} The way something is.}{}
\entry{naúḍ}{v.}{\pf{nôtre}}{To be ours.}{\s{fut} naúḍre, \s{subj} naús}
\entry{néḍ}{v. dep.}{\pf{naître}}{To be born. \s{This is a deponent verb whose subject takes the \s{acc} and which only takes passive affixes.}.}{\s{sub} néhs}
\entry{nérjẹ}{n.}{\pf{énergie}}{Energy.}{}
\entry{nés}{adj.}{from earlier \w{nésḍ}}{Left (side), left-handed.}{}
\entry{nésḍ}{adj. archaic}{\pf{senestre}}{Left (side), left-handed.}{}
\entry{ní}{v.}{\pf{nier}, \s{fut} from \pf{contrer}, \s{subj} from \pf{oposer}}{To deny, ref\-use, reject, rebut (+\s{acc} sbd./sth.).}{\s{fut} c’haúḍé, \s{subj} aubhaus}
\entry{ní}{conj.}{\pf{ni}}{Neither, nor.}{}
\entry{níb’hẹ}{n.}{\pf{niveau}}{\\Level, degree.\\\s{def iness + gen} On the level of.}{}
\entry{nór}{v.}{back-formation from *\w{nórâ} from \pf{ignorant}}{To be ignorant, unaware, oblivious.}{}
\entry{nóráv}{n.}{from archaic \w{ḅá nórávíc’h}}{Druid.}{}
\entry{núb’h}{v.}{\pf{nouveau}}{To be new.}{\s{fut} núb’he, \s{subj} núb’hs}
\refentry{p-}{ḅ-}
\refentry{ph-}{ḅ-}
\refentry{p’h-}{bh-}
\entry{R}{adj.}{from \pf{ré}}{\textit{Logic.} True, $\top$. \textit{Always capitalised}.}{}
\entry{r}{n.}{\pf{air}}{Air. \textit{Frequently plural}.}{}
\entry{ra}{conj.}{\pf{swa} > *\w{rá}}{\\Or. \textit{exclusive, see also~\w{u}}.\\\w{u}/\w{ra} \ldots\ \w{ra} \ldots\ ‘either \ldots\ or \ldots’ \textit{(exclusive)}.}{}
\entry{râ}{v.}{\pf{gagner}}{To win, gain, earn (+\s{acc} sth.).}{}
\entry{rá₁}{n.}{\pf{loi}}{Law, rule, regulation.}{}
\entry{rá₂}{adj.}{\pf{grand}}{Big, large, great.}{}
\entry{rá₃}{n.}{\pf{mois}}{Month.}{}
\entry{rá₄}{n.}{\pf{voix}}{Voice.}{}
\entry{rá₅}{n.}{\pf{bras}}{Arm.}{}
\entry{Ráb’h}{n.}{unknown; presumably the name of some celebrity or local deity}{\\\textit{indecl.} \s{def sg} \textit{always} \s{nom} \textit{or} \s{voc} Ráb’h. \textit{Main god of the ULTRAFRENCH pantheon; usually male. Old-fashioned also often all-caps \w{RÁB’H}.}.\ex \s{Snet’h}, \s{i.17}: \w{au lebálá daú RÁB’H} ‘and thus spake Ráb’h’.\\\w{Ráb’h sénýr} \s{def sg} Lord Ráb’h. \textit{Used for sense~{\bf 1} in all other cases; as with all names, only \w{sénýr} is inflected. Old-fashioned often \w{RÁB’H Sénýr}}.\ex \s{Snet’h}, \s{8.1}: \w{au labraúc RÁB’H naút B’héhénýr} ‘and they came to our Lord Ráb’h’.\\\textit{(rarely)} The main god of another culture. \textit{Only attested figuratively. Not capitalised in this sense, and declined like a regular word.}.\ex \s{Snet’h}, \s{ii.3}: \w{ledéraújá’z derévôt’he láráb’h} ‘their god demanded they return’.}{}
\refentry{ráb’h}{v́ár}
\entry{ráb’háy’}{v.}{\pf{travailler}, \s{fut} and \s{subj} from \pf{bos\-ser}}{To work.}{\s{fut} bohér, \s{subj} bos}
\entry{rác’hánár}{n.}{from \w{ráhe} + \w{c’hánár}}{Airship, dirigible.}{}
\entry{rác’hsaý’ad}{v.}{\pf{raconter des salades}}{To lie, tell tall tales, overexaggerate.}{\s{fut} rác’h\-sa\-ý’e, \s{subj} rác’hsaýs}
\entry{rád}{v. tr.}{\pf{rendre}}{To surrender +\s{acc} sth. (\s{dat} to sbd.).}{}
\entry{râd}{v.}{\pf{prendre}}{+\s{acc} \textit{or} \s{part} To grab. \textit{The \s{part} usually implies that only a part or some of a larger whole is grabbed, e.g. a handful of sand)}.}{}
\entry{râdrásôn}{v.}{\pf{prendre ses jambe à son cou}}{To run.}{\s{fut} râdrásônre, \s{subj} râdrásôns}
\entry{rádrénẹ́}{v. + \s{aci}}{\pf{les doigts dans le nez}}{To put no effort into.}{\s{fut} rádrénrẹ́, \s{subj} rádrénẹ́s}
\entry{râdvâ-}{prefix}{\pf{grandement}}{\textit{Superlative prefix. See grammar}.}{}
\entry{rád’hérn}{n.}{from \w{rá} + \w{dérny’é}}{\textit{(always definite)} Last month.}{}
\entry{rád’hsy’ô}{n.}{\pf{traditon}}{Tradition, custom.}{}
\entry{rád’hyc’hsy’ô}{n.}{\pf{traduction}}{Translation.}{}
\entry{râhaúḍ}{v.}{\pf{recontrer}}{To meet, encounter, come face to face (+\s{all} with sbd.).}{\s{fut} râhaúḍre, \s{subj} râhaús}
\entry{ráhe}{n.}{\pf{oiseau}}{Bird.}{}
\entry{ráhé}{n.}{from \w{ráhe} + \w{ráhó}}{Flying fish.}{}
\entry{ráhé}{n.}{\pf{voisin}}{Neighbour.}{}
\entry{ráhẹ}{conj.}{\pf{quoique}}{+\s{subj} Although, though.}{}
\entry{râhẹ}{n.}{\pf{français}}{Human, person.}{}
\entry{ráhis}{v.}{\pf{raciste}}{To be racist.}{\s{fut} ráhise, \s{subj} ráhiss}
\entry{ráhó}{n.}{\pf{poisson}}{Fish.}{}
\entry{ráhó}{n.}{\pf{gazon}}{Grassland, grassy field, meadow.}{}
\entry{ráhut’h}{n.}{\pf{grand} + \pf{couteau}}{Sword, blade \ex \w{áráhut’h’t ilý ly b’haúr} ‘the pen is mightier than the sword’ \textit{(originally a fossilised, obsolete ACI: \w{á\-hut’h\-rá éḍ ilý lẹb’haúr})}.}{}
\entry{rál}{n.}{\pf{toile}}{Canvas.}{}
\entry{rár}{v.}{\pf{voir}}{To see (+\s{part} sbd./sth.).}{\s{fut} b’hérẹ́, \s{subj} rárs}
\entry{rárd}{v.}{\pf{regarder}}{\\+\s{acc} To watch.\\+\s{part} To look at.}{\s{fut} rárdre, \s{subj} rárds}
\entry{râsír}{v.}{\pf{transpirer}}{+\s{aci}.\\To come to light, become known, transpire.\\\s{pres ant} For it to be clear, apparent, evident that \ldots \textit{Lit. ‘it has come to light that \ldots’}.}{\s{fut} râsírẹ́, \s{subj} râsírs}
\entry{rát’hẹ}{particle}{\pf{vois-tu}}{You see, you know.}{}
\entry{raû}{n. archaic}{\pf{tronc}}{Log (of a tree).}{}
\entry{raû}{interj.}{\pf{gône}}{Kid. \textit{This is grammatically a vocative—not that one could tell since it looks identical to the absolutive}.}{}
\entry{raú(b’hc’h)-}{prefix}{from \w{rób’hoc’h}}{\textit{Causative prefix, see §~\ref{subsec:diachrony-and-derivation}}.}{}
\entry{raúb’hẹ}{n.}{\pf{robot}}{Robot.}{}
\entry{raûc}{n.}{\pf{tronche}}{Head.}{}
\entry{raûd’hárb}{n.}{\pf{tronc d’arbre}}{Log (of a tree).}{}
\entry{raúhérẹ́}{v.}{from \w{raú-} + \w{sérẹ́}}{To tighten, make tighter (+\s{acc}).}{\s{fut} raúhérrẹ́, \s{subj} raúhérẹ́s}
\entry{raúhy’b’h}{v.}{from \w{raú-} + \w{sy’b’h}}{To raise, lift up (+\s{acc} sth.) (+\s{ela} from sth.).}{}
\entry{raúl}{n.}{\pf{parole}}{\\Language, speech, word.\\\w{Raúl} \textit{(definite only)} Short for \w{T’hebhaú Raúl}. \textit{\s{nom sg} irreg. \w{Raúl}; all other forms are regular}.}{}
\entry{raúvá}{n.}{\pf{fromage}}{Moon.}{}
\entry{ráv́â}{adv.}{\pf{rarement}}{\textit{neg. only} Seldom, rarely (ever).}{}
\entry{rávér}{n.}{\pf{grammaire}}{\\Grammar, the grammatical rules of a language.\\A textbook describing the grammar of a language.}{}
\entry{ráy’á}{v.}{\pf{voyage}}{\\To travel, go on a journey.\\\textit{n.} Travel, voyage, journey.}{}
\entry{ráy’é}{v.}{\pf{noyer}}{To drown.}{}
\entry{ráy’ê}{n.}{\pf{moyen}}{\\Way, means, method.\\\w{ráy’ê y’aúhý} + \s{aci} There is no way, that \ldots{}.\\\s{instr pl} \w{b’hehráy’ê} How, by what means, in this way.}{}
\entry{ráý’ẹ}{v.}{\pf{râler}}{To complain, grumble.}{}
\entry{ré}{v.}{\pf{vrai}}{To be true, correct, right.}{\s{fut} rẹ́, \s{subj} rés}
\entry{ré}{n.}{\pf{rai}}{Ray, beam.}{}
\entry{ré}{v.}{\pf{créer}, \s{subj} from \pf{fabriquer}}{To create, make (\s{+acc} sth.).}{\s{fut} rẹ́éré, \s{subj} faríc’hs}
\entry{ré}{adj.}{\pf{près}}{Near, close, nearby.}{}
\entry{ré}{v. intr.}{\pf{errer}}{To wander, roam (+\s{perl} across sth.).}{}
\entry{ré}{adv.}{en vain}{In vain, for nothing.\textit{Usually preceded directly by the verb it applies to}.}{}
\entry{ré}{n.}{\pf{souhait}}{Wish.}{}
\entry{rê}{conj.}{\pf{bien que}}{+\s{subj} Although, though.}{}
\entry{rê}{v.}{\pf{trine}}{To be composed of three parts or people; triune.}{\s{fut} rêrẹ́, \s{subj} rês}
\entry{rê}{n.}{\pf{airain}}{Copper.}{}
\entry{rê}{n.}{\pf{point}}{Point (in a score).}{}
\entry{ré-}{prefix}{\pf{très}}{\textit{Superlative prefix. See grammar}.}{}
\entry{rê-}{prefix}{\pf{moins}}{\textit{Neutral comparative prefix. See grammar}.}{}
\entry{réaû}{n.}{from \w{ré}}{Creation, making.}{}
\entry{rébh}{v.}{\pf{préparer}}{To anticipate (+\s{acc} sth.).}{}
\entry{rébhós}{n.}{\pf{réponse}}{Answer, response, reply.}{}
\entry{rẹ́b’h}{v. or n.}{\pf{rêver}}{\\To dream (+\s{gen} of sth.).\\Dream, a dreaming.}{\s{fut} rẹ́v́e, \s{subj} rẹ́b’hs}
\entry{réb’hní}{v.}{\pf{prévenir}}{\\To prevent, stop (+\s{acc} sth. from happening).\\To forewarn (+\s{part} of sth.).}{\s{fut} réb’hníre, \s{subj} réb’hnís}
\entry{réḍ}{v.}{\pf{souhaiter}}{To wish (+\s{acc/aci} for sth.).}{}
\entry{rêd}{v.}{\pf{craindre}}{+s{opt} To fear, lest \ldots \textit{Construed with the negated optative}.}{\s{fut} rêdrẹ́, \s{subj} rês}
\refentry{rêd}{ḅẹt’hẹ}
\entry{rêdrsýrśẹ}{v.}{\pf{prendre sur soi}}{\\+\s{aci} To take upon onself to do sth.\\+\s{pci} To take upon oneself to start doing sth.}{}
\entry{rẹ́dy’í}{v.}{\pf{réduire}}{To reduce (+\s{acc} \textit{or pass.} sbd./sth.) (+\s{all} to sth.).}{\s{fut} rẹ́dy’ré, \s{subj} rẹ́dy’ís}
\entry{rêd’hes}{particle}{\pf{bien sûr}}{Of course, certainly, surely.}{}
\entry{rẹ́flec̣}{v.}{\pf{réfléchir}}{To think (+\s{part} sth.).}{}
\entry{réhẹv́}{v.}{\pf{recevoir}}{To receive.}{\s{fut} réhẹv́é, \s{subj} rẹsy}
\entry{rêr}{n.}{\pf{fringues}}{\\An article of clothing, garment, piece of clothing.\\\textit{pl.} Clothes, garments.}{}
\entry{rés}{n.}{\pf{reste}}{Rest, remainder.}{}
\entry{rét’hád}{v.}{\pf{prétendre}}{To claim, allege.}{\s{fut} rét’hádrẹ́, \s{subj} rét’h\-ádes}
\entry{rét’hẹ}{v.}{\pf{traiter}}{To handle, take care of, deal with.}{\s{fut} rét’hẹre, \s{subj} rét’hes}
\entry{rét’hír}{v.}{\pf{retirer}}{\\(+\s{acc}) To pull, draw, withdraw.\\+\s{part} To pull on sth. without actually moving it; to try to pull sth.}{\s{fut} rét’hírẹ́, \s{subj} rét’hírs}
\entry{révôt’hẹ}{v.}{\pf{remonter}}{To return, come back.}{}
\entry{ríb’hy’ér}{n.}{\pf{rivière}}{River.}{}
\entry{rívnél}{n.}{\pf{criminel}}{Scoundrel, someone without virtue.}{}
\entry{ríy’ŷrệ}{n.}{\pf{prieuré}}{Priory.}{}
\entry{rjẹ}{n.}{\pf{Hergé}}{Comic book.}{}
\entry{rób’hoc’h}{v.}{\pf{provoquer}; future from \pf{infliger}}{\s{+acc} To cause, make happen.}{\s{fut} flijé, \s{subj} rób’hoc’hs}
\entry{rrá}{v.}{\pf{croire}}{Believe (something or someone).}{\s{fut} rrẹ́, \s{subj} rrás}
\entry{rráḍraúc}{n.}{\pf{droit} + \pf{gauche}}{Side.}{}
\entry{rrád’hahánár}{n.}{\pf{froid de canard}}{Extreme cold, coldness.}{}
\entry{rúb’h}{v.}{\pf{trouver}}{To find, discover.}{}
\entry{rvá}{interj.}{of unknown origin}{Alas, woe, oh. \textit{Exclamation of distress, surprise, sadness, or regret}.}{\textit{after words that end with ‘r’, this is spelt \w{-vá} instead}}
\entry{rýc̣ér}{v.}{\pf{requerir}}{To ask, question.}{}
\entry{rýd}{v.}{\pf{rude}}{To be uneven, rough, rugged.}{}
\entry{rýl}{v.}{\pf{brûler}}{\\\s{+acc} To burn.\\\s{+part} To scorch, singe.}{}
\entry{rýl}{n.}{\pf{gueule}}{Face.}{}
\entry{rýrŷ}{v.}{\pf{rugueux}}{To be rough, rugged.}{}
\entry{rýsḍ}{v.}{\pf{frustrer}}{To frustrate, vex, annoy.}{}
\entry{rývýr}{.n}{\pf{rumeur}}{History.}{}
\entry{rýý’ẹ́}{v.}{\pf{céruléen}}{To be cerulean, sky-blue.}{}
\entry{rzaúsḍ}{v.}{\pf{exhaustif}}{\\To be exhaustive, comprehensive, complete.\\To be finished, completed.}{\s{fut} rzaúsḍre, \s{subj} rzaúsḍs}
\entry{s}{conj.}{\pf{si}}{If, when, whenever.}{}
\entry{sá}{particle}{\pf{sans}}{Not, no. \textit{Always enclitic \w{s’} before vowels. This particle is used only in the subjunctive; see also \w{asý’ýâ}, \w{t’hé}}.}{}
\entry{sá}{conj.}{\pf{sans que}}{+\s{subj} Without (doing sth.).}{}
\entry{sáḍy’ér}{n.}{\pf{sanctuaire}}{Sanctuary, shrine.}{}
\entry{sáhẹ}{v.}{\pf{insensé}}{To be preposterous, absurd, nonsensical.}{\s{fut} sáhere, \s{subj} sáhes}
\entry{saj}{v.}{\pf{sage}}{To be wise, prudent.}{}
\entry{sajès}{n.}{\pf{sagesse}}{Wisdom.}{}
\entry{Sásc’hríḍ}{n. never lenited}{\pf{sanskrit}}{The Sanskrit language.}{}
\entry{sásy’él}{v.}{\pf{essentiel}}{To be essential.}{\s{fut} sásy’élẹ́, \s{subj} sásy’éls}
\entry{sauc’h}{conj.}{\pf{sauf que}}{+\s{subj} Except that.}{}
\entry{saul}{n.}{\pf{sol}}{Sun.}{}
\entry{saúr}{n.}{\pf{sorte}}{\\Kind, sort, type, form.\\\s{def + gen} (some) kind(s) of.}{}
\entry{saut’h}{v. intr. or tr.}{\pf{sauter}}{To teleport, translocate, warp (+\s{acc} sth.).}{}
\entry{sauz}{n.}{\pf{chose}}{Thing, object.}{}
\entry{sauz-aud}{adj.}{\pf{autre chose}}{Something else, another thing.}{}
\refentry{sauzaud}{sauz-aud}
\entry{sav́á}{v.}{\pf{savoir}}{To know (+\s{part/acc} sth. \textit{case depends on the depth of the speaker’s understanding}).}{\s{fut} saúr, \s{subj} sac}
\entry{Sávýy’él}{n.}{\pf{Samuel}}{\textit{Male given name}.}{}
\entry{sḅé}{v.}{\pf{espérer}}{\\To want (+\s{acc/inf} sth.).\\+\s{opt} To wish, want, desire.}{\s{fut} sḅérẹ́, \s{subj} sḅés}
\entry{sḅrí}{n.}{\pf{espirit}}{Soul.}{}
\entry{sb’hé}{v.}{\pf{se baigner}}{To bathe.}{}
\entry{sẹ}{particle}{\pf{ainsi}}{So, thus, as a result.}{}
\entry{séḅ}{v.}{\pf{simple}}{To be plain, simple.}{\s{fut} séḅrẹ́, \s{subj} séḅs}
\entry{seb’haúd}{v. intr.}{\pf{s’effondrer}}{To cave in, collapse.}{}
\entry{sèd’h}{part.}{from \pf{c’est du}}{It is due to (+\s{gen} sth. / +\s{aci} the fact that...).}{}
\entry{sẹh}{det.}{\pf{ceci}}{+\s{def} \textit{noun} This, these. \textit{Precedes and is attached to nouns}.}{}
\entry{sẹhérél}{v.}{\pf{se quereller}}{To quarrel, argue, fight about (+\s{part}).}{}
\entry{sehul}{v.}{\pf{s’écouler}}{To flow.}{}
\entry{sẹhúr}{v.}{\pf{secourir}}{To help, succour, give aid (+\s{dat} to sb.) (+\s{aci}/\s{acc} with sth.).}{\s{fut} sẹhúrre, \s{subj} sẹhús}
\entry{sénýr}{n.}{\pf{seigneur}}{\\Lord.\\\textit{Short for \w{Ráb’h sénýr}}.}{}
\entry{sẹrád}{v. intr.}{\pf{se rendre}}{To surrender.}{}
\entry{sérḍé}{det.}{\pf{certain}}{Certain, particular but not specified.}{}
\entry{sérẹ́}{v.}{\pf{serré}}{\\To be tight, close-fitting, snug.\\\s{indef} \textit{usually} \s{instr} \w{c’hýr sérệ} A heavy heart.}{\s{fut} sérrẹ́, \s{subj} sérẹ́s}
\entry{sèt’h}{v.}{\pf{sentir}}{To feel.}{\s{fut} sèt’he, \s{subj} sès}
\entry{séy’ẹ́}{v.}{\pf{essayer}}{+\s{part} \textit{or} \s{inf} To try, attempt.}{\s{fut} séy’ẹ́rẹ́, \s{subj} séy’ẹ́s}
\entry{siḍ}{n.}{\pf{site}}{Facility, site.}{}
\entry{sisḍé}{n.}{\pf{système}}{System.}{}
\entry{Sit’h}{n.}{\pf{Sith}}{Sith (Star Wars).}{}
\entry{sit’há}{conj.}{\pf{si tant est que}}{+\s{opt} Supposing that; if, assuming that.}{}
\entry{sívý’ér}{v.}{\pf{similaire}}{To be similar, alike (+\s{gen} to sth.).}{}
\entry{Snet’h}{n.}{}{\textit{Family name, equivalent to English ‘Smyth’}.}{}
\entry{sol}{n.}{\pf{sol}}{Ground, floor, earth, soil.\textit{The plural may be used to indicate a large quantity of soil}.}{}
\entry{suḍ}{v.}{\pf{soutenir}}{\\+\s{acc} To support, hold up.\\+\s{part} To help support, hold up part of.}{}
\entry{sud’hénvâ}{adv.}{\pf{soudainement}}{Suddenly.}{}
\entry{suf}{n.}{\pf{souffre}}{Pain.}{}
\entry{sufb’h}{n.}{\pf{souffle} + \pf{vie}}{Life.}{}
\entry{susy’é}{v.}{\pf{soucier}}{+\s{part, pci} To care about, worry about.}{\s{fut} susy’ére, \s{subj} susy’és}
\entry{swi}{det.}{\pf{celui}}{The one, that one, this one.}{}
\entry{sybhẹ́rýr}{v.}{\pf{supérieur}}{\textit{intr. or} +\s{gen} To be superior to, better than, higher than.}{\s{fut} sybhẹ́rýrẹ́, \s{subj} sybhẹ́rýrs}
\entry{syḅlẹ}{v.}{\pf{suppléer}}{\\To supplement (\s{acc} sth.) (+\s{instr} with sth.). \textit{If no \s{instr} is present, the subject is implied to be the supplement}.\\\w{syḅlâ} \textit{adj.} Additional, extra.}{}
\entry{syhyý’á}{v.}{\pf{succulent}}{To be succulent, delicious.}{\s{fut} syhyý’áré, \s{subj} syhyý’ás}
\entry{syl}{v.}{\pf{seul}}{\\To be the only one.\\To be lone, alone.}{\s{fut} syle, \s{subj} syls}
\entry{sy’b’h}{v. intr.}{\pf{se lever}}{To rise (+\s{ela} from sth.).}{}
\entry{sy’ê}{v.}{\pf{sien}}{To be his, hers, its.}{\s{fut} sy’êrẹ́, \s{subj} sy’ês}
\entry{sý’ẹ}{det.}{\pf{cela}}{+\s{def} \textit{noun} That, those. \textit{Precedes and is attached to nouns; often \w{sý’} before vowels, with one apostrophe, not two}.}{}
\refentry{s’}{sá}
\refentry{t-}{ḍ-}
\entry{t’hé}{conj.}{\pf{de peur que} > *\w{dbhýrc’h} > *\w{dýrc’h} > *\w{dc’hý} > \this}{Not, no. \textit{Always \w{t’h’\N} before vowels, but does not nasalise if the ‘é’ is still present. This particle is used only in the optative; see also \w{asý’ýâ}, \w{sá}}.}{}
\entry{T’hebhaú}{n. or adj.}{from \w{t’hebhaúz}}{(ULTRA-) France, (ULTRA-)French.}{}
\entry{T’hebhaú Raúl}{n. def. sg.}{from \w{t’hebhaúz} + \w{raúl}}{The ULTRAFR\-ENCH language. \textit{Only \w{T’hebhaú} is declined as though the entire phrase were one word. In informal speech and writing, this is typically shortened to \w{Raúl}}.}{\s{nom sg} \textit{irreg.} \w{T’hebhaú Raúl}}
\entry{t’hebhaúz}{v.}{\pf{jeter l’éponge}}{To be (ULTRA-)French.}{\s{fut} t’hebhaúźe, \s{subj} t’hebhaúś}
\entry{t’hiy’e}{v.}{from \w{yt’hiy’ihẹ}; \s{subj} via ba\-ck-formation from the \s{fut}}{+\s{part} To use, make use of.}{\s{fut} t’hiźe, \s{subj} \s{t’hizes}}
\entry{u}{conj.}{\pf{ou}}{\\Or. \textit{Inclusive, see also \w{ra}}.\\\w{u} \ldots\ \w{u} \ldots\ ‘\ldots\ or \ldots’ \textit{(inclusive)}.}{}
\entry{ub’h}{v.}{\pf{ouvrir}}{To open.}{\s{fut} uv́, \s{subj} ub’hs}
\entry{ub’hrá}{v.}{\pf{pouvoir}}{\\+\s{inf/aci} To be able to, can. \textit{Never construed with an \s{inf} if it in and of itself is the infinitive of an \s{aci} or \s{pci}, in which case the variant with the \s{part} (\senseref{2}) is used instead}.\\+\s{part} To be capable of \ldots\\\s{opt cond i + aci} To be possible; may. \textit{Dynamic or epistemic, never deontic; this and sense 4 are essentially a more emphatic optative}.\\\s{opt cond ii + aci} Might. \textit{Dynamic or epistemic, never deontic}.}{\s{fut} úrẹ́, \s{subj} ís}
\entry{ulíy’ẹ́}{v.}{\pf{oublier}}{To forget.}{\s{fut} ulíy’ẹ́rẹ́, \s{subj} ulíy’ẹ́s}
\entry{úrbh}{conj.}{\pf{pour peu que}}{+\s{opt} Provided that, so long as.}{}
\entry{urdálbhaúrḍ}{n.}{\pf{avoir un oursin dans le portefeuille}}{A very rich person; billionaire.}{}
\refentry{úrẹ́}{ub’hrá}
\entry{uy’ed’háb’hrí}{v.}{\pf{rouler dans la farine}}{To scam, cheat, swindle.}{\s{fut} uy’e\-d’háv́e, \s{subj} uy’ed’háb’hrís}
\entry{vá}{n.}{\pf{mât}}{Mast.}{}
\refentry{vá}{rvá}
\entry{vádłabhaud’hávúrsab’hád’háváb’hrárḍuẹ}{v. literary}{\pf{vendre la peau de ours avant de avoir tué}}{To depend on predictions of the future. \textit{Of disputed origin; first attested in the works of the Early UF comedian \s{J. A. B. Snet’h}}.}{\s{fut} vád\-ła\-bhau\-d’há\-vúr\-sa\-b’há\-d’há\-vá\-b’hrár\-ḍu\-re, \s{s} vád\-ła\-bhau\-d’há\-vúr\-sa\-b’há\-d’há\-vá\-b’hrár\-ḍus}
\entry{vâhẹ}{v.}{\pf{manquer}}{\\+\s{gen} To lack, want.\\+\s{part} \textit{or} \s{pass} To miss. \textit{The object and subject of this verb are swapped compared to English ‘to miss’, e.g. \w{b’hývvâhé} (\s{2pl.act} + \s{1sg.pass}) ‘I miss you (\s{pl})’, lit. roughly ‘you (\s{pl}) are wanting to me’)}.\\+\s{acc} To miss out on.}{\s{fut} vâhérẹ́, \s{subj} vâhés}
\entry{váj}{n.}{from \w{íváj}}{Image, picture.}{}
\entry{válḍrét’hás}{n.}{\pf{maltraitance}}{Torture.}{}
\entry{válfèz}{v.}{\pf{malfaisant}}{To be malfeasant, evil, malevolent.}{\s{fut} válfèź, \s{subj} válfès}
\entry{válv́áy’}{v.}{\pf{malvoyant}}{To be blind.}{}
\entry{válvê}{v.}{\pf{malmener}}{To mistreat, torture.}{\s{fut} válv́e, \s{subj} válvês}
\entry{v́ár}{v. irreg.}{\pf{devoir}}{\\\s{pass} +\s{aci} Must, have to, be obliged to. \textit{The subject is always in the passive in this sense only}.\\+\s{dat} To owe sbd. (+\s{acc} sth.).\\\s{cond i + aci} Even if; \textit{e.g.} \w{aúrdyssa dẹće} \textit{‘even if he should fail’}.}{\s{cond i, ii} dy, \s{fut} dv́e, \s{subj} ráb’h}
\entry{vás}{n. \s{pl def}}{\pf{masses}}{The masses, the people.}{}
\entry{vaúb’hẹ}{v. irreg.}{\pf{mauvais}}{\\To be bad.\\To be wrong, incorrect, inappropriate.}{\s{fut} bíré, \s{subj} bíres; \s{comp} lẹbír, y’ŷbír, rêbír; \s{sup} réb’hír, râdvâbír}
\entry{vaûd}{n.}{\pf{monde}}{World.}{}
\entry{vaûḍ}{v.}{\pf{montrer}}{To show, display (+\s{acc} sth.).}{}
\entry{vaúd’hér}{v.}{\pf{modérer}}{To be moderate.}{}
\entry{vaût’há}{n.}{\pf{montagne}}{Mountain.}{}
\entry{váý’eb’his}{n.}{\pf{maléfice}}{Vice.}{}
\entry{váý’ýr}{n.}{\pf{malheur}}{Tragedy, misfortune.}{}
\entry{váłé}{conj.}{\pf{malgré que}}{+\s{subj} Despite that, in spite of.}{}
\entry{vé}{conj.}{\pf{mais}}{But, however, although.}{}
\entry{vê₁}{adv.}{\pf{demain}}{Tomorrow.}{}
\entry{vê₂}{n.}{\pf{main}}{Hand.}{}
\entry{véc}{n.}{\pf{mèche}}{\\A strand of hair.\\\s{pl.} Hair.}{}
\entry{véḍ}{v.}{\pf{mettre}}{To lay, put, place (+\s{acc} sth.).}{}
\entry{véhýr}{conj.}{\pf{dans la mesure où}}{Insofar as.}{}
\entry{véhýr}{v/n.}{\pf{mesure}}{\\To measure.\\Measurement.}{\s{fut} véhýrẹ́, \s{subj} véhýrs}
\entry{vér}{n.}{\pf{mère}}{\textit{(informal)} Mum, mom.}{}
\entry{vérjet’hic’h}{v.}{\pf{énergétique}}{To be vigorous, energetic.}{}
\entry{vérr}{n.}{\pf{mer}}{Sea, ocean.}{}
\entry{vérs}{interj.}{\pf{merci}}{\\Thank you. (+\s{gen} for sth.).\\\w{dyvérs fér} To thank (+\s{dat} sbd.) (+\s{gen} for sth.).}{}
\refentry{vérvá}{vér + vá}
\entry{vêt’hnâ}{adv.}{from \pf{maintenant}, lenited for unknown reasons}{Now.}{}
\refentry{véy’ýr}{baú}
\entry{víd’hẹ}{n.}{\pf{midi}}{Noon, midday.}{}
\entry{Víd’hic’hlaúry’ê}{n.}{\pf{Midichlorien}}{Midichlorian (Star Wars).}{}
\entry{vísy’ô}{n.}{\pf{émission}}{\\Emission.\\Programme, broadcast, show.}{}
\entry{vnásḍér}{n.}{\pf{monastère}}{Castle.}{}
\entry{vú}{adj.}{\pf{moult}}{Many, much, a lot of.}{}
\entry{vúb’hvâ}{n.}{\pf{movement}}{Movement, motion.}{}
\entry{vúslihé}{n.}{\pf{mousse} + \pf{lichen}}{Moss.}{}
\entry{vvâ}{n.}{\pf{maman}}{Mother.}{}
\entry{vvâ}{n.}{\pf{moment}}{Moment, instant.}{}
\entry{vvaúríhe}{v. (in)tr.}{\pf{mémoriser}}{To remember.}{\s{fut} vvaúríźe, \s{subj} vvaúríhes}
\entry{vŷ}{v.}{\pf{mener}}{To lead.}{\s{fut} menre, \s{subj} mens}
\entry{w}{v.}{\pf{enlever}}{To remove (+\s{acc} sth.).}{}
\entry{ýr}{v.}{\pf{heurter}}{To hit, strike.}{\s{fut} ýrḍ, \s{subj} ýrs}
\entry{yt’hiy’ihẹ}{v.}{\pf{utiliser}}{+\s{part} \textit{Archaic}. To use, make use of.}{\s{fut} yt’hiy’iźe, \s{subj} yt’hiy’\-ihẹs}
\entry{Yý’is}{n.}{\pf{Ulysse}}{\textit{Male given name}.}{}
\entry{y’ác’hraúníc’h}{v.}{\pf{diachronique}}{To be diachronic.}{\s{fut} y’ác’hraú\-níc’hre, \s{subj} y’ác’hraúníc’hes}
\entry{y’aúhý}{part.}{\pf{il n’y a aucun}}{There is no, there are no, there is none.}{}
\entry{ý’aúhý}{part.}{\pf{il y a aucun}}{There is, there are.}{}
\entry{y’aúý’}{v.}{back-formation from \w{y’aúý’vâ}, displacing earlier \w{y’aúý’á}}{To be violent, vehement.}{}
\entry{y’aúý’á}{v. archaic}{back-formation from \w{y’aúý’ávâ}}{To be violent, vehement.}{}
\entry{y’aúý’ávâ}{adv. archaic}{\pf{violament}}{Violently, vehemently.}{}
\entry{y’aúý’vâ}{adv.}{back-formation from \w{y’aúý’á}, displacing earlier \w{y’aúý’ávâ}}{Violently, vehemently.}{}
\entry{y’é}{pron.}{\pf{rien}}{Nothing. \textit{Like most negative polarity items, this induces negation of the verb}.}{}
\entry{y’ẹ́}{v.}{\pf{nier}}{To forbid, deny.}{\s{fut} y’ẹ́rẹ́, \s{subj} y’ẹ́s}
\entry{y’ê}{v.}{\pf{mien}}{To be mine.}{\s{fut} y’êrẹ́, \s{subj} y’ês}
\entry{y’éjúré}{n.}{\pf{siège} + \pf{tabouret}}{Chair, seat.}{}
\entry{y’ér}{adv.}{\pf{hier}}{Yesterday.}{}
\entry{y’í}{n.}{\pf{nuit}}{Night.}{}
\entry{y’í}{n.}{\pf{puits}}{Well (water source).}{}
\entry{y’íhá}{v.}{\pf{puissant}}{To be powerful, mighty, puissant.}{}
\entry{y’ír}{v. (in)tr.}{\pf{ouïr}}{To understand, listen, \textit{(rarely)} hear.}{\s{fut} aúré, \s{subj} rás}
\entry{y’ís}{conj.}{\pf{puisque}}{Considering that, since, because. \textit{Unlike \w{c’haúr}, this does not take the subjunctive; it is used to indicate the (potential) cause of something}.}{}
\entry{y’úr}{n.}{\pf{jour}}{\\Day.\\\w{órdy’úr ád’y’úr} Day after day. \textit{Contracted \s{ela} and \s{ill}}.}{}
\entry{y’ŷ}{n.}{from \w{y’ŷvéłáfrí}}{Eye.}{}
\entry{y’ŷ-}{prefix}{\pf{mieux}}{\textit{Denying comparative prefix. See grammar}.}{}
\entry{y’ŷvéłáfrí}{n. pl. archaic}{\pf{yeux de merlan frit}}{Eyes.}{}
\entry{Zauḅ}{n.}{\pf{Ésope}}{Aesop.}{}
\refentry{’sý’ýâ}{asý’ýâ}






\end{document}

